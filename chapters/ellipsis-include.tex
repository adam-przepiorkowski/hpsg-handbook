%% -*- coding:utf-8 -*-

\abstract{This chapter provides an overview of the HPSG analyses of elliptical constructions. It first discusses three types of ellipsis (nonsentential utterances, predicate ellipsis, and non-constituent coordination) that have attracted much 
attention in HPSG. It then reviews existing evidence for and against the so-called direct interpretation or WYSIWYG (what you see is what you get) perspective to ellipsis, where no invisible material is posited at the ellipsis site. The chapter then recaps the key points of existing HPSG analyses applied to the three types of ellipsis.
}

%\usepackage[text={Do~not~edit!}, color=red]{draftwatermark}

\begin{document}
\maketitle
\label{chap-ellipsis}

{\avmoptions{center}


%\if0
\section{Introduction}
\label{ellipsis-sec-introduction}

Ellipsis is a phenomenon that involves a non-canonical mapping between syntax and semantics. What appears to be a syntactically incomplete utterance still receives a semantically complete representation, based on the features of the surrounding context, be it linguistic or nonlinguistic. The goal of syntactic theory is thus to account for how the complete semantics can be reconciled with the apparently incomplete syntax. One of the key questions here relates to the structure of the ellipsis site, that is, whether or not we should assume the presence of invisible syntactic material. Section~\ref{sec-three-types-of-ellipsis} introduces three types of ellipsis (nonsentential utterances, predicate ellipsis, and non-constituent coordination) that have attracted considerable attention and received treatment within HPSG (our focus here is on standard HPSG rather than Sign-Based Construction Grammar; \citealt{Sag2012}, see also \crossrefchapteralt[Section~\ref{prop:sec-sbcg}]{properties} and \crossrefchapteralt[Section~\ref{cxg:sec-sbcg}]{cxg} on SBCG and \crossrefchapteralt[Section~\ref{sec-non-constituent-coordination}]{coordination} on non-constituent coordination). 
In Section~\ref{sec-evidence-for-invisible-material} we overview existing evidence for and against the so-called WYSIWYG (what you see is what you get) approach to ellipsis, where no invisible material is posited at the ellipsis site. Finally in Sections~\ref{sec-analyses-of-NSUs}--\ref{sec-analyses-of-noncon}, we walk the reader through three types of HPSG analyses applied to the three types of ellipsis presented in Section~\ref{sec-three-types-of-ellipsis}. Our purpose is to highlight the nonuniformity of these analyses, along with the underlying intuition that ellipsis is not a uniform phenomenon. Throughout the chapter, we also draw the reader's attention to the key role that corpus and experimental data play in HPSG theorizing, which sets it apart from frameworks that primarily rely on intuitive judgments.



\section{Three types of ellipsis}
\label{sec-three-types-of-ellipsis}

Based on the type of analysis they receive in HPSG, elliptical phenomena can be broadly divided into three types:
         nonsentential utterances, predicate ellipsis, and non-constituent coordination.\footnote{For more detailed discussion, see \citet{Kim2020}.}
          We overview the key features of these types here before discussing in greater detail how they have been brought to bear on the question of whether there is invisible syntactic structure at the ellipsis site or not. We begin with stranded XPs, which HPSG treats as nonsentential utterances, and then move on to predicate and argument ellipsis, followed by phenomena known as non-constituent coordination.


\subsection{Nonsentential utterances}
This section introduces utterances smaller than a sentence, which we refer to as \emph{\isi{nonsentential utterances}} (NSUs). These range from \emph{\isi{Bare Argument Ellipsis}} (BAE)\footnote{This term is used in \citet{CJ2005a}.} as in (\ref{1}), through fragment answers as in (\ref{2})
to direct or embedded fragment questions (\isi{sluicing}) as in (\ref{3})--(\ref{4}):

\ea A: You were angry with them.\\ B: Yeah, angry with them and angry with the situation.\label{1}\z

\ea A: Where are we? \\B: In Central Park.\label{2}\z

\ea A: So what did you think about that?\\ B: About what? \label{3}\z

\ea A: There's someone at the door. \\B: Who?/I wonder who. \label{4}\z
%
As illustrated by these examples, sluicing hosts stranded \emph{wh}-phrases and has the function of an interrogative clause, while BAE hosts XPs representing various syntactic categories and typically has the function of a clause (\citealt[313]{Ginzburg:Sag:2000}, \citealt[233]{CJ2005a}).\footnote{Several subtypes of nonsentential utterances (NSUs) can be distinguished, based on their contextual functions, which we leave open here (for a recent taxonomy, see \citealt[217]{Ginzburg2012}).}

The key theoretical question NSUs raise is whether they are, on the one hand, parts of larger sentential structures or, on the other, nonsentential structures whose semantic and morphosyntactic features are licensed by the surrounding context. To adjudicate between these views, researchers have looked for evidence that NSUs in fact behave as if they were fragments of sentences. As we will see in Section~\ref{sec-evidence-for-invisible-material}, there is evidence to support both of these views. However, HPSG doesn't assume that NSUs are underlyingly sentential structures.

\subsection{Predicate ellipsis and argument ellipsis}
This section looks at four constructions whose syntax includes null or unexpressed elements. These constructions are \emph{Post-Auxiliary Ellipsis} (PAE),\footnote{The term PAE was introduced by \citet{Sag1976} and covers cases where a non-VP element is
elided after an auxiliary verb, as in \emph{You think
I am a superhero, but I am not.}} which a term we are using here for what is more typically referred to as \emph{Verb Phrase Ellipsis} (VPE); pseudogapping; \emph{Null Complement Anaphora} (NCA); and \emph{argument drop} (or \emph{pro}-drop). PAE features stranded auxiliary verbs as in (\ref{5}), while pseudogapping, also introduced by an auxiliary verb,
has a remnant right after the pseudo gap as in (\ref{pg}). 
%
%
%
%stranded auxiliary verbs followed by XPs corresponding to complements to verbal heads %resent in the antecedent or to adjuncts (\ref{pg}).
%
 NCA is characterized by omission of complements to some lexical verbs as in (\ref{6}), while argument drop refers to omission of a pronominal subject or an object argument, as illustrated in (\ref{7}) for Polish.

\ea A: I didn't ask George to invite you.\\B: Then who did? (PAE) \label{5}\z

\ea 
Larry might read the short story, but he won't the play.
%The dentist didn't call Sally today but they might %tomorrow.
(Pseudogapping) \label{pg}\z

\ea Some mornings you can't get hot water in the shower, but nobody complains. (NCA) \label{6} \z

\ea
\gll Pia p\'{o}\'{z}no wr\'{o}ci\l a do domu. {Od razu} posz\l a spa\'{c}.\\
Pia late got to home immediately went sleep\\
\glt `Pia got home late. She went straight to bed.'
(argument drop) \label{7}
\z
%

\begin{sloppypar}
One key question raised by such constructions
 is whether these unrealized null elements should be assumed to be underlyingly present in the syntax of these constructions, and the answer is rather negative (see Section \ref{sec-evidence-for-invisible-material}). Another question is whether theoretical analyses of constructions like PAE should be enriched with usage preferences, since these constructions compete with \textit{do it/that/so} anaphora in predictable ways (see \citealt{Miller2013a} for a proposal).
\end{sloppypar}

\subsection{Non-constituent coordination}

We now focus on three instances of non-constituent coordination -- gapping (\citealt{Ross1967}), right node raising (RNR), and argument cluster coordination (ACC) -- illustrated in (\ref{9}), (\ref{8}), and (\ref{acc}), respectively.

 \ea Ethan [gave away] his CDs and Rasmus his old guitar. (Gapping)\label{9}\z

 \ea %Ethan sold and Rasmus gave away [all his CDs]. (RNR) 
 Ethan prepares and Rasmus eats [the food]. (RNR)
 \label{8}\z

\ea Harvey [gave] a book to Ethan and a record to Rasmus. (ACC) \label{acc}\z

 %
In RNR, a single constituent located in the right-peripheral position is associated with both conjuncts. In both ACC and gapping, a finite verb is associated with both (or more) conjuncts but is only present in the leftmost one. Additionally in ACC, the subject of the first conjunct is also associated with the second conjunct but is only present in the former. These phenomena illustrate what appears to be coordination of standard constituents with elements not normally defined as constituents (a cluster of NPs in (\ref{9}), a stranded transitive verb in (\ref{8}), and a cluster of NP and PP in (\ref{acc})).
%

 To handle such constructions, the grammar must be permitted to (a) coordinate non-canonical constituents, (b) generate coordinated constituents parts of which are subject to an operation akin to deletion, or (c) coordinate VPs with nonsentential utterances. As we will see, HPSG analyses of these constructions make use of all three options, which we will see throughout this chapter.
 %, including the option expressed in (b), that coordinated structures may contain unpronounced material.

\section{Evidence for and against invisible material at the ellipsis site}
\label{sec-evidence-for-invisible-material}

This section is concerned with NSUs and PAE, since this is where the contentious issues arise of
 whether there is invisible syntactic material in an ellipsis site (Sections~\ref{sec-connectivity-effects} and~\ref{sec-island-effects}) and of where ellipsis is licensed (Sections~\ref{sec-structural-mismatches} and~\ref{sec-nonlinguistic-antecedents}).  Below, we consider evidence from the literature for and against invisible structure. As we will see, the evidence is based not only on intuitive judgments, but also on experimental and corpus data, the latter being more typical of the HPSG tradition.


\subsection{Connectivity effects}
\label{sec-connectivity-effects}

Connectivity effects refer to parallels between NSUs and their counterparts in sentential structures, thus speaking in favor of the existence of silent sentential structure. We focus on two kinds here: case-matching effects and preposition-stranding effects (for other examples of connectivity effects, see \citealt{Ginzburg2018}). It's been known since \citet{Ross1967} that NSUs exhibit case-matching effects, that is, they are typically marked for the same case that is marked on their counterparts in sentential structures. (\ref{10}) illustrates this for German, where case matching is seen between a \emph{wh}-phrase functioning as an NSU and its counterpart in the antecedent (\citealt[663]{Merchant2005-proc}):

\ea
\gll Er will jemandem schmeicheln, aber sie wissen nicht wem~/ *~wen.\\
     he will someone.\textsc{dat} flatter, but they know not who.\textsc{dat}  \hspaceThis{*~}who.\textsc{acc}\\
\glt `He wants to flatter someone, but they don't know whom.'\label{10}\z


Case-matching effects are crosslinguistically robust in that they are found in the vast majority of languages with overt case marking systems, and therefore, they have been taken as strong evidence for the reality of silent structure. The argument is that the pattern of case matching follows straightforwardly if an NSU is embedded in silent syntactic material whose content includes the same lexical head that assigns case to the NSU's counterpart in the antecedent clause (\citealt{Merchant2001, Merchant2005a}). However, a language like Hungarian poses a problem for this reasoning \citep{Jacobson2016}. While Hungarian has verbs that assign one of two cases to their object NPs in overt clauses with no meaning difference, case matching is still required between an NSU and its counterpart, whichever case is marked on the counterpart. To see this, consider (\ref{11}) from \citet[356]{Jacobson2016}. The verb \emph{hasonlit} `resembles' assigns either sublative (\textsc{subl}) or allative (\textsc{all}) case to its object, but if  the sublative is selected for an NSU's counterpart, the NSU  must match this case.

\ea
A: \gll Ki-re hasonlit P\'{e}ter?\\
        who-\textsc{subl} resembles Peter\\
   \glt  `Who does Peter resemble?'\\

B: \gll J\'{a}nos-ra / ? J\'{a}nos-hoz.\\
        J\'{a}nos-\textsc{subl} {} {} J\'{a}nos-\textsc{all}\\
\glt  `J\'{a}nos.'\label{11}
\z
%
\citet{Jacobson2016} notes that there is some speaker variation regarding the (un)ac\-cepta\-bi\-li\-ty of case mismatch here, while all speakers agree that either case is fine in a corresponding nonelliptical response to (\ref{11}A). This last point is important, because it shows that the requirement of---or at least a preference for---matching case features applies to NSUs to a greater extent than it does to their nonelliptical equivalents, challenging connectivity effects.

Similarly problematic for case-based parallels between NSUs and their sentential counterparts are some Korean data. Korean NSUs can drop case markers more freely than their counterparts in nonelliptical clauses can, a point made in \citet{Morgan1989} and \citet{Kim2015}. Observe the example in (\ref{12}) from \citet[237]{Morgan1989}.

  \ea
A: \gll Nwukwu-ka        ku  chaek-ul          sa-ass-ni?\\
        who-\textsc{nom} the book-\textsc{acc} buy-\textsc{pst}-\textsc{que}\\
\glt  `Who bought the book?'\\

B: \gll Yongsu-ka / Yongsu / * Yongsu-lul.\\
        Yongsu-\textsc{nom} {} Yongsu {} {} Yongsu-\textsc{acc}\\
\glt  `Yongsu.'

B$'$: \gll Yongsu-ka            /  *  Yongsu ku  chaek-ul          sa-ass-e.\\
           Yongsu-\textsc{nom}  {} {} Yongsu the book-\textsc{acc} buy-\textsc{pst}-\textsc{decl}\\
\glt  `Yongsu bought the book.'\\
\label{12}
\z
%
When an NSU  corresponds to a nominative subject in the antecedent (as in (\ref{12}B)), it can either be marked for nominative or be caseless.
However, replacing the same NSU  with a full sentential answer, as in (\ref{12}B$'$), rules out case drop from the subject. This strongly suggests that the case-marked and caseless NSUs couldn't have identical source sentences if they were to derive via PF-deletion (deletion in the phonological component).\footnote{Nominative (in Korean) differs in this respect from three other structural cases in the language---dative, accusative, and genitive---in that these three may be dropped from nonelliptical clauses \citep[see][]{Morgan1989, Lee2016, Kim2016}. However, see \citet{Mueller2002b} for a discussion of German dative and genitive as lexical cases.}  Data like these led \citet{Morgan1989} to propose that not all NSUs have a sentential derivation, an idea later picked up in \citet{Barton1998}.

%\iffalse{
The same pattern is associated with semantic case. That is, in (\ref{13}), if an NSU is case-marked, it needs
to be marked for comitative case like its counterpart in the A-sentence, but it may also simply be caseless. However,  
being caseless is not an option for the NSU's counterpart in a sentential response to A \citep{Kim2015}.
%\todostefan{glossing did not match for
%  \emph{ha-ess-e}. Please check}

\ea
A:
\gll Nwukwu-wa          hapsek-ul                     ha-yess-e?\\
     who-\textsc{com}   sitting.together-\textsc{acc} do-\textsc{pst}-\textsc{que}\\
\glt  `With whom did you sit together?'\\

B:
\gll Mimi-wa. 			/ Mimi.\\
     Mimi-\textsc{com} {} Mimi\\
\glt `With Mimi.' / `Mimi.' \label{13}\z
%
The generalization for Korean is then that NSUs may be optionally realized as caseless, but may never be marked for a different case than is marked on their counterparts.
%}\fi 

Overall, case-marking facts show that there is some morphosyntactic identity between NSUs and their antecedents, though not to the extent that NSUs have exactly the features that they would have if they were constituents embedded in sentential structures. The Hungarian facts also suggest that those aspects of the argument structure of the appropriate lexical heads present in the antecedent that relate to case licensing are relevant for an analysis of NSUs.\footnote{Hungarian and Korean are in fact not the only problematic languages; for a list, see \citet{Vicente2015}.}

The second kind of connectivity effects goes back to \citet{Merchant2001, Merchant2005a} and highlights apparent links between the features of NSUs and \emph{wh}- and focus movement (leftward movement of a focus-bearing expression). The idea is that prepositions behave the same under \emph{wh}- and focus movement as they do under clausal ellipsis, that is, they pied-pipe or strand in the same environments. If a language (e.g., English) permits preposition stranding under \emph{wh}- and focus movement (\emph{What did Harvey paint the wall with?} vs.\ \emph{With what did Harvey paint the wall?}), then NSUs may surface with or without prepositions, as illustrated in (\ref{14}) for sluicing and BAE (see Section~\ref{sec-analyses-of-NSUs} for a theoretical analysis of this variation).
%
\ea A: I know what Harvey painted the wall with.\\B: (With) what?/(With) primer.\label{14}\z
%
If there were indeed a link between preposition stranding and NSUs, then we would expect prepositionless NSUs to only be possible in languages with preposition stranding. This expectation is, however, disconfirmed by an ever-growing list of non-preposition stranding languages that do feature prepositionless NSUs: Brazilian Portuguese (\citealt{AlmeidaYoshida2007}), Spanish and French (\citealt{Rodrigues2006}), Greek (\citealt{Molimpakis2018}), Bahasa Indonesia (\citealt{Fortin2007}), %Emirati Arabic \citep{Leung2014},
 Russian \citep{Philippova2014}, Polish \citep{Szczegielniak2008, Sag2011, Nykiel2013}, %Czech \citep{Caha2011},
Bulgarian \citep{Abels2017}, Serbo-Croatian \citep{Stjepanovic2008, Stjepanovic2012}, Mauritian \citet{Abeille2019}, and %Saudi 
Arabic \citep{Leung2014, Alshaalan2020}. A few of these studies have presented experimental evidence that prepositionless NSUs are acceptable, though for reasons still poorly understood, they typically do not reach the same level of acceptability as their variants with prepositions do (see \citealt{Nykiel2013} for Polish, \citealt{Molimpakis2018} for Greek, and \citealt{Alshaalan2020} for Saudi Arabic). It is worth noting in this regard that the work on connectivity effects that follows the HPSG tradition is based on a solid foundation of empirical evidence to a larger extent than work grounded in the Minimalist tradition (see \citealt{Sag2011, Nykiel2013} for experimental work on Polish, and \citealt{Nykiel2015, Nykiel2017, Nykiel2020} for corpus work on English).

It is evident from this research that there is no grammatical constraint on NSUs that keeps track of what preposition-stranding possibilities exist in any given language. On the other hand, it doesn't seem sufficient to assume that NSUs can freely drop prepositions, given examples of sprouting like (\ref{15}), in which prepositions are not omissible (see \citealt{Chung1995}).\footnote{However, as noted by \citet{Ginzburg:Sag:2000} and \citet{Hardt2020}, there are cases where prepositions are dropped from NSUs that serve as adjuncts rather then arguments, as in \textit{A: I am going to the concert tonight. B: (At) What time?}}  As noted by \citet{Chung1995},
the difference between the
merger type of sluicing (\ref{14}) and the sprouting type of sluicing (\ref{15}) is that there is an explicit phrase that the NSU corresponds to in the former but not in the latter (in the HPSG literature, this phrase is termed a Salient Utterance by \citealt[313]{Ginzburg:Sag:2000} or a Focus-Establishing Constituent by \citealt{Ginzburg2012}). 

\ea A: I know Harvey painted the wall.\\B: *(With) what?/Yeah, *(with) primer.\label{15}\z
%
The challenge posed by (\ref{15}) is how to ensure that
the NSU is a PP matching the implicit PP argument in 
the A-sentence (see the discussion
around %joanna What's this example?
(\ref{spr}) for further detail). This challenge has not received much attention in the HPSG literature, though see \citet{Kim2015}.



\subsection{Island effects}
\label{sec-island-effects}

One of the predictions from the view that NSUs are underlyingly sentential is that they should respect \isi{island} constraints on long-distance movement (see \crossrefchapteralt{islands} for a discussion of islands in HPSG). But as illustrated below, NSUs (both sluicing and BAE) exhibit island-violating behavior.%joanna Why this footnote?
\footnote{As hinted earlier, the derivational approaches
need to move a remnant or NSU to the sentence initial position and delete a clausal constituent since only constitutents can be deleted. See \citet{Merchant2001,Merchant2010} for details.} The NSU  in (\ref{16}) would be illicitly extracted out of an adjunct (\textit{*Where does Harriet drink scotch that comes from?}) and the NSU  in (\ref{17}) would be extracted out of a complex NP (\textit{*The Gay Rifle Club, the administration has issued a statement that it is willing to meet with}).\footnote{\citet{Merchant2005a} argued that BAE, unlike sluicing, does respect island constraints, an argument that was later challenged \citep[see, e.g.,][239]{CJ2005a}, \citealt{Griffiths2014}. However, \citet{Merchant2005a} focused specifically on pairs of \emph{wh}-interrogatives and answers to them, running into the difficulty of testing for island-violating behavior, since a well-formed \emph{wh}-interrogative antecedent couldn't be constructed.}

\ea A: Harriet drinks scotch that comes from a very special part of Scotland.\\B: Where? \citep[245]{CJ2005a} \label{16}\z

\ea A: The administration has issued a statement that it is willing to meet with one of the student groups.\\B: Yeah, right---the Gay Rifle Club. \citep[245]{CJ2005a} \label{17}\z

Among \citeauthor{CJ2005a}'s (\citeyear[245]{CJ2005a}) examples of well-formed island-violating NSUs are also sprouted NSUs (those that correspond to implicit phrases in the antecedent) like (\ref{18})--(\ref{19}).

\ea A: John met a woman who speaks French.\\B: With an English accent?\label{18}\z

\ea A: For John to flirt at the party would be scandalous. \\B: Even with his wife?\label{19}\z
Other scholars assume that sprouted NSUs are one of the two kinds of NSUs that respect island constraints, the other kind being contrastive NSUs, illustrated in (\ref{20}) \citep{Chung1995, Merchant2005a, Griffiths2014}.

\ea A: Does Abby speak the same Balkan language that Ben speaks?\\
B: *No, Charlie. \citep[688]{Merchant2005a}  \label{20}\z
%
\citet{Schmeh2015} further explore the acceptability of NSUs preceded by the response particle \textit{no} like those in (\ref{20}) compared to NSUs introduced by the response particle \textit{yes}, depicted in (\ref{21}). (\ref{20}) and (\ref{21}) differ in terms of discourse function in that the latter supplements the antecedent rather than correcting it, a discourse function signaled by the response particle \textit{yes}.

\ea A: John met a guy who speaks a very unusual language. \\B: Yes, Albanian. \citep[245]{CJ2005a} \label{21}\z
%
\citet{Schmeh2015} find that corrections with \textit{no} lead to
lower acceptability ratings compared to 
supplementations with \textit{yes} and propose that this follows from the fact that corrections induce greater processing difficulty than supplementations do, hence the acceptability difference between (\ref{20}) and (\ref{21}). This finding makes it plausible that the perceived degradation of island-violating NSUs could ultimately be attributed to nonsyntactic factors, e.g., the difficulty of successfully computing a meaning for them.

In contrast to NSUs, many instances of PAE appear to respect island constraints, as would be expected if there were unpronounced structure from which material was extracted. An example of a relative clause island is depicted in (\ref{22}) (note that the corresponding sluicing NSU  is fine).


\ea[*]{
They want to hire someone who speaks a Balkan language, but I don't remember which they do [\sout{want to hire someone who speaks \jbtr}]. \citep[6]{Merchant2001}\label{22}
}
\z
(\ref{22}) contrasts with well-formed island-violating examples like (\ref{23}) and (\ref{24}), as observed by \citet[]{Miller2014} and \citet[]{Ginzburg2018}.\footnote{\citet{Miller2014} cites numerous corpus examples of island-violating pseudogapping.}
%
%
\eal
\ex{He managed to find someone who speaks a Romance language, but a Germanic language, he didn't [\sout{manage to find someone who 
speaks \jbtr}]. \citep[90]{Ginzburg2018}\label{23}}

\ex{He was able to find a bakery where they make good baguette, but croissants, he couldn't [\sout{find a bakery where they make 
good \jbtr}]. \citep[90]{Ginzburg2018}\label{24}}
\zl
%

As \citet{Ginzburg2018} rightly point out, we do not yet have a complete understanding of when or why island effects show up in PAE. Its behavior is at best inconsistent, failing to provide convincing evidence for silent structure.


\subsection{Structural mismatches}
\label{sec-structural-mismatches}

Because structural mismatches are rare or absent from NSUs \citep[see][]{Merchant2005a, Merchant2013},\footnote{Given the assumption that canonical sprouting NSUs have VP antecedents, as in (\ref{15}), \citet[95]{Ginzburg2018} cite examples---originally from \citet[13]{Beecher2008}---of sprouting NSUs with nominal, hence mismatched, antecedents, e.g., (i).
	\ea We're on to the semi-finals, though I don't know who against.\z
%
	Further examples where NSUs refer to an NP or AP antecedent appear in COCA (Corpus of Contemporary American English):
%
	\ea  A: Well, it's a defense mechanism. B: Defense against what?\z
	\ea Our Book of Mormon talks about the day of the Lamanite, when the church would make a special effort to build and reclaim a fallen people. And some people will say, Well, fallen from what? \z
%
	The NSUs in (ii)--(iii) repeat the lexical heads whose complements are being sprouted (\textit{defense} and \textit{fallen}), that is, they contain more material than is usual for NSUs (cf. (i)). It seems that without this additional material it would be difficult to integrate the NSUs into the propositions provided by the antecedents and hence to arrive at the intended interpretations.} this section focuses on PAE and developments surrounding the question of which contexts license it. In a seminal study of anaphora, \citet{Hankamer1976} classified PAE as a surface anaphor with syntactic features closely matching those of an antecedent present in the linguistic context. They argued in particular that PAE is not licensed if it mismatches its antecedent in voice. Compare the following two examples from \citet[327]{Hankamer1976}.
%\todosatz{no reference given}.

\eal
\ex[*]{
	The children asked to be squirted with the hose, so we did.  \label{25}
}
\ex[]{
	The children asked to be squirted with the hose, so they were. \label{26}
}
\zl
This proposal places tighter structural constraints on PAE than on other verbal anaphors (e.g., \textit{do it/that}) in terms of identity between an ellipsis site and its antecedent. This has prompted extensive evaluation in a number of corpus and experimental studies in the subsequent decades. Below are examples of acceptable structural mismatches reported in the literature, ranging from voice mismatch in (\ref{27}) to nominal antecedents in (\ref{28}) and to split antecedents in (\ref{29}).\footnote{\citet[87]{Miller2014} also reports cases of structural mismatch with English comparative pseudogapping, as in (i) from COCA:

\ea These savory waffles are ideal for brunch, served with a salad as you would a quiche. (Mag).\z
%
See also \citet{Abeille2016} for examples of voice mismatch in French RNR. 
}

\eal

\ex This information could have been released by Gorbachev, but he chose not to \jbtr. \citep[37]{Hardt1993} \label{27}

\ex Mubarak's survival is impossible to predict and, even if he does \jbtr, his plan to make his son his heir apparent is now in serious jeopardy. \citep[7]{Miller2014a} \label{28}

\ex Mary wants to go to Spain and Fred wants to go to Peru but because of limited resources only one of them can \jbtr. \citep[128]{Webber79a} \label{29}
\zl

There are two opposing views that have emerged from the empirical work regarding the acceptability and grammaticality of structural mismatches under PAE. The first view takes mismatches to be grammatical and connects degradation in acceptability to violation of certain independent constraints on discourse \citep{Kehler2002, Miller2011, %Kertz2013,
Miller2014, Miller2014a, Miller2014b} or processing \citep{Kim2011}. Two types of PAE have been identified on this view through extensive corpus work---auxiliary choice PAE and subject choice PAE---each with different discourse requirements with respect to the antecedent \citep{Miller2011, Miller2014a, Miller2014b}. The second view assumes that there is a grammatical ban on structural mismatch, but violations thereof may be repaired under certain conditions; repairs are associated with differential processing costs compared to matching ellipses and antecedents \citep{Arregui2006, Grant2012}. If we follow the first view, it is perhaps unexpected that voice mismatch should consistently incur a greater acceptability penalty under PAE than when no ellipsis is involved, as recently reported in \citet{Kim2018}.\footnote{But see \citet{Abeille2016} for experimental results
that show no acceptability penalty for voice mismatch in French Right Node Raising.} \citet{Kim2018} stop short of drawing firm conclusions regarding the grammaticality of structural mismatches, but one possibility is that the observed mismatch effects reflect a construction-specific constraint on PAE. HPSG analyses take structurally mismatched instances of PAE to be unproblematic and fully grammatical, while also recognizing construction-specific constraints: discourse or processing constraints formulated for PAE may or may not extend to other elliptical constructions, such as NSUs (see \citealt{Abeille2016,Ginzburg2018} for this point).


\subsection{Nonlinguistic antecedents}
\label{sec-nonlinguistic-antecedents}

Like structural mismatches, the availability of nonlinguistic (situational) antecedents for an ellipsis points to the fact that it needn't be interpreted by reference to and licensed by a structurally identical antecedent. Although this option is somewhat limited, PAE does tolerate nonlinguistic antecedents, as shown in (\ref{30})--(\ref{31}) \citep[see also][]{Hankamer1976, Schachter1977}.
\ea Mabel shoved a plate into Tate's hands before heading for the sisters' favorite table in the shop. ``You shouldn't have.'' She meant it. The sisters had to pool their limited resources
just to get by. \citep[ex.\ 23]{Miller2014b}\label{30}\z

\ea Once in my room, I took the pills out. ``Should I?'' I asked myself. \citep[ex.\ 22a]{Miller2014b}\label{31}\z
%
\citet{Miller2014b} note that such examples are exophoric PAE
involving no linguistic antecedent for the ellipsis but 
just a situation where the speaker articulates their opinion about the action involved. \citet{Miller2014b} provide an extensive critique of the earlier work on the ability of PAE to take nonlinguistic antecedents, arguing for a streamlined discourse-based explanation that neatly captures the attested examples as well as examples of structural mismatch like those discussed in Section~\ref{sec-structural-mismatches}. The important point here is again that PAE is subject to construction-specific constraints which limit its use with nonlinguistic antecedents.

NSUs appear in various nonlinguistic contexts as well. \citet{Ginzburg2018} distinguish three classes of such NSUs: sluices (\ref{32}), exclamative sluices (\ref{33}), and declarative fragments (\ref{34}).

\ea (In an elevator) What floor? \citep[298]{Ginzburg:Sag:2000}\label{32}\z

\ea It makes people ``easy to control and easy to handle,'' he said, ``but, God forbid, at what cost!'' \citep[96]{Ginzburg2018}
\label{33}\z

\ea BOBADILLA turns, gestures to one of the other men, who comes forward and gives him a roll of parchment, bearing the royal seal. ``My letters of appointment.'' (COCA)\label{34}\z
%
In addition to being problematic from the licensing point of view, NSUs like these have been put forward as evidence against the idea that they are underlyingly sentential, because it is unclear what the structure that underlies them would be. There could be many potential sources for
these NSUs 
\citep[see][306]{CJ2005a}. %joanna For instance, the NSU in (\ref{34}) has no determined source sentence, but could have more than one source sentences.
\footnote{This is not to say that a sentential analysis of fragments without linguistic antecedents hasn't been attempted. For details of a proposal involving a ``limited ellipsis'' strategy, see \citet{Merchant2005a} and \citet{Merchant2010}.}
%\todosatz{Merchant 2010: no reference given}


\section{Analyses of NSUs}
\label{sec-analyses-of-NSUs}

It is worth noting at the outset that the analyses of NSUs within the framework of HPSG are based on an elaborate theory of dialog  and on a wider range of data than is common practice in the ellipsis literature \citep{Ginzburg1994, Ginzburg2004, Ginzburg2014a, Larsson2002, Purver2006, Fernandez2006, Fernandez2002, Fernandez2007, Ginzburg2010, Ginzburg2014b, Ginzburg2012, Ginzburg2013, Kim2019}. Existing analyses of NSUs go back to \citet{Ginzburg:Sag:2000}, who recognize declarative fragments as in (\ref{34a}) and two kinds of sluicing NSUs: direct sluices as in (\ref{35}) and reprise sluices as  in (\ref{36}) (the relevant fragments are bolded). The difference between direct and reprise sluices lies in the fact that the latter are requests for clarification of any part of the antecedent. For instance, in (\ref{36}), the referent of \textit{that} is unclear to the interlocutor.

\ea ``I was wrong.'' Her brown eyes twinkled. ``Wrong about what?'' ``\textbf{That night}.'' (COCA) \label{34a}\z

\ea ``You're waiting,'' she said softly. ``\textbf{For what?}'' (COCA) \label{35} \z

\ea ``Can we please not say a lot about that?'' ``\textbf{About what?}'' (COCA) \label{36} \z


These different types of NSUs are derived from the \citet[333]{Ginzburg:Sag:2000} hierarchy of clausal types depicted in Figure~\ref{fig-cltypes}.


\begin{figure}
\centering
\begin{forest}
type hierarchy, instances
[phrase
  [clausality,partition
    [clause
      [core-cl
        [inter-cl
          [is-int-cl
            [dir-is-int-cl,tier=bottom % the types dir-is-int-cl, slu-int-cl and decl-frag-cl
                                       % will be drawn on the same line.
              [\textsc{Who?}\\\textsc{Jo}?\\\textbf{Jo?}
]]]]
        [decl-cl
          [slu-int-cl, % slu-int-cl is drawn as child of decl-cl but we do not want to have
                               % an edge to decl-cl
           tier=bottom,
           edge to=!r1111, % see explanation below for node pathes
           edge to=!r2111, % see explanation below
           no edge         % this has to be said after the other edges are drawn, if it is said
                           % before, no edge will be drawn ...
           [\textbf{Who?}\\\textbf{who}
]]]]]]
  [headedness,partition
    [hd-ph
      [hd-only-ph
        [hd-frag-ph
          [decl-frag-cl,tier=bottom,
                        edge to=!r1112 % draw an edge to the node which is the first child of
                                % the root node (!r1 = clausality) 's first child (clause) 's first
                                % child (core-cl) 's second child (2 = decl-cl)
           [\textbf{Bo}]]]]]]]
\end{forest}
\caption{Clausal hierarchy for fragments (\citealt[333]{Ginzburg:Sag:2000})}\label{fig-cltypes}
\end{figure}
%
%
 NSUs like declarative fragments (\type{decl-frag-cl}) are associated with type \type{hd-frag-ph} (headed-fragment phrase) and \type{decl-cl} (declarative clause), while direct sluices (\type{slu-int-cl}) and reprise sluices (\type{dir-is-int-cl}) are associated with type \type{hd-frag-ph} and \type{inter-cl} (interrogative clause). The type \type{slu-int-cl} is permitted to appear in independent and embedded clauses, hence it is underspecified for the head feature IC (independent clause). This specification contrasts with that of declarative fragments and reprise sluices, with both specified as [IC +], which  \citet[305]{Ginzburg:Sag:2000} use to block declarative fragments and reprise sluices from appearing in embedded clauses (e.g., \textit{A: What do they like? B: *I doubt bagels}).\footnote{This feature specification, however, needs to be remedied for  the speakers who accept examples like \textit{A: What does Kim take for breakfast? B: Lee says eggs.}}
%
\citet[304]{Ginzburg:Sag:2000} make use of the constraint shown in (\ref{hf-cx}), in which 
the two contextual attributes \textsc{sal-utt} and \textsc{max-qud} play key roles in ellipsis resolution
(we have added information about the \textsc{max-qud} to generate NSUs):
%\footnote{
%The notation \pi^{i}$, adopted from \citet{GS00}, indicates a set of %parameters.}
%
\eas
\label{hf-cx}
Head-Fragment Construction:\\
\avmtmp{
[cat & \normalsize{S}\\ %\[HEAD v\]\\
             %CONT &\[NUCL \@1\]\\
 ctxt & [max-qud & [params & \type{neset}]\\
         sal-utt & \{  [cat & \2\\
                        cont|ind & \type{i} ] \} ] ]}
$\rightarrow$
\avmtmp{
[ cat &     \2[head &  \type{nonverbal}]\\
  cont & [ind \type{i}]
]
}
%% \[CAT &S\\ %\[HEAD v\]\\
%%   %CONT &\[NUCL \@1\]\\
%%   CTXT & \[MAX-QUD $\lambda$\{$\pi$$^{i}$\}\\
%%   SAL-UTT \{  \[CAT \@2\\
%%                          CONT\ \[IND & \emph{i}\\
%%                                   \]\]\}\]\]

%%                     \ \ $\rightarrow$\ \
%% \[  CAT &  \@2\\
%%    CONT  &\[IND & \emph{i}\]
%% \]
%% \end{avm}
\zs
%
%
This constructional constraint first allows 
any non-verbal phrasal category (NP, AP, VP, PP, AdvP) to be mapped onto a sentential utterance as long as it corresponds to a Salient Utterance (\textsc{sal-utt}).\footnote{\citet{Ginzburg2012} uses the notion of the Dialogue Game Board (DGB) to keep track of all information relating to the common ground between interlocutors. The DGB is also the locus of contextual updates arising from each new question-under-discussion that is introduced.}
 This means that
the head daughter's syntactic category must match that of the \textsc{sal-utt}, which is an attribute supplied by the surrounding context as a (sub)utterance of another contextual attribute---the Maximal Question under Discussion (\textsc{max-qud}). The context gets updated with every new question-under-discussion, and \textsc{max-qud} represents the most recent question-under-discussion appropriately specified for the feature \textsc{params}, whose value is a nonempty set (\textit{neset}) of parameters.\footnote{As defined in \citet[304]{Ginzburg:Sag:2000}, the feature \textsc{max-qud} is also specified for \textsc{prop} (proposition) as its value. For the sake of simplicity, we suppress this feature here and further 
represent the value of
\textsc{max-qud} as a lambda abstraction, as in Figure~\ref{fig-the-mike}. 
See \citet[304]{Ginzburg:Sag:2000} for the
exact feature formulations of \textsc{max-qud}.} \textsc{sal-utt} is the (sub)utterance with the widest scope within \textsc{max-qud}. To put it informally, \textsc{sal-utt} represents a (sub)utterance of a \textsc{max-qud} that has not been resolved yet. Its feature \textsc{cat} supplies information relevant for establishing morphosyntactic identity with an NSU, that is, syntactic category and case information, and (\ref{hf-cx}) requires that an NSU  match this information. 

For illustration, consider the following exchange including 
a declarative fragment:
%
\ea
A: What did Barry break? \\
B: The mike.\label{37}
\z
%
In this dialog, the fragment \emph{The mike} 
corresponds to the \textsc{sal-utt} \emph{what}. Thus  
  the constructional constraint in (\ref{hf-cx}) would license an NSU structure
  like Figure~\ref{fig-the-mike}.
  
\begin{figure}[h!]
{\centering
\begin{forest}
sm edges without translation
[S\\
\avmtmp{
[cat & [ head & v]\\
  %CONT & \[NUCL \@1\]\\
 ctxt & [max-qud & !$\lambda_{i}[break(b,i)]$! \\
         sal-utt & \{ [cat  \2\\
                       cont|ind \type{i} ] \} ] ]}
[NP\\
 \avmtmp{
  [cat  & \2\\
   cont & [ ind \type{i} ] ]%\\
   %\PARAMS & \{$\pi$$^{i}$\}
  }
 [The mike, roof]]]
\end{forest}
}
\caption{Structure of a declarative fragment clause}\label{fig-the-mike}
\end{figure}
%
%
As illustrated in the figure, uttering the \emph{wh}-question in (\ref{37}A) evokes the QUD asking
the value of the variable \textit{i} linked to the object that Barry broke. The NSU 
\textit{The mike} matches that value.
The structured dialogue thus plays a key role in the retrieval of the propositional semantics for the NSU. %for the unexpressed expression. 

This constructional approach
has the advantage that it gives us a way of capturing the problems that \citet{Merchant2001, Merchant2005a} faces with respect to misalignments between preposition stranding under \emph{wh}- and focus movement and the realization of NSUs as NPs or PPs, as discussed in Section~\ref{sec-connectivity-effects}. Because the categories of \textsc{sal-utt} discussed in \citet{Ginzburg:Sag:2000} are limited to nonverbal, \textsc{sal-utt}s can surface either as NPs or PPs. As long as both of these syntactic categories are stored in the updated contextual information, an NSU's \textsc{cat} feature will be able to match either of them (See \citealt{Sag2011} for discussion of this possibility with respect to Polish and \citealt{Abeille2019} with respect to Mauritian).    

 Another advantage of this analysis of NSUs is that the content of \textsc{max-qud} can be supplied by either linguistic or nonlinguistic context. \textsc{max-qud} provides the propositional semantics for an NSU  and is, typically, a unary question. In the prototypical case, \textsc{max-qud} arises from the most recent \emph{wh}-question uttered in a given context,
as in (\ref{37}), but can also arise (via accommodation) from other forms found in the context, such as constituents in direct sluicing as in (\ref{39}), or from a nonlinguistic context as in (\ref{40}).
%
%
%\ea
%A: Barry broke the MIKE. \\
%B: Yes, the only one we had.\label{38}
%\z

\ea
A: A friend of mine broke the mike. \\
B: Who?\label{39}
\z

\ea
(Cab driver to passenger on the way to the airport)
A: Which airline?\label{40}
\z
%
The analysis of such direct sluices differs only slightly from that illustrated for (\ref{37}), and in fact all existing analyses of NSUs \citep{Sag2011,Ginzburg2012, Abeille2014, Kim2015, Abeille2019, Kim2019} are based on (\ref{hf-cx}). The direct sluice would have the structure given in Figure~\ref{fig-slu}.
%
%, we illustrate how it is applied to the declarative fragment in (\ref{37}) and the reprise sluice %n (\ref{39}).
%
%
\begin{figure}
{\centering
\begin{forest}
sm edges without translation
[S\\
\avmtmp{
[cat & [ head & v]\\
  %CONT & \[NUCL \@1\]\\
 ctxt & [max-qud & !$\lambda_{i}[break(i,m)]$! \\
         sal-utt & \{ [cat  \2\\
                       cont|ind \type{i} ] \} ] ]}
[NP\\
 \avmtmp{
  [cat  & \2\\
   cont & [ ind \type{i} ] ]%\\
   %\PARAMS & \{$\pi$$^{i}$\}
  }
 [Who]]]
\end{forest}
}
\caption{Structure of a sluiced interrogative clause}\label{fig-slu}
\end{figure}
%
%
\noindent The analyses in Figures~\ref{fig-the-mike} and~\ref{fig-slu} differ
only in the value of the feature \textsc{cont} (Content): in the former it is a proposition and in the latter a question.\footnote{In-situ languages like Korean and Mandarin allow pseudosluices (sluices with a copula
verb), which has lead to proposals that posit 
cleft clauses as their sources (\citealt{Merchant2001}).
However, \citet{Kim2015}
suggests that a cleft-source analysis does not hold for languages like
Korean since there is one clear difference between sluicing and cleft
constructions: the former allows multiple remnants, while clefts do not license multiple foci. 
See \citet{Kim2015} for a detailed discussion that differentiates
sluicing in embedded clauses (pseudosluices with the
copula verb) from direct sluicing in root clauses,
as \citet[329]{Ginzburg:Sag:2000} do.} 

This construction-based analysis, in which dialogue updating plays
a key role in the licensing of NSUs, also offers a direction
for handling the contrast between merger (\ref{paint-mer}) and sprouting (\ref{paint-spr}) examples (recall the discussion in Section \ref{sec-connectivity-effects}). %Consider similar examples in the following:
%like the following from COCA:
%
\eal
\ex A: I heard that the boy painted the wall with something. B: (With) what?  
\label{paint-mer}
\ex A: I heard that the boy painted the wall. B: *(With) what? \label{paint-spr}
\zl
%\zl
%
The difference between (\ref{paint-mer}) and (\ref{paint-spr})  is that the preceding antecedent clause in the
former includes an overt correlate for the NSU, but in (\ref{paint-spr}), 
all there is is just a PP that is implicitly provided
by the argument structure 
of the verb \textit{paint}.
%and \emph{losses}. 
%We illustrate this idea for the verb \emph{w}. 
%\citep[330]{Ginzburg:Sag:2000} %\citet{Kim2015} 
%
%%Adopting the analysis of %\citet{Ginzburg:Sag:2000}, 
\citet{Kim2015} suggests the following way of analyzing the contrast. Consider the argument structure of the lexeme \textit{paint}:

\ea
\label{lxm-paint}
Lexemic \textit{paint}:\\
\avmtmp{
[ phon   & \phonliste{ paint }\\
  cat  & [arg-st & < NP$_i$, NP$_j$,  PP[pform & with\\
                         ind   & x ]>]\\
  cont &     ! $paint(i,j,x) $ ! ]
}
\z
%
%
As represented in (\ref{lxm-paint}), the verb \emph{paint}
takes three arguments. But
note that the PP argument can be realized
either as an overt PP or a \textit{pro} expression. In the framework of HPSG, this optionality of an argument to be either realized as a complement or not expressed is represented as the Argument Realization Principle  (ARP; \citealt{ginzburg-miller-ellipsis-handbook}; \crossrefchapteralt[\page \pageref{page-argument-realization-principle}]{properties}):\footnote{This
ARP is an extended version of \citet[171]{Ginzburg:Sag:2000} and \citet[11]{Bouma:Malouf:Sag:01} where \textit{noncanon-ss} is specified to be one of its subtypes, \textit{gap-ss}.}
%
\inlinetodostefan{AVM: For Stefan, take care of the space after $\oplus$}

\ea
\label{arp}
Argument Realization Principle:\\
\type{v-wd} \impl
\avmtmp{
[ subj  & \1\\
 comps & \2\\
 arg-st & \1 \+ \2 \+ \type{list}(\type{noncanon-ss})
 ]
}
%% \begin{avm}
%% \emph{v-word} \;   $\Rightarrow$ \;
%% \[SYN\|VAL \[SUBJ & \@A\\
%%                  COMPS & \@B $\ominus $ list(\emph{pro})\]\\
%%   ARG-ST \@A $\oplus$ \@B\]
%%   \end{avm}
\z
The ARP tells us that the elements in the \textsc{arg-st} are realized as the \textsc{subj} and
\textsc{comps} elements and further that a noncanonical  element with syntactic-semantic information (including \textit{gap-ss} (gap-synsem) and \textit{pro}) in the argument structure need not be realized in the syntax, permitting mismatch between argument structure and syntactic valence features (see Section 5). 

In accordance with the ARP, the lexeme in (\ref{lxm-paint}) will then have at least the following two realizations, depending on the realization of the optional PP argument:

\ea
\label{paint2}
%Lexical item for \textit{wait}:\\
\avmtmp{
[ phon   & \phonliste{painted}\\
  cat  &[ subj  & < \1NP$_i$  >\\
          comps & <\2NP$_j$, \3PP[pform & \textit{with}\\
                            ind   &  x] >\\
                             arg-st & < \1NP, \2NP, \3PP>\\
 ]\\
  cont &     ! $paint(i,j,x) $ ! ]
}
\z

\ea
\label{paint}
%Lexical item for \textit{wait}:\\
\avmtmp{
[ phon   & \phonliste{ painted}\\
 cat  &[ subj  & < \1NP$_i$  >\\
                 comps & <\2NP$_j$ >\\
                   arg-st & < \1NP$_i$, \2NP$_j$, \3PP$_x$[\type{pro}]>\\
  ]\\
  cont &     ! $paint(i,j,x) $ ! ]
}\z
 %% \begin{avm}
 %% \[FORM \q<\emph{wait}\q>\\
 %%   ARG-ST \<NP\jbsub{\emph{i}}, PP\jbsub{\emph{x}}\>\\
 %%   SYN\[SUBJ \<NP[\emph{overt}]\>\\
 %%        COMPS \<PP[\emph{ini}]\>\]\\
 %%   SEM \emph{wait}(\emph{i, x})\]
 %%   \end{avm}
%\z
%
%The lexical information specifies that the second %argument of \textit{wait} can be an unrealized PP %while the first argument needs to be an overt NP.
%
The realization with an overt PP complement
in (\ref{paint2}) would project
a merger sentence like (\ref{paint-mer}) while the one with a covert PP in (\ref{paint}) would license the sprouting example in (\ref{paint-spr}). Each
of these two realizations would then license 
the following partial
VP structures, as given in Figure~\ref{paint-mer-st} and
Figure~\ref{paint-spr-st}.

\begin{figure}
\begin{forest}
[VP
  [ V\\
  \avmtmp{
      [ %head \2 \\
        subj < \1 >\\
        comps <\2, \3 >\\
        arg-st < \1NP, \2NP, \3PP> ]}
      [painted]]
  [\ibox{2}NP
     [the wall]]
  [\ibox{3}PP
    [P
      [with]]
    [NP
     [something]]
      ]]
\end{forest}
\caption{Structure of a merger antecedent}\label{paint-mer-st}
\end{figure}

\begin{figure}
\begin{forest}
[VP
  [ V\\
  \avmtmp{
      [ %head \2 \\
        subj < \1 >\\
        comps <\2>\\
        arg-st < \1NP, \2NP, PP[\type{pro}]> ]}
      [painted]]
  [\ibox{2}NP
         [the wall]]
      ]
\end{forest}
\caption{Structure of a sprouting antecedent}\label{paint-spr-st}
\end{figure}
%
Let us consider the NSU with the merger
antecedent in (\ref{paint-mer}). In this case, 
the NSU can be either the NP \emph{What?} or the PP
\emph{With what?} because of the available
DGB information triggered by the previous discourse.
As seen from the structure in Figure~\ref{paint-mer-st}, the
antecedent clause activates not only the PP information but also its
internal structure, including the NP within it. The NSU can thus be anchored to
either of these two, as given in the following:
%
%
\eal
\ex
\label{pp-dgb}
\avmtmp{
[
 ctxt|sal-utt \{[cat &  \upshape PP [%\type*{ini}
                          pform & with\\
                          ind   & x ]\\
              cont & !$\textit{paint}(i,j,x)$ ! ]\}]
}
\ex
\label{np-dgb}
\avmtmp{
[
 ctxt|sal-utt \{[cat &  \upshape NP [%\type*{ini}
                          %pform & with\\
                          ind   & x ]\\
              cont & !$\textit{paint}(i,j,x)$ ! ]\}]
}
\zl
%
The \textsc{sal-utt} in (\ref{pp-dgb}) is the PP
\emph{with something}, projecting  \emph{With what?} as a well-formed
NSU in accordance with (\ref{hf-cx}).  Since the overt PP also activates
its prepositional NP object, the discourse can supply the NP
as another possible \textsc{sal-utt} value as in (\ref{np-dgb}).
This information then projects \emph{What?} as a well-formed NSU in accordance with (\ref{hf-cx}).
Now consider (\ref{paint-spr}). Note that in Figure~\ref{paint-spr-st}  the
PP argument is not realized as a complement even though
the verb \emph{painted} takes
a PP as its argument value. The interlocutor can have access to this \textsc{arg-st} information, but nothing further: the PP argument has
no further specifications other than being served as an implicit argument of \emph{painted}. This means that,
only this implicit PP can be picked up as 
the \textsc{sal-utt}. This is why the sprouting example allows only a PP as a possible NSU. Thus the key difference between
merger and sprouting examples lies in what the previous discourse
activates via syntactic realizations.\footnote{We owe most of the ideas expressed here to the discussion
with Anne Abeill\'{e}.}
%When the optional argument of
%a verb like \emph{paint} is realized, the context can access to
%its internal structure as well. However, when the optional
%argument is not realized but stays as an implicit argument, the context
%can refer only to this as a whole, not the internal structure.
%
%
%The implied PP \textit{with something} functioning as the SAL-UTT here would %appear as a noncanonical synsem on the ARG-ST list of the verb \textit{paint}, %but not on its COMPS list (which contains only overtly realized arguments), and %thereby be able to provide appropriate morphosyntactic identity information.
%In contrast, the verb \emph{wait} in (\ref{wait}) selects an optional PP as its
%argument. This PP, not realized as a COMPS-list element, is a noncanonical %synsem with no internal structure.  This means that, unlike the overt PP case, %%
%
%
%
%
%The notable property, as mentioned, sprouting requires the presence of %prepositions in NSUs. That is, the NSUs in (\ref{paint-spr}) may not be NPs %consisting of the wh-phrase \emph{what} alone. The HPSG analysis presented %above offers a solution for
%this requirement. Consider again the lexical specification of
%\textit{paint} in (\ref{paint}).  Consider a partial structure
%%of this:
%
%[tree]

The advantages of the discourse-based analyses sketched here thus follow from their ability to capture limited morphosyntactic parallelism between NSUs and \textsc{sal-utt} without having to account for why NSUs behave differently from constituents of sentential structures. The island-violating behavior of NSUs is unsurprising on this analysis, as are attested cases of structural mismatch and situationally controlled NSUs.\footnote{The rarity of NSUs with nonlinguistic antecedents can be understood as a function of how easily a situational context can give rise to a \textsc{max-qud} and thus license ellipsis. See \citet{Miller2014b} for this point with regard to PAE.}
However, some loose ends still remain. (\ref{hf-cx}) %currently has no means of capturing certain connectivity %effects: it can't rule preposition drop out under sprouting
incorrectly rules out case mismatch in languages like Hungarian for speakers that do accept it (see discussion around example (\ref{11})).\footnote{See, however, \citet{Kim2015} for a proposal that introduces a case hierarchy specific to Korean to explain limited case mismatch in this language.}



\section{Analyses of predicate/argument ellipsis}
\label{sec-analyses-of-pred-ellipsis}
The first issue in the analysis of PAE is the status of the elided expression. It is assumed to be a \textit{pro} element due to its pronominal properties \citep[see][]{Lobeck1995, Lopez2000, Kim2006, Aelbrecht2015, Ginzburg2018}. For instance, PAE applies only to phrasal categories (\ref{43}), with the exception of pseudogapping as shown in (\ref{42}); it can cross utterance boundaries (\ref{44}); it can override island constraints ((\ref{45})--(\ref{46})); and it is subject to the Backwards Anaphora Constraint ((\ref{47})--(\ref{48})).

\ea[]{
Your weight affects your voice. It does mine, anyway. \citep[78]{Miller2014}\label{42}
}
\z
\ea[]{
Mary will meet Bill at Stanford because she didn't \jbtr at Harvard.\label{43}
}
\z
\ea[]{
A: Tom won't leave Seoul soon.\\
B: I don't think Mary will \jbtr either.\label{44}
}
\z
\ea[]{
John didn't hit a home run, but I know a woman who did. (CNPC: Complex Noun Phrase Constraint)\label{45}
}
\z
\ea[]{
That Betsy won the batting crown is not surprising, but that Peter didn't know she did \jbtr is
indeed surprising. (SSC: Sentential Subject Constraint)\label{46}
}
\z
\ea[*]{
Sue didn't \jbtr but John ate meat.\label{47}
}
\z
\ea[]{
Because Sue didn't \jbtr, John ate meat.\label{48}
}
\z

One way to account for PAE closely tracks analyses of \emph{pro}-drop phenomena. We do not need to posit a phonologically empty pronoun if a level of argument structure is
available where we can encode the required pronominal properties \citep[see][]{Bresnan1982a,Ginzburg:Sag:2000}. As we have seen, the ARP in (\ref{arp})
allows an 
argument to be a noncanonical  \emph{synsem}
such as \textit{pro} which need not be mapped onto \textsc{comps}. 
%\footnote{The synsem 
%objects Expressions have two subtypes: overt and covert ones, the latter of %which has two subtypes, \textit{pro} and \textit{gap}. See \citet{Sag2012a} for %details.}
%
For instance, the auxiliary
verb \textit{can}, bearing the feature AUX, has a \textit{pro} VP as its second argument in a sentence like \textit{John can't dance, but Sandy can.}, that is, this VP is not instantiated as a syntactic complement of the verb.\footnote{The rich body of HPSG work on English auxiliaries takes them not as special Infl categories, but as verbs bearing the feature AUX. See \citet{Kim:00, KS2002a, Sag:Wasow:Bender:2003, Sag2020a, kimmichaelis:2020}.} This possibility is represented formally in (\ref{51}) 
(see \citealt{Kim2006, ginzburg-miller-ellipsis-handbook}):
%joanna Anne says to cite GS2000 \& Sag et al. 2020. %\index{transitive}

\eas
\label{51}
Lexical description for \textit{can}:\\
\avmtmp{
[\type*{v-wd}
 phon & \phonliste{ can }\\
 cat & [ head & [vform & \type{fin}\\
                   aux &    +]\\
        subj &  < \1 >\\
        comps &   < > \\
 arg-st & < \1 NP, VP![\type{pro}]! >] ]
}
%% \begin{avm}
%% \[\emph{v-word}\\
%%  FORM \q<can\q>\\
%%  SYN\[HEAD \|VFORM \ \ \emph{fin}\\
%%       VAL \[SUBJ & \q<\@1\q>\\
%%            COMPS & \q<  \; \q>\]\]\\
%% ARG-ST  \q<\@1NP, VP[\emph{pro}]\q>\]
%%  \end{avm}
\zs
%
%
\iffalse{
Given this, English PAE can be analyzed as a language-particular VP \textit{pro} drop phenomenon, triggered
by a constraint like (\ref{52}) (see \citet{Kim:00, KimSag2005, KimSellsIntroduction, ginzburg-miller-ellipsis-handbook}). 
%joanna Are we keeping (\ref{52})? Anne says to not.

\inlinetodostefan{
AVM: type name too long and second column not used, brackets around \type{pro} should be normal text
brackets}


\ea\label{52}
Aux-Ellipsis Construction:\\
\avmtmp{
[\type*{aux-v-lxm}
   %\SYN\|\HEAD\|\AUX\ $+$\\
 arg-st < \1 XP,  YP > & ] } $\mapsto$
\avmtmp{
[\type*{aux-pae-wd}
 comps < > \\
 arg-st < \1 XP, YP[\type{pro}] > & ]
}
%% \begin{forest}
%%  [{\begin{avm}
%%  \[\emph{aux-v-lxm}\\
%%    %\SYN\|\HEAD\|\AUX\ $+$\\
%%    ARG-ST \q<\@1XP, \@2YP\q>\]
%%  \  \ $\mapsto$ \ \
%%  \[\emph{aux-ellipsis-wd}\\
%%    ARG-ST \q<\@1XP, \@2YP[\emph{pro}]\q>\]
%%  \end{avm}}]
%% \end{forest}
\z
What this tells us is that an auxiliary verb selecting two arguments
can be projected onto an elided auxiliary verb whose second argument
is realized as a small \textit{pro}. This argument is not mapped
onto any grammatical function on the \textsc{comps} list.}\fi
The output auxiliary
in (\ref{51}) will then project a structure like the one
in Figure~\ref{fig-53}.
%
%
The head daughter's \textsc{comps} list (VP[\textit{bse}]) is empty because the second element on the \textsc{arg-st} list
is a \textit{pro}.\footnote{%Though it is an open question if PAE is different from NCA (e.g., \textit{I asked Tracy to bring the horses into the barn but she refused}), there are some differences between the two (\citet{Miller2013, ginzburg-miller-ellipsis-handbook}). For instance, 
%
%in having an infinitival VP complement of
%a nonauxiliary verb (here the verb \textit{refused}) unexpressed. Both PAE and %NCA are licensed by the ARP, but  NCA %which \citet{Hankamer1976} take to be a deep anaphor,
The same line of analysis could be extended to 
NCA, which has received relatively little attention in modern syntactic theory, including in HPSG. However, NCA is sensitive only to a limited set of main verbs and its exact nature remains controversial.}
\begin{figure}
\begin{forest}
[S
  [\ibox{1} NP
      [Sandy]]
  [VP\\
   \avmtmp{
     [%\emph{head-only-cxt} \& \emph{ellip-cxt}\\
      head \2\\
      subj < \1 > ]}
    [V\\
     \avmtmp{
      [ head \2 [aux $+$ ]\\
        subj < \1 >\\
        comps < >\\
        arg-st < \1 NP, VP[\type{pro}] > ]}
      [can]]]]
\end{forest}
\caption{Structure of a VPE}\label{fig-53}
\end{figure}

We saw in Section~\ref{sec-structural-mismatches} that PAE does not require structural identity with its antecedent, which is supplied by the surrounding context. Therefore, ellipsis resolution is not based on syntactic reconstruction in HPSG analyses, but rather on structured discourse information \citep[see][295]{Ginzburg:Sag:2000}. %This discourse-based analysis then allows syntactic mismatches between the \textit{pro}-form and its antecedent since the key information to refer to is the relevant semantic or discourse information.Literature has noted examples with voice mismatches between  an elided VP and its antecedent VP (\citealt{Sag1976, Hardt1993, Kehler2002, Kim2011, Merchant2013, Kim2018}).  Observe the following examples from \citet{Merchant2013}:
%\begin{exe}
%\label{act-pass}
%\ex Active antecedent, passive ellipsis:
%\begin{xlist}
%\ex We also use the xpdf package in our examples, so you may %want to install that now
%if it isn't already \jbtr. $<$installed$>$ \label{install}
%\ex The janitor must remove the trash whenever it is apparent that it should
%be \jbtr. $<$removed$>$
%\end{xlist}
%\end{exe}
%\begin{exe}
%\label{pass-act}
%\ex Passive antecedent, active ellipsis:
%\begin{xlist}
%\ex The system can be used by anyone who wants to \jbtr. %$<$use it$>$   \label{use}
%\ex This problem was to have been looked into, but obviously
%nobody did \jbtr. $<$look into this problem$>$
%\end{xlist}
%\end{exe}
%
%In each of these examples, there is voice mismatch  between the understood (or
%elided) ellipsis and its putative antecedent. In (\ref{act-pass}), the elided passive VP is linked to the active antecedent, while in (\ref{pass-act}), the elided active VP is associated with the passive antecedent.  
The \emph{pro} analysis outlined above expects structural mismatches (and island violations), %For instance, the DGB evoked in (\ref{act-pass}a) and (\ref{pass-act}a) would be something like the following, respectively (ignoring the value of variables here):
%\eal
%\ex QUD: $\lambda$x$\lambda$y[install(x, y)]
%\ex QUD: $\lambda$x$\lambda$y[use(x, y)]
%\zl 
%
because the relevant antecedent information is the information that the DGB provides via the \textsc{max-qud} in each case, and hence no structural-match requirement is enforced on PAE.\footnote{In the derivational
analysis of \citet{Merchant2013}, cases of structural mismatch are licensed by
the postulation of the functional projection VoiceP above an IP: the understood
VP is linked to its antecedent under the IP.} This means in turn that HPSG analyses of PAE do not face the problem of having to rule out, or rule in, cases of structural mismatch or nonlinguistic antecedents, because their acceptability can be captured as reflecting discourse-based and construction-specific constraints on PAE.



\section{Analyses of non-constituent coordination and gapping}
\label{sec-analyses-of-noncon}

Constructions such as gapping, RNR, and
ACC have also often been taken to belong to elliptical constructions. Each of these constructions has received
relatively little attention in the research on elliptical constructions, possibly
because of their syntactic and semantic complexities. In this
section, we briefly review HPSG analyses of these 
three constructions, leaving more detailed discussion to 
\crossrefchapteralt{coordination} and references 
therein.\footnote{We also leave out discussion 
of HPSG analyses for pseudogapping: readers are referred to \citet{Miller92d-u, Kim2020} and \crossrefchapteralt[Section 4]{control-raising}.\inlinetodoobl{Add latex ref once this chapter materializes.}} 


\subsection{Gapping}

Gapping %is also a type of ellipsis that 
allows a finite
verb to be unexpressed in the non-initial conjuncts, as exemplified below. %of English coordination in (\ref{ex-gapping}).

\eal
\label{ex-gapping}
\ex Some ate bread, and others rice.\label{g1}
\ex Kim can play the guitar, and Lee the violin.\label{g2}
\zl
%
%
%
%
HPSG analyses of gapping fall into two kinds: one kind draws on \citeauthor{Beavers2004}'s (\citeyear{Beavers2004}) deletion-like analysis of non-constituent coordination \citep{Chaves2009} and the other on \citeauthor{Ginzburg:Sag:2000}'s (\citeyear{Ginzburg:Sag:2000}) analysis of NSUs \citep{Abeille2014}.\footnote{For a semantic approach to gapping, the reader is referred to \citet{Parketal2019}, who offer an analysis of scope ambiguities under gapping where the syntax assumed is of the NSU  type and the semantics is cast in the framework of Lexical Resource Semantics.} The latter analyses align gapping with analyses of NSUs, as discussed in Section~\ref{sec-analyses-of-NSUs}, more than with analyses of non-constituent coordination, and for this reason gapping could be classified together with other NSUs. We use the analysis in \citet{Abeille2014} for illustration below.


\citet{Abeille2014}, focusing on French and Romanian, offer a construction- and
discourse-based HPSG approach to gapping where the second headless gapped conjunct is taken to be an
NSU.  %type of fragment.
Their analysis places no syntactic parallelism requirements on the
first conjunct and the gapped conjunct, given English data like (\ref{65}) (note that the bracketed phrases differ in syntactic category).

\ea Pat has become [crazy]\sub{AP} and Chris [an incredible bore]\sub{NP}. \citep[248]{Abeille2014}  \label{65}\z
%
% assume an identity condition on gapping requiring that gapping remnants match major %constituents in the antecedent clause, which they term source clause. In other words,
Instead of requiring %strong 
syntactic parallelism between the two clauses, their analysis limits gapping remnants to elements of the argument structure of the verbal head present in the antecedent (i.e., the leftmost conjunct) and absent from the rightmost conjunct, which reflects the intuition articulated in \citet{Hankamer1971}. This analysis thus also licenses gapping remnants with implicit correlates, as illustrated in the
following Italian example, where the subject is implicit in the leftmost conjunct and overt in the rightmost conjunct (\citealt[251]{Abeille2014}).\footnote{Gapping is possible outside coordination
constructions like comparatives as well as in 
subordinate clauses. See \crossrefchaptert[Section~\ref{sec-non-constituent-coordination}]{coordination}.}
%(e.g., \textit{Robin speaks French better than Leslie German}, %\citet{Culicover:Jackendoff:05}Culicover & Jackendoff 2005) and certain %subordinate clauses (No doubt THEY will find US, before WE, THEM : Park et al %2019)
\ea
\label{italian}
\gll Mangio la pasta e Giovanni il riso.\\
eat.\textsc{1SG} \textsc{DET} pasta and Giovanni \textsc{DET} rice\\
\glt `I eat pasta and Giovanni eats rice.'
\z
%
The subject in the leftmost conjunct in (\ref{italian}) would be analyzed as a noncanonical \type{synsem} of type {\it pro} and the correlate for the remnant {\it Giovanni}. %and has two remnants in the gapped clause. 

%With observing such variations in gapping, 
\citet{Abeille2014} adopt two  key assumptions in their analysis: (a) coordination phrases are nonheaded constructions in which each conjunct shares the same
valence (\textsc{subj} and \textsc{comps}) and nonlocal (\textsc{slash}) features, while
 its head (\textsc{head}) value is not fixed but contains an upper bound (supertype) to accommodate examples like (\ref{65}), and (b) gapping is a special coordination construction in which the first (full) clause serves as the head 
 and some symmetric discourse relation holds between the conjuncts. 
 %
 %(the whole cx can be embedded like the first clause, while the gapped clause is %non finite)
 %
 %On this analysis, for instance,
 To illustrate, the gapped conjunct \emph{Chris an incredible
  bore} in (\ref{65}) is an NSU  with two cluster daughters, as represented by
  the simplified structure in 
  Figure~\ref{fig-gapping}:
 
 \begin{figure}
 \begin{forest}
[S
  [S
      [Pat has become crazy,roof]]
  [XP\\
     \avmtmp{[\type{hd-frag-ph}]}
   [XP+\\
      \avmtmp{[\type{cluster-ph}]}
    [NP
    [Chris]]
    [NP
     [an incredible bore]]
    ]]]
\end{forest}
\caption{Simplified structure of a gapping construction}\label{fig-gapping}
\end{figure} 
%joanna coordinator missing here?
  %
  The NSU that consists of the gapped conjunct in Figure~\ref{fig-gapping} has a single daughter, a cluster phrase with two cluster daughters.\footnote{The notion of a cluster refers to any sequence of dependents and was introduced in \citet{Mouret2006}'s analysis of ACC. In addition, see \crossrefchapterw{cg}\inlineaddpages for semantic issues arising from ACC.}
 The required
 syntactic parallelism between gapping remnants and their correlates in the antecedent is operationalized by adopting the contextual attribute \textsc{sal-utt}, which is introduced for all NSUs, as in (\ref{gap-hf-con}) (\citealt[(53)]{Abeille2014}) (for the definition of \textsc{sal-utt}, see Section \ref{sec-analyses-of-NSUs}).
 %\footnote{The interpretation of the remnants
 %follows from the high order unification algorithm.}

 \ea
\label{gap-hf-con}
Syntactic constraints on \emph{head-fragment-ph}:\\
\type{head-fragment-ph} \impl
\avmtmp{
[cnxt|sal-utt < [head  H$_{1}$\\
                 major +],...,[head  H$_{n}$\\
                                major +]>\\
 cat|head|cluster<[head  H$_{1}$],...,[head  H$_{n}$]>]
}
\z
 %
 %
 Syntactic parallelism between gapping remnants and their counterparts is achieved here by ensuring that each list member of the \textsc{sal-utt} structure-shares 
its \textsc{head} value with the corresponding cluster element.\footnote{The feature \textsc{major} makes each expression a major constituent functioning as a dependent of some verbal projection, blocking
remnants from being deeply embedded in the gapped clause.}
This analysis thus does not reconstruct a syntactic gapped clause and %is compatible with the island insensibility discussed in Section \ref{}. 
%The analysis also correctly 
predicts that a gapped clause may appear in contexts where a full finite clause cannot, as illustrated in (\ref{gcomp}).
 
 \ea 
 Bill wanted to meet Jane as well as Jane (*wanted to invite) him. \citep[242]{Abeille2014}\label{gcomp}
 \z
%

   With syntactic parallelism between the first and
   the gapped conjuncts captured this way, \citet{Abeille2014} also allow 
    gapping remnants to appear in a different order than their counterparts in the antecedent (\ref{64}) \citep[see][156--158]{Sag1985}:
    %, nor are they required to be the same syntactic category as their counterparts (\ref{65}).
%
\ea A policeman walked in at 11, and at 12, a fireman. \label{64}\z
%
%\ea Pat has become [crazy]$_{AP}$ and Chris [an incredible bore]$_{NP}$.  \label{65}\z
%
This ordering flexibility is licensed as long as some symmetric discourse relation holds between the two conjuncts. \citet{Abeille2014} 
localize this symmetric discourse relation to the \textsc{background} contextual feature of the Gapping Construction, which is a sub-construction of coordination. 

%the contextual BACKGROUND information requires each conjunct to hold some symmetric discourse relation (for a detailed discussion, see \citealt{Abeille2014}).

%joanna Maybe we want the text below back?
\iffalse{
\citet{Abeille2014} offer additional evidence from Romance (e.g., case mismatch between gapping remnants and their counterparts and even more possibilities of ordering remnants than is the case in English) to strengthen their point that syntactic identity is relaxed under gapping.\footnote{As
for scope ambiguities and ellipsis in gapping, see \citet{Yatabe2001} and \crossrefchapteralt{coordination}.}

There are three further key assumptions in Abeill\'{e} et al.'s (2014) analysis. First, two (or more) gapping remnants form a cluster whose mother has an underspecified syntactic category, that is, is a non-headed phrase (this information is represented by the Cluster head feature in \ref{66}). This phrase then serves as the head daughter of a head-fragment phrase, whose syntactic category is also underspecified. This means that there is no unpronounced verbal head in the phrase to which gapping remnants belong. Second, the meanings of the gapping remnants are computed from the meaning of the rightmost nonelliptical verbal conjunct, as represented by the Source feature in \ref{66}. Finally, the conjuncts are linked by a symmetric discourse relation (i.e.,  parallelism or contrast) that is part of the Background feature in \ref{66}.
}\fi



\subsection{Right Node Raising}

In typical examples of RNR, as shown below, the element to the immediate right of a parallel structure is shared with the left conjunct:

\eal
\label{ex-kim-prepares-and-Lee-eats-Kim-played-and-Lee-sang}
\ex  Kim prepares and Lee eats [the pasta].  \label{60}
\ex  Kim played and Lee sang [some Rock and Roll songs at Jane's party]. \label{rnr61}
\zl
%
The bracketed shared material can be either a constituent, as in (\ref{60}), or a non-constituent, as in (\ref{rnr61}).

RNR has consistently attracted HPSG analyses involving silent material (a detailed discussion of these can be found
in \crossrefchapteralt{coordination}). 
All existing analyses of RNR \citep{Abeille2016, Beavers2004, Chaves2014, Crysmann2003, Shiraishi2019, Yatabe2001, Yatabe2012} agree on this point,
although some of them propose more than one mechanism for accounting for different kinds of non-constituent coordination \citep{Chaves2014, Yatabe2001, Yatabe2012, Yatabe2019}. One strand of research within the RNR literatures adopts a linearization-based approach employed more generally in analyses of non-constituent coordination (NCC) (see \citealt{Yatabe2001, Yatabe2012}, for a general introduction to order domains see \crossrefchapteralt[Section~\ref{sec-domains}]{order}) and another proposes a deletion-like operation \citep{Abeille2016, Chaves2014, Shiraishi2019}.
%, the
%latter of which is our focus here.

%
% We focus on the latter here, along with the question that it has raised, that is, what kind % of identity constraints must be satisfied before RNR can apply.

The kind of material that may be RNRaised and the range of structural mismatches permitted between the left and right conjuncts have been the subject of recent debate.\footnote{Although we refer to the material on the left and right as conjuncts, it is been known since \citet{Hudson1976, Hudson1989} that RNR extends to other syntactic environments than coordination (see \citealt{Chaves2014}).} For instance, \citet[839--840]{Chaves2014} demonstrates that, besides more typical examples like (\ref{ex-kim-prepares-and-Lee-eats-Kim-played-and-Lee-sang}),
 there is a range of phenomena classifiable as RNR that exhibit various argument-structure mismatches as in (\ref{54})--(\ref{55}), and that can target material below the word level as in (\ref{56})--(\ref{57}).
%
%\ea Ethan sold and Rasmus gave away all his CDs. \label{RNR8} \z

\eal
\ex Sue gave me---but I don't think I will ever read---[a book about relativity]. \label{54}

\ex Never let me---or insist that I---[pick the seats].\label{55}

\ex We ordered the hard- but they got us the soft-[cover edition].\label{56}

\ex Your theory under- and my theory over[generates].\label{57}\zl
%
Furthermore, RNR can target strings that are not subject to any known syntactic operations, such as rightward movement \citep[865]{Chaves2014}.

\eal
\ex I thought it was going to be a good but it ended up being a very bad [reception].\label{58}

\ex Tonight a group of men, tomorrow night he himself, [would go out there somewhere and wait].\label{59}\zl
%\ea They were also as liberal or more liberal [than any other age group in the 1986 through 1989 surveys].\label{60}\z
RNRaised material can also be discontinuous, as in (\ref{ex-rnr-discontinuous}) (\citealt[868]{Chaves2014}; \citealt[238--240]{Whitman2009}).
%\todosatz{No reference given, but there is one in cg.bib and one in lfg.bib which can be %used by changing W to w.}

\eal
\label{ex-rnr-discontinuous}
\ex Please move from the exit rows if you are unwilling or unable [to perform the necessary actions] without injury.\label{61}

\ex The blast upended and nearly sliced [an armored Chevrolet Suburban] in half.\label{62}\zl
%
This evidence leads \citet{Chaves2014} to propose that RNR is a nonuniform phenomenon, comprising extraposition,  VP- or N$'$-ellipsis, and true RNR.
% besides 'true' RNR
% and requiring different kinds of theoretical analyses.
%
%
%
Of the three, only true RNR should be accounted for via the mechanism of optional surface-based deletion that is sensitive to morph form identity and targets any linearized strings, whether constituents or otherwise.\footnote{Whenever RNR can instead be analyzed as either VP or N$'$-ellipsis or extraposition, \citet{Chaves2014} proposes separate mechanisms for deriving them couched upon the direct interpretation approach described in the previous sections for NSUs and predicate/argument ellipsis, and an analysis employing the feature \textsc{extra} to record extraposed material along the lines of \citet{KimSag2005, Kay2012}.} \citeauthor{Chaves2014}' (\citeyear[874]{Chaves2014}) constraint licensing true RNR is given in (\ref{63}) as an informal version  (where $\alpha$ means a morphophonological constituent, and $^{+}$ the Kleene star (operator)):
%\inlinetodostefan{TODO for JB: Verify whether this %should be Kleene star (should then probably have symbol %*) or a Kleene plus (then + is OK)}
% JB: I checked Chaves's again. This is correct. 
%It permits the M(orpho)P(honology) feature of the mother to contain only one instance %(represented as $L_{3}$ in (\ref{63})) of the two morphophonologically identical sequences %[FORM $F_{1}$], \ldots, [FORM $F_{n}$] present in the daughters; the leftmost of these %sequences undergoes deletion. The final list in the mother, $L_{4}$, may be empty or %nonempty, depending on whether RNRaised material is discontinuous.
%
%
%
\ea
\label{63}
 Backward Periphery Deletion Construction:\\

Given a sequence of morphophonologic constituents $\alpha_{1}^{+}$ $\alpha_{2}^{+}$ $\alpha_{3}^{+}$ $\alpha_{4}^{+}$ $\alpha_{5}^{*}$, then the output
$\alpha_{1}^{+}$ $\alpha_{3}^{+}$ $\alpha_{4}^{+}$ $\alpha_{5}^{*}$
iff $\alpha_{2}^{+}$ and $\alpha_{4}^{+}$ are identical up to morph forms.
\z
(\ref{63}) takes the morphophonology of a phrase to be computed as the linear combination of the phonologies of the daughters, allowing deletion to apply locally.\footnote{For further detail on linearization-based analyses of RNR, the interested reader is referred to \citet{Yatabe2001, Yatabe2012} and to \crossrefchapteralt[Section~\ref{sec-domains}]{order} for details of linearization-based approaches in general.}
%, who distinguish between syntactic RNR and phonological RNR, based on the %amount of morphosyntactic identity holding between RNRaised material and %the requirements imposed on the slots it occupies in the structure, and %represent this distinction by treating the RNRaised material as either a %separate domain object on the mother's \textsc{dom} list (syntactic RNR) or %embedded in a larger domain object corresponding to the right conjunct %(phonological RNR).}



\iffalse{
\citeauthor{Chaves2014}' (\citeyear[874]{Chaves2014}) constraint licensing true RNR is given in \ref{bpd}. It permits the M(orpho)P(honology) feature of the mother to contain only one instance (represented as $L_{3}$ in (\ref{63})) of the two morphophonologically identical sequences [FORM $F_{1}$], \ldots, [FORM $F_{n}$] present in the daughters; the leftmost of these sequences undergoes deletion. The final list in the mother, $L_{4}$, may be empty or nonempty, depending on whether RNRaised material is discontinuous.
%
%\fi
\inlinetodostefan{AVM: long line and short type alignment, funny space after L2 and L3}



\ea
\label{bpd}
Backward periphery deletion construction:\\
\avmtmp{\small
[\type*{phrase}
  mp $L_{1}$:\type{ne-list} $\bigcirc L_{2}$:\type{ne-list} $\bigcirc L_{3} \bigcirc L_{4}$ & ]
} $\to$
\avmtmp{
 [ \type*{phrase}
    mp $L_{1} \bigcirc$ < [ form $F_{1}$ ], \ldots, [ form $F_{n}$ ] > $\bigcirc L_{2} \bigcirc L_{3}:$%\\ \hspace{50pt}
     <[ form $F_{1}$ ], \ldots, [form $F_{n}$ ] > $\bigcirc L_{4}$ & ]
}
\z
}\fi

Another deletion-based analysis of RNR is due to \citet{Abeille2016, Shiraishi2019}, differing from \citet{Chaves2014} in terms of identity conditions on deletion. \citet{Abeille2016} argue for a finer-grained analysis of French RNR without morphophonological identity. Their empirical evidence reveals a split between functional and lexical categories in French such that the former permit mismatch between the two conjuncts (where determiners or prepositions differ) under RNR, while the latter do not. \citet{Shiraishi2019} provide further corpus and experimental evidence that morphophonological identity is too strong a constraint on RNR, given the range of acceptable mismatches between the verbal forms of the material missing from the left conjunct and those of the material that is shared between both conjuncts.

\iffalse{
To illustrate, an English verb form mismatch is depicted in (\ref{verbform}), from \citep[see][5]{Shiraishi2019}, where the left conjunct requires a participle while the shared material contains an infinitive form of the verb \textit{appear}.

\ea Some new hybrid models already have, and others soon will appear in the automobile industry.\label{verbform}

\citep{Shiraishi2019} capture verb form mismatch of this kind by introducing into their analysis of RNR the feature LID, which carries lexeme identity information. That is, this feature ensures semantic and syntactic category identity but ignores differences introduced by inflectional suffixes, with the result that the participle and the infinitive in (\ref{verbform}) count as lexeme-identical. RNR is licensed by including the LID feature in the MP feature also used in \citet{Chaves2014} (see (\ref{bpd})).
%Shiraishi2019 rnr-cx here \label{rnrcx}
%(\ref{rnrcx}) ensures that the content of the $l_{2}$ list in the left conjunct, which is elided and hence not represented in the mother, is shared with the $r_{2}$ list in the right conjunct via the LID feature. The lists $l_{1}$, $r_{1}$ and $r_{3}$ in (\ref{rnrcx}) represent material present overtly in the left and right conjuncts.
}\fi


\subsection{Argument Cluster Coordination}

ACC is a type of non-constituent coordination (NCC), as
illustrated in (\ref{ex-acc}):
%
%
\eal
\label{ex-acc}
\ex John gave [a book to Mary] and [a record to Jane].   \label{acc-here}
\ex John gave [Mary a book] and [Jane a record].  \label{acc-1}
\zl
%
%Our focus here is on HPSG analyses of ACC, which departs from those
%relying on the notion of ellipsis where silent material is permitted as part of the %structure.
%

%Departing from the traditional assumption that such examples involve non-constituent
%coordination, 
As for the treatment of ACC, the existing HPSG analyses have articulated two main views: ellipsis
(\citealt{Yatabe2001, Crysmann2003, Beavers2004}) and non-standard constituents (\citealt{Mouret2006}). For discussion of
the nonelliptical view, which takes ACC to be a special type of coordination,
we refer the reader to \crossrefchapteralt{coordination} and references 
therein. Here we just focus on the ellipsis view, which better fits this chapter.

The ellipsis analysis set forth by \citet{Beavers2004} gains its motivation 
from examples like  (\ref{ex-acc-eliptical}):%\footnote{For an example of an analysis of ACC that coordinates noncanonical constituents and doesn't posit the existence of silent material, see \citep{Mouret2006}.}

\eal
\label{ex-acc-eliptical}
\ex Jan travels to Rome tomorrow, [to Paris on Friday], and will fly to Tokyo
on Sunday. \label{acc2}
\ex Jan wanted to study medicine when he was 11, [law when he was 13],
and to study nothing at all when he was 18. \label{acc3}\zl
%
As pointed out by \citet{Beavers2004}, such examples challenge non-ellipsis analyses within the assumption that only 
constituents of like category can coordinate.\footnote{As discussed in 
 \crossrefchapteralt[Section~\ref{unlikessec}]{coordination} and references therein, there are numerous examples (e.g., \textit{Fred became wealthy and a Republican}) where unlike categories are coordinated.}
The status of the bracketed conjuncts in (\ref{ex-acc-eliptical}) is quite questionable, since they are not VPs like the other two fellow conjuncts. \citegen{Beavers2004} proposal is to treat such examples as standard VP coordination with ellipsis of the verb in the second conjunct, as
given in the following:

\ea
\label{strike-ex}
 Jan [[travels to Rome tomorrow], [[\sout{travels}] to Paris on Friday], and [will fly to Tokyo on Sunday]]].
\z 
%
\citet{Beavers2004} further adopt the \textsc{dom} list
machinery proposed as part of the linearization theory (see  \citealt{Crysmann2003a} for this proposal), and
allow some elements in the daughters' \textsc{dom} lists to be absent from 
the mother's \textsc{dom} list (\citealt{Yatabe2001,Crysmann2003a}).\footnote{For detailed discussion of the feature \textsc{dom}, see \crossrefchapteralt[Section~\ref{sec-domains}]{order}).} 
%
%where the level of an order domain is operationalized as the \textsc{dom} list.
%
This idea is encoded in the Coordination Construction, given in (\ref{CC}), which is a simplified
version of the one in \citep[27]{Beavers2004}:\footnote{For simplicity, we represent only the \textsc{dom} value, suppressing all the other information and further add the parentheses for \avmtmp{\tag{A} and \tag{D}}. 
For the exact formulation, see \citet{Beavers2004}. Further, for 
more details on the role of the \textsc{dom} list in HPSG accounts of constituent order, the reader is referred to \crossrefchapteralt{order}.}
%

\ea\label{CC}
Syntactic constraints on \emph{cnj-cxt} \citep[27]{Beavers2004}:\\
\type{cnj-cxt} \impl
\avmtmp{
[mtr [dom \tag{A} \+ \tag{B$_{1}$} \+ \tag{C}  \+ \tag{B$_{2}$} \+ \tag{D}]\\
 dtrs  <[dom \tag{A} \+ \tag{B$_{1}$}[\type{ne-list}] \+ (\tag{D})],\smallskip\\
        [dom  \tag{C}[(\type{conj})] \+ (\tag{A}) \+ \tag{B$_{2}$}[\type{ne-list}] \+ \tag{D}]>]
}
\z
%
%
% \ea
%\label{conj-cxt}
%\type{cnj-cxt} \impl
%\avmtmp{
%[mtr [dom \@A \oplus \@B$_{1}$ \oplus \@C \oplus  \@B$_{2}$ \oplus \@D]\\
% dtrs \<[dom \@A \oplus \@B$_{1}$[\textit{ne-list}] \oplus \@X],
%[dom   <\@C[(\emph{conj})] \oplus \@X \oplus  \@B$_{g}$[\textit{ne-list}] \oplus \@D]\>
%]
%}
%\z
%
%
As specified in this construction, there are two conjuncts with the \textsc{dom} value.
The mother \textsc{dom} value has the potentially empty material \avmtmp{\tag{A}} from the left conjunct (the corresponding material in the right conjunct is elided), a unique element \avmtmp{\tag{B$_{1}$}} from the left
conjunct, the coordinator \avmtmp{\tag{C}}, a unique element \avmtmp{\tag{B$_{2}$}} from
the right conjunct, and some material \avmtmp{\tag{D}} from the right 
conjunct (the corresponding material in the left conjunct is elided). (\ref{CC}) licenses various types of coordination. For instance, 
when \avmtmp{\tag{A}} is empty, it licenses examples like
\emph{Kim and Pat}, but when \avmtmp{\tag{A}} is non-empty, it licenses examples like \emph{John gave a book to Mary and a record
to Jane}. %in (\label{acc-here}).
When
both \avmtmp{\tag{A}} and \avmtmp{\tag{D}} are non-empty, it allows examples 
like (\ref{strike-ex}). The content of the \textsc{dom} list consists of prosodic constituents (i.e., constituents with no information about their internal morphosyntax) and this offers a way of accounting for coordination of noncanonical constituents as a type of ellipsis.
%\footnote{See \citet{Beavers2004} for discussion of %semantic issues in NCC.}
 
%In analyses of ACC, the elements present on the mother's \textsc{dom} list are those present %overtly on the \textsc{dom} lists of both conjuncts, as well as those present overtly on the %\textsc{dom} list of the left, but not the right, conjunct. The \textsc{dom} value of the mother in %(\ref{CC}) begins with material A (empty or otherwise) from the left conjunct, some %material from the left conjunct B$_{1}$, the conjunct's
%coordinator C (if present), some material B$_{2}$
%from the right conjunct, and ends with some material D from the right conjunct. To %derive NCC as in (\ref{ex-acc}), the left-most element on the mother's \textsc{dom} list, %representing material present overtly only in the left conjunct (here the verb \emph{gave}), may not be empty.
%
%joanna Anne's asked that Johnson and Park et al 2019 be added/expanded on.
 
%
% a type of nonconsitutent cluster. 
%
%One advantage of this analysis comes from the following data set (\citealt{Beavers2004}):
%
%
%\ea  Jan [walks [talks and [chews gum]]]. \z
%\ea Jan [[walks and talks] [or [walks and [chews gum]]]]. \z
%\ea *Jan [walks [chews gum] \z
%
%NOTE: Is this a complete thought?

%
%
%
%
%
%
% To make this more precise, consider the \citep{Beavers2004} schema in (\ref{acc2}), which % derives not just ACC, but also RNR and constituent coordination.
%
%
%
%BeaversSag2004 here \label{acc2}
%
%
%This assumption allows us to coordinate VPs where left- and/or right-most elements on the %mother's \textsc{dom} list may or may not be empty to capture different types of coordination.
%
%
%\ea Harvey gave a book to Ethan and a record to Rasmus. \label{acc3}\z

%The schema in (\ref{acc2}) also permits derivation of RNR, provided the right-most element %on the mother's \textsc{dom} list (the correspondent of the material present overtly only in the %right conjunct) is not empty. We now take a closer look at analyses of RNR in the next %section.




\section{Summary}
\label{sum}
This chapter has reviewed three types of ellipsis---nonsentential utterances, predicate ellipsis, and non-constituent coordination---which correspond to three kinds of analysis within HPSG. The pattern that emerges from this overview is that HPSG favors the ``what you see is what get'' approach to ellipsis,
%and limits a deletion-based %approach, common in the %minimalist literature on %ellipsis, to a phenomena.
 accouting for a wider variety of data,  from corpora 
 as well as from experiments,  than other derivation-based approaches common in the minimalist literature.



%\citep{Chomsky1957}.
%\citep{Comrie1981}




%\begin{table}
%\caption{Frequencies of word classes}
%\label{tab:1:frequencies}
% \begin{tabular}{lllll}
%  \lsptoprule
            %& nouns & verbs & adjectives & adverbs\\
 % \midrule
  %absolute  &   12 &    34  &    23     & 13\\
%  relative  &   3.1 &   8.9 &    5.7    & 3.2\\
 % \lspbottomrule
 %\end{tabular}
%\end{table}




\section*{Abbreviations}

\begin{tabularx}{.99\textwidth}{@{}lX}
ACC & Argument Cluster Coordination\\
BAE & Bare Argument Ellipsis\\
DGB & Dialogue Game Board\\
IND & Index\\
MAX-QUD & Maximal-Question-under-Discussion\\
NCA & Null Complement Anaphora\\
NCC & Non-constituent Coordination\\ 
NSU & Nonsentential utterance\\
PAE & Predicate Argument Ellipsis\\
RNR & Right-Node Raising\\
SAL-UTT & Salient Utterance\\
VPE & Verb Phrase Ellipsis\\
\end{tabularx}


\section*{Acknowledgements}
We thank Anne Abeill\'{e} and Stefan M\"{u}ller
for substantive discussion and helpful suggestions. We also thank  
Yusuke Kubota for helpful comments.  

{\sloppy
\printbibliography[heading=subbibliography,notkeyword=this]
}
%
}% AVM options


\end{document}
