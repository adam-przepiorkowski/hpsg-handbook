\documentclass[output=paper
	        ,collection
	        ,collectionchapter
 	        ,biblatex
                ,babelshorthands
                ,newtxmath
                ,draftmode
                ,colorlinks, citecolor=brown
]{langscibook}

\IfFileExists{../localcommands.tex}{%hack to check whether this is being compiled as part of a collection or standalone
  % add all extra packages you need to load to this file 

% the ISBN assigned to the digital edition
\usepackage[ISBN=9783961102556]{ean13isbn} 

\usepackage{graphicx}
\usepackage{tabularx}
\usepackage{amsmath} 

%\usepackage{tipa}      % Davis Koenig
\usepackage{xunicode} % Provide tipa macros (BC)

\usepackage{multicol}

% Berthold morphology
\usepackage{relsize}
%\usepackage{./styles/rtrees-bc} % forbidden forest 08.12.2019


\usepackage{langsci-optional} 
% used to be in this package
\providecommand{\citegen}{}
\renewcommand{\citegen}[2][]{\citeauthor{#2}'s (\citeyear*[#1]{#2})}
\providecommand{\lsptoprule}{}
\renewcommand{\lsptoprule}{\midrule\toprule}
\providecommand{\lspbottomrule}{}
\renewcommand{\lspbottomrule}{\bottomrule\midrule}
\providecommand{\largerpage}{}
\renewcommand{\largerpage}[1][1]{\enlargethispage{#1\baselineskip}}


\usepackage{langsci-lgr}

\newcommand{\MAS}{\textsc{m}\xspace} % \M is taken by somebody

%\usepackage{./styles/forest/forest}
\usepackage{langsci-forest-setup}

\usepackage{./styles/memoize/memoize} 
\memoizeset{
  memo filename prefix={chapters/hpsg-handbook.memo.dir/},
  register=\todo{O{}+m},
  prevent=\todo,
}

\usepackage{tikz-cd}

\usepackage{./styles/tikz-grid}
\usetikzlibrary{shadows}


% removed with texlive 2020 06.05.2020
% %\usepackage{pgfplots} % for data/theory figure in minimalism.tex
% % fix some issue with Mod https://tex.stackexchange.com/a/330076
% \makeatletter
% \let\pgfmathModX=\pgfmathMod@
% \usepackage{pgfplots}%
% \let\pgfmathMod@=\pgfmathModX
% \makeatother

\usepackage{subcaption}

% Stefan Müller's styles
\usepackage{./styles/merkmalstruktur,german,./styles/makros.2020,./styles/my-xspace,./styles/article-ex,
./styles/eng-date}

\selectlanguage{USenglish}

\usepackage{./styles/abbrev}


% Has to be loaded late since otherwise footnotes will not work

%%%%%%%%%%%%%%%%%%%%%%%%%%%%%%%%%%%%%%%%%%%%%%%%%%%%
%%%                                              %%%
%%%           Examples                           %%%
%%%                                              %%%
%%%%%%%%%%%%%%%%%%%%%%%%%%%%%%%%%%%%%%%%%%%%%%%%%%%%
% remove the percentage signs in the following lines
% if your book makes use of linguistic examples
\usepackage{langsci-gb4e} 


%% St. Mü.: 03.04.2020
%% these two versions of the command can be used for series of sets of examples:
%% \eal
%% \ex
%% \ex
%% \zlcont
%% \ealcont
%% \ex
%% \ex
%% \zl

\let\zlcont\z
\def\ealcont{\exnrfont\ex\begin{xlist}[iv.]\raggedright}

% original version of \z
%\def\z{\ifnum\@xnumdepth=1\end{exe}\else\end{xlist}\fi}
% \zcont just removes \end{exe}
%\def\zcont{\ifnum\@xnumdepth=1\else\end{xlist}\fi}
\def\zcont{}

% Crossing out text
% uncomment when needed
%\usepackage{ulem}

\usepackage{./styles/additional-langsci-index-shortcuts}

% this is the completely redone avm package
\usepackage{./styles/langsci-avm}
\avmsetup{columnsep=.3ex,style=narrow}

%\let\asort\type*


\usepackage{./styles/avm+}


\renewcommand{\tpv}[1]{{\avmjvalfont\itshape #1}}

% no small caps please
\renewcommand{\phonshape}[0]{\normalfont\itshape}

\regAvmFonts

\usepackage{theorem}

\newtheorem{mydefinition}{Def.}
\newtheorem{principle}{Principle}

{\theoremstyle{break}
%\newtheorem{schema}{Schema}
\newtheorem{mydefinition-break}[mydefinition]{Def.}
\newtheorem{principle-break}[principle]{Principle}
}


%% \newcommand{schema}[2]{
%% \begin{minipage}{\textwidth}
%% {\textbf{Schema~\theschema}}]\hspace{.5em}\textbf{(#1)}\\
%% #2
%% \end{minipage}}


% This avoids linebreaks in the Schema
\newcounter{schemacounter}
\makeatletter
\newenvironment{schema}[1][]
  {%
   \refstepcounter{schemacounter}%
   \par\bigskip\noindent
   \minipage{\linewidth}%
   \textbf{Schema~\theschemacounter\hspace{.5em} \ifx&#1&\else(#1)\fi}\par
  }{\endminipage\par\bigskip\@endparenv}%
\makeatother

%\usepackage{subfig}





% Davis Koenig Lexikon

\usepackage{tikz-qtree,tikz-qtree-compat} % Davis Koenig remove

\usepackage{shadow}



\usepackage[english]{isodate} % Andy Lücking
\usepackage[autostyle]{csquotes} % Andy
%\usepackage[autolanguage]{numprint}

%\defaultfontfeatures{
%    Path = /usr/local/texlive/2017/texmf-dist/fonts/opentype/public/fontawesome/ }

%% https://tex.stackexchange.com/a/316948/18561
%\defaultfontfeatures{Extension = .otf}% adds .otf to end of path when font loaded without ext parameter e.g. \newfontfamily{\FA}{FontAwesome} > \newfontfamily{\FA}{FontAwesome.otf}
%\usepackage{fontawesome} % Andy Lücking
\usepackage{pifont} % Andy Lücking -> hand

\usetikzlibrary{decorations.pathreplacing} % Andy Lücking
\usetikzlibrary{matrix} % Andy 
\usetikzlibrary{positioning} % Andy
\usepackage{tikz-3dplot} % Andy

% pragmatics
\usepackage{eqparbox} % Andy
\usepackage{enumitem} % Andy
\usepackage{longtable} % Andy
\usepackage{tabu} % Andy              needs to be loaded before hyperref as of texlive 2020

% tabu-fix
% to make "spread 0pt" work
% -----------------------------
\RequirePackage{etoolbox}
\makeatletter
\patchcmd
	\tabu@startpboxmeasure
	{\bgroup\begin{varwidth}}%
	{\bgroup
	 \iftabu@spread\color@begingroup\fi\begin{varwidth}}%
	{}{}
\def\@tabarray{\m@th\def\tabu@currentgrouptype
    {\currentgrouptype}\@ifnextchar[\@array{\@array[c]}}
%
%%% \pdfelapsedtime bug 2019-12-15
\patchcmd
	\tabu@message@etime
	{\the\pdfelapsedtime}%
	{\pdfelapsedtime}%
	{}{}
%
%
\makeatother
% -----------------------------


% Manfred's packages

%\usepackage{shadow}

\usepackage{tabularx}
\newcolumntype{L}[1]{>{\raggedright\arraybackslash}p{#1}} % linksbündig mit Breitenangabe


% Jong-Bok

%\usepackage{xytree}

\newcommand{\xytree}[2][dummy]{Let's do the tree!}

% seems evil, get rid of it
% defines \ex is incompatible with gb4e
%\usepackage{lingmacros}

% taken from lingmacros:
\makeatletter
% \evnup is used to line up the enumsentence number and an entry along
% the top.  It can take an argument to improve lining up.
\def\evnup{\@ifnextchar[{\@evnup}{\@evnup[0pt]}}

\def\@evnup[#1]#2{\setbox1=\hbox{#2}%
\dimen1=\ht1 \advance\dimen1 by -.5\baselineskip%
\advance\dimen1 by -#1%
\leavevmode\lower\dimen1\box1}
\makeatother


% YK -- CG chapter

%\usepackage{xspace}
\usepackage{bm}
\usepackage{ebproof}


% Antonio Branco, remove this
\usepackage{epsfig}

% now unicode
%\usepackage{alphabeta}





\usepackage{pst-node}


% fmr: additional packages
%\usepackage{amsthm}


% Ash and Steve: LFG
\usepackage{./styles/lfg/dalrymple}

\RequirePackage{graphics}
%\RequirePackage{./styles/lfg/trees}
%% \RequirePackage{avm}
%% \avmoptions{active}
%% \avmfont{\sc}
%% \avmvalfont{\sc}
\RequirePackage{./styles/lfg/lfgmacrosash}

\usepackage{./styles/lfg/glue}

%%%%%%%%%%%%%%%%%%%%%%%%%%%%%%
%% Markup
%%%%%%%%%%%%%%%%%%%%%%%%%%%%%%
\usepackage[normalem]{ulem} % For thinks like strikethrough, using \sout

% \newcommand{\high}[1]{\textbf{#1}} % highlighted text
\newcommand{\high}[1]{\textit{#1}} % highlighted text
%\newcommand{\term}[1]{\textit{#1}\/} % technical term
\newcommand{\qterm}[1]{`{#1}'} % technical term, quotes
%\newcommand{\trns}[1]{\strut `#1'} % translation in glossed example
\newcommand{\trnss}[1]{\strut \phantom{\sqz{}} `#1'} % translation in ungrammatical glossed example
\newcommand{\ttrns}[1]{(`#1')} % an in-text translation of a word
%\newcommand{\feat}[1]{\mbox{\textsc{\MakeLowercase{#1}}}}     % feature name
%\newcommand{\val}[1]{\mbox{\textsc{\MakeLowercase{#1}}}}    % f-structure value
\newcommand{\featt}[1]{\mbox{\textsc{\MakeLowercase{#1}}}}     % feature name
\newcommand{\vall}[1]{\mbox{\textsc{\textup{\MakeLowercase{#1}}}}}    % f-structure value
\newcommand{\mg}[1]{\mbox{\textsc{\MakeLowercase{#1}}}}    % morphological gloss
%\newcommand{\word}[1]{\textit{#1}}       % mention of word
\providecommand{\kstar}[1]{{#1}\ensuremath{^*}}
\providecommand{\kplus}[1]{{#1}\ensuremath{^+}}
\newcommand{\template}[1]{@\textsc{\MakeLowercase{#1}}}
\newcommand{\templaten}[1]{\textsc{\MakeLowercase{#1}}}
\newcommand{\templatenn}[1]{\MakeUppercase{#1}}
\newcommand{\tempeq}{\ensuremath{=}}
\newcommand{\predval}[1]{\ensuremath{\langle}\textsc{#1}\ensuremath{\rangle}}
\newcommand{\predvall}[1]{{\rm `#1'}}
\newcommand{\lfgfst}[1]{\ensuremath{#1\,}}
\newcommand{\scare}[1]{`#1'} % scare quotes
\newcommand{\bracket}[1]{\ensuremath{\left\langle\mathit{#1}\right\rangle}}
\newcommand{\sectionw}[1][]{Section#1} % section word: for cap/non-cap
\newcommand{\tablew}[1][]{Table#1} % table word: for cap/non-cap
\newcommand{\lfgglue}{LFG+Glue}
\newcommand{\hpsgglue}{HPSG+Glue}
\newcommand{\gs}{GS}
%\newcommand{\func}[1]{\ensuremath{\mathbf{#1}}}
\newcommand{\func}[1]{\textbf{#1}}
\renewcommand{\glue}{Glue}
%\newcommand{\exr}[1]{(\ref{ex:#1}}
\newcommand{\exra}[1]{(\ref{ex:#1})}


%%%%%%%%%%%%%%%%%%%%%%%%%%%%%%
% Notation
%\newcommand{\xbar}[1]{$_{\mbox{\textsc{#1}$^{\raisebox{1ex}{}}$}}$}
\newcommand{\xprime}[2][]{\textup{\mbox{{#2}\ensuremath{^\prime_{\hspace*{-.0em}\mbox{\footnotesize\ensuremath{\mathit{#1}}}}}}}}
\providecommand{\xzero}[2][]{#2\ensuremath{^0_{\mbox{\footnotesize\ensuremath{\mathit{#1}}}}}}



\let\leftangle\langle
\let\rightangle\rangle

%\newcommand{\pslabel}[1]{}



  %add all your local new commands to this file


% Don't do this at home. I do not like the smaller font for captions.
% I just removed loading the caption packege in langscibook.cls
%% \captionsetup{%
%% font={%
%% stretch=1%.8%
%% ,normalsize%,small%
%% },%
%% width=.8\textwidth
%% }

\makeatletter
\def\blx@maxline{77}
\makeatother


\newcommand{\page}{}

\newcommand{\todostefan}[1]{\todo[color=orange!80]{\footnotesize #1}\xspace}
\newcommand{\todosatz}[1]{\todo[color=red!40]{\footnotesize #1}\xspace}

\newcommand{\inlinetodostefan}[1]{\todo[color=green!40,inline]{\footnotesize #1}\xspace}

\newcommand{\addpages}{\todostefan{add pages}}
\newcommand{\addglosses}{\todostefan{add glosses}}


\newcommand{\spacebr}{\hspaceThis{[}}

\newcommand{\danish}{\jambox{(\ili{Danish})}}
\newcommand{\english}{\jambox{(\ili{English})}}
\newcommand{\german}{\jambox{(\ili{German})}}
\newcommand{\yiddish}{\jambox{(\ili{Yiddish})}}
\newcommand{\welsh}{\jambox{(\ili{Welsh})}}

% Cite and cross-reference other chapters
\newcommand{\crossrefchaptert}[2][]{\citet*[#1]{chapters/#2}, Chapter~\ref{chap-#2} of this volume} 
\newcommand{\crossrefchapterp}[2][]{(\citealp*[#1]{chapters/#2}, Chapter~\ref{chap-#2} of this volume)}
\newcommand{\crossrefchapteralt}[2][]{\citealt*[#1]{chapters/#2}, Chapter~\ref{chap-#2} of this volume}
\newcommand{\crossrefchapteralp}[2][]{\citealp*[#1]{chapters/#2}, Chapter~\ref{chap-#2} of this volume}
% example of optional argument:
% \crossrefchapterp[for something, see:]{name}
% gives: (for something, see: Author 2018, Chapter~X of this volume)

\let\crossrefchapterw\crossrefchaptert



% Davis Koenig

\let\ig=\textsc
\let\tc=\textcolor

% evolution, Flickinger, Pollard, Wasow

\let\citeNP\citet

% Adam P

%\newcommand{\toappear}{Forthcoming}
\newcommand{\pg}[1]{p.\,#1}
\renewcommand{\implies}{\rightarrow}

\newcommand*{\rref}[1]{(\ref{#1})}
\newcommand*{\aref}[1]{(\ref{#1}a)}
\newcommand*{\bref}[1]{(\ref{#1}b)}
\newcommand*{\cref}[1]{(\ref{#1}c)}

\newcommand{\msadam}{.}
\newcommand{\morsyn}[1]{\textsc{#1}}

\newcommand{\nom}{\morsyn{nom}}
\newcommand{\acc}{\morsyn{acc}}
\newcommand{\dat}{\morsyn{dat}}
\newcommand{\gen}{\morsyn{gen}}
\newcommand{\ins}{\morsyn{ins}}
%\newcommand{\aploc}{\morsyn{loc}}
\newcommand{\voc}{\morsyn{voc}}
\newcommand{\ill}{\morsyn{ill}}
\renewcommand{\inf}{\morsyn{inf}}
\newcommand{\passprc}{\morsyn{passp}}

%\newcommand{\Nom}{\msadam\nom}
%\newcommand{\Acc}{\msadam\acc}
%\newcommand{\Dat}{\msadam\dat}
%\newcommand{\Gen}{\msadam\gen}
\newcommand{\Ins}{\msadam\ins}
\newcommand{\Loc}{\msadam\loc}
\newcommand{\Voc}{\msadam\voc}
\newcommand{\Ill}{\msadam\ill}
\newcommand{\PassP}{\msadam\passprc}

\newcommand{\Aux}{\textsc{aux}}

\newcommand{\princ}[1]{\textnormal{\textsc{#1}}} % for constraint names
\newcommand{\notion}[1]{\emph{#1}}
\renewcommand{\path}[1]{\textnormal{\textsc{#1}}}
\newcommand{\ftype}[1]{\textit{#1}}
\newcommand{\fftype}[1]{{\scriptsize\textit{#1}}}
\newcommand{\la}{$\langle$}
\newcommand{\ra}{$\rangle$}
%\newcommand{\argst}{\path{arg-st}}
\newcommand{\phtm}[1]{\setbox0=\hbox{#1}\hspace{\wd0}}
\newcommand{\prep}[1]{\setbox0=\hbox{#1}\hspace{-1\wd0}#1}

%%%%%%%%%%%%%%%%%%%%%%%%%%%%%%%%%%%%%%%%%%%%%%%%%%%%%%%%%%%%%%%%%%%%%%%%%%%

% FROM FS.STY:

%%%
%%% Feature structures
%%%

% \fs         To print a feature structure by itself, type for example
%             \fs{case:nom \\ person:P}
%             or (better, for true italics),
%             \fs{\it case:nom \\ \it person:P}
%
% \lfs        To print the same feature structure with the category
%             label N at the top, type:
%             \lfs{N}{\it case:nom \\ \it person:P}

%    Modified 1990 Dec 5 so that features are left aligned.
\newcommand{\fs}[1]%
{\mbox{\small%
$
\!
\left[
  \!\!
  \begin{tabular}{l}
    #1
  \end{tabular}
  \!\!
\right]
\!
$}}

%     Modified 1990 Dec 5 so that features are left aligned.
%\newcommand{\lfs}[2]
%   {
%     \mbox{$
%           \!\!
%           \begin{tabular}{c}
%           \it #1
%           \\
%           \mbox{\small%
%                 $
%                 \left[
%                 \!\!
%                 \it
%                 \begin{tabular}{l}
%                 #2
%                 \end{tabular}
%                 \!\!
%                 \right]
%                 $}
%           \end{tabular}
%           \!\!
%           $}
%   }

\newcommand{\ft}[2]{\path{#1}\hspace{1ex}\ftype{#2}}
\newcommand{\fsl}[2]{\fs{{\fftype{#1}} \\ #2}}

\newcommand{\fslt}[2]
 {\fst{
       {\fftype{#1}} \\
       #2 
     }
 }

\newcommand{\fsltt}[2]
 {\fstt{
       {\fftype{#1}} \\
       #2 
     }
 }

\newcommand{\fslttt}[2]
 {\fsttt{
       {\fftype{#1}} \\
       #2 
     }
 }


% jak \ft, \fs i \fsl tylko nieco ciasniejsze

\newcommand{\ftt}[2]
% {{\sc #1}\/{\rm #2}}
 {\textsc{#1}\/{\rm #2}}

\newcommand{\fst}[1]
  {
    \mbox{\small%
          $
          \left[
          \!\!\!
%          \sc
          \begin{tabular}{l} #1
          \end{tabular}
          \!\!\!\!\!\!\!
          \right]
          $
          }
   }

%\newcommand{\fslt}[2]
% {\fst{#2\\
%       {\scriptsize\it #1}
%      }
% }


% superciasne

\newcommand{\fstt}[1]
  {
    \mbox{\small%
          $
          \left[
          \!\!\!
%          \sc
          \begin{tabular}{l} #1
          \end{tabular}
          \!\!\!\!\!\!\!\!\!\!\!
          \right]
          $
          }
   }

%\newcommand{\fsltt}[2]
% {\fstt{#2\\
%       {\scriptsize\it #1}
%      }
% }

\newcommand{\fsttt}[1]
  {
    \mbox{\small%
          $
          \left[
          \!\!\!
%          \sc
          \begin{tabular}{l} #1
          \end{tabular}
          \!\!\!\!\!\!\!\!\!\!\!\!\!\!\!\!
          \right]
          $
          }
   }



% %add all your local new commands to this file

% \newcommand{\smiley}{:)}

% you are not supposed to mess with hardcore stuff, St.Mü. 22.08.2018
%% \renewbibmacro*{index:name}[5]{%
%%   \usebibmacro{index:entry}{#1}
%%     {\iffieldundef{usera}{}{\thefield{usera}\actualoperator}\mkbibindexname{#2}{#3}{#4}{#5}}}

% % \newcommand{\noop}[1]{}



% Rui

\newcommand{\spc}[0]{\hspace{-1pt}\underline{\hspace{6pt}}\,}
\newcommand{\spcs}[0]{\hspace{-1pt}\underline{\hspace{6pt}}\,\,}
\newcommand{\bad}[1]{\leavevmode\llap{#1}}
\newcommand{\COMMENT}[1]{}


% Rui coordination
\newcommand{\subl}[1]{$_{\scriptstyle \textsc{#1}}$}



% Andy Lücking gesture.tex
\newcommand{\Pointing}{\ding{43}}
% Giotto: "Meeting of Joachim and Anne at the Golden Gate" - 1305-10 
\definecolor{GoldenGate1}{rgb}{.13,.09,.13} % Dress of woman in black
\definecolor{GoldenGate2}{rgb}{.94,.94,.91} % Bridge
\definecolor{GoldenGate3}{rgb}{.06,.09,.22} % Blue sky
\definecolor{GoldenGate4}{rgb}{.94,.91,.87} % Dress of woman with shawl
\definecolor{GoldenGate5}{rgb}{.52,.26,.26} % Joachim's robe
\definecolor{GoldenGate6}{rgb}{.65,.35,.16} % Anne's robe
\definecolor{GoldenGate7}{rgb}{.91,.84,.42} % Joachim's halo

\makeatletter
\newcommand{\@Depth}{1} % x-dimension, to front
\newcommand{\@Height}{1} % z-dimension, up
\newcommand{\@Width}{1} % y-dimension, rightwards
%\GGS{<x-start>}{<y-start>}{<z-top>}{<z-bottom>}{<Farbe>}{<x-width>}{<y-depth>}{<opacity>}
\newcommand{\GGS}[9][]{%
\coordinate (O) at (#2-1,#3-1,#5);
\coordinate (A) at (#2-1,#3-1+#7,#5);
\coordinate (B) at (#2-1,#3-1+#7,#4);
\coordinate (C) at (#2-1,#3-1,#4);
\coordinate (D) at (#2-1+#8,#3-1,#5);
\coordinate (E) at (#2-1+#8,#3-1+#7,#5);
\coordinate (F) at (#2-1+#8,#3-1+#7,#4);
\coordinate (G) at (#2-1+#8,#3-1,#4);
\draw[draw=black, fill=#6, fill opacity=#9] (D) -- (E) -- (F) -- (G) -- cycle;% Front
\draw[draw=black, fill=#6, fill opacity=#9] (C) -- (B) -- (F) -- (G) -- cycle;% Top
\draw[draw=black, fill=#6, fill opacity=#9] (A) -- (B) -- (F) -- (E) -- cycle;% Right
}
\makeatother


% pragmatics
\newcommand{\speaking}[1]{\eqparbox{name}{\textsc{\lowercase{#1}\space}}}
\newcommand{\alname}[1]{\eqparbox{name}{\textsc{\lowercase{#1}}}}
\newcommand{\HPSGTTR}{HPSG$_{\text{TTR}}$\xspace}

\newcommand{\ttrtype}[1]{\textit{#1}}
\newcommand{\avmel}{\q<\quad\q>} %% shortcut for empty lists in AVM
\newcommand{\ttrmerge}{\ensuremath{\wedge_{\textit{merge}}}}
\newcommand{\Cat}[2][0.1pt]{%
  \begin{scope}[y=#1,x=#1,yscale=-1, inner sep=0pt, outer sep=0pt]
   \path[fill=#2,line join=miter,line cap=butt,even odd rule,line width=0.8pt]
  (151.3490,307.2045) -- (264.3490,307.2045) .. controls (264.3490,291.1410) and (263.2021,287.9545) .. (236.5990,287.9545) .. controls (240.8490,275.2045) and (258.1242,244.3581) .. (267.7240,244.3581) .. controls (276.2171,244.3581) and (286.3490,244.8259) .. (286.3490,264.2045) .. controls (286.3490,286.2045) and (323.3717,321.6755) .. (332.3490,307.2045) .. controls (345.7277,285.6390) and (309.3490,292.2151) .. (309.3490,240.2046) .. controls (309.3490,169.0514) and (350.8742,179.1807) .. (350.8742,139.2046) .. controls (350.8742,119.2045) and (345.3490,116.5037) .. (345.3490,102.2045) .. controls (345.3490,83.3070) and (361.9972,84.4036) .. (358.7581,68.7349) .. controls (356.5206,57.9117) and (354.7696,49.2320) .. (353.4652,36.1439) .. controls (352.5396,26.8573) and (352.2445,16.9594) .. (342.5985,17.3574) .. controls (331.2650,17.8250) and (326.9655,37.7742) .. (309.3490,39.2045) .. controls (291.7685,40.6320) and (276.7783,24.2380) .. (269.9740,26.5795) .. controls (263.2271,28.9013) and (265.3490,47.2045) .. (269.3490,60.2045) .. controls (275.6359,80.6368) and (289.3490,107.2045) .. (264.3490,111.2045) .. controls (239.3490,115.2045) and (196.3490,119.2045) .. (165.3490,160.2046) .. controls (134.3490,201.2046) and (135.4934,249.3212) .. (123.3490,264.2045) .. controls (82.5907,314.1553) and (40.8239,293.6463) .. (40.8239,335.2045) .. controls (40.8239,353.8102) and (72.3490,367.2045) .. (77.3490,361.2045) .. controls (82.3490,355.2045) and (34.8638,337.3259) .. (87.9955,316.2045) .. controls (133.3871,298.1601) and   (137.4391,294.4766) .. (151.3490,307.2045) -- cycle;
\end{scope}%
}


% KdK
\newcommand{\smiley}{:)}

\renewbibmacro*{index:name}[5]{%
  \usebibmacro{index:entry}{#1}
    {\iffieldundef{usera}{}{\thefield{usera}\actualoperator}\mkbibindexname{#2}{#3}{#4}{#5}}}

% \newcommand{\noop}[1]{}

% chngcntr.sty otherwise gives error that these are already defined
%\let\counterwithin\relax
%\let\counterwithout\relax

% the space of a left bracket for glossings
\newcommand{\LB}{\hspaceThis{[}}

\newcommand{\LF}{\mbox{$[\![$}}

\newcommand{\RF}{\mbox{$]\!]_F$}}

\newcommand{\RT}{\mbox{$]\!]_T$}}





% Manfred's

\newcommand{\kommentar}[1]{}

\newcommand{\bsp}[1]{\emph{#1}}
\newcommand{\bspT}[2]{\bsp{#1} `#2'}
\newcommand{\bspTL}[3]{\bsp{#1} (lit.: #2) `#3'}

\newcommand{\noidi}{§}

\newcommand{\refer}[1]{(\ref{#1})}

%\newcommand{\avmtype}[1]{\multicolumn{2}{l}{\type{#1}}}
\newcommand{\attr}[1]{\textsc{#1}}

\newcommand{\srdefault}{\mbox{\begin{tabular}{c}{\large <}\\[-1.5ex]$\sqcap$\end{tabular}}}

%% \newcommand{\myappcolumn}[2]{
%% \begin{minipage}[t]{#1}#2\end{minipage}
%% }

%% \newcommand{\appc}[1]{\myappcolumn{3.7cm}{#1}}


% Jong-Bok


% clean that up and do not use \def (killing other stuff defined before)
%\if 0
\newcommand\DEL{\textsc{del}}
\newcommand\del{\textsc{del}}

\newcommand\conn{\textsc{conn}}
\newcommand\CONN{\textsc{conn}}
\newcommand\CONJ{\textsc{conj}}
\newcommand\LITE{\textsc{lex}}
\newcommand\lite{\textsc{lex}}
\newcommand\HON{\textsc{hon}}

%\newcommand\CAUS{\textsc{caus}}
%\newcommand\PASS{\textsc{pass}}
\newcommand\NPST{\textsc{npst}}
%\newcommand\COND{\textsc{cond}}



\newcommand\hdlite{\textsc{head-lex construction}}
\newcommand\hdlight{\textsc{head-light} Schema}
\newcommand\NFORM{\textsc{nform}}

\newcommand\RELS{\textsc{rels}}
%\newcommand\TENSE{\textsc{tense}}


%\newcommand\ARG{\textsc{arg}}
\newcommand\ARGs{\textsc{arg0}}
\newcommand\ARGa{\textsc{arg}}
\newcommand\ARGb{\textsc{arg2}}
\newcommand\TPC{\textsc{top}}
%\newcommand\PROG{\textsc{prog}}

\newcommand\LIGHT{\textsc{light}\xspace}
\newcommand\pst{\textsc{pst}}
%\newcommand\PAST{\textsc{pst}}
%\newcommand\DAT{\textsc{dat}}
%\newcommand\CONJ{\textsc{conj}}
\newcommand\nominal{\textsc{nominal}}
\newcommand\NOMINAL{\textsc{nominal}}
\newcommand\VAL{\textsc{val}}
%\newcommand\val{\textsc{val}}
\newcommand\MODE{\textsc{mode}}
\newcommand\RESTR{\textsc{restr}}
\newcommand\SIT{\textsc{sit}}
\newcommand\ARG{\textsc{arg}}
\newcommand\RELN{\textsc{rel}}
%\newcommand\REL{\textsc{rel}}
%\newcommand\RELS{\textsc{rels}}
%\newcommand\arg-st{\textsc{arg-st}}
\newcommand\xdel{\textsc{xdel}}
\newcommand\zdel{\textsc{zdel}}
\newcommand\sug{\textsc{sug}}
%\newcommand\IMP{\textsc{imp}}
%\newcommand\conn{\textsc{conn}}
%\newcommand\CONJ{\textsc{conj}}
%\newcommand\HON{\textsc{hon}}
\newcommand\BN{\textsc{bn}}
\newcommand\bn{\textsc{bn}}
\newcommand\pres{\textsc{pres}}
\newcommand\PRES{\textsc{pres}}
\newcommand\prs{\textsc{pres}}
%\newcommand\PRS{\textsc{pres}}
\newcommand\agt{\textsc{agt}}
%\newcommand\DEL{\textsc{del}}
%\newcommand\PRED{\textsc{pred}}
\newcommand\AGENT{\textsc{agent}}
\newcommand\THEME{\textsc{theme}}
%\newcommand\AUX{\textsc{aux}}
%\newcommand\THEME{\textsc{theme}}
%\newcommand\PL{\textsc{pl}}
\newcommand\SRC{\textsc{src}}
\newcommand\src{\textsc{src}}
\newcommand{\FORMjb}{\textsc{form}}
\newcommand{\formjb}{\FORM}
\newcommand\GCASE{\textsc{gcase}}
\newcommand\gcase{\textsc{gcase}}
\newcommand\SCASE{\textsc{scase}}
\newcommand\PHON{\textsc{phon}}
%\newcommand\SS{\textsc{ss}}
\newcommand\SYN{\textsc{syn}}
%\newcommand\LOC{\textsc{loc}}
\newcommand\MOD{\textsc{mod}}
\newcommand\INV{\textsc{inv}}
%\newcommand\L{\textsc{l}}
%\newcommand\CASE{\textsc{case}}
\newcommand\SPR{\textsc{spr}}
\newcommand\COMPS{\textsc{comps}}
%\newcommand\comps{\textsc{comps}}
\newcommand\SEM{\textsc{sem}}
\newcommand\CONT{\textsc{cont}}
\newcommand\SUBCAT{\textsc{subcat}}
\newcommand\CAT{\textsc{cat}}
%\newcommand\C{\textsc{c}}
%\newcommand\SUBJ{\textsc{subj}}
\newcommand\subjjb{\textsc{subj}}
%\newcommand\SLASH{\textsc{slash}}
\newcommand\LOCAL{\textsc{local}}
%\newcommand\ARG-ST{\textsc{arg-st}}
%\newcommand\AGR{\textsc{agr}}
\newcommand\PER{\textsc{per}}
%\newcommand\NUM{\textsc{num}}
%\newcommand\IND{\textsc{ind}}
\newcommand\VFORM{\textsc{vform}}
\newcommand\PFORM{\textsc{pform}}
\newcommand\decl{\textsc{decl}}
%\newcommand\loc{\textsc{loc   }}
% \newcommand\   {\textsc{  }}

%\newcommand\NEG{\textsc{neg}}
\newcommand\FRAMES{\textsc{frames}}
%\newcommand\REFL{\textsc{refl}}

\newcommand\MKG{\textsc{mkg}}

%\newcommand\BN{\textsc{bn}}
\newcommand\HD{\textsc{hd}}
\newcommand\NP{\textsc{np}}
\newcommand\PF{\textsc{pf}}
%\newcommand\PL{\textsc{pl}}
\newcommand\PP{\textsc{pp}}
%\newcommand\SS{\textsc{ss}}
\newcommand\VF{\textsc{vf}}
\newcommand\VP{\textsc{vp}}
%\newcommand\bn{\textsc{bn}}
\newcommand\cl{\textsc{cl}}
%\newcommand\pl{\textsc{pl}}
\newcommand\Wh{\ital{Wh}}
%\newcommand\ng{\textsc{neg}}
\newcommand\wh{\ital{wh}}
%\newcommand\ACC{\textsc{acc}}
%\newcommand\AGR{\textsc{agr}}
\newcommand\AGT{\textsc{agt}}
\newcommand\ARC{\textsc{arc}}
%\newcommand\ARG{\textsc{arg}}
\newcommand\ARP{\textsc{arc}}
%\newcommand\AUX{\textsc{aux}}
%\newcommand\CAT{\textsc{cat}}
%\newcommand\COP{\textsc{cop}}
%\newcommand\DAT{\textsc{dat}}
\newcommand\NEWCOMMAND{\textsc{def}}
%\newcommand\DEL{\textsc{del}}
\newcommand\DOM{\textsc{dom}}
\newcommand\DTR{\textsc{dtr}}
%\newcommand\FUT{\textsc{fut}}
\newcommand\GAP{\textsc{gap}}
%\newcommand\GEN{\textsc{gen}}
%\newcommand\HON{\textsc{hon}}
%\newcommand\IMP{\textsc{imp}}
%\newcommand\IND{\textsc{ind}}
%\newcommand\INV{\textsc{inv}}
\newcommand\LEX{\textsc{lex}}
\newcommand\Lex{\textsc{lex}}
%\newcommand\LOC{\textsc{loc}}
%\newcommand\MOD{\textsc{mod}}
\newcommand\MRK{{\nr MRK}}
%\newcommand\NEG{\textsc{neg}}
\newcommand\NEW{\textsc{new}}
%\newcommand\NOM{\textsc{nom}}
%\newcommand\NUM{\textsc{num}}
%\newcommand\PER{\textsc{per}}
%\newcommand\PST{\textsc{pst}}
\newcommand\QUE{\textsc{que}}
%\newcommand\REL{\textsc{rel}}
\newcommand\SEL{\textsc{sel}}
%\newcommand\SEM{\textsc{sem}}
%\newcommand\SIT{\textsc{arg0}}
%\newcommand\SPR{\textsc{spr}}
%\newcommand\SRC{\textsc{src}}
\newcommand\SUG{\textsc{sug}}
%\newcommand\SYN{\textsc{syn}}
%\newcommand\TPC{\textsc{top}}
%\newcommand\VAL{\textsc{val}}
%\newcommand\acc{\textsc{acc}}
%\newcommand\agt{\textsc{agt}}
\newcommand\cop{\textsc{cop}}
%\newcommand\dat{\textsc{dat}}
\newcommand\foc{\textsc{focus}}
%\newcommand\FOC{\textsc{focus}}
\newcommand\fut{\textsc{fut}}
\newcommand\hon{\textsc{hon}}
\newcommand\imp{\textsc{imp}}
\newcommand\kes{\textsc{kes}}
%\newcommand\lex{\textsc{lex}}
%\newcommand\loc{\textsc{loc}}
\newcommand\mrk{{\nr MRK}}
%\newcommand\nom{\textsc{nom}}
%\newcommand\num{\textsc{num}}
\newcommand\plu{\textsc{plu}}
\newcommand\pne{\textsc{pne}}
%\newcommand\pst{\textsc{pst}}
\newcommand\pur{\textsc{pur}}
%\newcommand\que{\textsc{que}}
%\newcommand\src{\textsc{src}}
%\newcommand\sug{\textsc{sug}}
\newcommand\tpc{\textsc{top}}
%\newcommand\utt{\textsc{utt}}
%\newcommand\val{\textsc{val}}
%% \newcommand\LITE{\textsc{lex}}
%% \newcommand\PAST{\textsc{pst}}
%% \newcommand\POSP{\textsc{pos}}
%% \newcommand\PRS{\textsc{pres}}
%% \newcommand\mod{\textsc{mod}}%
%% \newcommand\newuse{{`kes'}}
%% \newcommand\posp{\textsc{pos}}
%% \newcommand\prs{\textsc{pres}}
%% \newcommand\psp{{\it en\/}}
%% \newcommand\skes{\textsc{kes}}
%% \newcommand\CASE{\textsc{case}}
%% \newcommand\CASE{\textsc{case}}
%% \newcommand\COMP{\textsc{comp}}
%% \newcommand\CONJ{\textsc{conj}}
%% \newcommand\CONN{\textsc{conn}}
%% \newcommand\CONT{\textsc{cont}}
%% \newcommand\DECL{\textsc{decl}}
%% \newcommand\FOCUS{\textsc{focus}}
%% %\newcommand\FORM{\textsc{form}} duplicate
%% \newcommand\FREL{\textsc{frel}}
%% \newcommand\GOAL{\textsc{goal}}
\newcommand\HEAD{\textsc{head}}
%% \newcommand\INDEX{\textsc{ind}}
%% \newcommand\INST{\textsc{inst}}
%% \newcommand\MODE{\textsc{mode}}
%% \newcommand\MOOD{\textsc{mood}}
%% \newcommand\NMLZ{\textsc{nmlz}}
%% \newcommand\PHON{\textsc{phon}}
%% \newcommand\PRED{\textsc{pred}}
%% %\newcommand\PRES{\textsc{pres}}
%% \newcommand\PROM{\textsc{prom}}
%% \newcommand\RELN{\textsc{pred}}
%% \newcommand\RELS{\textsc{rels}}
%% \newcommand\STEM{\textsc{stem}}
%% \newcommand\SUBJ{\textsc{subj}}
%% \newcommand\XARG{\textsc{xarg}}
%% \newcommand\bse{{\it bse\/}}
%% \newcommand\case{\textsc{case}}
%% \newcommand\caus{\textsc{caus}}
%% \newcommand\comp{\textsc{comp}}
%% \newcommand\conj{\textsc{conj}}
%% \newcommand\conn{\textsc{conn}}
%% \newcommand\decl{\textsc{decl}}
%% \newcommand\fin{{\it fin\/}}
%% %\newcommand\form{\textsc{form}}
%% \newcommand\gend{\textsc{gend}}
%% \newcommand\inf{{\it inf\/}}
%% \newcommand\mood{\textsc{mood}}
%% \newcommand\nmlz{\textsc{nmlz}}
%% \newcommand\pass{\textsc{pass}}
%% \newcommand\past{\textsc{past}}
%% \newcommand\perf{\textsc{perf}}
%% \newcommand\pln{{\it pln\/}}
%% \newcommand\pred{\textsc{pred}}


%% %\newcommand\pres{\textsc{pres}}
%% \newcommand\proc{\textsc{proc}}
%% \newcommand\nonfin{{\it nonfin\/}}
%% \newcommand\AGENT{\textsc{agent}}
%% \newcommand\CFORM{\textsc{cform}}
%% %\newcommand\COMPS{\textsc{comps}}
%% \newcommand\COORD{\textsc{coord}}
%% \newcommand\COUNT{\textsc{count}}
%% \newcommand\EXTRA{\textsc{extra}}
%% \newcommand\GCASE{\textsc{gcase}}
%% \newcommand\GIVEN{\textsc{given}}
%% \newcommand\LOCAL{\textsc{local}}
%% \newcommand\NFORM{\textsc{nform}}
%% \newcommand\PFORM{\textsc{pform}}
%% \newcommand\SCASE{\textsc{scase}}
%% \newcommand\SLASH{\textsc{slash}}
%% \newcommand\SLASH{\textsc{slash}}
%% \newcommand\THEME{\textsc{theme}}
%% \newcommand\TOPIC{\textsc{topic}}
%% \newcommand\VFORM{\textsc{vform}}
%% \newcommand\cause{\textsc{cause}}
%% %\newcommand\comps{\textsc{comps}}
%% \newcommand\gcase{\textsc{gcase}}
%% \newcommand\itkes{{\it kes\/}}
%% \newcommand\pass{{\it pass\/}}
%% \newcommand\vform{\textsc{vform}}
%% \newcommand\CCONT{\textsc{c-cont}}
%% \newcommand\GN{\textsc{given-new}}
%% \newcommand\INFO{\textsc{info-st}}
%% \newcommand\ARG-ST{\textsc{arg-st}}
%% \newcommand\SUBCAT{\textsc{subcat}}
%% \newcommand\SYNSEM{\textsc{synsem}}
%% \newcommand\VERBAL{\textsc{verbal}}
%% \newcommand\arg-st{\textsc{arg-st}}
%% \newcommand\plain{{\it plain}\/}
%% \newcommand\propos{\textsc{propos}}
%% \newcommand\ADVERBIAL{\textsc{advl}}
%% \newcommand\HIGHLIGHT{\textsc{prom}}
%% \newcommand\NOMINAL{\textsc{nominal}}

\newenvironment{myavm}{\begingroup\avmvskip{.1ex}
  \selectfont\begin{avm}}%
{\end{avm}\endgroup\medskip}
\newcommand\pfix{\vspace{-5pt}}


\newcommand{\jbsub}[1]{\lower4pt\hbox{\small #1}}
\newcommand{\jbssub}[1]{\lower4pt\hbox{\small #1}}
\newcommand\jbtr{\underbar{\ \ \ }\ }


%\fi

% cl

\newcommand{\delphin}{\textsc{delph-in}}


% YK -- CG chapter

\newcommand{\grey}[1]{\colorbox{mycolor}{#1}}
\definecolor{mycolor}{gray}{0.8}

\newcommand{\GQU}[2]{\raisebox{1.6ex}{\ensuremath{\rotatebox{180}{\textbf{#1}}_{\scalebox{.7}{\textbf{#2}}}}}}

\newcommand{\SetInfLen}{\setpremisesend{0pt}\setpremisesspace{10pt}\setnamespace{0pt}}

\newcommand{\pt}[1]{\ensuremath{\mathsf{#1}}}
\newcommand{\ptv}[1]{\ensuremath{\textsf{\textsl{#1}}}}

\newcommand{\sv}[1]{\ensuremath{\bm{\mathcal{#1}}}}
\newcommand{\sX}{\sv{X}}
\newcommand{\sF}{\sv{F}}
\newcommand{\sG}{\sv{G}}

\newcommand{\syncat}[1]{\textrm{#1}}
\newcommand{\syncatVar}[1]{\ensuremath{\mathit{#1}}}

\newcommand{\RuleName}[1]{\textrm{#1}}

\newcommand{\SemTyp}{\textsf{Sem}}

\newcommand{\E}{\ensuremath{\bm{\epsilon}}\xspace}

\newcommand{\greeka}{\upalpha}
\newcommand{\greekb}{\upbeta}
\newcommand{\greekd}{\updelta}
\newcommand{\greekp}{\upvarphi}
\newcommand{\greekr}{\uprho}
\newcommand{\greeks}{\upsigma}
\newcommand{\greekt}{\uptau}
\newcommand{\greeko}{\upomega}
\newcommand{\greekz}{\upzeta}

\newcommand{\Lemma}{\ensuremath{\hskip.5em\vdots\hskip.5em}\noLine}
\newcommand{\LemmaAlt}{\ensuremath{\hskip.5em\vdots\hskip.5em}}

\newcommand{\I}{\iota}

\newcommand{\sem}{\ensuremath}

\newcommand{\NoSem}{%
\renewcommand{\LexEnt}[3]{##1; \syncat{##3}}
\renewcommand{\LexEntTwoLine}[3]{\renewcommand{\arraystretch}{.8}%
\begin{array}[b]{l} ##1;  \\ \syncat{##3} \end{array}}
\renewcommand{\LexEntThreeLine}[3]{\renewcommand{\arraystretch}{.8}%
\begin{array}[b]{l} ##1; \\ \syncat{##3} \end{array}}}

\newcommand{\hypml}[2]{\left[\!\!#1\!\!\right]^{#2}}

%%%%for bussproof
\def\defaultHypSeparation{\hskip0.1in}
\def\ScoreOverhang{0pt}

\newcommand{\MultiLine}[1]{\renewcommand{\arraystretch}{.8}%
\ensuremath{\begin{array}[b]{l} #1 \end{array}}}

\newcommand{\MultiLineMod}[1]{%
\ensuremath{\begin{array}[t]{l} #1 \end{array}}}

\newcommand{\hypothesis}[2]{[ #1 ]^{#2}}

\newcommand{\LexEnt}[3]{#1; \ensuremath{#2}; \syncat{#3}}

\newcommand{\LexEntTwoLine}[3]{\renewcommand{\arraystretch}{.8}%
\begin{array}[b]{l} #1; \\ \ensuremath{#2};  \syncat{#3} \end{array}}

\newcommand{\LexEntThreeLine}[3]{\renewcommand{\arraystretch}{.8}%
\begin{array}[b]{l} #1; \\ \ensuremath{#2}; \\ \syncat{#3} \end{array}}

\newcommand{\LexEntFiveLine}[5]{\renewcommand{\arraystretch}{.8}%
\begin{array}{l} #1 \\ #2; \\ \ensuremath{#3} \\ \ensuremath{#4}; \\ \syncat{#5} \end{array}}

\newcommand{\LexEntFourLine}[4]{\renewcommand{\arraystretch}{.8}%
\begin{array}{l} \pt{#1} \\ \pt{#2}; \\ \syncat{#4} \end{array}}

\newcommand{\ManySomething}{\renewcommand{\arraystretch}{.8}%
\raisebox{-3mm}{\begin{array}[b]{c} \vdots \,\,\,\,\,\, \vdots \\
\vdots \end{array}}}

\newcommand{\lemma}[1]{\renewcommand{\arraystretch}{.8}%
\begin{array}[b]{c} \vdots \\ #1 \end{array}}

\newcommand{\lemmarev}[1]{\renewcommand{\arraystretch}{.8}%
\begin{array}[b]{c} #1 \\ \vdots \end{array}}

\newcommand{\p}{\ensuremath{\upvarphi}}

% clashes with soul package
\newcommand{\yusukest}{\textbf{\textsf{st}}}

\newcommand{\shortarrow}{\xspace\hskip-1.2ex\scalebox{.5}[1]{\ensuremath{\bm{\rightarrow}}}\hskip-.5ex\xspace}

\newcommand{\SemInt}[1]{\mbox{$[\![ \textrm{#1} ]\!]$}}

\newcommand{\HypSpace}{\hskip-.8ex}
\newcommand{\RaiseHeight}{\raisebox{2.2ex}}
\newcommand{\RaiseHeightLess}{\raisebox{1ex}}

\newcommand{\ThreeColHyp}[1]{\RaiseHeight{\Bigg[}\HypSpace#1\HypSpace\RaiseHeight{\Bigg]}}
\newcommand{\TwoColHyp}[1]{\RaiseHeightLess{\Big[}\HypSpace#1\HypSpace\RaiseHeightLess{\Big]}}

\newcommand{\LemmaShort}{\ensuremath{ \ \vdots} \ \noLine}
\newcommand{\LemmaShortAlt}{\ensuremath{ \ \vdots} \ }

\newcommand{\fail}{**}
\newcommand{\vs}{\raisebox{.05em}{\ensuremath{\upharpoonright}}}
\newcommand{\DerivSize}{\small}

\def\maru#1{{\ooalign{\hfil
  \ifnum#1>999 \resizebox{.25\width}{\height}{#1}\else%
  \ifnum#1>99 \resizebox{.33\width}{\height}{#1}\else%
  \ifnum#1>9 \resizebox{.5\width}{\height}{#1}\else #1%
  \fi\fi\fi%
\/\hfil\crcr%
\raise.167ex\hbox{\mathhexbox20D}}}}

\newenvironment{samepage2}%
 {\begin{flushleft}\begin{minipage}{\linewidth}}
 {\end{minipage}\end{flushleft}}

\newcommand{\cmt}[1]{\textsl{\textbf{[#1]}}}
\newcommand{\trns}[1]{\textbf{#1}\xspace}
\newcommand{\ptfont}{}
\newcommand{\gp}{\underline{\phantom{oo}}}
\newcommand{\mgcmt}{\marginnote}

\newcommand{\term}[1]{\emph{#1}}

\newcommand{\citeposs}[1]{\citeauthor{#1}'s \citeyearpar{#1}}

% for standalone compilations Felix: This is in the class already
%\let\thetitle\@title
%\let\theauthor\@author 
\makeatletter
\newcommand{\togglepaper}[1][0]{ 
\bibliography{../Bibliographies/stmue,../localbibliography,
../Bibliographies/properties,
../Bibliographies/np,
../Bibliographies/negation,
../Bibliographies/ellipsis,
../Bibliographies/binding,
../Bibliographies/complex-predicates,
../Bibliographies/control-raising,
../Bibliographies/coordination,
../Bibliographies/morphology,
../Bibliographies/lfg,
collection.bib}
  %% hyphenation points for line breaks
%% Normally, automatic hyphenation in LaTeX is very good
%% If a word is mis-hyphenated, add it to this file
%%
%% add information to TeX file before \begin{document} with:
%% %% hyphenation points for line breaks
%% Normally, automatic hyphenation in LaTeX is very good
%% If a word is mis-hyphenated, add it to this file
%%
%% add information to TeX file before \begin{document} with:
%% \include{localhyphenation}
\hyphenation{
A-la-hver-dzhie-va
anaph-o-ra
ana-phor
ana-phors
an-te-ced-ent
an-te-ced-ents
affri-ca-te
affri-ca-tes
ap-proach-es
Atha-bas-kan
Athe-nä-um
Bona-mi
Chi-che-ŵa
com-ple-ments
con-straints
Cope-sta-ke
Da-ge-stan
Dor-drecht
er-klä-ren-de
Ginz-burg
Gro-ning-en
Jap-a-nese
Jon-a-than
Ka-tho-lie-ke
Ko-bon
krie-gen
Le-Sourd
moth-er
Mül-ler
Nie-mey-er
Par-a-digm
Prze-piór-kow-ski
phe-nom-e-non
re-nowned
Rie-he-mann
un-bound-ed
with-in
}

% listing within here does not have any effect for lfg.tex % 2020-05-14

% why has "erklärende" be listed here? I specified langid in bibtex item. Something is still not working with hyphenation.


% to do: check
%  Alahverdzhieva


% biblatex:

% This is a LaTeX frontend to TeX’s \hyphenation command which defines hy- phenation exceptions. The ⟨language⟩ must be a language name known to the babel/polyglossia packages. The ⟨text ⟩ is a whitespace-separated list of words. Hyphenation points are marked with a dash:

% \DefineHyphenationExceptions{american}{%
% hy-phen-ation ex-cep-tion }

\hyphenation{
A-la-hver-dzhie-va
anaph-o-ra
ana-phor
ana-phors
an-te-ced-ent
an-te-ced-ents
affri-ca-te
affri-ca-tes
ap-proach-es
Atha-bas-kan
Athe-nä-um
Bona-mi
Chi-che-ŵa
com-ple-ments
con-straints
Cope-sta-ke
Da-ge-stan
Dor-drecht
er-klä-ren-de
Ginz-burg
Gro-ning-en
Jap-a-nese
Jon-a-than
Ka-tho-lie-ke
Ko-bon
krie-gen
Le-Sourd
moth-er
Mül-ler
Nie-mey-er
Par-a-digm
Prze-piór-kow-ski
phe-nom-e-non
re-nowned
Rie-he-mann
un-bound-ed
with-in
}

% listing within here does not have any effect for lfg.tex % 2020-05-14

% why has "erklärende" be listed here? I specified langid in bibtex item. Something is still not working with hyphenation.


% to do: check
%  Alahverdzhieva


% biblatex:

% This is a LaTeX frontend to TeX’s \hyphenation command which defines hy- phenation exceptions. The ⟨language⟩ must be a language name known to the babel/polyglossia packages. The ⟨text ⟩ is a whitespace-separated list of words. Hyphenation points are marked with a dash:

% \DefineHyphenationExceptions{american}{%
% hy-phen-ation ex-cep-tion }

  \memoizeset{
    memo filename prefix={hpsg-handbook.memo.dir/},
    % readonly
  }
  \papernote{\scriptsize\normalfont
    \@author.
    \@title. 
    To appear in: 
    Stefan Müller, Anne Abeillé, Robert D. Borsley \& Jean-Pierre Koenig (eds.)
    HPSG Handbook
    Berlin: Language Science Press. [preliminary page numbering]
  }
  \pagenumbering{roman}
  \setcounter{chapter}{#1}
  \addtocounter{chapter}{-1}
}
\makeatother

\makeatletter
\newcommand{\togglepaperminimal}[1][0]{ 
  \bibliography{../Bibliographies/stmue,
                ../localbibliography,
  ../Bibliographies/coordination,
collection.bib}
  %% hyphenation points for line breaks
%% Normally, automatic hyphenation in LaTeX is very good
%% If a word is mis-hyphenated, add it to this file
%%
%% add information to TeX file before \begin{document} with:
%% %% hyphenation points for line breaks
%% Normally, automatic hyphenation in LaTeX is very good
%% If a word is mis-hyphenated, add it to this file
%%
%% add information to TeX file before \begin{document} with:
%% \include{localhyphenation}
\hyphenation{
A-la-hver-dzhie-va
anaph-o-ra
ana-phor
ana-phors
an-te-ced-ent
an-te-ced-ents
affri-ca-te
affri-ca-tes
ap-proach-es
Atha-bas-kan
Athe-nä-um
Bona-mi
Chi-che-ŵa
com-ple-ments
con-straints
Cope-sta-ke
Da-ge-stan
Dor-drecht
er-klä-ren-de
Ginz-burg
Gro-ning-en
Jap-a-nese
Jon-a-than
Ka-tho-lie-ke
Ko-bon
krie-gen
Le-Sourd
moth-er
Mül-ler
Nie-mey-er
Par-a-digm
Prze-piór-kow-ski
phe-nom-e-non
re-nowned
Rie-he-mann
un-bound-ed
with-in
}

% listing within here does not have any effect for lfg.tex % 2020-05-14

% why has "erklärende" be listed here? I specified langid in bibtex item. Something is still not working with hyphenation.


% to do: check
%  Alahverdzhieva


% biblatex:

% This is a LaTeX frontend to TeX’s \hyphenation command which defines hy- phenation exceptions. The ⟨language⟩ must be a language name known to the babel/polyglossia packages. The ⟨text ⟩ is a whitespace-separated list of words. Hyphenation points are marked with a dash:

% \DefineHyphenationExceptions{american}{%
% hy-phen-ation ex-cep-tion }

\hyphenation{
A-la-hver-dzhie-va
anaph-o-ra
ana-phor
ana-phors
an-te-ced-ent
an-te-ced-ents
affri-ca-te
affri-ca-tes
ap-proach-es
Atha-bas-kan
Athe-nä-um
Bona-mi
Chi-che-ŵa
com-ple-ments
con-straints
Cope-sta-ke
Da-ge-stan
Dor-drecht
er-klä-ren-de
Ginz-burg
Gro-ning-en
Jap-a-nese
Jon-a-than
Ka-tho-lie-ke
Ko-bon
krie-gen
Le-Sourd
moth-er
Mül-ler
Nie-mey-er
Par-a-digm
Prze-piór-kow-ski
phe-nom-e-non
re-nowned
Rie-he-mann
un-bound-ed
with-in
}

% listing within here does not have any effect for lfg.tex % 2020-05-14

% why has "erklärende" be listed here? I specified langid in bibtex item. Something is still not working with hyphenation.


% to do: check
%  Alahverdzhieva


% biblatex:

% This is a LaTeX frontend to TeX’s \hyphenation command which defines hy- phenation exceptions. The ⟨language⟩ must be a language name known to the babel/polyglossia packages. The ⟨text ⟩ is a whitespace-separated list of words. Hyphenation points are marked with a dash:

% \DefineHyphenationExceptions{american}{%
% hy-phen-ation ex-cep-tion }

  \memoizeset{
    memo filename prefix={hpsg-handbook.memo.dir/},
    % readonly
  }
  \papernote{\scriptsize\normalfont
    \@author.
    \@title. 
    To appear in: 
    Stefan Müller, Anne Abeillé, Robert D. Borsley \& Jean-Pierre Koenig (eds.)
    HPSG Handbook
    Berlin: Language Science Press. [preliminary page numbering]
  }
  \pagenumbering{roman}
  \setcounter{chapter}{#1}
  \addtocounter{chapter}{-1}
}
\makeatother




% In case that year is not given, but pubstate. This mainly occurs for titles that are forthcoming, in press, etc.
\renewbibmacro*{addendum+pubstate}{% Thanks to https://tex.stackexchange.com/a/154367 for the idea
  \printfield{addendum}%
  \iffieldequalstr{labeldatesource}{pubstate}{}
  {\newunit\newblock\printfield{pubstate}}
}

\DeclareLabeldate{%
    \field{date}
    \field{year}
    \field{eventdate}
    \field{origdate}
    \field{urldate}
    \field{pubstate}
    \literal{nodate}
}

%\defbibheading{diachrony-sources}{\section*{Sources}} 

% if no langid is set, it is English:
% https://tex.stackexchange.com/a/279302
\DeclareSourcemap{
  \maps[datatype=bibtex]{
    \map{
      \step[fieldset=langid, fieldvalue={english}]
    }
  }
}


% for bibliographies
% biber/biblatex could use sortname field rather than messing around this way.
\newcommand{\SortNoop}[1]{}


% Doug Ball

\newcommand{\elist}{\q<\ \ \q>}

\newcommand{\esetDB}{\q\{\ \ \q\}}


\makeatletter

\newcommand{\nolistbreak}{%

  \let\oldpar\par\def\par{\oldpar\nobreak}% Any \par issues a \nobreak

  \@nobreaktrue% Don't break with first \item

}

\makeatother


% intermediate before Frank's trees are fixed
% This will be removed!!!!!
%\newcommand{\tree}[1]{} % ignore them blody trees
%\usepackage{tree-dvips}


\newcommand{\nodeconnect}[2]{}
\newcommand{\nodetriangle}[2]{}



% Doug relative clauses
%% I've compiled out almost all my private LaTeX command, but there are some
%% I found hard to get rid of. They are defined here.
%% There are few others which defined in places in the document where they have only
%% local effect (e.g. within figures); their names all end in DA, e.g. \MotherDA
%% There are a lot of \labels -- they are all of the form \label{sec:rc-...} or
%% \label{x:rc-...} or similar, so there should be no clashes.

% Subscripts -- scriptsize italic shape lowered by .25ex 
\newcommand{\subscr}[1]{\raisebox{-.5ex}{\protect{\scriptsize{\itshape #1\/}}}}
% A boxed subscript, for avm tags in normal text
\newcommand{\subtag}[1]{\subscr{\idx{#1}}}

%% Sets and tuples: I use \setof{} to get brackets that are upright, not slanted
%\newcommand{\setof}[1]{\ensuremath{\lbrace\,\mathit{#1}\,\rbrace}}
% 11.10.2019 EP: Doug requested replacement of existing \setof definition with the following:
%\newcommand{\setof}[1]{\begin{avm}\{\textcolor{red}{#1}\}\end{avm}}
% 31.1.2019 EP: Doug requested re-replacement of the above \textcolour version with the following:
\newcommand{\setof}[1]{\begin{avm}\{#1\}\end{avm}}

\newcommand{\tuple}[1]{\ensuremath{\left\langle\,\mbox{\textit{#1}}\,\right\rangle}}

% Single pile of stuff, optional arugment is psn (e.g. t or b)
% e.g. to put a over b over c in a centered column, top aligned, do:
%   \cPile[t]{a\\b\\c} 
\newcommand{\cPile}[2][]{%
  \begingroup%
  \renewcommand{\arraystretch}{.5}\begin{tabular}[#1]{c}#2\end{tabular}%
  \endgroup%
}

%% for linguistic examples in running text (`linguistic citation'):
\newcommand{\lic}[1]{\textit{#1}}

%% A gap marked by an underline, raised slightly
%% Default argument indicates how long the line should be:
\newcommand{\uGap}[1][3ex]{\raisebox{.25em}{\underline{\hspace{#1}}}\xspace}

%% \TnodeDA{XP}{avmcontents} -- in a Tree, put a node label next to an AVM
\newcommand{\TnodeDA}[2]{#1~\begin{avm}{#2}\end{avm}}

%% This allows tipa stuff to be put in \emph -- we need to change to cmr first.
%% It is used in the discussion of Arabic.
\newcommand{\emphtipa}[1]{{\fontfamily{cmr}\emph{\tipaencoding #1}}} 



 
 
\definecolor{lsDOIGray}{cmyk}{0,0,0,0.45}


% morphology.tex:
% Berthold

\newcommand{\dnode}[1]{\rnode{#1}{\fbox{#1}}}
\newcommand{\tnode}[1]{\rnode{#1}{\textit{#1}}}

\newcommand{\tl}[2]{#2}

\newcommand{\rrr}[3]{%
  \psframebox[linestyle=none]{%
    \avmoptions{center}
    \begin{avm}
      \[mud & \{ #1 \}\\
      ms & \{ #2 \}\\
      mph & \<  #3 \> \]
    \end{avm}
  }
}
\newcommand{\rr}[2]{%
  \psframebox[linestyle=none]{%
    \avmoptions{center}
    \begin{avm}
      \[mud & \{ #1 \}\\
      mph & \<  #2 \> \]
    \end{avm}
  }
}
 

% Frank Richter
\newtheorem{mydef}{Definition}

\long\def\set[#1\set=#2\set]%
{%
\left\{%
\tabcolsep 1pt%
\begin{tabular}{l}%
#1%
\end{tabular}%
\left|%
\tabcolsep 1pt%
\begin{tabular}{l}%
#2%
\end{tabular}%
\right.%
\right\}%
}

\newcommand{\einruck}{\\ \hspace*{1em}}


%\newcommand{\NatNum}{\mathrm{I\hspace{-.17em}N}}
\newcommand{\NatNum}{\mathbb{N}}
\newcommand{\Aug}[1]{\widehat{#1}}
%\newcommand{\its}{\mathrm{:}}
% Felix 14.02.2020
\DeclareMathOperator{\its}{:}

\newcommand{\sequence}[1]{\langle#1\rangle}

\newcommand{\INTERPRETATION}[2]{\sequence{#1\mathsf{U}#2,#1\mathsf{S}#2,#1\mathsf{A}#2,#1\mathsf{R}#2}}
\newcommand{\Interpretation}{\INTERPRETATION{}{}}

\newcommand{\Inte}{\mathsf{I}}
\newcommand{\Unive}{\mathsf{U}}
\newcommand{\Speci}{\mathsf{S}}
\newcommand{\Atti}{\mathsf{A}}
\newcommand{\Reli}{\mathsf{R}}
\newcommand{\ReliT}{\mathsf{RT}}

\newcommand{\VarInt}{\mathsf{G}}
\newcommand{\CInt}{\mathsf{C}}
\newcommand{\Tinte}{\mathsf{T}}
\newcommand{\Dinte}{\mathsf{D}}

% this was missing from ash's stuff.

%% \def \optrulenode#1{
%%   \setbox1\hbox{$\left(\hbox{\begin{tabular}{@{\strut}c@{\strut}}#1\end{tabular}}\right)$}
%%   \raisebox{1.9ex}{\raisebox{-\ht1}{\copy1}}}



\newcommand{\pslabel}[1]{}

\newcommand{\addpagesunless}{\todostefan{add pages unless you cite the
 work as such}}

% dg.tex
% framed boxes as used in dg.tex
% original idea from stackexchange, but modified by Saso
% http://tex.stackexchange.com/questions/230300/doing-something-like-psframebox-in-tikz#230306
\tikzset{
  frbox/.style={
    rounded corners,
    draw,
    thick,
    inner sep=5pt,
    anchor=base,
  },
}

% get rid of these morewrite messages:
% https://tex.stackexchange.com/questions/419489/suppressing-messages-to-standard-output-from-package-morewrites/419494#419494
\ExplSyntaxOn
\cs_set_protected:Npn \__morewrites_shipout_ii:
  {
    \__morewrites_before_shipout:
    \__morewrites_tex_shipout:w \tex_box:D \g__morewrites_shipout_box
    \edef\tmp{\interactionmode\the\interactionmode\space}\batchmode\__morewrites_after_shipout:\tmp
  }
\ExplSyntaxOff


% This is for places where authors used bold. I replace them by \emph
% but have the information where the bold was. St. Mü. 09.05.2020
\newcommand{\textbfemph}[1]{\emph{#1}}



% Felix 09.06.2020: copy code from the third line into localcommands.tex: https://github.com/langsci/langscibook#defined-environments-commands-etc
\patchcmd{\mkbibindexname}{\ifdefvoid{#3}{}{\MakeCapital{#3} }}{\ifdefvoid{#3}{}{#3 }}{}{\AtEndDocument{\typeout{mkbibindexname could not be patched.}}}

  \togglepaper[19]
}{}


\usepackage{avm}
\usepackage{linguex}
\usepackage{lsp-gb4e}
\usepackage{gb4e}
\usepackage{avm+}
\usepackage[normalem]{ulem}


\author{%
	Joanna Nykiel\affiliation{Kyung Hee University, Seoul}%
	\lastand Jong-Bok Kim\affiliation{Kyung Hee University, Seoul}%
}
\title{Ellipsis}

% \chapterDOI{} %will be filled in at production

%\epigram{Change epigram in chapters/03.tex or remove it there }

\abstract{This chapter provides an overview of HPSG analyses of ellipsis. The structure of the chapter follows three types of ellipsis, nonsentential utterances, predicate ellipsis (including VP ellipsis), and nonconstituent coordination, with three types of analyses applied to them. These analyses characteristically do not admit silent syntactic material for any ellipsis phenomena with the exception of certain types of nonconstituent coordination.}


\begin{document}
\maketitle
\label{chap-ellipsis}

{\avmoptions{center}

%\if0
\section{Introduction}
\label{ellipsis-sec-introduction}

Ellipsis is a phenomenon that involves a noncanonical mapping between syntax and semantics. What appears to be a syntactically incomplete utterance still receives a semantically complete representation, based on the features of the surrounding context, be the context linguistic or nonlinguistic. The goal of syntactic theory is thus to account for how the complete semantics can be reconciled with the apparently incomplete syntax. One of the key questions here relates to the structure of the ellipsis site, that is, whether or not we should assume the presence of invisible syntactic material. Section~\ref{sec-three-types-of-ellipsis} introduces three types of ellipsis (nonsentential utterances, predicate ellipsis, and nonconstituent coordination) that have attracted considerable attention and received treatment within HPSG (our focus here is on standard HPSG rather than Sign-Based Construction Grammar, \citealt{Sag2012,Abeille2020}). 
 In Section~\ref{sec-evidence-for-invisible-material} we overview existing evidence for and against the so-called WYSIWYG (`What You See Is What You Get') approach to ellipsis, where no invisible material is posited at the ellipsis site. Finally in Sections~\ref{sec-analyses-of-NSUs}--\ref{sec-analyses-of-noncon}, we walk the reader through three types of HPSG analyses applied to the three types of ellipsis presented in Section~\ref{sec-three-types-of-ellipsis}. Our purpose is to highlight the nonuniformity of these analyses, along with the underlying intuition that ellipsis is not a uniform phenomenon. Throughout the chapter we also draw the reader's attention to the key role that corpus and experimental data play in HPSG theorizing, setting it aside from frameworks that primarily rely on intuitive judgments.


%\inlinetodostefan{refer to other sections as well. This is usually done in introductions.}


\section{Three types of ellipsis}
\label{sec-three-types-of-ellipsis}

Depending on the type of analysis by means of which HPSG handles them, elliptical phenomena can be broadly divided into three types:
         nonsentential utterances, predicate ellipsis, and nonconstituent coordination.\footnote{Part of the discussion here is evolved from \citet{Kim2020}.}
          We overview the key features of these types here before discussing in greater detail how they have been brought to bear on the question of whether there is invisible syntactic structure at the ellipsis site or not. We begin with stranded XPs, which HPSG treats as nonsentential utterances, and then move on to predicate and argument ellipsis, followed by phenomena known as nonconstituent coordination.


\subsection{Nonsentential utterances}
This section introduces utterances smaller than a sentence, which we refer to as \emph{\isi{nonsentential utterances}} (NSUs). These range from \emph{\isi{Bare Argument Ellipsis}} BAE, a term used in \citealt{CJ2005a} (\ref{1}), to fragment answers (\ref{2})
and direct or embedded \isi{sluicing} (\ref{3})--(\ref{4}):

\ea A: You were angry with them.\\ B: Yeah, angry with them and angry with the situation.\label{1}\z

\ea A: Where are we? \\B: In Central Park.\label{2}\z

\ea A: So what did you think about that?\\ B: About what? \label{3}\z

\ea A: There's someone at the door. \\B: Who?/I wonder who. \label{4}\z
%
As illustrated by these examples, sluicing hosts stranded \emph{wh}-phrases and has the function of an interrogative clause, while BAE hosts XPs representing various syntactic categories and typically has the function of a declarative clause (\citealt{Ginzburg:Sag:2000,CJ2005a}).\footnote{Several subtypes of nonsentential utterances can be distinguished, based on their contextual functions, which we leave it open here (for a recent taxonomy, see \citealt[217]{Ginzburg2012}).}

The key theoretical question NSUs raise is whether they are parts of larger sentential structures or rather nonsentential structures whose semantic and morphosyntactic features are licensed by the surrounding context. To adjudicate between these views, researchers have looked for evidence that NSUs in fact behave as if they were fragments of sentences. As we will see in Section~\ref{sec-evidence-for-invisible-material}, there is evidence to support both of these views. However, HPSG doesn't assume that NSUs are underlyingly sentential structures.

\subsection{Predicate ellipsis and argument ellipsis}
The section looks at three constructions whose syntax includes null or unexpressed elements. They are Post-Auxiliary Ellipsis (PAE)\footnote{The term PAE, introduced by \citealt{Sag1976}, can cover examples like where a non-VP is
elided after an auxiliary verb as in \textit{You think
I am dumb, but I am not.}}  introduced for what is more typically referred to as Verb Phrase Ellipsis (VPE)} and pseudogapping, Null Complement Anaphora (NCA), and argument drop (or pro drop). PAE features stranded auxiliary verbs (\ref{5}) while pseudogapping, also introduced by an auxiliary verb,
has a remnant right after the pseudo gap (\ref{pg}):
%
%
%
%stranded auxiliary verbs followed by XPs corresponding to complements to verbal heads %resent in the antecedent or to adjuncts (\ref{pg}).
%
 NCA is characterized by omission of complements to some lexical verbs (\ref{6}), while argument drop refers to omission of a pronominal subject or an object argument, as illustrated in (\ref{7}) for Polish.

\ea A: I didn't ask George to invite you.\\B: Then who did?\label{5}\z

\ea The dentist didn't call Sally today but they might tomorrow. \label{pg}\z

\ea Some mornings you can't get hot water in the shower, but nobody complains. \label{6} \z

\ea
\gll Pia p\'{o}\'{z}no wr\'{o}ci\l a do domu. Od razu posz\l a spa\'{c}.\\
Pia late got to home right away went sleep\\
\glt `Pia got home late. She went straight to bed.'
\label{7}
\z
%
One key question raised from such constructions
 is whether these unrealized null elements should be assumed to be underlyingly present in the syntax of theses constructions, and the answer is no. Another question is whether theoretical analyses of constructions like PAE should be enriched with usage preferences since these constructions compete with \textit{do it/that/so} anaphora in predictable ways (see \citealt{Miller2011} for a proposal).


\subsection{Nonconstituent coordination}

We focus on three instances of nonconstituent coordination --- right node raising (RNR), argument
cluster coordination (ACC), and gapping (\citealt{Ross1967}) --- illustrated in (\ref{8}), (\ref{acc}), and (\ref{9}), respectively.

 \ea Ethan sold and Rasmus gave away [all his CDs]. (RNR) \label{8}\z

\ea Harvey [gave] a book to Ethan and a record to Rasmus. (ACC) \label{acc}\z

 \ea Ethan [gave away] his CDs and Rasmus his old guitar. (Gapping)\label{9}\z
 %
In RNR, a single constituent located in the right-peripheral position is associated with both conjuncts. In both ACC and gapping, a finite verb is associated with both (or more) conjuncts but only present in the leftmost one. Additionally in ACC, the subject of the first conjunct is also associated with the second conjunct but only present in the former. These phenomena illustrate what appears to be coordination of standard constituents with elements not normally defined as constituents (a stranded transitive verb in (\ref{8}), a cluster of NP and PP in (\ref{acc}), and a cluster of NPs in (\ref{9})).
%

 To handle such constructions the grammar must be permitted to (a) coordinate noncanonical constituents, (b) generate coordinated constituents parts of which are subject to an operation akin to deletion, or (c) coordinate VPs with nonsentential utterances. As we will see, HPSG analyses of these constructions make use of all three options, including the option expressed in (b), that coordinated structures may contain unpronounced material.

\section{Evidence for and against invisible material at the ellipsis site}
\label{sec-evidence-for-invisible-material}

This section is concerned with NSUs and PAE since this is where the contentious issues arise of where ellipsis is licensed (Sections~\ref{sec-structural-mismatches} and~\ref{sec-nonlinguistic-antecedents}) and whether there is invisible syntactic material in an ellipsis site (Sections~\ref{sec-connectivity-effects} and~\ref{sec-island-effects}). Below we consider evidence for and against invisible structure found in the ellipsis literature. As we will see, the evidence is based not only on intuitive judgments, but also on experimental and corpus data, the latter being more typical of the HPSG tradition.


\subsection{Connectivity effects}
\label{sec-connectivity-effects}

Connectivity effects refer to parallels between NSUs and their counterparts in sentential structures, thus speaking in favor of the existence of silent sentential structure. We focus on two kinds here: case-matching effects and preposition-stranding effects (for other examples of connectivity effects, see \citealt{Ginzburg2018}). It's been known since \citet{Ross1967} that NSUs exhibit case-matching effects, that is, they are typically marked for the same case that is marked on their counterparts in sentential structures. (\ref{10}) illustrates this for German, where case matching is seen between a \emph{wh}-phrase functioning as an NSU and its counterpart in the antecedent (\citealt[663]{Merchant2005-proc}):

\ea
\gll Er will jemandem schmeicheln, aber sie wissen nicht wem~/ *~wen.\\
     he will someone.\textsc{dat} flatter, but they know not who.\textsc{dat}  \hspaceThis{*~}who.\textsc{acc}\\
\glt `He wants to flatter someone, but they don't know whom.'\label{10}\z


Case-matching effects are crosslinguistically robust in that they are found in the vast majority of languages with overt case marking systems, and therefore, they have been taken as strong evidence for the reality of silent structure. The argument is that the pattern of case matching follows straightforwardly if an NSU is embedded in silent syntactic material whose content includes the same lexical head that assigns case to the NSU's counterpart in the antecedent clause to assign case to the NSU (\citealt{Merchant2001, Merchant2005a}). However, a language like Hungarian poses a problem for this reasoning \citep{Jacobson2016}. While Hungarian has verbs that assign one of two cases to their object NPs in overt clauses with no meaning difference, case matching is still required between an NSU and its counterpart, whichever case is marked on the counterpart. To see this, consider (\ref{11}) from \citet[356]{Jacobson2016}. The verb \emph{hasonlit} assigns either sublative (SUBL) or allative (ALL) case to its object, but if SUBL is selected for an NU's counterpart, the NUS must match this case.

\ea
A: \gll Ki-re hasonlit P\'{e}ter?\\
        who.\textsc{subl} resembles Peter\\
   \glt  `Who does Peter resemble?'\\

B: \gll J\'{a}nos-ra / * J\'{a}nos-hoz.\\
        J\'{a}nos.\textsc{subl} {} {} J\'{a}nos.\textsc{all}\\
\glt  `J\'{a}nos.'\label{11}
\z
%
\citet{Jacobson2016} notes that there is some speaker variation regarding the (un)ac\-cepta\-bi\-li\-ty of case mismatch here at the same time that all speakers agree that either case is fine in a corresponding nonelliptical response to (\ref{11}A). This last point is important, because it shows that the requirement of---or at least a preference for---matching case features applies to NSUs to a greater extent than it does to their nonelliptical equivalents, challenging connectivity effects.

Similarly problematic for case-based parallels between NSUs and their sentential counterparts are some Korean data. Korean NSUs can drop case markers more freely than their counterparts in nonelliptical clauses can, a point made in \citet{Morgan1989} and \citet{Kim2015}. Observe the example in (\ref{12}) from \citet[237]{Morgan1989}.

  \ea
A: \gll Nwukwu-ka        ku  chaek-ul          sa-ass-ni?\\
        who-\textsc{nom} the book-\textsc{acc} buy-\textsc{pst}-\textsc{que}\\
\glt  `Who bought the book?'\\

B: \gll Yongsu-ka / Yongsu / * Yongsu-lul.\\
        Yongsu-\textsc{nom} {} Yongsu {} {} Yongsu-\textsc{acc}\\
\glt  `Yongsu.'

B$'$: \gll Yongsu-ka            /  *  Yongsu ku  chaek-ul          sa-ass-e\\
           Yongsu-\textsc{nom}  {} {} Yongsu the book-\textsc{acc} buy-\textsc{pst}-\textsc{decl}\\
\glt  `Yongsu bought the book'\\
\label{12}
\z
%
When an NUS corresponds to a nominative subject in the antecedent (as in \ref{12}B), it can be either marked for nominative or caseless.
However, replacing the same NUS with a full sentential answer, as in (\ref{12}B$'$), rules out case drop from the subject. This strongly suggests that the case-marked and caseless NSUs couldn't have identical source sentences if they were to derive via PF-deletion.\footnote{Nominative differs in this respect from three other structural cases, dative, accusative and genitive, in that the latter may also be dropped from nonelliptical clauses \citep[see][]{Morgan1989, Lee2016, Kim2016}.}  Data like these led \citet{Morgan1989} to propose that not all NSUs have a sentential derivation, an idea later picked up in \citet{Barton1998}.

%\iffalse{
The same pattern is associated with semantic case. That is, in (\ref{13}), an NUS need
to be marked for comitative like its counterpart in the A-sentence, but being caseless is not an option for the NU's counterpart \citep{Kim2015}.
%\todostefan{glossing did not match for
%  \emph{ha-ess-e}. Please check}

\ea
A:
\gll Nwukwu-wa          hapsek-ul                     ha-yess-e?\\
     who-\textsc{com}   sitting.together-\textsc{acc} do-\textsc{pst}-\textsc{que}\\
\glt  `With whom did you sit together?'\\

B:
\gll Mimi-wa / * Mimi.\\
     Mimi-\textsc{src} {} {} Mimi\\
\glt `With Mimi.' / * `Mimi.' \label{13}\z
%
The generalization for Korean is then that NSUs may be optionally realized as caseless but may never be marked for a different case than is marked on their counterparts.
%}\fi 

Overall, case-marking facts show that there is some morphosyntactic identity between NSUs and their antecedents, though not to the extent that NSUs have exactly the features that they would have if they were constituents embedded in sentential structures. The Hungarian facts also suggest that those aspects of the argument structure of the appropriate lexical heads present in the antecedent that relate to case licensing are relevant for an analysis of NSUs.\footnote{Hungarian and Korean are in fact not the only problematic languages; for a list, see \citet{Vicente2015}.}

The second kind of connectivity effects goes back to \citet{Merchant2001, Merchant2005a} and highlights apparent links between the features of NSUs and \emph{wh}- and focus movement (movement of a focus-bearing expression). The idea is that prepositions behave the same under \emph{wh}- and focus movement as they do under clausal ellipsis, that is, they pied-pipe or strand in the same environments. If a language (e.g., English) permits preposition stranding under \emph{wh}- and focus movement (\emph{What did Harvey paint the wall with?} vs \emph{With what did Harvey paint the wall?}), then NSUs may surface with or without prepositions, as illustrated in (\ref{14}) for sluicing and BAE.
%
\ea A: I know what Harvey painted the wall with.\\B: (With) what?/(With) primer.\label{14}\z
%
If there indeed was a link between between preposition stranding and NSUs, then we would not expect prepositionless NSUs in languages without preposition stranding. This expectation is disconfirmed by an ever-growing list of non-preposition stranding languages that do feature prepositionless NSUs: Brazilian Portuguese (\citealt{AlmeidaYoshida2007}), Spanish and French (\citealt{Rodrigues2006}), Greek (\citealt{Molimpakis2018}), Bahasa Indonesia (\citealt{Fortin2007}), %Emirati Arabic \citep{Leung2014},
 Russian \citep{Philippova2014}, Polish \citep{Szczegielniak2008, Sag2011, Nykiel2013}, %Czech \citep{Caha2011},
Bulgarian \citep{Abels2017}, Serbo-Croatian \citep{Stjepanovic2008, Stjepanovic2012}, and %Saudi 
Arabic \citep{Leung 2014, Alshaalan2020}. A few of these studies have presented experimental evidence that prepositionless NSUs are acceptable, though --- for reasons still poorly understood --- they typically do not reach the same level of acceptability as their variants with prepositions do (see \citealt{Nykiel2013} for Polish, \citealt{Molimpakis2018} for Greek, and \citealt{Alshaalan2020} for Saudi Arabic). It is worth noting in this regard that the work following the HPSG tradition is based on a solid foundation of experimental evidence to a larger extent than work grounded in the Minimalist tradition (see \citealt{Sag2011, Miller2014}).

It is evident from this research that there is no grammatical constraint on NSUs that keeps track of what preposition-stranding possibilities exist in any given language. On the other hand, it doesn't seem sufficient to assume that NSUs can freely drop prepositions, given examples of sprouting like (\ref{15}), in which prepositions are not omissible (see \citealt{Chung1995} on the non-omissibility of prepositions under sprouting). The difference between (\ref{14}) and (\ref{15}) is that there is an explicit phrase the NUS corresponds to (in the HPSG literature this phrase is termed a Salient Utterance \citep[313]{Ginzburg:Sag:2000} or a Focus-Establishing Constituent \citealt{Ginzburg2012}) in the former but not in the latter.

\ea A: I know Harvey painted the wall.\\B: *(With) what?/Yeah, *(with) primer.\label{15}\z
%
The issue in such examples is then how to ensure
the NSU to have the proper PP that matches with the implicit PP argument in 
the antecedent clause A (see the discussion
around (\ref{spr})). This issue has not received much attention in the HPSG literature, though see \citet{Kim2015}.



\subsection{Island effects}
\label{sec-island-effects}

One of the predictions of the view that NSUs are underlyingly sentential is that they should respect island constraints on long-distance movement. But as illustrated below, NSUs (both sluicing and BAE) exhibit island-violating behavior. The NUS in (\ref{16}) would be illicitly extracted out of an adjunct (\textit{*Where does Harriet drink scotch that comes from?}) and the NUS in (\ref{17}) would be extracted out of a complex NP (\textit{*The Gay Rifle Club, the administration has issued a statement that it is willing to meet with}).\footnote{\citet{Merchant2005a} argued that BAE, unlike sluicing, does respect island constraints, an argument that was later challenged \citep[see e.g,][]{CJ2005a, Griffiths2014}. However, \citet{Merchant2005-proc} focused specifically on pairs of \emph{wh}-interrogatives and answers to them, running into the difficulty of testing for island-violating behavior, since a well-formed \emph{wh}-interrogative antecedent couldn't be constructed.}

\ea A: Harriet drinks scotch that comes from a very special part of Scotland.\\B: Where? \citep[245]{CJ2005a} \label{16}\z

\ea A: The administration has issued a statement that it is willing to meet with one of the student groups.\\B: Yeah, right---the Gay Rifle Club. \citep[245]{CJ2005a} \label{17}\z

Among \citeauthor{CJ2005a}'s (\citeyear[245]{CJ2005a}) examples of well-formed island-violating NSUs are also sprouted NSUs (those that correspond to implicit phrases in the antecedent) like (\ref{18})--(\ref{19}).

\ea A: John met a woman who speaks French.\\B: With an English accent?\label{18}\z

\ea A: For John to flirt with at the party would be scandalous. \\B: Even with his wife?\label{19}\z
Other scholars assume that sprouted NSUs are one of the two kinds of NSUs that respect island constraints, the other kind being contrastive NSUs, illustrated in (\ref{20}) \citep{Chung1995, Merchant2001, Griffiths2014}.

\ea A: Does Abby speak the same Balkan language that Ben speaks?\\
B: *No, Charlie. \citep{Merchant2001}  \label{20}\z
%
\citet{Schmeh2015} further explore the acceptability of NSUs preceded by the response particle \textit{no} like those in (\ref{20}) compared to NSUs introduced by the response particle \textit{yes} depicted in (\ref{21}). (\ref{20}) and (\ref{21}) differ in terms of discourse function in that the latter supplements the antecedent rather than correcting it, a discourse function signaled by the response particle \textit{Yes}.

\ea A: John met a guy who speaks a very unusual language. \\B: Yes, Albanian. \citep[245]{CJ2005a} \label{21}\z
%
\citet{Schmeh2015} find that corrections cause to
lower acceptability ratings compared to supplementations and propose that this follows from the fact that corrections induce greater processing difficulty than supplementations do, and hence the acceptability difference between (\ref{20}) and (\ref{21}). This finding makes it plausible that the perceived degradation of island-violating NSUs could ultimately be attributed to nonsyntactic factors, e.g., the difficulty of successfully computing a meaning for them.

In contrast to NSUs, many instances of PAE appear to respect island constraints, as would be expected if there was unpronounced structure from which material was extracted. An example of a relative clause island is depicted in (\ref{22}) (note that the corresponding sluicing NUS is fine).


\ea[*]{
They want to hire someone who speaks a Balkan language, but I don't remember which they do
[\sout{want to hire someone who speaks t}]. \citep[6]{Merchant2001}\label{22}
}
\z
(\ref{22}) contrasts with well-formed island-violating examples like (\ref{23}) and (\ref{24}), as observed by \citet{Miller2014, Ginzburg2018}.
%
%
\eal
\ex{He managed to find someone who speaks a Romance language, but a Germanic language, he didn't [\sout{manage to find someone who speaks t}].\label{23}}

\ex{He was able to find a bakery where they make good baguette, but croissants, he couldn't [\sout{find a bakery where they make good t}].\label{24}}
\zl
%
As \citet{Ginzburg2018} rightly point out, we do not yet have a complete understanding of when or why island effects show up in PAE. Its behavior is at best inconsistent, failing to provide convincing evidence for silent structure.


\subsection{Structural mismatches}
\label{sec-structural-mismatches}

Because structural mismatches are rare or absent from NSUs \citep[see][]{Merchant2005a, Merchant2013},\footnote{\citet{Ginzburg2018} cite examples---originally from \citet{Beecher2008}---of sprouting NSUs with nominal, hence mismatched, antecedents, e.g., (i).
	\ea We're on to the semi-finals, though I don't know who against.\z
%
	Somewhat similar examples, where NSUs need to refer to an AP antecedent, appear in COCA:
%
	\ea  A: Well, it's a defense mechanism. B: Defense against what?\z
	\ea Our Book of Mormon talks about the day of the Lamanite, when the church would make a special effort to build and reclaim a fallen people. And some people will say, Well, fallen from what? \z
%
	The NSUs in (ii)--(iii) repeat the lexical heads whose complements are being sprouted (\textit{defense} and \textit{fallen}), that is, they contain more material than is usual for NSUs (cf. (i)). It seems that without this additional material it would be difficult to integrate the NSUs into the propositions provided by the antecedents and hence to arrive at the intended interpretations.} this section focuses on PAE and developments surrounding the question of which contexts license it. In a seminal study of anaphora, \citet{Hankamer1976} classified PAE as a surface anaphor with syntactic features closely matching those of an antecedent present in the linguistic context. They argued in particular that PAE is not licensed if it mismatches its antecedent in voice. Compare the following two examples from \citet[327]{Hankamer1976}.
%\todosatz{no reference given}.

\eal
\ex[*]{
	The children asked to be squirted with the hose, so we did.  \label{25}
}
\ex[]{
	The children asked to be squirted with the hose, so they were. \label{26}
}
\zl
This proposal places tighter structural constraints on PAE than on other verbal anaphors (e.g., \textit{do it/that}) in terms of identity between an ellipsis site and its antecedent and has prompted extensive evaluation in a number of corpus and experimental studies in the decades following \citet{Hankamer1976}. Below are examples of acceptable structural mismatches reported in the literature, ranging from voice mismatch (\ref{27}) to nominal antecedents (\ref{28}) to split antecedents (\ref{29}).

\eal

\ex This information could have been released by Gorbachev, but he chose not to [release it]. \citep[37]{Hardt1993} \label{27}

\ex Mubarak's survival is impossible to predict and, even if he does [survive], his plan to make his son his heir apparent is now in serious jeopardy. \citep{Miller2014a} \label{28}

\ex Wendy is eager to sail around the world and Bruce is eager to climb Mt. Kilimanjaro, but neither of them can [do the things they want], because money is too tight. \citep{Webber79a} \label{29}
\zl

There are two opposing views that have emerged from the empirical work regarding the acceptability and grammaticality of structural mismatches under PAE. The first view takes mismatches to be grammatical and connects degradation in acceptability to violation of certain independent discourse \citep{Kehler2002, Miller2011, %Kertz2013,
Miller2014, Miller2014a, Miller2014b} or processing constraints \citep{Kim2011}. Two types of PAE have been identified on this view through extensive corpus work (a characteristic of the HPSG research style)---auxiliary choice PAE and subject choice PAE---each with different discourse requirements with respect to the antecedent \citep{Miller2011, Miller2014a, Miller2014b}. The second view assumes that there is a grammatical ban on structural mismatch but violations thereof may be repaired under certain conditions; repairs are associated with differential processing costs compared to matching ellipses and antecedents \citep{Arregui2006, Grant2012}. If we follow the first view, it is perhaps unexpected that voice mismatch should consistently incur a greater acceptability penalty under PAE than when no ellipsis is involved, as recently reported in \citet{Kim2011}. \citet{Kim2011} stops short of drawing firm conclusions regarding the grammaticality of structural mismatches, but one possibility is that the observed mismatch effects reflect a construction-specific constraint on PAE. HPSG analyses take structurally mismatched instances of PAE to be unproblematic and fully grammatical, while also recognizing construction-specific constraints: discourse or processing constraints formulated for PAE may or may not extend to other elliptical constructions, such as NSUs (see \citealt{Abeille2016,Ginzburg2018} for this point).


\subsection{Nonlinguistic antecedents}
\label{sec-nonlinguistic-antecedents}

Like structural mismatches, the availability of nonlinguistic antecedents for an ellipsis points to the fact that it needn't be interpreted by reference to and licensed by a structurally identical antecedent. Although this option is somewhat limited, PAE does tolerate nonlinguistic antecedents, as shown in (\ref{30})--(\ref{31}) \citep[see also][]{Hankamer1976, Schachter1977}.
\ea Mabel shoved a plate into Tate's hands before heading for the sisters' favorite table in the shop. ``You shouldn't have.'' She meant it. The sisters had to pool their limited resources
just to get by. \citep[ex. 23][]{Miller2014b}\label{30}\z

\ea Once in my room, I took the pills out. ``Should I?'' I asked myself. \citep[ex. 22a][]{Miller2014b}\label{31}\z
\citet{Miller2014b} provide an extensive critique of the earlier work on the ability of PAE to take nonlinguistic antecedents, arguing for a streamlined discourse"=based explanation that neatly captures the attested examples as well as examples of structural mismatch like those discussed in Section~\ref{sec-structural-mismatches}. The important point here is again that PAE is subject to construction-specific constraints which limit its use with nonlinguistic antecedents.

NSUs appear in various nonlinguistic contexts as well. \citet{Ginzburg2018} distinguish three classes of such NSUs: sluices (\ref{32}), exclamative sluices (\ref{33}), and declarative fragments (\ref{34}).

\ea (In an elevator) What floor? \citep[298]{Ginzburg:Sag:2000}\label{32}\z

\ea It makes people ``easy to control and easy to handle,'' he said, ``but, God forbid, at what a cost!''
%(Ginzburg \& Miller To appear, ex. 34a)\todosatz{no reference given}
\label{33}\z

\ea BOBADILLA turns, gestures to one of the other men, who comes forward and gives him a roll of parchment, bearing the royal seal. ``My letters of appointment.'' (COCA)\label{34}\z
In addition to being problematic from the licensing point of view, NSUs like these have been put forward as evidence against the idea that they are underlyingly sentential, because it is unclear what the structure that underlies them would be \citep[see][]{Ginzburg:Sag:2000, CJ2005a, Stainton2006}.\footnote{This is not to say that a sentential analysis of fragments without linguistic antecedents hasn't been attempted. For details of a proposal involving a `limited ellipsis' strategy, see \citet{Merchant2005a} and \citet{Merchant2010}.}
%\todosatz{Merchant 2010: no reference given}


\section{Analyses of NSUs}
\label{sec-analyses-of-NSUs}

It is worth noting at the outset that the analyses of NSUs within the framework of HPSG are based on an elaborate theory of dialog \citep{Ginzburg1994, Ginzburg2004, Ginzburg2014a, Larsson2002, Purver2006, Fernandez2006, Fernandez2002, Fernandez2007, Ginzburg2010, Ginzburg2014b, Ginzburg2012, Ginzburg2013, Kim2019} and on a wider range of data than is common practice in the ellipsis literature. Existing analyses of NSUs go back to \citet{Ginzburg:Sag:2000}, who recognize declarative fragments (\ref{34a}) and two kinds of sluicing NSUs, direct sluices (\ref{35}) and reprise sluices (\ref{36}) (the relevant fragments are bolded). The difference between direct and reprise sluices lies in the fact that the latter are requests for clarification of any part of the antecedent. For instance, in (\ref{36}) the referent of \textit{that} is unclear to the interlocutor.

\ea ``I was wrong.'' Her brown eyes twinkled. ``Wrong about what?'' ``\textbf{That night}.'' (COCA) \label{34a}\z

\ea ``You're waiting,'' she said softly. ``\textbf{For what?}'' (COCA) \label{35} \z

\ea ``Can we please not say a lot about that?'' ``\textbf{About what?}'' (COCA) \label{36} \z


The different types of fragments are derived from the \citet[333]{Ginzburg:Sag:2000} hierarchy of clausal types depicted in Figure~\ref{fig-cltypes}:


\begin{figure}
\centering
\begin{forest}
type hierarchy, instances
[phrase
  [clausality,partition
    [clause
      [core-cl
        [inter-cl
          [is-int-cl
            [dir-is-int-cl,tier=bottom % the types dir-is-int-cl, slu-int-cl and decl-frag-cl
                                       % will be drawn on the same line.
              [\textsc{Who?}\\\textsc{Jo}?\\\textbf{Jo?}
]]]]
        [decl-cl
          [slu-int-cl, % slu-int-cl is drawn as child of decl-cl but we do not want to have
                               % an edge to decl-cl
           tier=bottom,
           edge to=!r1111, % see explanation below for node pathes
           edge to=!r2111, % see explanation below
           no edge         % this has to be said after the other edges are drawn, if it is said
                           % before, no edge will be drawn ...
           [\textbf{Who?}\\\textbf{who}
]]]]]]
  [headedness,partition
    [hd-ph
      [hd-only-ph
        [hd-frag-ph
          [decl-frag-cl,tier=bottom,
                        edge to=!r1112 % draw an edge to the node which is the first child of
                                % the root node (!r1 = clausality) 's first child (clause) 's first
                                % child (core-cl) 's second child (2 = decl-cl)
           [\textbf{Bo}]]]]]]]
\end{forest}
\caption{Clausal hierarchy for fragments (\citealt[333]{Ginzburg:Sag:2000})}\label{fig-cltypes}
\end{figure}
%
%
%
%
%
 NSUs like declarative fragments (decl-frag-cl) are associated with type hd-frag-ph (headed-fragment phrase) and decl-cl (declarative clause), while direct sluices (slu-int-cl) and reprise sluices (dir-is-int-cl) are associated with type hd-frag-ph and inter-cl (interrogative clause). The type slu-int-cl is permitted to appear in independent and embedded clauses, hence it is underspecified for the head feature IC (independent clause). This specification contrasts with that of declarative fragments and reprise sluices, with both specified as [IC +].
  %(see \citealt[333]{Ginzburg:Sag:2000} for the overall organization of the clausal types %including fragment types).
%
%GShierarchy here \label{cltypes}
%
%
%
\citet[304]{Ginzburg:Sag:2000} make use of the constraint shown in (\ref{hf-cx}) (we have added information about the \textsc{max-qud} to generate NSUs.
%
%

\ea
\label{hf-cx}
Head-Fragment Construction:\\
\avmtmp{
[cat & s\\ %\[HEAD v\]\\
             %CONT &\[NUCL \@1\]\\
 ctxt & [max-qud & !$\lambda\{\pi^{i}\}$!\\
         sal-utt & \{  [cat & \2\\
                        cont|ind & \type{i} ] \} ] ]}
$\rightarrow$
\avmtmp{
[ cat      \2\\
  cont|ind \type{i}
]
}
%% \[CAT &S\\ %\[HEAD v\]\\
%%   %CONT &\[NUCL \@1\]\\
%%   CTXT & \[MAX-QUD $\lambda$\{$\pi$$^{i}$\}\\
%%   SAL-UTT \{  \[CAT \@2\\
%%                          CONT\ \[IND & {\it i}\\
%%                                   \]\]\}\]\]

%%                     \ \ $\rightarrow$\ \
%% \[  CAT &  \@2\\
%%    CONT  &\[IND & {\it i}\]
%% \]
%% \end{avm}
\z
Let us see how this constructional constraint allows us to
license NSUs and capture their properties, including the connectivity effects we discussed in Section~\ref{sec-connectivity-effects}.

Note first that any phrasal category
can function as an NU, that is, can be mapped onto a sentential utterance as long as it corresponds to a Salient Utterance (SAL-UTT). This means that
the head daughter's syntactic category must match that of a SAL-UTT, which is an attribute supplied by the surrounding context as a (sub)utterance of another contextual attribute---the Maximal Question under Discussion (MAX-QUD). The two contextual attributes SAL-UTT and MAX-QUD are introduced specifically for the purpose of analyzing NSUs. The context gets updated with every new question-under-discussion, and MAX-QUD represents the most recent question-under-discussion, while SAL-UTT is the (sub)utterance with the widest scope within MAX-QUD. To put it informally, SAL-UTT represents a (sub)utterance of a MAX-QUD that has not been resolved yet. Its feature CAT supplies information relevant for establishing morphosyntactic identity with an NU, that is, syntactic category and case information, and (\ref{hf-cx}) requires that an NUS match this information. Because the permissible categories of SAL-UTT are nominal, SAL-UTTs can surface either as NPs or PPs, and so can NSUs. This gives us a way of capturing the problems that \citet{Merchant2001, Merchant2005a} faces with respect to misalignments between preposition stranding under \emph{wh}- and focus movement and the realization of NSUs as NPs or PPs, as discussed in Section~\ref{sec-connectivity-effects}. Meanwhile, MAX-QUD provides the propositional semantics for an NUS and is, typically, a unary question. The content of MAX-QUD can be supplied by linguistic or nonlinguistic context. In the prototypical case, MAX-QUD arises from the most recent \emph{wh}-question uttered in a given context (\ref{37}), but can also arise (via accommodation) from other forms found in the context, such as constituents bearing focal accent ({\it MIKE} in (\ref{38})) and constituents in need of clarification (\ref{39}), or from a nonlinguistic context (\ref{40}).\footnote{\citet{Ginzburg2012} uses the notion of the Dialog Game Board (DGB) to keep track of all information relating to the common ground between interlocutors. The DGB is also the locus of contextual updates arising from each new question-under-discussion that is introduced.}
\ea
A: What did Barry break? \\
B: The mike.\label{37}
\z

\ea
A: Barry broke the MIKE. \\
B: Yes, the only one we had.\label{38}
\z

\ea
A: Barry broke the mike. \\
B: Who?\label{39}
\z

\ea
(Cab driver to passenger on the way to airport)
A: Which airline?\label{40}
\z

The existing analyses of NSUs \citep{Ginzburg2012, Sag2011, Kim2015, Abeille2014, Abeille2019, Kim2019} are based on \citeauthor{Ginzburg:Sag:2000}'s (\citeyear{Ginzburg:Sag:2000}) constraint. Below in Figure~\ref{fig-the-mike} and Figure~\ref{fig-slu}, we illustrate how it is applied to the declarative fragment in (\ref{37}) and the reprise sluice in (\ref{39}). The analyses in Figure~\ref{fig-the-mike} and Figure~\ref{fig-slu} differ in the value of the feature CONT (Content), the former being a proposition and the latter a question.

\begin{figure}
{\centering
\begin{forest}
sm edges without translation
[S\\
\avmtmp{
[cat & [ head & v]\\
  %CONT & \[NUCL \@1\]\\
 ctxt & [max-qud & !$\lambda\{\pi^{i}\}[break(b,i)]$! \\
         sal-utt & \{ [cat  \2\\
                       cont|ind \type{i} ] \} ] ]}
[NP\\
 \avmtmp{
  [cat  & \2\\
   cont & [ ind \type{i} ] ]%\\
   %\PARAMS & \{$\pi$$^{i}$\}
  }
 [The mike]]]
\end{forest}
}
\caption{Structure of the declarative fragment clause}\label{fig-the-mike}
\end{figure}

\begin{figure}
{\centering
\begin{forest}
sm edges without translation
[S\\
\avmtmp{
[cat & [ head & v]\\
  %CONT & \[NUCL \@1\]\\
 ctxt & [max-qud & !$\lambda\{\pi^{i}\}[break(i,m)]$! \\
         sal-utt & \{ [cat  \2\\
                       cont|ind \type{i} ] \} ] ]}
[NP\\
 \avmtmp{
  [cat  & \2\\
   cont & [ ind \type{i} ] ]%\\
   %\PARAMS & \{$\pi$$^{i}$\}
  }
 [Who]]]
\end{forest}
}
\caption{Structure of the sluiced interrogative clause}\label{fig-slu}
\end{figure}


This construction-based analysis, in which dialog updating plays
a key role in the licensing of NSUs, can also offer a simple account of sprouting examples like (\ref{35}), repeated here for convenience as (\ref{spr}).

\ea ``You're waiting,'' she said softly. ``\textbf{For what?}'' (COCA) \label{spr} \z
%
\citet[331]{Ginzburg:Sag:2000} suggest the following way of analyzing such sprouted NSUs. The implied PP \textit{for someone} functioning as SAL-UTT here would appear as a noncanonical synsem on the ARG-ST list of the verb \textit{wait}, but not on the COMPS list, and thereby be able to provide appropriate morphosyntactic identity information. The lexical entry for \textit{wait} would look like the one given in (\ref{wait}):


%\inlinetodostefan{Use feature geometry of \citet{Sag97a}}
\ea
\label{wait}
Lexical item for \textit{wait}:\\
\avmtmp{
[ phon   & \phonliste{ wait }\\
  arg-st & < NP$_i$, PP$_x$ >\\
  cat  &[ subj  & < NP$_i$  >\\
                 comps & <  > ]\\
  cont &     ! $wait(i, x) $ ! ]
}
 %% \begin{avm}
 %% \[FORM \q<{\it wait}\q>\\
 %%   ARG-ST \<NP\jbsub{{\it i}}, PP\jbsub{{\it x}}\>\\
 %%   SYN\[SUBJ \<NP[{\it overt}]\>\\
 %%        COMPS \<PP[{\it ini}]\>\]\\
 %%   SEM {\it wait}({\it i, x})\]
 %%   \end{avm}
\z
%
The lexical information specifies that the second argument of \textit{wait} can be an unrealized PP while the first argument needs to be an overt NP. Now consider the dialog in (\ref{spr}). Uttering
the sentence \textit{You're waiting.} would then update the DGB with a SAL-UTT represented by the unrealized PP, as in (\ref{updateddgb}).
%
\ea
\label{updateddgb}
\avmtmp{
[%{\it dgb}\\
 ctxt|sat-utt \{[cat &  \upshape PP [%\type*{ini}
                          pform & for\\
                          ind   & x ]\\
              cont & !$\textit{wait.for}(i, x)$ ! ]\}]
}
%% \begin{avm}
%% \[%{\it dgb}\\
%%  DGB \[SAT-UTT \[SYN  PP\[{\it ini}\\
%%                           PFORM {\it for}\\
%%                         \IND\ {\it  x}\]\\
%%                  SEM {\it wait.for}({\it i,x})\]\]\]
%% \end{avm}
\z
%
The NUS \textit{For what?}, matching this SAL-UTT,
projects a well-formed NUS in accordance with the Head-Fragment Construction. An alternative to this approach is \citet{Kim2015}, who takes the unrealized oblique argument of the verb \textit{wait} as an instance of indefinite null instantiation (INI), following \citep[see][]{Ruppenhofer2014}, with the result that this PP appears on the COMPS list with the annotation \textit{ini}.

The advantages of the nonsentential analyses sketched here follow from their ability to capture limited morphosyntactic parallelism between NSUs and SAL-UTT without having to account for why NSUs behave differently from constituents of sentential structures. The island-violating behavior of NSUs is unsurprising on this analysis, as are attested cases of structural mismatch and situationally controlled NSUs.\footnote{The rarity of NSUs with nonlinguistic antecedents can be understood as a function of how easily a situational context can give rise to a MAX-QUD and thus license ellipsis (see \citealt{Miller2014b} for this point with regard to PAE).} However, some loose ends still remain. (\ref{hf-cx}) currently has no means of capturing certain connectivity effects: it can't rule preposition drop out under sprouting and it incorrectly rules out case mismatch in languages like Hungarian for speakers that do accept it (see discussion around example (\ref{11})).\footnote{See, however, \citet{Kim2015} for a proposal that introduces  a case hierarchy specific to Korean to explain case mismatch and introducing an additional constraint to block preposition drop under sprouting.}
%
%NOTE from review: Please explain how Kim 2015 handles pseudosluices (with a copula) in wh in %situ languages such as Korean (and mandarin), not that these languages
%allow direct sluices



\section{Analyses of predicate/argument ellipsis}
\label{sec-analyses-of-pred-ellipsis}
The first issue in the analysis of PAE is the status of the elided expression. It is assumed to be a \textit{pro} element due to its pronominal properties \citep[see][]{Lobeck1995, Lopez2000, Kim2006, Aelbrecht2015, Ginzburg2018}. For instance, PAE applies only to phrasal categories ((\ref{42})--(\ref{43})),
can cross utterance boundaries (\ref{44}), can override island constraints ((\ref{45})--(\ref{46})), and is subject to the Backwards Anaphora Constraint ((\ref{47})--(\ref{48})).

\ea[*]{
Mary will meet Bill at Stanford because she didn't  \jbtr John.\label{42}
}
\z
\ea[]{
Mary will meet Bill at Stanford because she didn't \jbtr at Harvard.\label{43}
}
\z
\ea[]{
A: Tom won't leave Seoul soon.\\
B: I don't think Mary will \jbtr either.\label{44}
}
\z
\ea[]{
John didn't hit a home run, but I know a woman who did. (CNPC)\label{45}
}
\z
\ea[]{
That Betsy won the batting crown is not surprising, but that Peter didn't know she did \jbtr is
indeed surprising. (SSC)\label{46}
}
\z
\ea[*]{
Sue didn't [e] but John ate meat.\label{47}
}
\z
\ea[]{
Because Sue didn't [e], John ate meat.\label{48}
}
\z

%Argument ellipsis we find in languages like Polish and Korean can also be taken to be ellipsis of a pronominal %expression, as in (\ref{49}).
%
%\ea
%\gll Mimi-ka {\it pro} po-ass-ta.\\
%Mimi.NOM  pro see.PST.DECL\\
%\glt `Mimi saw (him)'\label{49}\z


One way to account for PAE closely tracks analyses of {\it pro}-drop phenomena. We do not need to posit a phonologically empty pronoun if a level of argument structure is
available where we can encode the required pronominal properties\citep[see][]{Bresnan1982a, Ginzberg2000,Kim2006, Ginzburg2018}. In the framework of HPSG, we represent this possibility as the Argument Realization Constraint (\ref{50}), permitting mismatch between argument structure and syntactic valence features:\footnote{Expressions have two subtypes: overt and covert ones, the latter of which has two subtypes, \textit{pro} and \textit{gap}. See \citet{Sag2012a} for details.}

%\inlinetodostefan{AVM: Space after $\ominus$}
\ea
\label{50}
Argument Realization Constraint (ARC):\\
\type{v-wd} \impl
\avmtmp{
[cat [ subj  & \1\\
           comps & \2 \ $\ominus$ \ !$list(pro)$!]\\
 arg-st \1 \+ \2 ]
}
%% \begin{avm}
%% \emph{v-word} \;   $\Rightarrow$ \;
%% \[SYN\|VAL \[SUBJ & \@A\\
%%                  COMPS & \@B $\ominus$ list(\emph{pro})\]\\
%%   ARG-ST \@A $\oplus$ \@B\]
%%   \end{avm}
\z
The Argument Realization Constraint tells us that a \textit{pro} element
in the argument structure need not be realized in the syntax.
 For
example, as represented in (\ref{51}), the auxiliary
verb \textit{can} in examples like \textit{John can't dance, but Sandy can.}
has a \textit{pro} VP as its second argument, that is, this VP is not instantiated as the syntactic
complement of the verb (\citealt{Kim2006}): %\index{transitive}

%\inlinetodostefan{AVM: space before VP}
\ea
\label{51}
Lexical entry for \textit{can}:\\
\avmtmp{
[\type*{v-wd}
 phon & \phonliste{ can }\\
 cat & [ head|vform & \type{fin}\\
        subj  & < \1 >\\
               comps &  < >  ]\\
 arg-st & < \1 NP, !VP[\type{pro}]! > ]
}
%% \begin{avm}
%% \[{\it v-word}\\
%%  FORM \q<can\q>\\
%%  SYN\[HEAD \|VFORM \ \ {\it fin}\\
%%       VAL \[SUBJ & \q<\@1\q>\\
%%            COMPS & \q<  \; \q>\]\]\\
%% ARG-ST  \q<\@1NP, VP[{\it pro}]\q>\]
%%  \end{avm}
\z
%
%
Given this, English PAE can be analyzed as a language-particular VP \textit{pro} drop phenomenon, trigged
by a constraint like (\ref{52}).

%\inlinetodostefan{
%AVM: type name too long and second column not used, brackets around \type{pro} should be %normal text
%brackets}


\ea\label{52}
Aux-Ellipsis Construction:\\
\avmtmp{
[\type*{aux-v-lxm}
   %\SYN\|\HEAD\|\AUX\ $+$\\
 arg-st < \1 XP, \2 YP > & ] } $\mapsto$
\avmtmp{
[\type*{aux-pae-wd}
 arg-st < \1 XP, \2 YP[\type{pro}] > & ]
}
%% \begin{forest}
%%  [{\begin{avm}
%%  \[{\it aux-v-lxm}\\
%%    %\SYN\|\HEAD\|\AUX\ $+$\\
%%    ARG-ST \q<\@1XP, \@2YP\q>\]
%%  \  \ $\mapsto$ \ \
%%  \[{\it aux-ellipsis-wd}\\
%%    ARG-ST \q<\@1XP, \@2YP[{\it pro}]\q>\]
%%  \end{avm}}]
%% \end{forest}
\z
What this tells us is that an auxiliary verb selecting two arguments
can be projected onto an elided auxiliary verb whose second argument
is realized as a small \textit{pro}. This argument is not mapped
onto any grammatical function on the COMPS list. The output auxiliary
in (\ref{51}) will then project a structure like the one
in Figure~\ref{fig-53}.
%
\begin{figure}
\begin{forest}
[S
  [\ibox{1} NP
      [Sandy]]
  [VP\\
   \avmtmp{
     [%{\it head-only-cxt} \& {\it ellip-cxt}\\
      head \2\\
      subj < \1 > ]}
    [V\\
     \avmtmp{
      [ head \2 [aux $+$ ]\\
        subj < \1 >\\
        comps < >\\
        arg-st < \1 NP, VP[\type{pro}] > ]}
      [can]]]]
\end{forest}
%% \begin{forest}
%% [S
%%   [\begin{avm}\@1NP\end{avm}
%%       [Sandy]]
%%   [\begin{avm} \avml \hfil VP\\
%%       \[%{\it head-only-cxt} \& {\it ellip-cxt}\\
%%       HEAD \@2\\
%%       SUBJ \q< \@1\q>\]\avmr \end{avm}
%%     [{\begin{avm} \avml \hfil  V\\
%%         \[HEAD \@2\[\AUX\ +\]\\
%%         SUBJ \q<\@1\q>\\
%%         COMPS \q<\; \;\q>\\
%%         ARG-ST \q<\@1NP, VP[{\it pro}]\q>\] \avmr \end{avm}}
%%       [can]]]]
%% \end{forest}
\caption{Structure of a VPE}\label{fig-53}
\end{figure}
%
The head daughter's COMPS list (VP[bse]) is empty because the second element in the ARG-ST
is a \textit{pro}.\footnote{PAE is basically different from NCA, as in examples like \textit{I asked Tracy to bring the horses into the barn but she refused}, where the infinitival VP complement of
\textit{refused} is unexpressed. NCA, which \citet{Hankamer1976} take to be a deep anaphor, is sensitive only to a limited set of main verbs, and its exact nature is still controversial. NCA has received relatively little attention in modern syntactic theory, including in HPSG.}

As mentioned in Section~\ref{sec-structural-mismatches}, HPSG analyses of PAE do not face the problem of having to rule out, or rule in, cases of structural mismatch or nonlinguistic antecedents, because their acceptability can be captured as reflecting discourse-based and construction-specific constraints on PAE.

%NOTE from review: Do you want to distinguish PAE from other cases of pro VP (I tried, I
%promise?
%Please say how the analysis you cite (Chaves 2014 ?) handles the cases
%of mismatches you discussed in the previous sections, as well as
%insensitivity to islands, split antecedents etc


\section{Analyses of nonconstituent coordination and gapping}
\label{sec-analyses-of-noncon}

Constructions such as
RNR (right node raising), ACC (Argument cluster coordination), and gapping have also often been taken to belong
to elliptical constructions. Each of these constructions has received
relatively little attention in the research on elliptical constructions, possibly
because of their syntactic and semantic complexities. In this
section, we briefly discuss the direction of surface-based HPSG analyses
for these that take them as a subtype of ellipsis. For the 
detailed discussion of these constructions, we refer the
reader to \citet{Abeill{\'e}2020a} in this volume and references 
therein. 
%
%
%, whose analyses we address in separate subsections below.

\subsection{Right Node Raising}

In typical examples of RNR, as shown in the examples below, the element to the immediate right of a parallel structure is shared with the left conjunct:

\eal
\label{ex-kim-prepares-and-Lee-eats-Kim-played-and-Lee-sang}
\ex  Kim prepares and Lee eats [the pasta].  \label{60}
\ex  Kim played and Lee sang [some Rock and Roll songs at Jane's party]. \label{rnr61}
\zl
%
The bracketed shared material can be either a constituent as in (\ref{60}) or a non-constituent as in (\ref{rnr61}).

RNR has consistently attracted HPSG analyses involving silent material.
All existing analyses of RNR \citep{Abeille2016, Beavers2004, Chaves2008-in-lexicon, Chaves2014, Crysmann2003, Shiraishi2019, Yatabe2001, Yatabe2012} agree on this point,
although some of them propose more than one mechanism for accounting for different kinds of nonconstituent coordination \citep{Chaves2014, Yatabe2001, Yatabe2012, Yatabe2019}. One strand of research within the RNR literatures adopts a linearization-based approach employed more generally in analyses of NCC \citep{Yatabe2001, Yatabe2012} and another proposes a deletion-like operation \citep{Abeille2016, Chaves2014, Shiraishi2019}.
%, the
%latter of which is our focus here.

%
% We focus on the latter here, along with the question that it has raised, that is, what kind % of identity constraints must be satisfied before RNR can apply.

The kind of material that may be RNRaised and the range of structural mismatches permitted between the left and right conjuncts have been the subject of recent debate.\footnote{Although we refer to the material on the left and right as conjuncts, it is been known since \citet{Hudson1976, Hudson1989} that RNR extends to other syntactic environments than coordination (see \citealt{Chaves2014} for stressing this point).} For instance, \citet[839--840]{Chaves2014} demonstrates that, besides more typical examples like (\ref{ex-kim-prepares-and-Lee-eats-Kim-played-and-Lee-sang}),
 there is a range of phenomena classifiable as RNR that exhibit various argument-structure mismatches ((\ref{54})--(\ref{55})) and can target material below the word level ((\ref{56})--(\ref{57})).
%
%\ea Ethan sold and Rasmus gave away all his CDs. \label{RNR8} \z

\eal
\ex Sue gave me---but I don't think I will ever read---[a book about relativity]. \label{54}

\ex Never let me---or insist that I---[pick the seats].\label{55}

\ex We ordered the hard- but they got us the soft-[cover edition].\label{56}

\ex Your theory under- and my theory over[generates].\label{57}\zl
%
Furthermore, RNR can target strings that are not subject to any known syntactic operations, such as rightward movement \citep[865]{Chaves2014}.

\eal
\ex I thought it was going to be a good but it ended up being a very bad [reception].\label{58}

\ex Tonight a group of men, tomorrow night he himself, [would go out there somewhere and wait].\label{59}\zl

%\ea They were also as liberal or more liberal [than any other age group in the 1986 through 1989 surveys].\label{60}\z
RNRaised material can also be discontinuous, as in (\ref{ex-rnr-discontinuous}) (\citealt[868]{Chaves2014}; \citealt[238--240]{Whitman2009}).
%\todosatz{No reference given, but there is one in cg.bib and one in lfg.bib which can be %used by changing W to w.}

\eal
\label{ex-rnr-discontinuous}
\ex Please move from the exit rows if you are unwilling or unable [to perform the necessary actions] without injury.\label{61}

\ex The blast upended and nearly sliced [an armored Chevrolet Suburban] in half.\label{62}\zl
%
This evidence leads \citet{Chaves2014} to propose that RNR is a nonuniform phenomenon, comprising extraposition,  VP\/N$'$-ellipsis, and true RNR.
% besides 'true' RNR
% and requiring different kinds of theoretical analyses.
%
%
%
Of the three, only true RNR should be accounted for via the mechanism of optional surface-based deletion that is sensitive to morph form identity and targets any linearized strings, whether constituents or otherwise.\footnote{Whenever RNR can instead be analyzed as either VP\/N'-ellipsis or extraposition, \citeauthor{Chaves2014} proposes separate mechanisms for deriving them: the direct interpretation approach described in the previous sections for NSUs and predicate/argument ellipsis and an analysis employing the feature EXTRA to record extraposed material along the lines of \citeauthor{KimSag2005, Kay2012}), respectively.} \citeauthor{Chaves2014}' (\citeyear[874]{Chaves2014}) constraint licensing true RNR is given in (\ref{63}) as an informal version  ($\alpha$
= a morphophonologic constituent, $^{+}$ = a
 $^{+}$ = a Kleene plus):
%
%It permits the M(orpho)P(honology) feature of the mother to contain only one instance %(represented as $L_{3}$ in (\ref{63})) of the two morphophonologically identical sequences %[FORM $F_{1}$], \ldots, [FORM $F_{n}$] present in the daughters; the leftmost of these %sequences undergoes deletion. The final list in the mother, $L_{4}$, may be empty or %nonempty, depending on whether RNRaised material is discontinuous.
%
%
%
\ea
\label{63}
 Backward Periphery Deletion Construction:\\

Given a sequence of morphophonologic constituents $\alpha_{1}^{+}$ $\alpha_{2}^{+}$ $\alpha_{3}^{+}$ $\alpha_{4}^{+}$ $\alpha_{5}^{*}$, then the output
$\alpha_{1}^{+}$ $\alpha_{3}^{+}$ $\alpha_{4}^{+}$ $\alpha_{5}^{*}$
iff $\alpha_{2}^{+}$ and $\alpha_{4}^{+}$ are identical up to morph forms.
\z
(\ref{63}) takes the morphophonology of a phrase to be computed as the linear combination of the phonologies of the daughters, allowing deletion to apply locally.\footnote{For more detail on linearization-based analyses of RNR, the interested reader is referred to \citet{Yatabe2001, Yatabe2012}, who distinguish between syntactic RNR and phonological RNR, based on the amount of morphosyntactic identity holding between RNRaised material and the requirements imposed on the slots it occupies in the structure, and represent this distinction by treating the RNRaised material as either a separate domain object on the mother's DOM list (syntactic RNR) or embedded in a larger domain object corresponding to the right conjunct (phonological RNR).}



\iffalse{
\citeauthor{Chaves2014}' (\citeyear[874]{Chaves2014}) constraint licensing true RNR is given in \ref{bpd}. It permits the M(orpho)P(honology) feature of the mother to contain only one instance (represented as $L_{3}$ in (\ref{63})) of the two morphophonologically identical sequences [FORM $F_{1}$], \ldots, [FORM $F_{n}$] present in the daughters; the leftmost of these sequences undergoes deletion. The final list in the mother, $L_{4}$, may be empty or nonempty, depending on whether RNRaised material is discontinuous.
%
%\fi
%\inlinetodostefan{AVM: long line and short type alignment, funny space after L2 and L3}



\ea
\label{bpd}
Backward periphery deletion construction:\\
\avmtmp{\small
[\type*{phrase}
  mp $L_{1}$:\type{ne-list} $\bigcirc L_{2}$:\type{ne-list} $\bigcirc L_{3} \bigcirc L_{4}$ & ]
} $\to$
\avmtmp{
 [ \type*{phrase}
    mp $L_{1} \bigcirc$ < [ form $F_{1}$ ], \ldots, [ form $F_{n}$ ] > $\bigcirc L_{2} \bigcirc L_{3}:$%\\ \hspace{50pt}
     <[ form $F_{1}$ ], \ldots, [form $F_{n}$ ] > $\bigcirc L_{4}$ & ]
}
\z
}\fi

Another deletion-based analysis of RNR is due to \citep{Abeille2016, Shiraishi2019}, differing from \citet{Chaves2014} in terms of identity conditions on deletion. \citet{Abeille2016} argue for a finer-grained analysis of French RNR without morphophonological identity. Their empirical evidence reveals a split between functional and lexical categories in French such that the former permit mismatch between the two conjuncts (where determiners or prepositions differ) under RNR, while the latter do not. \citet{Shiraishi2019} provide further corpus and experimental evidence that morphophonological identity is too strong a constraint on RNR, given the range of acceptable mismatches between the verbal forms of the material missing from the left conjunct and those of the material that is shared between both conjuncts.

\iffalse{
To illustrate, an English verb form mismatch is depicted in (\ref{verbform}), from \citep[see][5]{Shiraishi2019}, where the left conjunct requires a participle while the shared material contains an infinitive form of the verb \textit{appear}.

\ea Some new hybrid models already have, and others soon will appear in the automobile industry.\label{verbform}

\citep{Shiraishi2019} capture verb form mismatch of this kind by introducing into their analysis of RNR the feature LID, which carries lexeme identity information. That is, this feature ensures semantic and syntactic category identity but ignores differences introduced by inflectional suffixes, with the result that the participle and the infinitive in (\ref{verbform}) count as lexeme-identical. RNR is licensed by including the LID feature in the MP feature also used in \citet{Chaves2014} (see (\ref{bpd})).
%Shiraishi2019 rnr-cx here \label{rnrcx}
%(\ref{rnrcx}) ensures that the content of the $l_{2}$ list in the left conjunct, which is elided and hence not represented in the mother, is shared with the $r_{2}$ list in the right conjunct via the LID feature. The lists $l_{1}$, $r_{1}$ and $r_{3}$ in (\ref{rnrcx}) represent material present overtly in the left and right conjuncts.
}\fi


\subsection{Argument Cluster Coordination}

As noted earlier, ACC is a type of non-constituent coordination, as
illustrated in (\ref{ex-acc}):
%
%
\eal
\label{ex-acc}
\ex John gave [a book to Mary] and [a record to Jane].   \label{acc-here}
\ex John gave [Mary a book] and [Jane a record].  \label{acc-1}
\zl
%
%Our focus here is on HPSG analyses of ACC, which departs from those
%relying on the notion of ellipsis where silent material is permitted as part of the %structure.
%

%Departing from the traditional assumption that such examples involve non-constituent
%coordination, 
For the treatment of ACC, the existing HPSG analyses have two main views: ellipsis
(\citealt{Yatabe2001, Crysmann2003, Beavers2004}) and
non-standard constituents (\citealt{Mouret2006}). As for the discussion of
the non-elliptical view that takes ACC to be a special type of coordination,
we refer the reader to  \citet{Abeill{\'e}2020a} in this volume and references 
therein but focus on the ellipsis view that fits in this chapter. 
 The ellipsis view is based on examples like the ones in (\ref{ex-acc-eliptical}):\footnote{For an example of an analysis of ACC that coordinates noncanonical constituents and doesn't posit the existence of silent material, see \citep{Mouret2006}.}

\eal
\label{ex-acc-eliptical}
\ex Jan travels to Rome tomorrow, [to Paris on Friday], and will fly to Tokyo
on Sunday. \label{acc2}
\ex Jan wanted to study medicine when he was 11, [law when he was 13],
and to study nothing at all when he was 18. \label{acc3}\zl
%
As pointed out by \citet{Beavers2004}, such examples challenge non-ellipsis analyses, given the traditional assumption that only 
constituents of like category can coordinate. The status of the bracketed conjuncts
is quite questionable, since they are not VPs like the other two fellow conjuncts.
To address this issue, surface-oriented HPSG analyses employ a key idea from linearization theory
where the level of an order domain is operationalized as the DOM list obeying the
Coordination Construction given in (\ref{CC}), which is a simplified
version of the one in \citet[(27)]{Beavers2004}:\footnote{For simplicity, we represent only the DOM value, suppressing all the other information. For more details on the role of the DOM list in HPSG accounts of constituent order, the reader is referred to \citealt{Mueller2020} in this volume.}
%

\ea\label{CC}
\begin{avm}
\[mtr [dom \@A $\oplus$ \@B$_{1}$ $\oplus$ \@C $\oplus$  \@B$_{2}$ $\oplus$ \@D]\\
 dtrs \<[dom \@A $\oplus$ \@B$_{1}$[\textit{ne-list}] $\oplus$ \@X],
        [dom  \@C[({\it conj})] $\oplus$ \@Y $\oplus$  \@B$_{g}$[\textit{ne-list}] $\oplus$ \@D]\> \]
\end{avm}
\z
%
%
% \ea
%\label{conj-cxt}
%\type{cnj-cxt} \impl
%\avmtmp{
%[mtr [dom \@A \oplus \@B$_{1}$ \oplus \@C \oplus  \@B$_{2}$ \oplus \@D]\\
% dtrs \<[dom \@A \oplus \@B$_{1}$[\textit{ne-list}] \oplus \@X],
%[dom   <\@C[({\it conj})] \oplus \@X \oplus  \@B$_{g}$[\textit{ne-list}] \oplus \@D]\>
%]
%}
%\z
%
%
The content of the DOM list consists of prosodic constituents (constituents with no information about their internal morphosyntax) and offers a way of accounting for coordination of noncanonical constituents. In analyses of ACC, the elements present on the mother's DOM list are those present overtly on the DOM lists of both conjuncts, as well as those present overtly on the DOM list of the left, but not the right, conjunct. The DOM value of the mother in (\ref{CC}) begins with material A (empty or otherwise) from the left conjunct, some material from the left conjunct B$_{1}$, the conjunct's
coordinator C (if present), some material B$_{2}$
from the right conjunct, and ends with some material D from the right conjunct. To derive NCC as in (\ref{ex-acc}), the left-most element on the mother's DOM list, representing material present overtly only in the left conjunct (here the verb {\it gave}), may not be empty.\footnote{See \citet{Beavers2004} for the discussion of semantic issues
in NCC.}
 
%
% a type of nonconsitutent cluster. 
%
%One advantage of this analysis comes from the following data set (\citealt{Beavers2004}):
%
%
%\ea  Jan [walks [talks and [chews gum]]]. \z
%\ea Jan [[walks and talks] [or [walks and [chews gum]]]]. \z
%\ea *Jan [walks [chews gum] \z
%
%NOTE: Is this a complete thought?

%
%
%
%
%
%
% To make this more precise, consider the \citep{Beavers2004} schema in (\ref{acc2}), which % derives not just ACC, but also RNR and constituent coordination.
%
%
%
%BeaversSag2004 here \label{acc2}
%
%
%This assumption allows us to coordinate VPs where left- and/or right-most elements on the %mother's DOM list may or may not be empty to capture different types of coordination.
%
%
%\ea Harvey gave a book to Ethan and a record to Rasmus. \label{acc3}\z

%The schema in (\ref{acc2}) also permits derivation of RNR, provided the right-most element %on the mother's DOM list (the correspondent of the material present overtly only in the %right conjunct) is not empty. We now take a closer look at analyses of RNR in the next %section.


\subsection{Gapping}

Gapping is also a type of ellipsis that allows a finite
verb to be unexpressed in the non-initial conjuncts of English coordination in (\ref{ex-gapping}).

\eal
\label{ex-gapping}
\ex Some ate bread, and others rice.\label{g1}
\ex Kim can play the guitar, and Lee the violin.\label{g2}
\zl
%
%
%
%

HPSG analyses of gapping in English fall into two kinds: one kind draws on \citeauthor{Beavers2004}'s (\citeyear{Beavers2004}) deletion-based analysis of nonconstituent coordination \citep{Chaves2009} and the other on \citeauthor{Ginzburg:Sag:2000}'s (\citeyear{Ginzburg:Sag:2000}) analysis of NSUs \citep{Abeille2014}.\footnote{For a semantic approach to gapping, the reader is referred to \citet{Parketal2019}, who offer an analysis of scope ambiguities under gapping where the syntax assumed is of the NUS type and the semantics is cast in the framework of Lexical Resource Semantics.} The latter analyses align gapping with analyses of NSUs, as discussed in Section~\ref{sec-analyses-of-NSUs}, more than with analyses of nonconstituent coordination, and for this reason gapping could be classified together with other NSUs. We use the analysis in \citet{Abeille2014} for illustration below.


\citet{Abeille2014}, focusing on French and Romanian, offer a construction- and
discourse-based HPSG approach to gapping where the second headless gapped conjunct is taken to be an
NUS type of fragment. Their analysis places no syntactic parallelism requirements on the
first conjunct and the gapped conjunct, given data like (\ref{65}).

\ea Pat has become [crazy]$_{AP}$ and Chris [an incredible bore]$_{NP}$.  \label{65}\z
%
% assume an identity condition on gapping requiring that gapping remnants match major %constituents in the antecedent clause, which they term source clause. In other words,
Instead of requiring strong syntactic parallelism between the two clauses, their analysis limits gapping remnants to elements of the argument structure of the verbal head present in the antecedent and absent from the rightmost conjunct, which reflects the intuition articulated in \citet{Hankamer1971}. Their analysis starts with the assumption that coordination phrases are nonheaded constructions in which each conjunct shares the same
valence (SUBJ and COMPS) and nonlocal (SLASH) features while
 its head (HEAD) value is not fixed but contains an upper bound (supertype) to accommodate
 examples like (\ref{65}). On this analysis, the gapped conjunct {\it Chris an incredible
  bore} in (\ref{65}) is an NUS fragment with two cluster daughters. The required
 syntactic parallelism between gapping remnants and their counterparts in the antecedent is operationalized by adopting the contextual attribute SAL-UTT, which is introduced for all NSUs, as in (\ref{gap-hf-con}).
 %\footnote{The interpretation of the remnants
 %follows from the high order unification algorithm.}

 \ea
\label{gap-hf-con}
Syntactic constraints on {\it head-fragment-ph} (\citealt[(53)]{Abeille2014}):\\
\type{head-fragment-ph} \impl
\avmtmp{
[cnxt|sal-utt < [head  H$_{1}$\\
                 major +],...,[head  H$_{n}$\\
                                major +]>\\
 cat|head|cluster<[head  H$_{1}$],...,[head  H$_{n}$]>]
}
\z
 %
 %
The syntactic identity between gapping remnants and their counterparts is achieved
by ensuring that each list member of the SAL-UTT bears the specification [MAJOR +] as part of its HEAD feature and is coindexed with a gapping remnant.\footnote{The feature MAJOR makes each expression a major constituent functioning as a dependent of some verbal projection, blocking
remnants from being deeply embedded in the gapped clause.}
   With syntactic identity captured this way, we predict correctly that gapping remnants needn't appear in the same order as their counterparts in the antecedent (\ref{64}) \citep[see][156--158]{Sag1985}, nor are they required to be the same syntactic category as their counterparts (\ref{65}).

\ea A policeman walked in at 11, and at 12, a fireman. \label{64}\z
%
%\ea Pat has become [crazy]$_{AP}$ and Chris [an incredible bore]$_{NP}$.  \label{65}\z
%


The unique properties of gapping compared to other types of ellipsis are captured by
gapping as a sup-construction of coordination and assigning its own constructional
constraints to it: the contextual background information requires each conjunct to hold
some symmetric discourse relation (for a detailed discussion, see
\citealt{Abeille2014}).


\iffalse{
\citet{Abeille2014} offer additional evidence from Romance (e.g., case mismatch between gapping remnants and their counterparts and even more possibilities of ordering remnants than is the case in English) to strengthen their point that syntactic identity is relaxed under gapping.

There are three further key assumptions in Abeill\'{e} et al.'s (2014) analysis. First, two (or more) gapping remnants form a cluster whose mother has an underspecified syntactic category, that is, is a non-headed phrase (this information is represented by the Cluster head feature in \ref{66}). This phrase then serves as the head daughter of a head-fragment phrase, whose syntactic category is also underspecified. This means that there is no unpronounced verbal head in the phrase to which gapping remnants belong. Second, the meanings of the gapping remnants are computed from the meaning of the rightmost nonelliptical verbal conjunct, as represented by the Source feature in \ref{66}. Finally, the conjuncts are linked by a symmetric discourse relation (i.e.,  parallelism or contrast) that is part of the Background feature in \ref{66}.
}\fi

\iffalse{

With these ingredients of the analysis in place, we reproduce the gapping construction in (\ref{66}). The construction represents asymmetric coordination in the sense that the daughters include both nonelliptical verbal conjuncts and head-fragment phrases with an underspecified syntactic category. The mother only shares its syntactic category with the nonelliptical conjuncts so that its own category is specified to be verbal.


\ea
\label{66}
Gapping construction\\
\type{gapping-ph} \impl \type{coord-ph} \&\\
\begin{avm}
%$\Langle$
\< \[HEAD \fbox{H} \type{verbal}\\
CNXT | BCKG \{ ..., sym-disc-rel(\fbox{M$_{1}$},..., \fbox{M$_{j}$}, \fbox{M$_{j+1}$},..., \fbox{M$_{n}$} ), ... \}\\

DTRS $\langle$ \[ HEAD \fbox{H} \[ verbal \\ CLUSTER elist \]\\CONT \fbox{M$_{1}$} \] ,..., \[
HEAD \fbox{H} \[ verbal \\ CLUSTER elist \]\\CONTENT \fbox{M$_{j}$} \]\] \> %$\Rangle
$\bigoplus$
\end{avm}
\begin{avm}
%$\Langle$
\<\[ HEAD \[CLUSTER $\langle \fbox{1},...,\fbox{n}\rangle$\] \\
             SOURCE \fbox{M$_{j}$}\\
             CONTENT \fbox{M$_{j+1}$} \],..., \[ HEAD \[CLUSTER $\langle$ \fbox{1$'$},...,\fbox{n$'$}$\rangle$\]\\
                                                SOURCE \fbox{M$_{j}$}\\
                                                CONTENT \fbox{M$_{n}$}\]\>%$\rangle$
\end{avm}
\z
}\fi



\section{Summary}
\label{sum}
This chapter has reviewed three types of ellipsis, nonsentential utterances, predicate ellipsis, and nonconstituent coordination, corresponding to three kinds of analysis within HPSG. The pattern that emerges from this overview is that HPSG favors the `what you see is what get' approach to ellipsis and limits a deletion-based approach, common in the minimalist literature on ellipsis, to a subset of nonconstituent coordination phenomena.

%\citep{Chomsky1957}.
%\citep{Comrie1981}




%\begin{table}
%\caption{Frequencies of word classes}
%\label{tab:1:frequencies}
% \begin{tabular}{lllll}
%  \lsptoprule
            %& nouns & verbs & adjectives & adverbs\\
 % \midrule
  %absolute  &   12 &    34  &    23     & 13\\
%  relative  &   3.1 &   8.9 &    5.7    & 3.2\\
 % \lspbottomrule
 %\end{tabular}
%\end{table}




\section*{Abbreviations}

\begin{tabularx}{.99\textwidth}{@{}lX}
NSUs & Nonsentential utterances\\
BAE & Bare Argument Ellipsis\\
VPE & Verb Phrase Ellipsis\\
NCA & Null Complement Anaphora\\
SAL-UTT & Salient Utterance\\
MAX-QUD & Maximal-Question-under-Discussion\\
DGB & Dialog Game Board\\
\end{tabularx}


\section*{Acknowledgements}
We thank the editors of this handbook and Yusuke Kubota for helpful comments.

{\sloppy
\printbibliography[heading=subbibliography,notkeyword=this]
}
%
}% AVM options


\end{document}

%      <!-- Local IspellDict: en_US-w_accents -->
