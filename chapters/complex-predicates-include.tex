%% -*- coding:utf-8 -*-
\author{%
	Danièle Godard\affiliation{Université Paris Diderot}%
	\lastand Pollet Samvelian\affiliation{Université Sorbonne Nouvelle}
}
\title{Complex predicates}


% \chapterDOI{} %will be filled in at production

%\epigram{Change epigram in chapters/03.tex or remove it there }
\abstract{Complex predicates are constructions in which a head attracts arguments from its
  predicate complement. Auxiliaries, copulas, predicative verbs, certain
  control or raising verbs, perception verbs, causative verbs and light verbs can head complex predicates. This phenomenon has been studied in HPSG in different languages, including Romance and Germanic languages, Korean and Persian. They each illustrate different aspects of complex predicate formation. Romance languages show that argument inheritance is compatible with different phrase structures. German, Dutch and Korean show that argument inheritance can induce different word order properties, and Persian shows that a complex predicate can be preserved by a derivation rule (nominalization from a verb), and, most importantly in Persian, which has relatively few simplex verbs, that light verb constructions are used to turn a noun into a verb. }

\begin{document}
\maketitle
\label{chap-complex-predicates}

{\avmoptions{center}

% \textcolor{red}{Do not edit now. Stefan is editing offline.}
% \newpage
% Do not edit now. Stefan is editing offline.
% \newpage
% Do not edit now. Stefan is editing offline.
% \newpage

% why is this different?

\section{Introduction}

Words such as verbs, nouns, adjectives or prepositions typically denote predicates that are associated with arguments, and those arguments are typically syntactically realized as the subject, complements or specifier of those words. For instance, a verb such as \textit{to eat} has two arguments, realized as its subject and its object, and understood as agent (the eater) and patient (what is eaten). Usually, arguments are associated with just one predicate (one word). However, in constructions called \textit{complex predicates}, two or more predicates associated with words behave as if they formed just one predicate, while keeping their status as different words in the syntax. 
For instance, tense auxiliaries in Romance languages form a complex predicate with the participle
which follows, but they are different words, since they can be separated by an adverb, as in French \emph{Lucas a rapidement lu ce livre} `Lucas has quickly read this book'; see (\ref{GSexemple1}).
Several properties set apart complex predicates from ordinary predicates, and those properties can differ from one language to another. In HPSG, complex predicates are analyzed as constructions in which one predicate, the head, ``attracts'' the arguments of the other, that is, the syntactic arguments of one word or predicate include the syntactic arguments of another word or predicate. This chapter is devoted to the various analyses of complex predicates that have been proposed within HPSG and some of the cross-linguistic variation in the behavior of complex predicates, focusing on French, German, Korean and Persian. 

\section{What are complex predicates?}


The term \textit{complex predicate} does not have a universally accepted definition. In this section, we explain
how it is used in HPSG to name a syntactic phenomenon where two (or more) words 
form what appears to be a single predicate because the head is attracting the (syntactic) arguments of its complement.
We then mention the work that has been done in different languages on this aspect of natural language grammars and the constructions in which it manifests itself. 
Finally, we contrast our use of the term \textit{complex predicates} with other uses of the term and with related phenomena, in particular Serial Verb Constructions. 

%\inlinetodostefan{Stefan: To make the volume as such more uniform, please start with an introduction explaining what you are doing. A section should never start with a subsection right away.} % EP 25.07.2020: Taken care of.

\subsection{Definition}\label{GSsection1.1}

In the HPSG tradition, a complex predicate is composed of two or more words, each of which is itself a predicate. By predicate, we mean either a verb or a word of a different category (noun, adjective, preposition) which is associated with an argument structure. A complex predicate is a construction in which the head attracts the arguments of the other predicate, which is its complement: the arguments selected by the complement predicate ``become'' the arguments of the head \citep{HN89b, HN98a}. The phenomenon is called \emph{argument attraction}, \emph{composition}, \emph{inheritance} or \emph{sharing}.

To take an example, tense auxiliaries and the participle in Romance languages are two different words, since they can be separated by adverbs, as in the French examples in (\ref{GSexemple1}), but the two verbs belong to the same clause, and, more precisely, the syntactic arguments belong to one argument structure. We admit that the property of monoclausality can manifest itself differently in different languages \citep{Butt2010a}. In the case of Romance auxiliary constructions, the first verb (the auxiliary) hosts the clitics which pronominalize the arguments of the participle: corresponding to the NP complement \emph{son livre} `his book' in (\ref{GSexemple1a}), the pronominal clitic \emph{l(e)} is hosted by the auxiliary \emph{a} `has' in (\ref{GSexemple1b}) and (\ref{GSexemple1c}). This contrasts with the construction of a control verb such as \emph{vouloir} `to want', where the clitic corresponding to the argument of the infinitive is hosted by the infinitive, as in (\ref{GSexemple2}) \citep[from][406]{AG2002b-u}: 

\eal 
	\label{GSexemple1} 
	\ex[]{
	\gll Paul a   rapidement lu   son livre.\\
		 Paul has quickly    read his book\\\jambox*{(\ili{French})}
	\glt `Paul has quickly read his book.'}\label{GSexemple1a} 
		
	\ex[]{
	\gll Paul l'a    rapidement lu.\\
		 Paul it.has quickly    read\\
	\glt `Paul has quickly read it.'}\label{GSexemple1b}  
		
	\ex[*]{
	\gll Paul a   rapidement le lu.\\
		 Paul has quickly    it read\\
	\glt Intended: `Paul has quickly read it.'}\label{GSexemple1c} 
\zl


\eal 
	\label{GSexemple2} 
	\ex[]{
	\gll Paul veut  lire son livre.\\
		 Paul wants read his book\\\jambox*{(\ili{French})}
	\glt `Paul wants to read his book.'}\label{GSexemple2a}
		
	\ex[]{
	\gll Paul veut  le lire.\\
		 Paul wants it read\\
	\glt `Paul wants to read it.'}\label{GSexemple2b}
		
		
	\ex[*]{
	\gll Paul le veut lire.{\footnotemark}\\
		 Paul it wants  read\\
	\glt Intended: `Paul wants to read it.' \footnotetext{Possible in an earlier stage of French.}}\label{GSexemple2c}			
\zl


This approach to complex predicates goes back to Relational Grammar \citep{aissen1983clause}: although formalized in a different way, their analysis of causative constructions in Romance languages relies on such argument attraction, under the name of \emph{clause union}. Similarly, in Lexical Functional Grammar, \cite{andrews1999complex} speak of complex predicates as building a domain of grammatical relations sharing. It is also present in Categorial Grammar \citep{Geach70a}, with complex categories whose definition takes into account the nature of the argument they combine with and the operation of function attraction. In particular, \cite[301]{kraak1998deductive} accommodates complex predicates by introducing a specific mode of combination called \emph{clause union mode}, where two verbs (two lexical heads) are combined. But, in this account, there is no argument attraction in general, the mechanism being specifically defined in order to account for clitic climbing.


There are other definitions of complex predicates. The term has been used to describe the complex content of a word, when it can be decomposed. For instance, the verb \emph{dance} has been analyzed as incorporating the noun \emph{dance} and considered a ``complex predicate'' \citep{HK97a-u}. In the sense adopted here, complex predicates involve at least two words, and are syntactic constructions. Closer to what we consider here to be complex predicates is the case of Japanese passive or causative verbs, illustrated in (\ref{GSexemple3}).

\ea[]{
\label{GSexemple3}
\gll tabe-rare-sasete-i-ta.\\
     eat-\textsc{pass}-\textsc{caus}-\textsc{prog}-\textsc{pst}\\\jambox*{(\ili{Japanese})}
\glt `(someone) was causing (something) to be eaten.'}
\z

The causative morpheme adds a causer argument, and behaves as if it took the verb stem as its complement (more precisely, the verb stem with the passive morpheme, in this case), whose expected subject appears as the object of the causative verb. This operation is like argument attraction. However, it happens in the lexicon rather than in syntax: the elements in (\ref{GSexemple3}) are bound morphemes, and they form a word \citep{manning1999lexical, gunji2012topics}. Thus, we do not consider causative verbs in Japanese to constitute complex predicates.

Complex predicates are sometimes given a semantic definition: the two elements together describe one situation \citep{butt1995structure}. Such a semantic definition does not coincide with the syntactic one. It is true that the head verb of a complex predicate tends to add tense, aspectual or modal information, while the other element describes a situation type. Thus, in (\ref{GSexemple1}), the two verbs jointly describe one situation, the auxiliary adding tense and aspect information. But the semantics of a complex predicate is not always different from that of ordinary verbal complements. Thus, there is no evident semantic distinction depending on whether the Italian restructuring verb \emph{volere} `to want' is the head of a complex predicate (\ref{GSexemple4a}) or not (\ref{GSexemple4b}), and the two verbs do not seem to describe just one situation \citep[314]{Monachesi98a}.  

\eal 
	\label{GSexemple4} 
	\ex[]{ 
	\gll Anna lo vuole comprare.\\ 
	     Anna it wants buy\\\jambox*{(\ili{Italian})}
	\glt `Anna wants to buy it.'}\label{GSexemple4a}
		
	\ex[]{ 
	\gll Anna vuole comprarlo.\\
	     Anna wants buy.it\\
    \glt `Anna wants to buy it.'}\label{GSexemple4b} 
\zl

The same point is made for Hindi in \cite{poornima2009hindi}. They show that there exist two structures combining an aspectual verb and a main verb; in one of them, the aspectual verb is the head of a complex predicate while, in the other one, it is a modifier of the main verb. In more general terms, complex predicates show that syntax and semantics are not always isomorphic in a language. Thus, although the semantic definition of complex predicates may be useful for some purposes, we will ignore it here.

The distinction between complex predicates and \emph{serial verb constructions} (SVCs), for example the one illustrated in (\ref{GSexemple5}) (from \citealt[294]{MH2016}), is not evident (e.g.\ \citealt{andrews1999complex, MH2016}). The main reason is that the constructions which have been dubbed SVCs are different in different languages; we agree with \cite{andrews1999complex} that they do not share a grammatical mechanism, but they do share more superficial tendencies, such as their resemblance to paratactic constructions due to the absence of marking of complementation or coordination, and they also involve more semantic relations than are usually associated with complementation and coordination.

\ea[]{
	\label{GSexemple5}
	\gll \`Oz\'o s\`a\'an  rr\'a  \'ogb\`a.\\
	     Ozo     jump      cross  fence\\\jambox*{(\ili{Edo})}
	\glt `Ozo jumped over the fence.'} 
\z

Accordingly, SVCs are not within the purview of complex predicates, and will not be studied in this chapter.

\subsection{Constructions involving complex predicates}\label{GSsection1.2}

Complex predicates enter into a number of constructions across languages. They differ from ordinary constructions in different ways, depending on the construction, such as the position of pronominal clitics in Romance languages (``clitic climbing''), word order or special semantic combinations. 

The following have been particularly studied in HPSG:

\begin{itemize}
	
	\item Romance languages' tense auxiliaries, copulas and other verbs taking predicative complements, restructuring verbs headed by certain subject raising or control verbs, as well as certain causative and perception verbs \citep{abeille1994complementation, abeille2000french, abeille2001deux, abeille2001varieties, AG2002b-u, AG2010, abeille1995doublestructure, AGMS98a, AGS1998, Monachesi98a};
	
	\item certain constructions in German and Dutch, called coherent constructions, headed by tense auxiliaries, certain raising and control verbs, certain verbs with predicative complements, as well as the copula and particle verbs \citep{HN89b, HN94a, Rentier94, Kiss94, Kiss95a, BvN98a, HN98a, Kathol98b, Kathol2000a, Meurers2000b-Short, DM2002, dKM2001a,  Mueller2002b, Mueller2003a, muller2018clause};
	
	\item Korean auxiliaries, control verbs, \emph{ha} causative verbs and light verb constructions \citep{ Sells1991, Ryu:93, Chung98a-u, lee2001argument, choi2001mixed, Yoo2003, Kim2016a-u};
	
	\item Hindi aspectual predicates \citep{poornima2009hindi}; 
	
	\item Persian light verb constructions (combinations of a semantically light verb with a predicate belonging to diverse categories; \citealt{bonami2010persian, MuellerPersian, pollet2012grammaire, bonami2015diversity});   
	
	\item causatives in various languages (among them German, Italian, Turkish), including both analytical causatives (complex predicates in the sense adopted her) and synthetic causatives \citep{Webelhuth98a-u}. 
	
\end{itemize}


In this chapter, we examine some of these constructions which illustrate the different ways in which complex predicates differ from ordinary verbs.


\section{The basic mechanism in HPSG: Argument attraction}\label{GSsection2}
\label{complex-predicates-sec-argument-attraction}

In HPSG, complex predicates are analyzed in the following way: one of the predicates is the head of the construction, and it attracts the syntactic arguments of the other predicate, that is, its complements and, possibly, its subject. The phenomenon is called \emph{argument attraction}, \emph{composition}, \emph{inheritance}, \emph{raising} or \emph{sharing}. We illustrate it with tense auxiliaries in French \citep{abeille1994complementation, AG2002b-u}.

In French, auxiliary constructions consist of a tense auxiliary (\emph{avoir} `to have' or \emph{\^etre} `to be') followed by a past participle and its complements, as illustrated in (\ref{GSexemple1}). The auxiliary is the head.
It bears inflectional affixes (for tense and person) like any other verb, and if the sentence is declarative, it is in the indicative form as expected; for example, the auxiliary in (\ref{GSexemple1}) has the form of a present indicative third person. 
The auxiliary also hosts pronominal clitics, as verbal heads in general do, as shown in (\ref{GSexemple1b}) and (\ref{GSexemple1c}). Moreover, it can be gapped alone, as (\ref{GSexemple6a}) shows, while the participle can only be gapped with the auxiliary, as illustrated by (\ref{GSexemple6b}) and (\ref{GSexemple6c});\footnote{Note that (\ref{GSexemple6c}) is acceptable with the possession verb \emph{avoir}.} this is expected if the auxiliary is the head, since it behaves like \emph{pense} `think' in (\ref{GSexemple6d}), while the participle behaves like the infinitive in (\ref{GSexemple6e}) and (\ref{GSexemple6f}). 

\eal
	\label{GSexemple6}
	\ex[]{
	\gll Lola a   achet\'e des   pommes, et  Alice (a)             cueilli des   p\^eches.\\
	     Lola has bought   some  apples  and Alice \spacebr{}has   picked  some  peaches\\\hspace{-5pt}
	\glt `Lola has bought apples, and Alice (has) picked peaches.'}\label{GSexemple6a}
		
	\ex[]{
	\gll Lola a   achet\'e des   pommes, et  Alice (a             achet\'e) des   p\^eches.\\
	     Lola has bought   some  apples  and Alice \spacebr{}has  bought    some  peaches\\
	\glt `Lola has bought apples, and Alice (has bought) peaches.'}\label{GSexemple6b}
		
	\ex[\#]{
	\gll Lola a   achet\'e des   pommes, et  Alice a   des   p\^eches.\\
		 Lola has bought   some  apples  and Alice has some  peaches\\
	\glt `Lola has bought apples, and Alice has peaches.'}\label{GSexemple6c}
				
	\ex[]{
	\gll Lola pense  acheter des   pommes, et  Alice (pense)            cueillir des  p\^eches.\\
	     Lola thinks buy     some  apples  and Alice \spacebr{}thinks   pick     some peaches\\
	\glt `Lola is thinking of buying apples, and Alice (is thinking of) picking peaches.'}\label{GSexemple6d}
		
	\ex[]{
	\gll Lola pense  acheter des  pommes, et  Alice (pense            acheter) des  p\^eches.\\
	     Lola thinks buy 	 some apples  and Alice \spacebr{}thinks  buy      some peaches\\
	\glt `Lola is thinking of buying apples, and Alice (is thinking of picking) peaches.'}\label{GSexemple6e}
		
	\ex[*]{
	\gll Lola pense  cueillir des  pommes et  Alice pense  des  p\^eches.\\
	     Lola thinks pick     some apples and Alice thinks some peaches\\
	\glt Intended: `Lola is thinking of picking apples and Alice is thinking of (picking) peaches.'}\label{GSexemple6f}
\zl

The auxiliary construction in French is a complex predicate: The clitic corresponding to a complement of the participle is hosted by the auxiliary (it is said to ``climb'') as in (\ref{GSexemple1b}). Moreover, it occurs in bounded dependencies such as the infinitival complement of adjectives like \emph{facile} `easy' or \emph{impossible} `impossible', whose nominal complement is unexpressed, as in (\ref{GSexemple7a}); this unexpressed complement can be that of a participle (\ref{GSexemple7c}) but not that of an infinitive complement (\ref{GSexemple7b}). This follows if the unexpressed complement is in fact treated as the complement of the auxiliary.

\eal 
	\label{GSexemple7}
	\ex[]{
	\gll Cette technique est impossible \`a ma\^itriser en un  jour.\\
	     this  technique is  impossible to  master      in one day\\\jambox*{(\ili{French})}
	\glt `This technique is impossible to master in one day.'}\label{GSexemple7a}
		
	\ex[*]{
	\gll Cette technique est impossible \`a r\'eussir \`a ma\^itriser en un  jour.\\
		 this  technique is  impossible to  manage    to  master      in one day\\
	\glt Intended: `This technique is impossible to manage to master in one day.'}\label{GSexemple7b}
		
	\ex[]{
	\gll Cette technique est impossible \`a avoir ma\^itris\'e en un  jour.\\
	     this  technique is  impossible to  have  mastered     in one day\\
	\glt `This technique is impossible to have mastered in one day.'}\label{GSexemple7c}
\zl

These two properties (clitic climbing and occurrence in bounded dependencies) follow if the complements of the participle become those of \emph{avoir} `to have'.
In fact, both clitic climbing and the dependency found in easy/impossible constructions belong to the set of bounded dependencies.
In addition, the tense auxiliary \emph{avoir} `to have' is a subject raising verb (see \crossrefchapterw{control-raising}): the subject is selected by the participle and shared by the auxiliary. For instance, \emph{Paul} is an agent in (\ref{GSexemple1a}) (\emph{Paul a lu son livre}, `Paul has read his book') because \emph{lire} `to read' requires an agent subject, and in e.g.\ \emph{Il a fait froid} (lit.\ It has made cold, `It [the weather] was cold'), the subject is the impersonal subject \emph{il}, because that is the subject of the participle \emph{fait froid}.
Thus, the auxiliary \emph{avoir} (like tense auxiliary \emph{\^etre} `to be') is, in fact, a generalized raising verb: its whole argument structure is identified with that of the participle. A simplified description of subject raising verbs and tense auxiliaries is given in (\ref{GSexemple8}) (for the feature [\light$\pm$], see Section~\ref{GSsection3}).

\begin{exe}
	\ex 	\label{GSexemple8}
	\begin{xlist}
        \ex{Subject raising verb:\\
     	\begin{avm}
		\[{} arg-st \ibox{1} $\oplus$ \< \,\[{}subj \ibox{1} \]\, \> \,$\oplus$ \ibox{2} \] 
	\end{avm}\label{GSexemple8a}
	}
	
	    \ex{Tense auxiliary:\\
	\begin{avm}
		\[arg-st \ibox{1} $\oplus$ \< \,\[arg-st \ibox{1} $\oplus$ \ibox{2}\\ {}light +\\ \]\, \> \,$\oplus$ \ibox{2} \] 
	\end{avm}\label{GSexemple8b}
	}
	\end{xlist}
\end{exe}

The subject raising verb takes a complement saturated complement, which is described as the second
element of the argument structure, expecting a subject \ibox{1} identified with the subject of the
raising verb. The notation \ibox{1} without $\left< \, \right>$ indicates that this element may be
absent: it is meant to accommodate subjectless verbs. In addition, the raising verb may have its own
complements, noted here as \ibox{2}. On the other hand, the auxiliary is not only a subject raising
verb, but takes as a complement a participle which has not combined with any complements and  only has attracted complements.

The arguments of a word are made up of subject and complements. The relation between (expected) arguments and realized subject and complements is as in (\ref{GSexemple9}) (see \citealt[171]{GSag2000a-u}; \citealt{BMS2001a}). The arguments include the subject, the complements and the specifier, but also a list of non-canonical elements (possibly empty; see below).

\begin{exe}
        \ex[]{Argument Realization Principle\\	
        \begin{avm}
		{{\rm \emph{word}} \impl \[ synsem \| loc
		            \[cat \[subj & \ibox{1}\\ comps & \ibox{2}\\ spr & \ibox{3}\\ \]\\
		            arg-st \ibox{1} $\oplus$ \ibox{2} $\oplus$ \ibox{3} $\bigcirc$ {\rm list (\emph{non-canon})}\\ \]
		        \]}
          	\end{avm}\label{GSexemple9}
	}
\end{exe}

In (\ref{GSexemple10a}), the participle \emph{lu} `read' selects the argument \emph{son livre} `her book', which is attracted by the auxiliary \emph{a} `has'. Accordingly, it is realized as the complement of the auxiliary \emph{a}. The structure of the VP in (\ref{GSexemple10a}) is given in Figure~\ref{GSfigure1}.

\eal
	\label{GSexemple10}
	\ex[]{
	\gll Marie a   lu   son livre.\\
	     Marie has read her book\\
	\glt `Marie has read her book.'}\label{GSexemple10a}
		
	\ex[]{
	\gll Marie l'a    lu.\\
	     Marie it.has read\\
	\glt `Mary has read it.'}\label{GSexemple10b}
\zl

%%%%%%%%%%%%%%%%%%%%%%%%%%%%%%%%%%%%%%%%%%%%%%%%%%%%%%%%%%%%%%%%%

\begin{figure}
    {\centering
\begin{forest}
 [VP
 [V [\ms{
            head & \ms{\normalfont{\emph{basic-verb}}\\
                        vform \normalfont{\emph{indic.}}}\\
            subj & \liste{ \ibox{1} } \\
            comps & \liste{ \ibox{3}, \ibox{2} }\\
            arg-st & \liste{ \ibox{1}, \ibox{3}, \ibox{2} }
            }[a\\has, align=center, base=bottom]]] 
 [\ibox{3} V [\ms{
            head & \ms{\normalfont{\emph{basic-verb}}\\
                        vform \normalfont{\emph{pst-ptcp}}}\\
            subj & \liste{ \ibox{1} }\\
            comps & \liste{ \ibox{2} }\\
            arg-st & \liste{ \ibox{1}, \ibox{2} }
            }[lu\\read, align=center, base=bottom, tier=word]]]
 [\ibox{2} NP, before computing xy={s'+=30pt} 
            [son livre\\her book, base=bottom, tier=word, roof]]]
\end{forest}} \caption{VP structure in French}
    \label{GSfigure1}
\end{figure}

%%%%%%%%%%%%%%%%%%%%%%%%%%%%%%%%%%%%%%%%%%%%%%%%%%%%%%%%%%%%%%%%%
\begin{figure}
\begin{forest}
type hierarchy
 [synsem
 [non-canon
    [aff]
    [gap,calign with current]
    [null-pro]]
 [canon]]
\end{forest}
\caption{Subtypes of \type{synsem}}\label{GSexemple11}
\end{figure}

Let us turn to pronominal clitics. The arguments are of type \emph{synsem}, which can have different subtypes (Figure \ref{GSexemple11}). Usually, these subtypes are not specified on lexemes, but they are on
words occurring in sentences.

Romance clitics, illustrated by \emph{l(e)} in (\ref{GSexemple10b}), are analyzed as affixes (\emph{aff}) on verbs, which correspond to arguments of the verb \citep{MS97a-u}. They belong to the argument structure of the participle, and are attracted by the auxiliary, although they are not realized as complements. In (\ref{GSexemple10b}) and Figure~\ref{GSfigure2}, the arguments of the auxiliary are the subject \ibox{1}, the participle \ibox{3}, and \ibox{2}; \ibox{2} is typed as an affix, third person, masculine singular. It belongs to the argument structure, but not to the complement list of the auxiliary (see (\ref{GSexemple9})).


We distinguish between \emph{basic verbs} and \emph{reduced verbs}, following \cite{AGS1998}. With basic verbs, the argument list is simply the concatenation of the subject and complements, while reduced verbs have at least one affix argument which belongs to the argument list, but not to the complement list. Such verbs are subject to a morphological rule which realizes this affixal argument as an affix, the so-called clitic pronoun \emph{l(e)}. Thus, in Figure~\ref{GSfigure1}, both the auxiliary \emph{a} `has' and the participle \emph{lu} `read' are basic verbs: the arguments tagged \ibox{3} and \ibox{2} are also complements. On the other hand, in Figure~\ref{GSfigure2}, the participle is a basic verb -- argument \ibox{2} is typed as an affix, but is also a complement -- while the auxiliary is a reduced verb: argument \ibox{2} is not a complement of the auxiliary, and the verb hosts the affix \emph{l(e)}.

In French, past participles never host clitics, as we saw in (\ref{GSexemple1c}), which we assume to be a morphological property. But in Italian, past participles may host clitics, although never when they combine
with the auxiliary. The specification that the participle complement of the auxiliary is a basic verb accounts for this property, because basic verbs are not the target of the morphological rule realizing the affixal argument as an affix. Although both verbs in Figure~\ref{GSfigure2} have an affixal argument, one is a basic verb (the participle), the affixal argument being also an expected complement, and the other is a reduced verb (the auxiliary), this affixal argument not being an expected complement.\footnote{It is worth noting that tense auxiliaries can take as complement a coordination of participles:
	
\begin{exe}
	\label{FootExample1}
	\ex[]{
	\gll Jean a   acheté et  lu   ce   livre.\\
	     Jean has bought and read this book\\
	\glt `Jean bought and read this book.'}\label{FootExample1a}
		
	\ex[]{
	\gll Jean l'a    acheté et  lu. \\
	     Jean it.has bought and read\\
	\glt `Jean bought and read it'}\label{FootExample1b}
\end{exe}    

This may be seen as raising a difficulty for the analysis of their complement based on argument structure sharing, since argument structure characterizes words rather than phrases. However, coordinations of words are a special kind of phrases, since the conjuncts must share their argument structure. It is plausible that such coordinations inherit an argument structure from the conjuncts (for further discussion of coordination, see \crossrefchapteralt{coordination}).}

%%%%%%%%%%%%%%%%%%%%%%%%%%%%%%%%%%%%%%%%%%%%%%%%%%%%%%%%%%%%%%%%%

\begin{figure}
    {\centering
\begin{forest}
 [VP
 [V [\ms{
            head & reduced-verb\\
            subj & \liste{ \ibox{1} }\\
            comps & \liste{ \ibox{3} }\\
            arg-st & \liste{ \ibox{1}, \ibox{3}, \ibox{2} }
            }[l'a\\it.has, align=center, base=bottom]]] 
 [\ibox{3} V [\ms{
            head & basic-verb\\
            subj & \liste{ \ibox{1} }\\
            comps & \liste{ \ibox{2} }\\
            arg-st & \liste{ \ibox{1}, \ibox{2} \normalfont{[\emph{aff, 3\textsuperscript{rd}, msg}]} } }
            [lu\\read, align=center, base=bottom, tier=word]]] ]
\end{forest}}\caption{Clitic climbing in French}
    \label{GSfigure2}
\end{figure}

%%%%%%%%%%%%%%%%%%%%%%%%%%%%%%%%%%%%%%%%%%%%%%%%%%%%%%%%%%%%%%%%%

\section{Different structures for complex predicates: Restructuring verbs and the copula in Romance languages}\label{GSsection3}

In addition to tense auxiliaries, Romance languages have other cases of complex predicates that are headed by restructuring verbs, by the copula and other verbs taking predicative complements, and by certain causative and perception verbs. We focus here on restructuring
verbs and the copula. An analysis of causative and perception verbs is proposed in \citet{abeille1995doublestructure, AGMS98a, AG2010}. 

A comparison of the properties of constructions headed by restructuring verbs in different Romance languages illustrates an important aspect of the phenome\-non: argument attraction is compatible with different syntactic structures. Restructuring verbs enter either a flat structure or a verbal complex \citep{Monachesi98a, abeille2001deux, AG2010}. As for the copula, it differs from tense auxiliaries and restructuring verbs in two respects: its complement always behaves like a phrase, although it can be fully saturated for its complements, partially saturated or not saturated at all \citep{abeille2001varieties, AG2002b-u}; and it has a uniform behavior and analysis across the Romance languages.

\subsection{Romance restructuring verbs as head of complex predicates} \label{GSsection3.1}

Certain verbs in Romance languages, called \emph{restructuring verbs}, exhibit two behaviors: either as ordinary verbs taking a VP complement or as heads of complex predicates \citep{rizzi1982issues, aissen1983clause}. Restructuring verbs are modal, aspectual or movement verbs (such as \emph{venire} `to come’, \emph{andare} `to go’, \emph{correre} `to run’, \emph{tornare} `to come back’ in Italian). However, it must be kept in mind that this behavior is lexical: verbs which are close semantically may not be heads of complex predicates. 

Several properties show that such verbs can head complex predicates \citep[323-328]{Monachesi98a}. The first is clitic climbing, which is possible with restructuring verbs, though optional (while it is obligatory with tense auxiliaries). The examples in (\ref{GSexemple12}) all mean `John wants to eat them’ (examples from \citealt[113]{AG2010}). For each language, the first example illustrates the complex predicate, and the second one the VP complement construction, with the clitic downstairs.

\eal 
\settowidth\jamwidth{(Portuguese)}
\label{GSexemple12} 
\ex{
\gll Giovanni \emph{le} vuole mangiare.\\
     Giovanni them      wants eat\\\jambox{(\ili{Italian})}
\glt `Giovanni wants to eat them.'} \label{GSexemple 12a}  
		
\ex{
\gll Giovanni vuole mangiar\emph{le}.\\
     Giovanni wants eat.them\\ 
\glt `Giovanni wants to eat them.'} \label{GSexemple12b} 

\ex{
\gll Juan \emph{las} quiere comer.\\
     Juan them       wants  eat\\\jambox{(\ili{Spanish})}
\glt `Juan wants to eat them.'} \label{GSexemple12c} 
		
\ex{
\gll Juan quiere comer\emph{las}.	\\
     Juan wants  eat.them \\ 
\glt `Juan wants to eat them.'}\label{GSexemple12d} 
		
\ex{
\gll O            Jo\~ao quere-\emph{as} comer.	\\
     \textsc{det} Jo\~ao wants-them      eat \\\jambox*{(\ili{Portuguese})} 
\glt `Jo\~ao wants to eat them.'}\label{GSexemple12e} 
	
\ex{
\gll O            Jo\~ao quer  comê-\emph{las}.	\\
     \textsc{det} Jo\~ao wants eat-them \\
\glt `Jo\~ao wants to eat them.'}\label{GSexemple12f}
		
\ex{
\gll En           Joan \emph{les} vol   menjar.	\\
     \textsc{det} Joan them       wants eat \\\jambox*{(\ili{Catalan})}
\glt `Joan wants to eat them.'}\label{GSexemple12g} 
	
\ex{
\gll En           Joan vol   menjar-\emph{les}.\\
     \textsc{det} Joan wants eat-them\\
\glt `Joan wants to eat them.'}\label{GSexemple12h} 
\zl

The second property showing restructuring verbs' complex predicate status is the medio-passive or middle \emph{si} construction, where the verb hosts the reflexive clitic \emph{si} or \emph{se} (\ref{GSexemple13b}) (depending on the language), and the subject corresponds to the object of the active construction (\ref{GSexemple13a}), with an interpretation close to that of middles in English. The construction is possible with restructuring verbs such as \emph{potere} `to be able to' (\ref{GSexemple13c}) and (\ref{GSexemple13d}) (see \citealt[333-336]{Monachesi98a}), but not with verbs only taking an infinitival VP complement such as \emph{parere} `to appear' (\ref{GSexemple13e}) (examples (\ref{GSexemple13d}) and (\ref{GSexemple13e}) from \citealt[122]{AG2010}).


\eal
 	\label{GSexemple13} 
    \ex[]{
	\gll Giovanni stira queste camicie facilmente.\\
	     Giovanni irons these  shirts  easily\\\jambox*{(\ili{Italian})}
	\glt `Giovanni irons these shirts easily.'}\label{GSexemple13a} 
		
	\ex[]{ 
	\gll Queste camicie si          stirano facilmente.\\ 
 	     these  shirts  \textsc{si} iron    easily\\
	\glt `These shirts iron easily.'}\label{GSexemple13b}
		
	\ex[]{
	\gll Giovanni pu\`o stirare queste camicie facilmente.\\
	     Giovanni can   iron    these  shirts  easily\\
	\glt `Giovanni can iron these shirts easily.'} \label{GSexemple13c} 
		
	\ex[]{ 
	\gll Queste camicie si          possono stirare facilmente.\\ 
	     these  shirts  \textsc{si} can     iron    easily\\
	\glt `These shirts can be ironed easily.'}\label{GSexemple13d}		
		
	\ex[*]{ 
	\gll Queste camicie si          paiono stirare facilmente.\\ 
	     these  shirts  \textsc{si} appear iron    easily\\
	\glt Intended: `These shirts appear to be ironed easily.'}\label{GSexemple13e}			
\zl


The medio-passive verb alternates with a transitive verb: it is the result of a Lexical Rule, shown in (\ref{GSexemple14}), which takes a transitive verb like \emph{stirare} as in (\ref{GSexemple13a}) to give a verb whose subject corresponds to the expected object of the transitive verb and which acquires a reflexive clitic noted as \emph{a-aff} (realized \emph{si} or \emph{se}) as in (\ref{GSexemple13b}) (\citealt[31]{AGS1998}; \citealt{Monachesi98a}). 

\begin{exe} 
        \ex{Medio-Passive Lexical Rule \label{GSexemple14}\\	
        \begin{avm}
		\[{}arg-st \<np, np [\normalfont{\emph{acc}}]$_j$ \> \, $\oplus$ \ibox{1}\] \, $\mapsto$ \[{}arg-st \<np$_j$, [\normalfont{\emph{a-aff, acc}}]$_j$  \>\, $\oplus$ \ibox{1}\]
	\end{avm}}
\end{exe}

What is crucial here is that the input is a verb taking an accusative NP complement. Hence, a verb taking a VP complement like Italian \emph{potere} `to be able to' or \emph{parere} `to appear' cannot be the input, since it lacks an NP complement. On the other hand, the corresponding restructuring verb \emph{potere} can be the input, since it inherits such a complement from the infinitive: the verb \emph{potere} in (\ref{GSexemple13c}) inherits \emph{queste camicie} `these shirts' from \emph{stirare} `to iron', allowing it to be the input to Rule (\ref{GSexemple14}), which gives the verb occurring in (\ref{GSexemple13d}). 
%On the other hand, the verb \emph{parere} which is not a restructuring verb does not have an NP object and cannot be the input to this Rule (\ref{GSexemple14}).

The third relevant property of restructuring verbs is their acceptability in bounded dependencies, as illustrated in (\ref{GSexemple7}) for tense auxiliaries and (\ref{GSexemple15}) for restructuring verbs. (\ref{GSexemple15b}) (from \citealt[341]{Monachesi98a}) relies on \emph{cominciare} `to begin' being a restructuring verb, while \emph{promettere} `to promise' is not (\ref{GSexemple15c}).  

\eal 
	\label{GSexemple15} 
    \ex[]{
	\gll Questa canzone \`e facile da apprendere.\\
		 this   song    is  easy   to learn\\\jambox*{(\ili{Italian})}
	\glt `This song is easy to learn.'}\label{GSexemple15a} 
		
	\ex[]{ 
	\gll Questa canzone \`e facile da cominciare a  apprendere.\\
		 this   song    is  easy   to begin      to learn\\
	\glt `This song is easy to begin to learn.'}\label{GSexemple15b}
		
	\ex[*]{
	\gll Questa canzone \`e facile da promettere di apprendere.\\
	     this   song    is  easy   to promise    to learn\\
	\glt Intended: `This song is easy to promise to learn.'}\label{GSexemple15c} 	
\zl

The complement of adjectives such as `easy' in Romance languages is a bound\-ed dependency: they take an infinitival complement whose own expected complement (we analyze it as a null pronoun; see Figure~\ref{GSexemple11}) is coindexed with its subject (\citealt{AGS1998, Monachesi98a}).\footnote{Forms such as \emph{a}, \emph{da} and \emph{di}, which introduce infinitival complements in (\ref{GSexemple15}), are not analyzed as heads, but as markers, a part of speech which has the feature \textsc{marking} and whose value is specific to the form. Markers select the head with which they combine (for instance, \emph{da} selects an infinitival VP in (\ref{GSexemple15a})), and the feature is shared by the whole VP. Hence, the adjective \emph{facile} `easy' in Italian takes as a complement an infinitival VP [\textsc{marking} \emph{da}].}

\ea
\avmtmp{
[head   & adjective\\
 arg-st & < xp$_j$, vp [vform & infinitive\\ 
                        marking & da\\
                        comps   & < \type{null-pro} [\type{acc} ]$_j$ >  $\oplus$ \2 \\ ] > ]
}
\z

Complex predicates can occur in this construction because their head attracts the complement of their complement. Thus, in (\ref{GSexemple15b}), \emph{cominciare} `to begin' is expecting the same object as \emph{apprendere} `to learn', which is coindexed with the subject of the copular construction, in the same way as \emph{apprendere} is expecting an object in (\ref{GSexemple15a}). 

Fourth and finally, the possibility of preposing the verbal complement of a verb which can take a VP complement or be the head of a complex predicate disappears when there is evidence of a complex predicate. For the sake of simplification, we now concentrate on Italian and Spanish. The data in (\ref{GSexemple17}), with a preposed VP, contrast with those in (\ref{GSexemple18}) (both examples from \citealt[132]{AG2010}), where the head verb bears a clitic corresponding to the expected complement of the infinitive. Preposing of the verbal complement is associated with pronominalization (\emph{lo}) in Italian (\ref{GSexemple17a}) but not in Spanish (\ref{GSexemple17b}), where it is more natural in contrastive contexts.

\begin{exe}
	\ex {[Context] Does he want to talk to Mary?}\label{GSexemple17} 
	\begin{xlist}
	\ex[]{
	\gll Parlare a  Maria, certamente lo vuole.\\
	     talk    to Maria  certainly  it wants\\\jambox*{(Italian)}
	\glt `Talk to Maria, certainly he wants to.'}\label{GSexemple17a}
		 
	\ex[]{ 
	\gll Hablarle 	 a  Mar\'ia, seguramente quiere (pero          no  a  su  madre).\\
	     talk.to.her to Mar\'ia  certainly   wants  \spacebr{}but  not to her mother\\\jambox*{(\ili{Spanish})}
	\glt `Talk to Maria, certainly he wants to (but not to her mother).'}\label{GSexemple17b}
	\end{xlist}
\end{exe}

\eal
	\label{GSexemple18} 
    \ex[*]{
	\gll Parlare, certamente glielo    vuole.\\
	     talk     certainly  to.him/it wants\\\jambox*{(\ili{Italian})}
	\glt Intended: `Talk to him, he certainly wants to.'}\label{GSexemple18a} 	
	
	\ex[*]{ 
	\gll Hablar, le         quiere (pero           no  mucho      tiempo).\\
	     talk    to.him/her wants  \spacebr{}but   not a.long time\\\jambox*{(\ili{Spanish})}
	\glt Intended: `Talk to him/her he wants to (but not for a long time).'}\label{GSexemple18b}
\zl

We assume that restructuring verbs have two possible descriptions: as ordinary verbs taking an infinitival VP complement, or as heads of complex predicates. They are related by the Argument Attraction Lexical Rules given in (\ref{GSexemple19}) (adapted from \citealt[331]{Monachesi98a}).\footnote{We leave aside the object control and object raising verbs (verbs of influence or perception verbs) which can also be the head of a complex predicate, and hence be the target of a similar Lexical Rule \citep{AGMS98a, AG2010}.}  

\begin{exe}
	\ex {Argument attraction lexical rules for Romance restructuring verbs}\label{GSexemple19} 
	\begin{xlist}
        \ex{
          \label{GSexemple19a}Subject control verbs \\
        \begin{avm}
		{\[head \normalfont{\emph{verb}}\\
        arg-st \<xp\normalfont{\textsubscript{\emph{i}}}, \ibox{2} 
                    \[head \[\normalfont{\emph{verb}}\\
                    vform \normalfont{\emph{inf.}}\\\]\\ 
                    subj \<xp\normalfont{\textsubscript{\emph{i}}}\>\\
                    comps \< \>\]\,\>\, $\oplus$ \ibox{3}\\\]}
          	\end{avm}\vspace{.2cm}          	$\mapsto$ \\
        \begin{avm}
		{\[arg-st \<xp\normalfont{\textsubscript{\emph{i}}}, v
                    \[\normalfont{\emph{basic-verb}}\\
                    light $+$ \\
                    comps \ibox{4}\\\]\,\> \,$\oplus$ \ibox{4} $\oplus$ \ibox{3}\]}
          	\end{avm}}
	
	    \ex{\label{GSexemple19b}
        Subject raising verbs\\
        \begin{avm}
		{\[head \normalfont{\emph{verb}}\\
        arg-st \ibox{1} $\oplus$ \<\ibox{2}
                    \[head \[\normalfont{\emph{verb}}\\
                    vform \normalfont{\emph{inf.}}\\\]\\ 
                    subj \ibox{1}\\
                    comps \< \>\]\,\>\, $\oplus$ \ibox{3}\\\]}
          	\end{avm}\vspace{.2cm} $\mapsto$ \\
        \begin{avm}
		{\[arg-st \ibox{1} $\oplus$ \<v
                    \[\normalfont{\emph{basic-verb}}\\
                    light $+$ \\
                    comps \ibox{4}\\\]\,\> \,$\oplus$ \ibox{4} $\oplus$ \ibox{3}\]}
          	\end{avm}}
	\end{xlist}
\end{exe}

In the input description, the verbal complement is saturated for its complements. The verb may have other complements in addition to the saturated infinitival VP, noted as list \ibox{3}. We distinguish between subject control verbs and subject raising verbs to accommodate the case where the complement verb is subjectless, but with complements that can be attracted. In (\ref{GSexemple20a}), the verb \emph{sembra} `seems' is a raising verb, and the infinitive \emph{piacere} `to please' is an impersonal verb with no subject, but with a complement, realized by \emph{gli} on the head verb \emph{sembra} (there is another interpretation where \emph{gli} is the complement of \emph{sembra}, which is irrelevant).\footnote{Alternatively, in a grammar with null pronouns, impersonal and unaccusative verbs in Romance languages could be analyzed as having a null pronoun subject, a representation which allows a common input for subject control and raising verbs in the Argument Attraction Lexical Rule (as in \citealt[331]{Monachesi98a}).} Note that there is inter-speaker variation: \emph{sembrare} `to seem' is not a restructuring verb for all Italian speakers (hence \% on the examples).

The category of the subject of control verbs is not specified: it can be an infinitival VP as well as an NP (or even a sentence); in the first case, the index is that of the situation (\ref{GSexemple20c}), in the second, it is the index of the nominal entity (\ref{GSexemple20b}). Again, the upstairs clitic \emph{gli} corresponds to the argument of \emph{piacere} `to please':

\eal
\judgewidth{\%}
	\label{GSexemple20}
    \ex[\%]{
	\gll Gli    sembra piacere molto.\\ 
	  	 to.him seems  please  a.lot\\\jambox*{(\ili{Italian})}
	\glt `It seems that he likes it a lot.'}\label{GSexemple20a}

	\ex[\%]{ 
	\gll [Questo           regalo] gli    sembra piacere.\\ 
		 \spacebr{}this    gift    to.him seems  please\\
	\glt `This gift seems to please him.'}\label{GSexemple20b}		
		
	\ex[\%]{ 
	\gll [Andare           in vacanza]  gli    sembra piacere\\
		 \spacebr{}go.away on vacation  to.him seems  please\\
		\glt `To go away on vacation seems to please him.'}\label{GSexemple20c}	
\zl

\subsection{The different structures of complex predicates with restructuring verbs} \label{GSsection3.2}

The point of this section is to show that argument attraction is compatible with different structures: complex predicate formation and structure are two different aspects of the grammar. In Romance languages, restructuring verbs can take a VP complement, or be the head of a complex predicate. In the latter case, there are two possible structures: the restructuring verbs enter either a flat structure or a verbal complex. We speak of a flat structure when the complement verb as well as the complements that it subcategorizes for are all sisters of the head. We speak of a verbal complex when the head verb and the complement verb form a constituent by themselves, to the exclusion of their complements (see Figure~\ref{GSfigure3}).

We contrast Italian and Spanish.\footnote{In Portuguese, restructuring verb constructions are also a flat structure, but with different ordering constraints than Italian; the variety of Spanish not described here is similar to Portuguese. Except for the copula (see Section~\ref{GSsection3.4}), complex predicate constructions with head verbs entering only one structure also distribute between these two structures among Romance languages: tense auxiliaries in French, Italian and Portuguese, as well as Romanian modal \emph{a putea} `can', are the head of a flat structure, while tense auxiliaries in the variety of Spanish described here and in Romanian enter a verbal complex \citep{AG2010}.} Note that in Spanish, there is variation among speakers: we describe here one usage of Spanish complex predicates. 

The impossibility of preposing illustrated in (\ref{GSexemple18}) for both languages shows that the sequence of the complement verb and its complements does not form a constituent (a VP) when there is a complex predicate, a point made by \cite{rizzi1982issues} for Italian, on the basis of a series of constructions (pied-piping, clefting, Right Node Raising, Complex NP shift). However, the two languages differ with respect to other properties. In what follows, the fact that there is a complex predicate is indicated by the presence of a clitic on the head verb.
 
First, adverbs occur between the restructuring verb and the infinitive in Italian (\ref{GSexemple21a}), but not in Spanish in a general way (\ref{GSexemple21b}) (though a few adverbs, such as \emph{casi} `nearly', \emph{ya} `already' and \emph{apenas} `barely' are possible). In Spanish, an adverb may occur after the verb and before the infinitive if the complement is a VP (\ref{GSexemple21c}) (examples in (\ref{GSexemple21}) from \citealt[139]{AG2010}).

\eal
	\label{GSexemple21} 
	\ex[]{
	\gll Giovanni \emph{lo} vuole spesso leggere.\\ 
	     Giovanni it        wants often  read\\\jambox*{(\ili{Italian})}
	\glt `Giovanni wants to read it often.'}\label{GSexemple21a}

	\ex[*]{ 
	\gll Juan \emph{lo} quiere {a menudo} leer.\\ 
	     Juan it        wants  often      read\\\jambox*{(\ili{Spanish})}
	\glt Intended: `Juan wants to read it often.'}\label{GSexemple21b}		
	
	\ex[]{ 
	\gll Juan quiere {a menudo} leer\emph{lo}.\\
	     Juan wants  often      read.it\\
	\glt `Juan wants to read it often.'}\label{GSexemple21c}	
\zl

Second, an inverted subject NP can occur between the two verbs of a complex predicate in Italian (\ref{GSexemple22a}), but not in Spanish (\ref{GSexemple22b}). The subject can occur postverbally in interrogative sentences. In Italian, it can occur between the two verbs with a special prosody, indicated by the small capitals in (\ref{GSexemple22a}), and with inter-speaker variation \citep{salvi1980ausiliari}. In Spanish, this is not possible (except for the pronominal subject; \citealt{suner1982syntax}).

\eal
\judgewidth{\%}
	\label{GSexemple22} 
	\ex[\%]{
	\gll Lo comincia \textsc{Maria} a  capire,     il problema, oppure no?\\ 
		 it begins   Maria          to understand  the problem  or     no\\\jambox*{(\ili{Italian})}
	\glt `Maria, she's beginning to understand it, the problem, yes or no?'}\label{GSexemple22a}

	\ex[*]{ 
	\gll ?`Lo         comienza Juan a  comprender?\\
		 \spacebr{}it begins   Juan to understand\\\jambox*{(\ili{Spanish})}
	\glt `Is Juan beginning to understand it?'}\label{GSexemple22b}		
	
	\ex[]{ 
	\gll ?`Comienza       Juan a  comprenderlo?\\
		 \spacebr{}begins Juan to understand.it\\
	\glt `Is Juan beginning to understand it?'}\label{GSexemple22c}	
\zl

Finally, Italian heads of complex predicates can have scope over the coordination of infinitives with their complements (\ref{GSexemple23a}), while this is not the case in Spanish (\ref{GSexemple23b}). Again, the presence of a clitic on the head verb (\emph{lo vuole} lit.\ it wants, \emph{le volvi\'o} lit.\ to.him started.again) shows that this is a complex predicate construction (examples from \citealt[136--137]{AG2010}).

\eal
\judgewidth{\%}
	\label{GSexemple23} 
	\ex[\%]{
	\gll Giovanni lo vuole comprare subito      e   dare a  Maria.\\ 
         Giovanni it wants   buy      immediately and give to Maria\\\jambox*{(\ili{Italian})}
	\glt `Giovanni wants to buy it immediately and give it to Maria.'}\label{GSexemple23a}

	\ex[*]{ 
	\gll Le         volvi\'o      a  pedir un aut\'ografo y   a  hacer proposiciones.\\
         to.him/her started.again to ask   an autograph   and to make  proposals\\\jambox*{(\ili{Spanish})}
	\glt Intended: `He started again to ask him for an autograph and to make proposals to him/her.'}\label{GSexemple23b}	
\zl

Constituency tests such as preposing, as in (\ref{GSexemple18}), show that the verbal complement is not a VP in either language. The verbal complex, in which the two verbs form a constituent without the complements, is well-suited to account for the absence of adverbs (in a general way) and of subject NPs, if such combinations exclude elements other than verbs (adverbs in particular). This constraint can be captured by the feature [\light $+$], which has been used in Romance languages for other phenomena as well (\citealt{abeille2000french}; see Section~\ref{GSsection3.3}).%
    \footnote{The adverbs admissible in the Spanish verbal complex are light.}
Hence, complex predicate constructions in Spanish contain a verbal complex, while they form a flat structure in Italian containing the complement verb and its complements. 

This is illustrated with examples in Figure~\ref{GSfigure3}, which all mean `Marco wants to give it to Maria'. The verb takes a VP complement in Figure~\ref{GSfigure3a} in both languages, it is the head of a flat VP in Italian in Figure~\ref{GSfigure3b}, and it enters a verbal V-V complex in Spanish in Figure~\ref{GSfigure3c} (from \citealt[146]{AG2010}).

\begin{figure}
\begin{subfigure}{.495\textwidth}
\begin{forest} 
for tree={%
    l sep=10pt}
[S
   [SN
      [Marco\\Marco\\Marco\\Marco, align=center, base=bottom, tier=word]]
   [VP
      [V[vuole\\wants\\quiere\\wants, align=center, base=bottom, tier=word]]
      [N[lo-dare a Maria\\it-give to Maria\\darlo a María\\give.it to María, align=center, base=bottom, tier=word, roof]]
]]
\end{forest}
\caption{VP complement}
\label{GSfigure3a}
\end{subfigure}
\hfill
\begin{subfigure}{.495\textwidth}
\begin{forest} 
for tree={%
    l sep=19pt}
[S
   [NP
      [Marco\\Marco, align=center, base=bottom, tier=word]]
   [VP
      [V[lo-vuole\\it-wants, align=center, base=bottom, tier=word]]
      [V[dare\\give, align=center, base=bottom, tier=word]]
      [PP[a Maria\\to Maria, align=center, base=bottom, tier=word, roof]]
      ]]
\end{forest}
\caption{Flat structure}
\label{GSfigure3b}
\end{subfigure}
\\
\vspace{20pt}

\begin{subfigure}{.5\textwidth}
\centering
\begin{forest} 
sm edges
[S
   [NP
      [Marco;Marco]]
   [VP
      [V[V [lo-quiere;it-wants]] [V[dar;give]]]
      [PP[a María;to María, roof]]
]]]
\end{forest}
\caption{Verbal complex}
\label{GSfigure3c}
\end{subfigure}
\caption{Three complementations for Romance restructuring verbs}
\label{GSfigure3}
\end{figure}

The possibility of the coordination in (\ref{GSexemple23a}) has been viewed as an argument in favor
of a complement VP, even when there is argument attraction \citep{andrews1999complex}. The data go
against such an analysis for Spanish, since the coordination is not acceptable. For Italian,
although such sequences as (\ref{GSexemple23a}) can be analyzed as coordinations of VP, they can
also be Non-Constituent Coordinations (NCCs; an English example would be \emph{John gives a book to Maria and discs to her brother}; see \crossrefchapteralt[Section 7]{coordination}). 
So, the question becomes: why is (\ref{GSexemple23b}) not an acceptable NCC in Spanish?
\cite{AG2010} propose that NCCs are subject to a general constraint in Romance
languages: the parallel elements of the coordination must be at the same syntactic level, otherwise
the acceptability is degraded. An example is the contrast between (\ref{GSexemple24a}) and
(\ref{GSexemple24b}) in Spanish. The structure of (\ref{GSexemple23b}), repeated in
(\ref{GSexemple24c}), is similar to that of (\ref{GSexemple24b}), if it is a verbal complex ((\ref{GSexemple24}) from \citealt[137, 144]{AG2010}).

\eal
	\judgewidth{??}
	\label{GSexemple24} 
	\ex[]{
	\gll Juan da    [el           libro de Proust] [a           Mar\'ia] y   [el          (libro)            de Camus] [a           Pablo].\\
		 Juan gives \spacebr{}the book  of Proust  \spacebr{}to Mar\'ia  and \spacebr{}the \spacebr{}book    of Camus  \spacebr{}to Pablo\\\jambox*{(\ili{Spanish})}
	\glt `Juan gives the book by Proust to Mar\'ia and the book by Camus to Pablo.'}\label{GSexemple24a}
	
	\ex[??]{ 
	\gll Juan da    [el           libro de Proust] [a           Mar\'ia] y   [de          Camus] [a Pablo].\\
		 Juan gives \spacebr{}the book  of Proust  \spacebr{}to Mar\'ia  and \spacebr{}of Camus  \spacebr{}to Pablo\\
	\glt Intended: `Juan gives the book  by Proust to Mar\'ia and the book by Camus to Pablo.'}\label{GSexemple24b}
	
	\ex[*]{ 
	\gll [Le                  volvi\'o      a  pedir] [un          aut\'ografo] y   [a           hacer]         [proposiciones].\\
    	 \spacebr{}to.him/her started.again to ask    \spacebr{}an autograph    and \spacebr{}to make \spacebr{}proposals\\
	\glt Intended: `He started again to ask him an autograph and to make proposals to him/her.'}\label{GSexemple24c}
\zl

In (\ref{GSexemple24a}), the NP \emph{el de Camus} `the one by Camus' is parallel to and at the same level as \emph{el libro de Proust} `the book by Proust', the PP \emph{a Pablo} `to Pablo' is parallel to and at the same level as \emph{a Mar\'ia} `to Mar\'ia', and the NP and the PP are both complements of \emph{da} `gives'. But, in (\ref{GSexemple24b}), \emph{de Camus} `by Camus' is parallel to \emph{de Proust} `by Proust', and not at the same level as \emph{el libro de Proust} or as \emph{a Pablo}: \emph{a Pablo} corresponds to the complement of \emph{da} `gives' while \emph{de Camus} corresponds to the complement of the noun \emph{libro} `book'. Thus, the acceptability is degraded. 

If the structure of a complex predicate is that of a verbal complex in Spanish, the structure of (\ref{GSexemple24c}) is similar to that of (\ref{GSexemple24b}): \emph{a hacer} corresponds to a \emph{a pedir}, which is the complement V of \emph{volvi\'o} in a V-V constituent, and is not at the same level as \emph{proposiciones}, which corresponds to \emph{un aut\'ografo}, which is outside the V-V constituent.   

\subsection{Analysis of Romance restructuring verb constructions in HPSG} \label{GSsection3.3}
\label{sec-romance-complex-predicates}

It has been shown in Section~\ref{GSsection3.1} that the different Romance languages all have
complex predicate constructions, and, in Section~\ref{GSsection3.2}, that, although they share some
properties (such as clitic climbing and occurrence in other bounded dependencies), they also show syntactic differences amongst themselves (separability of the head and the infinitive or participle in Italian, but not in Spanish, and the possibility of coordination of the complement verb with its complements in Italian, but not in Spanish). The flexibility of HPSG grammars allows us to describe both the commonalities and the differences. The common behavior follows from the fact that they share the mechanism of argument attraction, which characterizes certain classes of verbs; the differences follow from a different phrase structure: the restructuring verb enters a flat structure in Italian (Figure~\ref{GSfigure3b}), while it enters a verbal complex in Spanish (Figure~\ref{GSfigure3c}). This analysis contrasts with that of \cite{andrews1999complex} in LFG, who propose that complex predicates in Romance languages arise when two verbs have a common domain of grammatical functions, but correspond to just one phrase structure, all these verbs taking a VP complement. It is not clear how they can account for the differences between the two languages.

Two phrase structure rules combining a head with its complements account for the distinction between the flat structure and the verbal complex: the usual head-complements phrase, and a different one, the head-cluster phrase, which is also used in German (see Section~\ref{GSsection4.1.2}). The difference between the flat structure and the verbal complex is attributed to the feature [\light $\pm$]. 

The \type{head-complements-phrase} is defined as follows:

\ea
\label{GSexemple25}
\type{head-complements-phrase} (Romance languages) \impl \\	
\avmtmp{
[synsem|loc|cat [head  & \1 \\
		 comps & \3\\
		 light & $-$]\\
 head-dtr|synsem|loc|cat [head  & \1 \\
		          comps & \2 \shuffle\ \3\\
		          light & $+$]\\
		  non-head-dtrs \texttt{synsem2sign}(\ibox{2}) \type{non-empty list}\\
		]}
\z
The \compsl is a list of \type{synsem} objects. It is converted into a list of signs by the
relational constraint \texttt{synsem2sign} (see \citealt[\page 34]{GSag2000a-u} for a similar
proposal using \texttt{synsem2sign}).

%%%%%%%%%%%%%%%%%%%%%%%%%%%%%%%MAS LARG%%%%%%%%%%%%%%%%%%%%%%%%%%%%%%%%%%

\begin{figure}
\centerfit{%
\begin{forest}
sm edges
 [S
 [\ibox{1} NP$_j$
            [Marco;Marco]]
  [VP \ms{
            comps & \liste{  }\\
            light & $-$}, before computing xy={s'-=10pt}
    [V \ms{
            arg-st & \liste{ \ibox{1}, \ibox{2}, \ibox{3}, \ibox{4} }\\
            comps & \liste{ \ibox{2}, \ibox{3}, \ibox{4} }\\
            light & $+$
            }, before computing xy={s'+=7pt}[vuole;wants]]
    [\ibox{2} V \ms{
            vform & inf. \\
            arg-st & \liste{ \normalfont{\textsc{np}}$_j$, \ibox{3}, \ibox{4}}\\
            comps & \liste{ \ibox{3}, \ibox{4} }\\
            light & $+$}[dare;give]]
     [\ibox{3} NP
            [questo libro;this book, roof]]
     [\ibox{4} PP
            [a Maria;to Maria, roof]]]]
\end{forest}
}
\caption{Flat VP structure with an Italian restructuring verb}
    \label{GSfigure4}
\end{figure}

%%%%%%%%%%%%%%%%%%%%%%%%%%%%%%%MAS LARG%%%%%%%%%%%%%%%%%%%%%%%%%%%%%%%%%%



The \emph{head-complements-phrase} is usually saturated for the expected complements, but not always: list \ibox{3} is usually empty, but does not have to be (see the case of the copula in Section~\ref{GSsection3.4}). An example of the flat structure with a restructuring verb is given in Figure~\ref{GSfigure4}.

In the flat structure, the head verb takes as complements the infinitival verb and the canonical complements expected by the infinitive, and combines with them. The VP, corresponding to the \emph{head-complements-phrase}, is complement saturated.

The verbal complex corresponds to another kind of \emph{head-complements-phrase}, called the \emph{head-cluster-phrase}, given in (\ref{GSexemple26}) (see \citealt[6]{Mueller2002b}; \citealt[39]{muller2018clause}).\footnote{This rule is also used in Romanian. As in German, we do not specify the category of the complement (which can be a noun in Spanish, for instance).} 

\begin{exe}
        \ex[]{\type{head-cluster-phrase}  (Spanish) \impl\\	
        \begin{avm}
		{\[synsem | loc | cat &
		            \[comps & \ibox{1}\\
		            light & $+$\\\]\\
		  head-dtr | synsem | loc | cat &
		            \[head & verb\\
		            comps & \ibox{1} $\oplus$ \<\ibox{2}\>\\
		            light & $+$\\\]\\
		  non-head-dtrs & \< \, \[synsem \ibox{2} \[light $+$\]\] \,\>\\
		\]}\label{GSexemple26}
          	\end{avm}}
\end{exe}

This differs from the usual \emph{head-complements-phrase} on two accounts: here, there is only one non-head daughter, and all the constituents are [\light +]. The \light feature \citep{bonami2012phrase} renames the \textsc{weight} feature proposed in \cite{abeille2000french}, as well as the \lex feature used in German (e.g.\ \citealt{HN89b, HN94a, Kiss95a, Meurers2000b-Short, Mueller2002b, hohle2018spuren}). The \light feature has ordering as well as structural consequences \citep{abeille2000french, AG2010}. It is appropriate both for words and phrases. Words can be light or non-light; lexical verbs (finite verbs, participles or infinitives without complements) are light. Most phrases are non-light; in particular, the VP, that is, the phrase which combines with the subject in Romance languages, is non-light.\footnote{Note that the head-only phrase is non-light. Hence, the VP which dominates a lexical verb only is non-light.} But some phrases can be light if they are composed of light constituents. Such is the case for the \emph{head-cluster-phrase}. 

The \emph{head-cluster-phrase} is illustrated in Figure~\ref{GSfigure5}: the phrase \emph{quiere dar} corresponds to the \emph{head-cluster-phrase} in (\ref{GSexemple26}), while the whole VP (\emph{quiere dar aquel libro a Mar\'ia} `wants to give that book to Mar\'ia') corresponds to the usual \emph{head-complements-phrase} in (\ref{GSexemple25}).

%%%%%%%%%%%%%%%%%%%%%%%%%%%%%%%MAS LARG%%%%%%%%%%%%%%%%%%%%%%%%%%%%%%%%%%

\begin{figure}
    \centering
\begin{forest}
sm edges
 [S
 [{\ibox{1} NP$_j$}
            [Marco;Marco]]
  [VP \ms{
            comps & \liste{ }\\
            light & $-$} 
    [V \ms{
            comps & \liste{ \ibox{3}, \ibox{4} }\\
            light & $+$}, before computing xy={s'-=10pt} 
    [V \ms{
            comps & \liste{ \ibox{2}, \ibox{3}, \ibox{4} }\\
            arg-st & \liste{ \ibox{1}, \ibox{2}, \ibox{3}, \ibox{4} }\\
            light & $+$}, before computing xy={s'+=4pt} [quiere;wants]]
    [\ibox{2} V \ms{
            comps & \liste{\ibox{3}, \ibox{4} }\\
            arg-st & \liste{ NP$_j$, \ibox{3}, \ibox{4} }\\
            light & $+$}, before computing xy={s'-=4pt} [dar;give]]]
     [\ibox{3} NP
            [aquel libro;that book, roof]]
     [\ibox{4} PP
            [a María; to María, roof]]]]
\end{forest}
\caption{VP with a verbal complex with a Spanish restructuring verb}
    \label{GSfigure5}
\end{figure}

%%%%%%%%%%%%%%%%%%%%%%%%%%%%%%%MAS LARG%%%%%%%%%%%%%%%%%%%%%%%%%%%%%%%%%%

Regarding the canonical complements in the verbal complex construction, the requirement is passed up by the verbal complex, according to the description in (\ref{GSexemple26}) (the list \ibox{1} is non-empty). The verbal complex itself combines with the canonical complements expected by the infinitive (here, \ibox{3} and \ibox{4}).

More has to be said regarding the clitic \emph{lo} in Italian \emph{Marco lo-vuole dare a Maria} `wants to give it to Maria' and Spanish \emph{Marco lo-quiere dar a Mar\'ia} `wants to give it to Mar\'ia') in Figure \ref{GSfigure3}. The infinitive is a basic verb: there is no difference between the complements and the arguments (except for the subject); its complement list contains an affixal element (see Section~\ref{GSsection2}). Following Rule (\ref{GSexemple19a}), this element is attracted to the argument list of the head verb, but it is not realized as a complement, as is expected given Principle (\ref{GSexemple9}); the head verb is then a reduced verb (see Figure~\ref{GSfigure6}), which is the target of a morphological rule of cliticization, hence the clitic \emph{lo} `it' on the head verb \emph{vuole} or \emph{quiere} `wants'. 


%%%%%%%%%%%%%%%%%%%%%%%%%%%%%%%MAS LARG%%%%%%%%%%%%%%%%%%%%%%%%%%%%%%%%%%

\begin{figure}
\begin{subfigure}[b]{\textwidth}
\centering
\begin{forest}
sm edges
  [VP \ms{comps & \liste{ }} 
    [V \ms{
            \normalfont{\emph{reduced verb}}\\
            comps & \liste{ \ibox{2}, \ibox{4} }\\
            arg-st & \liste{ \normalfont{\textsc{np}\emph{\textsubscript{j}}}, \ibox{2}, \ibox{3}, \ibox{4} }\\
            light & $+$}[lo-vuole;it-wants]]
    [\ibox{2} V \ms{
            \normalfont{\emph{basic verb}}\\
            comps & \liste{ \ibox{3}, \ibox{4} }\\
            arg-st & \liste{ \normalfont{\textsc{np}\emph{\textsubscript{j}}}, \ibox{3} \emph{aff}, \ibox{4} }\\
            light & $+$}[dare;give]]
     [\ibox{4} PP
            [a Maria;to Maria, roof]]]
\end{forest}
\caption{Italian clitic climbing}
\label{GSfigure6a}
\end{subfigure}
\\
\vspace{20pt}
\begin{subfigure}[b]{\textwidth}
\centering
\begin{forest}
sm edges
  [VP \ms{comps & \liste{ }} 
  [V \ms{
            comps & \ibox{4}\\
            light & $+$} 
    [V \ms{
            \normalfont{\emph{reduced verb}}\\
            comps & \liste{ \ibox{2}, \ibox{4} }\\
            arg-st & \liste{ \normalfont{\textsc{np}\emph{\textsubscript{j}}}, \ibox{2}, \ibox{3}, \ibox{4} } \\
            light & $+$}[lo-quiere;it-wants]]
    [\ibox{2} V \ms{
            \normalfont{\emph{basic verb}}\\
            comps & \liste{ \ibox{3}, \ibox{4} }\\
            arg-st & \liste{ \normalfont{\textsc{np}\emph{\textsubscript{j}}}, \ibox{3} \emph{aff}, \ibox{4} } \\
            light & $+$}[dar;give]]] 
     [\ibox{4} PP, before computing xy={s'+=15pt}
            [a María;to María, roof]]]
\end{forest}
\caption{Spanish clitic climbing}
\label{GSfigure6b}
\end{subfigure}
\caption{Italian and Spanish clitic climbing with Italian and Spanish restructuring verbs}
\label{GSfigure6}
\end{figure}

It remains to ensure that Spanish restructuring verbs are characterized by a verbal complex, and Italian ones by a flat structure. We assume an additional constraint on phrases in Spanish. According to (\ref{GSexemple27}), if the phrase is light, it follows that the non-head daughters are also light, and, conversely, if the phrase is non-light, the non-head daughters are non-light.

% \begin{exe}
%         \ex{	
%         \begin{avm}
% 		\[\normalfont{\emph{phrase}}\\
% 		light \ibox{1}\] \, \impl \[non-head-dtrs | light \ibox{1}\]
% 	\end{avm}\jambox*{(for \ili{Spanish})}\label{GSexemple27} }
% \end{exe}
\ea	
\label{GSexemple27}
\type{phrase} \impl\\
\avmtmp{
[light \1\\
 non-head-dtrs \normalfont{\textit{list}}( [light \1] ) \\ ] % todo avm
}\jambox*{(for \ili{Spanish})}
\z
The structure of the flat VP does not obey this constraint: the infinitival verb which is a non-head daughter is light, while the other complements are non-light (see Figure~\ref{GSfigure4}). When constraint (\ref{GSexemple27}) applies, the head of a restructuring verb cannot enter a flat structure. 

In order to prevent a verbal complex in Italian (and French), we assume a different additional constraint.

\begin{exe}
        \ex[]{	
        \begin{avm}
		\[\normalfont{\emph{phrase}}\\
		light $+$\] \, \impl \[non-head-dtrs \normalfont{list} ([\textsc{head} $\neg$\type{verbal}])\]
	\end{avm}\jambox*{(for \ili{Italian})}\label{GSexemple28} }
\end{exe}

If the structure included a verbal complex in Italian, it would be light, but this is not possible because constraint (\ref{GSexemple28}) says that the non-head daughter in a light phrase cannot be verbal, which excludes the participle or the infinitive.

Romance languages follow the general constraints on ordering in non-head-final languages. According to constraint (\ref{GSexemple29}), the verb precedes the complements it subcategorizes for. This is relevant not only for the head of the complex predicate, but also for the participle complement of the tense auxiliary or the infinitive complement of a restructuring verb. Although the latter do not combine with their expected complements, they still subcategorize for them.  

\begin{exe} 
        \ex[]{	
        \begin{avm}
		V \[comps \< \ldots, \ibox{1}, \ldots{} \>\] \, < 	\[synsem \ibox{1}\]
\end{avm}\jambox*{(for \ili{Romance languages})}}\label{GSexemple29}
\end{exe}

\subsection{The complements of the copula in Romance languages}\label{GSsection3.4}

It is an interesting fact that, while Romance restructuring verbs enter two different structures (the flat structure and the verbal complex), the copula has the same complementation across Romance languages \citep{abeille2001varieties, AG2010}.\footnote{We concentrate on the predicative use of the copula.} Moreover, this complementation differs both from the flat structure and the verbal complex: the copula takes a non-light complement, which can be saturated or not. 

The complement of the copula is underspecified: it is predicative (noted [\prd +]), but it can be an adjective, a noun, a preposition or a passive participle (for the passive construction, see \citealt{AG2002b-u}). We illustrate clitic climbing with the same example in different Romance languages (examples from \citealt[120]{AG2010}).

\eal
	\label{GSexemple30} 
	\ex{
	\gll Jean lui        \'etait fid\`ele.\\ 
		 Jean to.him/her was     faithful\\\jambox*{(\ili{French})}
	\glt `Jean was faithful to him/her.'}\label{GSexemple30a}

	\ex{ 
	\gll Giovanni le     era     fedele.\\
		 Giovanni to.her was     faithful\\\jambox*{(\ili{Italian})} 
	\glt `Giovanni was faithful to her.'}\label{GSexemple30b}
		
	\ex{ 
	\gll Juan le         era     fiel.\\
		 Juan to.him/her was     faithful\\\jambox*{(\ili{Spanish})}
	\glt `Juan was faithful to him/her.'}\label{GSexemple30c}
		
	\ex{ 
	\gll O Jo\~an era-lhe        fiel.\\ 
		 \textsc{det} Jo\~an was-to.him/her faithful\\\jambox*{(\ili{Portuguese})}
	\glt `Jo\~an was faithful to him/her.'}\label{GSexemple30d}
		
	\ex{ 
	\gll En Joan li      era fidel.\\ 
		 \textsc{det} Joan to.him/her was faithful\\\jambox*{(\ili{Catalan})}
	\glt `Joan was faithful to him/her.'}\label{GSexemple30e}
		
	\ex{ 
	\gll Ion \^ii       era credincios.\\
		 Ion to.him/her was faithful\\\jambox*{(\ili{Romanian})}
	\glt `Ion was faithful to him/her.'}\label{GSexemple30f}
\zl

The properties of the construction differentiate it clearly from tense auxiliaries and restructuring verbs. For the sake of simplification, we restrict the examples to French, Italian and Spanish. The sequence of the head of the complement with its complements is a constituent, since, for instance, it can be dislocated and pronominalized (\ref{GSexemple31}) (examples in (\ref{GSexemple31}) and (\ref{GSexemple32}) from \citealt[133-134]{AG2010}).

\begin{exe}
	\ex{
[Context] Is John faithful to his friends?}\label{GSexemple31} 
	\begin{xlist}
        \ex[]{
		\gll Fid\`ele \`a ses amis,   il l'est plus qu'\`a    ses convictions politiques.\\
		     faithful to  his friends he it.is more {than to} his convictions political\\\jambox*{(\ili{French})}
		\glt `Faithful to his friends, he is, more than to his political ideas.'}\label{GSexemple31a} 
		
        \ex[?]{
		\gll {Fedele} ai     suoi amici,  (lo)         \`e pi\`u che  alle   sue idee  politiche.\\
		     faithful to.the his  friends \spacebr{}it is  more  than to.the his ideas political\\\jambox*{(\ili{Italian})}
		\glt `Faithful to his friends, he is, more than to his political ideas.'}\label{GSexemple31b} 
		
		\ex[]{
		\gll Fiel     a  sus amigos,  lo es m\'as que  a  sus convicciones pol\'iticas.\\
		     faithful to his friends  it is more  than to his convictions  political\\\jambox*{(\ili{Spanish})}
		\glt `Faithful to his friends, he is, more than to his political ideas.'}\label{GSexemple31c} 
	\end{xlist}
\end{exe}

Crucially, the construction differs from that of restructuring verbs in that the dislocated constituent can leave behind its complements (\ref{GSexemple32}).

\eal 
	\label{GSexemple32}
	\ex[]{
	\gll Fid\`ele, il l'est plus \`a ses amis    qu'\`a  ses convictions politiques.\\
	     faithful  he it.is more to  his friends than.to his convictions political\\\jambox*{(\ili{French})}
	\glt `As for being faithful, he is to his friends more than to his political convictions.’}\label{GSexemple32a} 
		
    \ex[]{
	\gll Fedele,  lo \`e ai     sui amici   pi\`u che  alle   sue idee politiche.\\
	     faithful it is  to.the his friends more  than to.the his ideas political\\\jambox*{(\ili{Italian})}
	\glt `As for being faithful, he is to his friends more than to his political convictions.’}\label{GSexemple32b} 
		
	\ex[]{
	\gll Fiel,    lo es m\'as a  sus amigos  que  a  sus convicciones pol\'iticas.\\
	     faithful it is more  to his friends than to his convictions  political\\\jambox*{(\ili{Spanish})}
	\glt `As for being faithful, he is to his friends more than to his political convictions.’}\label{GSexemple32c} 
\zl

Similarly, the predicative complement can be extracted with its complements or it can leave them behind. In the latter case, it can be cliticized, as shown in (\ref{GSexemple33c}) (compare with examples (\ref{GSexemple17}) and (\ref{GSexemple18}) with restructuring verbs). In (\ref{GSexemple33}), the adjective is extracted (it corresponds to the predicative complement of \emph{\^etre} `to be') as part of a concessive adjunct (examples (\ref{GSexemple33}) and (\ref{GSexemple34}) from \citealt[146, 148]{AG2010}).


\begin{exe}
	\ex{
[Context] Is he really faithful to his friends?}\label{GSexemple33} 
	\begin{xlist}
    \ex[]{
	\gll Aussi fid\`ele \`a ses amis    qu'il   soit, il ne          perd  pas de vue   ses int\'er\^ets.\\
		 as    faithful to  his friends {as he} is    he \textsc{ne} lose  not of sight his interests\\\jambox*{(\ili{French})}
	\glt `As faithful to his friends as he is, he does not lose sight of his interests.'}\label{GSexemple33a} 
		
    \ex[]{
	\gll Aussi fid\`ele qu'il   soit \`a ses amis,   il ne          perd  pas de vue   ses int\'er\^ets.\\
		 as    faithful {as he} is   to  his friends he \textsc{ne} lose  not of sight his interests\\
	\glt `As faithful as he is to his friends, he does not lose sight of his interests.'}\label{GSexemple33b} 
		
	\ex[]{
	\gll Aussi fid\`ele  qu'il  leur    soit, il ne          perd  pas de vue   ses int\'er\^ets.\\
		 as    faithful {as he} to.them is    he \textsc{ne} lose  not of sight his interests\\
	\glt `As faithful to them as he is, he does not lose sight of his interests.'}\label{GSexemple33c} 
	\end{xlist}
\end{exe}

Moreover, an adverb may intervene between the copula and the adjective, not only in French or Italian, where it is expected (it is possible with tense auxiliaries and restructuring verbs), but also in Spanish, where it is not expected, if the structure is the same as with restructuring verbs. We illustrate this possibility with cliticization, in order to make the contrast with restructuring verbs clearer.

\eal
	\label{GSexemple34} 
	\ex[]{
	\gll Rom\'eo lui        sera    probablement fid\`ele.\\
	 	 Rom\'eo   to.him/her will.be probably     faithful\\\jambox*{(\ili{French})}
	\glt `Rom\'eo will probably be faithful to him/her.'}\label{GSexemple34a}

	\ex[]{ 
	\gll Romeo le     sar\`a  probabilmente fedele.\\
		 Romeo to.her will.be probably      faithful\\\jambox*{(\ili{Italian})}
	\glt `Romeo will probably be faithful to her.'}\label{GSexemple34b}
		
	\ex[]{ 
	\gll Romeo le 		  ser\'a  probablemente fiel.\\
		 Romeo to.him/her will.be probably      faithful\\\jambox*{(\ili{Spanish})}
	\glt `Romeo will probably be faithful to him/her.'}\label{GSexemple34c}
\zl

The data show that, contrary to restructuring verbs, the copula in Romance languages has only one complementation. \citet{AG2002b-u,AG2010} propose that the copula takes a ``phrasal'' complement, which can be saturated or not. This analysis is implemented by saying that the predicative complement is non-light, whether it is saturated or not, and that it is underspecified with respect to complement saturation or attraction.

\begin{exe}
        \ex[]{Description of the copula in Romance languages \\
        \begin{avm}
		{\[arg-st \<\ibox{1},
                    \[head [prd +] \\
                    subj \ibox{1}\\
                    comps \ibox{2}\\
                    light $-$\\\]\,\> \,\,$\oplus$ \ibox{2}\]}
          	\end{avm}}\label{GSexemple35}
\end{exe}

Like tense auxiliaries, the copula is a subject raising verb, hence the identical value \ibox{1} for its subject and that of the complement, which allows it to be empty. Its complement differs from that of a tense auxiliary (\ref{GSexemple8b}) on several accounts: it is predicative, which is not the case for tense auxiliaries, and it is non-light; in addition, it is not specified for its category. Being non-light, it may have combined with its complements or some of them, while the complement of the auxiliary is light, hence all its complements are attracted.\footnote{Note that the complements included in a predicative PP are not attracted to the copula in a general way.}  

\begin{figure}
    \centering
\begin{forest}
sm edges
  [VP \ms{comps & \liste{ }}
 [V \ms{
            subj & \liste{ \ibox{1} }\\
            comps & \liste{ \ibox{3} }\\
            arg-st & \liste{ \ibox{1}, \ibox{3} }
            }[sera;will.be]] 
 [\ibox{3} AP \ms{
            head & \normalfont{[\textsc{prd} $+$]}\\
            light & $-$\\
            subj & \liste{ \ibox{1} }\\
            comps & \liste{ }}[fid\`ele \`a ses amis;faithful to his friends, roof]]]
\end{forest}    \label{GSfigure7}
    \caption{The Romance copula with a saturated complement}
\end{figure}{}

Figure~\ref{GSfigure8} illustrates a case where the affix complement of the adjective is attracted to the copula. For cliticization and the notion of reduced verb, see Section~\ref{GSsection2}. 

\begin{figure}
    \centering
\begin{forest}
sm edges
  [VP \ms{comps & \liste{ }}
 [V \ms{
             \normalfont{\emph{reduced-verb}}\\
            subj & \liste{ \ibox{1} }\\
            comps & \liste{ \ibox{3} }\\
            arg-st & \liste{ \ibox{1}, \ibox{3}, \ibox{2} }}[leur-sera;to.them-will.be]] 
 [\ibox{3} AP \ms{
            head & \rm [\textsc{prd} $+$]\\
            light & $-$\\
            subj & \liste{ \ibox{1} }\\
            comps & \liste{ \ibox{2} \rm \emph{aff} }}[fid\`ele;faithful]]]
\end{forest}
    \caption{Clitic climbing with the Romance copula}
    \label{GSfigure8}
\end{figure}

\section{Complex predicates and word order}\label{GSsection4}


In certain languages, a complex verb construction signals itself essentially by properties of word order. This is the case for instance in German \citep{HN89b, HN94a, Kiss94, Kiss95a, HN98a, Kathol98b, HN99d, Kathol2000a, Meurers2000b-Short, DM2002, dKM2001a, Mueller2002b, Mueller2003a, MuellerCopula} and Dutch \citep{Rentier94, BvN98a}, as well as Korean \citep{ Sells1991, Chung98a-u, Yoo2003, Kim2016a-u}. We concentrate on coherent constructions in German, and on Korean auxiliaries.    

\subsection{Verbal complexes in German}\label{GSsection4.1}

The contrast in German between coherent and incoherent constructions is reinterpreted in terms of complex predicate formation: coherent constructions constitute a complex predicate, as does the copula with its predicative complement. In this case, the two predicates cannot be separated and form a verbal complex.

\subsubsection{Coherent and incoherent constructions in German}\label{GSsection4.1.1}

Among verbs with an infinitival complement, German distinguishes between coherent and incoherent constructions \citep{gunnar1955studien}. We speak of constructions rather than verbs, because, although the constructions are triggered by lexical properties of verbs, many verbs can be constructed either way. Verbs entering coherent constructions, obligatorily or optionally, belong to different classes: they might be tense auxiliaries (where the verbal complement is an infinitive or a participle), modals, subject and object raising verbs, subject and object control verbs, copulas, predicative verbs or particle verbs (see \citealt{Mueller2002b}, Chapters 2 and 6).

Coherent and incoherent constructions differ with respect to several properties (separability of the head verb and the infinitive, extraposition of the infinitive with its complements, pied-piping in relative clauses and scope of adjuncts). In incoherent constructions, an adverb such as \emph{nicht} `not' may occur between the two verbs as in (\ref{GSexemple36a}) (from \citealt[42]{Mueller2002b}), the infinitival phrase can be extraposed as in (\ref{GSexemple36b}) and (\ref{GSexemple36c}), and the infinitive is pied-piped with its relative pronoun complement as in (\ref{GSexemple36d}) (examples from \citealt[117--118]{HN98a}).

\eal
	\label{GSexemple36} 
	\ex[]{
	\gll \ldots{} dass Karl zu schlafen nicht versucht.\\ 
		 {}       that Karl to sleep    not tries \\\jambox*{(\ili{German})}
	\glt `that Karl does not try to sleep'}\label{GSexemple36a}

    \ex[]{
	\gll \ldots{} dass Peter Maria das Auto zu kaufen \"uberredet.\\ 
	 	 {}       that Peter Maria the car  to buy    persuades \\
	\glt `that Peter persuades Maria to buy the car'}\label{GSexemple36b}

	\ex[]{ 
	\gll \ldots{} dass Peter Maria \"uberredet, [das          Auto zu kaufen].\\
		 {}       that Peter Maria persuades    \spacebr{}the car  to buy\\
	\glt `that Peter persuades Maria to buy the car'}\label{GSexemple36c}	
		
	\ex[]{ 
	\gll Das  ist das Auto, das   zu kaufen er Peter  \"uberreden wird.\\
		 that is  the car   which to buy    he Peter  persuade    will\\
	\glt `That is the car, which he will persuade Peter to buy.'}\label{GSexemple36d}
\zl

On the other hand, coherent constructions, of which the combination of the future auxiliary \emph{wird} `will' in (\ref{GSexemple37a}) or the raising verb \emph{scheinen} `to seem’ with an infinitival complement in (\ref{GSexemple37d}) are typical examples, do not allow for a non-verbal element between the two verbs, as shown in (\ref{GSexemple37b}), nor for extraposition of the infinitive with its complements, as shown in (\ref{GSexemple37c}) and (\ref{GSexemple37e}) (examples (\ref{GSexemple37a}), (\ref{GSexemple37c}), (\ref{GSexemple37d}) and (\ref{GSexemple37e}) from \citealt[43]{Mueller2002b}), nor for pied-piping of the infinitive in relative clauses (\ref{GSexemple37f}) and (\ref{GSexemple37g}) (examples adapted from \citealt[66]{HN99d}).\footnote{The head verb in coherent constructions is italicized.}    

\eal
	\label{GSexemple37} 
	\ex[]{
	\gll \ldots{} dass Karl das Buch lesen \emph{wird}.\\ 
		 {}       that Karl the book read  will \\\jambox*{(\ili{German})}
	\glt `that Karl will read the book'}\label{GSexemple37a}

	\ex[*]{
	\gll \ldots{} dass Karl das Buch lesen nicht wird.\\ 
		 {}       that Karl the book read  not   will \\
	\glt Intended: `that Karl will not read the book'}\label{GSexemple37b}

	\ex[*]{ 
	\gll \ldots{} dass Karl wird das Buch lesen.\\
		 {}       that Karl will the book read\\
	\glt Intended: `that Karl will read the book'}\label{GSexemple37c}	
		
	\ex[]{ 
	\gll \ldots{} weil Karl das Buch zu lesen \emph{scheint}.\\
		 {}       because Karl the book to read seems\\
	\glt `because Karl seems to read the book'}\label{GSexemple37d}	
		
	\ex[*]{ 
	\gll \ldots{} weil Karl scheint das Buch zu lesen.\\
		 {}       because Karl seems   the book to read\\
	\glt Intended: `because Karl seems to read the book'}\label{GSexemple37e}
		
	\ex[*]{ 
	\gll Das  ist das Buch das  lesen Karl wird.\\
		 this is  the book that read  Karl will\\
	\glt Intended: `This is the book that Karl will read.'}\label{GSexemple37f}
		
	\ex[*]{ 
	\gll Das  ist das Buch das  zu lesen Karl scheint.\\
		 this is  the book that to read  Karl seems\\
	\glt Intended: `This is the book that Karl seems to read.'}\label{GSexemple37g}
\zl

Scrambling of the complements of the two verbs, or of the subject of the head verb with the complements of the infinitival, is possible in a coherent construction. In (\ref{GSexemple38a}) the complements of \emph{sehen} `see' (\emph{Peter}) and of \emph{kaufen} `buy' (\emph{das Auto} `the car') are not interleaved. In (\ref{GSexemple38b}), \emph{Peter}, the complement of \emph{sehen}, occurs between \emph{das Auto}, which is the complement of \emph{kaufen}, and \emph{kaufen} (example (\ref{GSexemple38b}) from \citealt[117]{HN98a}).

\eal
	\label{GSexemple38} 
	\ex[]{
	\gll \ldots{} dass er Peter das Auto kaufen \emph{sehen} \emph{wird}.\\ 
		 {}       that       he Peter the car  buy    see            will \\\jambox*{(\ili{German})}
	\glt `that he will see Peter buy the car'}\label{GSexemple38a}

	\ex[]{ 
	\gll \ldots{} dass er das Auto Peter kaufen \emph{sehen}  \emph{wird}.\\
		 {}       that       he the car  Peter buy    see             will\\
	\glt `that he will see Peter buy the car'}\label{GSexemple38b}
\zl

In the complex predicate approach of this chapter, these data point to the following analysis: incoherent constructions involve a saturated VP complement, while coherent constructions do not; rather, they involve a complex predicate, with a verb attracting the complements of its complement. We assume here a verbal complex for the complex predicate. Figure~\ref{GSfigure9a} represents example (\ref{GSexemple36b}), and Figure~\ref{GSfigure9b} represents example (\ref{GSexemple38b}).

\begin{figure}
\begin{subfigure}[b]{\textwidth}
\centering
\begin{forest}
sm edges
[S
   [NP [Peter;Peter]]
   [V$'$
      [NP [Maria;Maria]]
      [V$'$ 
        [VP [das Auto zu kaufen;the car to buy, roof] ]
        [V [überredet;persuades]]]]]
\end{forest}
\caption{Incoherent construction (embedded clause)}
\label{GSfigure9a}
\end{subfigure}
\\
\vspace{20pt}
\begin{subfigure}[b]{\textwidth}
\centering
\begin{forest}
sm edges
[S
   [NP [er;he]]
   [V$'$ 
     [NP [das Auto;the car, roof]]
     [V$'$ 
       [NP [Peter;Peter] ]
       [V
         [V [V [kaufen;buy]]
            [V [sehen;see]]] 
         [V [wird;will]]]]]]
 \end{forest}
\caption{Coherent construction (embedded clause)}
\label{GSfigure9b}
\end{subfigure}
\caption{Incoherent and coherent constructions in German}
\label{GSfigure9}
\end{figure}

\subsubsection{Coherent constructions in HPSG}\label{GSsection4.1.2}

One might wonder whether it is possible to analyze the data in terms of word order instead of
structure: a verb governing a coherent construction would trigger a modification of the ordering
domain. More precisely, it would induce domain union of the two ordering domains associated with the
two verbal projections (see \crossrefchapteralt{order} for a discussion of order domains). Usually, the domain in which constituents are ordered is identical with the phrase or the sentence which dominates them. In the linearization approach \citep{Reape94a}, dominance and ordering can be distinguished. In certain circumstances, the domain for ordering is larger than the domain of constituency, so that the elements belonging to different phrases can be reordered and interleaved, a phenomenon called domain union. Domain union could be responsible for the order in (\ref{GSexemple38b}): the structure would be the same as in incoherent constructions (see Figure~\ref{GSfigure9a}), but the ordering domain would be the whole sentence.

The existence of the remote (or long) passive goes against such an analysis \citep{HN94a, Kathol98b, Mueller2002b}. A complex predicate construction can be passivized in such a way that the subject (in the nominative case) of the passive auxiliary corresponds to the object of the active infinitive complement. An (impersonal) passive construction like (\ref{GSexemple39a}) with an infinitival VP containing an accusative object (\emph{den Wagen} `the car') alternates with a coherent construction such as (\ref{GSexemple39b}), with a corresponding nominative (examples (\ref{GSexemple39a}) and (\ref{GSexemple39b}) from \citealt[137]{Mueller2002b}, (\ref{GSexemple39c}) and (\ref{GSexemple39d}) from \citealt[40]{Mueller2003a}). 

\eal
	\label{GSexemple39} 
	\ex[]{
	\gll \ldots{} weil       oft   versucht wurde, [den          Wagen zu reparieren].\\ 
		 {}       because    often tried    was    \spacebr{}the car   to repair\\
	\glt `because many attempts were made to repair the car'}\label{GSexemple39a}

	\ex[]{
	\gll \ldots{} weil       der Wagen oft   zu reparieren \emph{versucht} \emph{wurde}.\\ 
		 {}       because    the car   often to repair     tried             was\\
	\glt `because many attempts were made to repair the car'}\label{GSexemple39b}

	\ex[]{ 
	\gll Karl darf       nicht versuchen zu schlafen.\\
		 Karl is.allowed not   try       to sleep\\
	\glt `Karl is not allowed to try to sleep.'\\
	\glt `Karl is allowed to not try to sleep.'}\label{GSexemple39c}	
		
	\ex[]{ 
	\gll Karl darf       versuchen, nicht zu schlafen.\\
		 Karl is.allowed try        not   to sleep\\
	\glt `Karl is allowed to try not to sleep.’}\label{GSexemple39d}
\zl

In (\ref{GSexemple39a}), the infinitival VP is extraposed. In (\ref{GSexemple39b}), there is no infinitival VP, as shown by the position of the adverb \emph{oft} `often', which occurs before \emph{zu reparieren} `to repair', while modifying \emph{versucht} `tried'. In a coherent construction, an adverb can scope over any of the verbs that belong to it. In (\ref{GSexemple39c}), \emph{zu schlafen} `to sleep' is not part of the coherent construction, because it is extraposed; \emph{nicht} `not' can have scope over \emph{darf} `is allowed' or \emph{versuchen} `to try', not over \emph{schlafen} `to sleep'. In (\ref{GSexemple39d}), \emph{nicht} belongs to the extraposed infinitival; accordingly, it can only scope over that. The fact that \emph{oft} can scope over \emph{versucht} in (\ref{GSexemple39b}) shows that they belong to the same coherent construction, which allows for passivization: \emph{versuchen}, attracting the complement of \emph{reparieren}, can be passivized.

German differs from Romance languages in not distinguishing structurally between the subject and the complements: the subject can be considered as a complement, and introduced by the same rule. The structure of the sentence is usually represented as having binary branching daughters (see Figure~\ref{GSfigure9}). The constraint is as follows (\citealt[21]{muller2018clause}).

\begin{exe}
    \ex[] {\emph{head-complement-phrase} (German) \impl \\
    \begin{avm}
      {\[synsem & \[loc | cat | comps \ibox{1} $\oplus$ \ibox{3}\\
      light $-$\]\\
      head-dtr | synsem & \[loc | cat | comps \ibox{1} $\oplus$ \<\ibox{2}\>\, $\oplus$ \ibox{3}\]\\
      non-head-dtrs & \<[synsem \ibox{2}]\>\]}\label{GSexemple40}
    \end{avm}}
\end{exe}

Following constraint (\ref{GSexemple40}), the head combines with one complement at a time, noted as
\ibox{2}. The presentation of the list as composed of three parts, with the relevant one in any
position, allows for a free order. The lexical verb is \mbox{[\light$+$]}, and the phrase combining the verb
with a complement is [\light $-$].\footnote{The feature \light is the equivalent of \lex used in German
  studies, although the properties of light elements may differ depending on the language. It does
  not belong to \textsc{local} features in (\ref{GSexemple40}), because an extracted constituent may
  differ from its trace as regards lightness (\citealt{muller2018clause}; see \crossrefchaptert{udc} for discussion of extraction).} The structure of (\ref{GSexemple41}) is exemplified in Figure~\ref{GSfigure10} \citep[22]{muller2018clause}.

\ea[]{
	\label{GSexemple41}
	\gll \ldots{} weil    das Buch jeder     kennt.\\
		 {}       because the book everybody knows\\
	\glt `because everybody knows the book'}

\z

%%%%%%%%%%%%%%%%%%%%%%%%%%%%%%%MAS LARG%%%%%%%%%%%%%%%%%%%%%%%%%%%%%%%%%%

\begin{figure}
    \centering
	\begin{forest}
	sm edges
 	[CP 
    [C [weil;because]]
    [V \ms{{\normalfont{\emph{fin}}}, comps \liste{ }, \normalfont{\textsc{light $-$}}} 
        [\ibox{2} NP [das Buch;the book, roof]]    
        [V \ms{{\normalfont{\emph{fin}}}, comps \liste{ \ibox{2} }, \normalfont{\textsc{light $-$}}} 
            [\ibox{1} NP [jeder;everybody]]
            [V \ms{{\normalfont{\emph{fin}}}, comps \liste{\ibox{1}, \ibox{2} }, \normalfont{\textsc{light $+$}}} [kennt;knows]]]]]
\end{forest}
    \caption{Clause structure in German}
    \label{GSfigure10}
\end{figure}


%%%%%%%%%%%%%%%%%%%%%%%%%%%%%%%MAS LARG%%%%%%%%%%%%%%%%%%%%%%%%%%%%%%%%%%

Turning to complex predicates, they form a verbal complex phrase: they cannot be separated by an adverb or an NP, as shown in (\ref{GSexemple37b}) and (\ref{GSexemple37c}). Given the structure of the German sentence with binary branching, illustrated in Figure~\ref{GSfigure9}, this verbal complex only shows up structurally when there is a series of verbs attracting the complements of their complements, as in (\ref{GSexemple38}) (see Figure~\ref{GSfigure9b}).

The phrase structure constraint allowing complex predicates is as in (\ref{GSexemple42}) (\citealt{MuellerCopula}; \citealt[39]{muller2018clause}). It is called \type{head-cluster-phrase}, rather than \type{verbal"=complex"=phrase}, because it is not specialized for verbal heads.\footnote{Following \cite{HN94a} and \cite{dKM2001a}, but contrary to \cite{muller2018clause}, we mention the lightness of the mother. Müller’s decision is motivated by the fact that infinitive intransitive verbs may be analyzed as argument saturated (and non-light) if their subject is represented as a head feature rather than a complement. However, it leads to formal complications which are best ignored in this presentation. Hence, we assume here, for the sake of simplification, that the subject of the infinitival verb is a complement in German, like the subject of a finite verb.  }  

\begin{exe}
    \ex {\emph{head-cluster-phrase} (German) \impl \\
    \begin{avm}
      {\[synsem & \[loc | cat | comps \ibox{1}\\
      light $+$\]\\
      head-dtr | synsem & \[loc | cat | comps \ibox{1} $\oplus$ \<\ibox{2}\>\\ light $+$\]\\
      non-head-dtrs & \<[synsem \ibox{2} [light $+$]]\>\]} \end{avm}\label{GSexemple42}}
\end{exe}

We illustrate the analysis with sentence (\ref{GSexemple38b}) (\ldots{} \emph{dass er das Auto Peter kaufen sehen wird} `that he will see Peter buy the car’), elaborating on Figure~\ref{GSfigure9b}. The description of \emph{werden} (the future auxiliary), a subject raising verb, is as in (\ref{GSexemple43}) (from \citealt[39]{muller2018clause}), and that of \emph{sehen} `to see', an object raising verb and an obligatorily coherent verb, is as in (\ref{GSexemple44}) (simplified from \citealt[102]{Mueller2002b}). The subject is assumed here (for simplification) to be part of the list of complements of infinitives as well as of finite verbs; hence the raised subject of the infinitive complement of \emph{werden} `will' is included in list \ibox{1}, and that of \emph{sehen} is included in list \ibox{2}. The subject is distinguished from the other elements of this list by its semantic role (it is the first semantic argument of the infinitive).
The infinitive is analyzed as having the feature [\vform \textit{base}], noted as \textit{bse}.

\begin{exe}
    \ex {\emph{werden} (future auxiliary): \\
    \begin{avm}
      {\[head {\normalfont{\emph{verb}}}\\
      comps \ibox{1} $\oplus$ \< V[{\normalfont{\emph{bse}}}, comps \ibox{1}, light $+$]\>\]}
    \end{avm}}\label{GSexemple43}
\end{exe}

\begin{exe}
    \ex {\emph{sehen} (obligatory coherent verb): \\
    \begin{avm}
      {\[head {\normalfont{\emph{verb}}}\\
      comps \ibox{1} $\oplus$ \ibox{2} $\oplus$ \< V[{\normalfont{\emph{bse}}}, comps \ibox{2}, light $+$]\>\]}
    \end{avm}}\label{GSexemple44}
\end{exe}

Sentence (\ref{GSexemple38b}) is represented in Figure~\ref{GSfigure11}. 


%%%%%%%%%%%%%%%%%%%%%%   LARG    &&&&&&&&&&&&&&&&&&&&&&&
%\inlinetodostefan{Stefan: I changed this figure. It is now CP and \vform is under head.}

\begin{figure}
%    \centering
\oneline{%
\begin{forest}
sm edges
%    fairly nice empty nodes,
	[CP
		[C[dass;that]]
    	[{V[\comps \eliste]}
          [\ibox{1} NP [er;he]]
     		[{V[\comps \sliste{ \ibox{1} }]}
                    [\ibox{2} NP [das Auto;the car, roof]]
     			[{V[\comps \sliste{ \ibox{1}, \ibox{2} }]}
                          [\ibox{3} NP [Peter;Peter]]
     			  [{V[\head \ibox{4}, \comps \sliste{ \ibox{1}, \ibox{3}, \ibox{2} } ]}
			     				[\ibox{7} V \ms{head \ibox{5}\\
                					comps \sliste{ \ibox{1}, \ibox{3}, \ibox{2} }}
                		[\ibox{6} V \ms{vform {\normalfont{\emph{bse}}}\\
                						comps \sliste{ \ibox{3}, \ibox{2} }} 
                			[kaufen;buy]]
                		[V \ms{head \ibox{5} [ \vform \type{bse} ]\\
                                       comps \sliste{ \ibox{1}, \ibox{3}, \ibox{2} }  $\oplus$ \ibox{6}} [sehen;see]]]
     [V \ms{head \ibox{4} [ \vform \type{fin} ]\\
                    comps \sliste{ \ibox{1}, \ibox{3}, \ibox{2} } $\oplus$ \sliste{ \ibox{7} }}%,   before computing xy={s'-=35pt} 
    [wird;will]]]]]]]
 \end{forest}}    
    \caption{Coherent construction with verbal complexes in German}
    \label{GSfigure11}
\end{figure}

%%%%%%%%%%%%%%%%%%%%%%   LARG    &&&&&&&&&&&&&&&&&&&&&&&


\subsubsection{The German copula}\label{GSsection4.1.3}

The copula in German, with an adjectival argument, is also the head of a complex predicate.\footnote{As in Romance languages, the German copula accepts nominal and prepositional predicative complements. However, they are complement saturated.} The subject of the copula and the complements of the adjectives can be permuted (examples from \citealt[68]{Mueller2002b}; see (\ref{GSexemple38}) for coherent verbs):

\eal 
	\label{GSexemple45} 
    \ex[]{
	\gll \ldots{} dass die Sache                 dem Minister                ganz       klar  war.\\ 
	     {}       that the {matter.\textsc{nom}} the {minister.\textsc{dat}} completely clear was\\
	\glt `that the matter was completely clear to the minister'}\label{GSexemple45a}

	\ex[]{ 
	\gll \ldots{} dass dem Minister                die Sache                 ganz       klar  war. \\
		 {}       that the {minister.\textsc{dat}} the {matter.\textsc{nom}} completely clear was\\
	\glt `that the matter was completely clear to the minister'}\label{GSexemple45b}
\zl

Adverbs can have different scopings: in (\ref{GSexemple46}) (from \citealt[68]{Mueller2002b}), \emph{immer} `always' can modify the modal or the adjective. This follows if there is just one coherent construction, both the modal and the copula being the head of a complex predicate (see Section~\ref{GSsection4.1.2}, example (\ref{GSexemple39b}) for coherent verbs).

\ea[]{
	\label{GSexemple46}
	\gll \ldots{} weil    der Mann             ihr              immer  treu     sein wollte.\\
	     {}       because the man.\textsc{nom} her.\textsc{dat} always faithful be   wanted.to\\
	\glt `because the man always wanted to be faithful to her'\\
	\glt `because the man wanted to be faithful to her forever'}
\z

\cite{Mueller2002b} also shows that the copula does not take a saturated AP complement. Contrary to a construction with an incoherent verb, this AP cannot be extraposed, as shown in (\ref{GSexemple47b}), or pied piped with a relative pronoun, as shown in (\ref{GSexemple47d}) (from \citealt[70]{Mueller2002b}; compare with (\ref{GSexemple36c}), (\ref{GSexemple36d})).   

\eal
	\label{GSexemple47} 
	\ex[]{
	\gll Karl ist auf seinem Sohn stolz gewesen.\\ 
		 Karl is  on  his    son  proud been\\
	\glt `Karl was proud of his son.'}\label{GSexemple47a}

	\ex[*]{ 
	\gll Karl ist gewesen auf seinem Sohn stolz.\\
		 Karl is  been    on  his    son  proud\\
	\glt Intended: `Karl was proud of his son.'}\label{GSexemple47b}
 
    \ex[]{
	\gll der Sohn, auf den  Karl stolz gewesen ist\\ 
		 the son   on  whom Karl proud been    is\\
	\glt `the son of whom Karl was proud'}\label{GSexemple47c}
		     
	\ex[*]{
	\gll der Sohn, auf den stolz Karl gewesen ist\\ 
		 the son on whom proud Karl been is\\
	\glt Intended: `the son of whom Karl was proud'}\label{GSexemple47d}
\zl

In addition, the German copula, like the Romance copula, is a subject raising verb: the semantic properties of the subject depend on the adjective (a human is proud or faithful, and a matter is clear, as shown also by the nominalizations, cf.\ the man's faithfulness, the clarity of the matter); moreover, the sentence can be subjectless (from \citealt[72]{Mueller2002b}): 

\ea[]{
	\label{GSexemple48}
	\gll Am     Montag ist schulfrei.\\
		 at.the Monday is  school.free\\
	\glt `There is no school on Monday.'}
\z

The description of the German copula, restricted to its predicative use and to its syntactic part, is as follows:
\ea
\label{GSexemple49}
{\emph{sein} (copula): \\}
    \begin{avm}
      {\[
      head {\normalfont{\emph{verb}}}\\
      comps \ibox{1} $\oplus$ \<\ \[head [prd $+$]\\
      comps \ibox{1}\]\,\>\]}
    \end{avm}
\z

It differs from the Romance copula in not specifying the lightness of its predicative complement. The \comps list includes the subject, while subject and complements are distinguished in Romance languages.


\subsection{Argument attraction with Korean auxiliaries}\label{GSsection4.2}

Like German complex predicates, Korean auxiliary constructions allow the arguments of the auxiliary and its verb complement to be interleaved. 
Other properties show that the auxiliary forms a complex predicate with its verbal complement, and such properties are absent with control verbs: in spite of the flexible word order which they sometimes allow, the latter do not belong to the complex predicates. 
As in German again, the auxiliary and its verbal complement constitute a verbal complex.

\subsubsection{Scrambling with auxiliaries in Korean}\label{GSsection4.2.1}
 
 
Korean resembles German in that a complex predicate signals itself notably by its word order properties (see \citealt{Sells1991, Chung98a-u, Yoo2003, Kim2016a-u}). We illustrate here the case of auxiliaries.\footnote{\cite{Chung98a-u} also considers control verbs to be the head of complex predicates, and \citegen{Kim2016a-u} study includes serial verb and light verb constructions.} 

Korean auxiliaries semantically resemble aspectual or modal verbs rather than tense auxiliaries:
they include such verbs as \emph{iss-} `to be in the process/state of', \emph{chiwu-} `to do
resolutely’, \emph{siph-} `to want', but also the verb of negation \emph{anh-} `not' (see also \crossrefchapteralt[Section~\ref{sec-negative-auxiliary-verb}]{negation}). They bear the tense marking for the sentence (\ref{GSexemple50a}), impose a certain ending to their verbal complement (\emph{-e} in (\ref{GSexemple50a})), and, when they also have a use as ordinary verbs (\ref{GSexemple50b}), they have an argument structure which is absent in their auxiliary use (examples from \citealt[85--86]{Kim2016a-u}).

\eal
	\label{GSexemple50} 
    \ex[]{
	\gll Mia-ka             wul-e               pely-ess-ta.\\ 
		{Mia-\textsc{nom}} {cry-\textsc{conn}} {end.up-\textsc{pst}-\textsc{decl}}\\
	\glt `Mia ended up crying.'}\label{GSexemple50a}

    \ex[]{
	\gll Mimi-nun            congi-lul            hyucithong-ey            pely-ess-ta.\\ 
		{Mimi-\textsc{top}} {paper-\textsc{acc}} {trash.can-\textsc{loc}} {throw.away-\textsc{pst}-\textsc{decl}}\\
	\glt `Mimi threw away the paper in the trash can.'}\label{GSexemple50b}
\zl

In (\ref{GSexemple50b}), the verb has three arguments: agent subject, theme object, and location complement. This argument structure is absent in (\ref{GSexemple50a}).

Consider the sentences in (\ref{GSexemple51}). There is no evidence of scrambling in (\ref{GSexemple51a}): the subject \emph{Maryka} (`Mary' + nominative) starts the sentence, and the complement of the verb \emph{ilkko} `read' immediately precedes it. However, in (\ref{GSexemple51b}), the subject of the head verb \emph{issta} `be in the process of', namely \emph{Maryka}, occurs between the complement of \emph{ilkko}, namely \emph{ku chaykul} (`the book' + accusative), and the verb \emph{ilkko} itself.
\eal
	\label{GSexemple51}
	\ex[]{
	\gll Mary-ka           ku  chayk-ul          ilk-ko  	        iss-ta.\\ 
             Mary-\textsc{nom} the book-\textsc{acc} read-\textsc{conn} be.in.the.process.of-\textsc{decl}\\
	\glt `Mary is in the process of reading the book.'}\label{GSexemple51a}
		
	\ex[]{
	\gll Ku  chayk-ul          Mary-ka           ilk-ko  	        iss-ta.\\ 
	     the book-\textsc{acc} Mary-\textsc{nom} read-\textsc{conn} be.in.the.process.of-\textsc{decl}\\
	\glt `Mary is in the process of reading the book.'}\label{GSexemple51b}

\zl

A priori, these data could be explained in two ways: either the auxiliary always takes a VP complement, and scrambling is due to linearization, in which case the domains of the two verbs are unioned \citep{Reape94a}; or there is a complex predicate: the complement of the embedded verb (\emph{ku chaykul} `the book' + accusative) is attracted by the auxiliary verb. 

There are several properties which show that scrambling is due to argument attraction. First, the presence of the auxiliary allows for case alternation: the argument of a verb like \emph{mek-} `to eat' is assigned accusative case, as shown in (\ref{GSexemple52a}); however, when the verb is the complement of the auxiliary verb \emph{siph-} `to want' in (\ref{GSexemple52b}), it can be either accusative or nominative (examples (\ref{GSexemple52}) from \citealt[87]{Kim2016a-u}).

\eal
	\label{GSexemple52}
	\ex[]{
	\gll Mimi-ka			 sakwa-lul/*ka		      mek-ess-ta.\\ 
		 {Mimi-\textsc{nom}} {apple-\textsc{acc/nom}} {eat-\textsc{pst-decl}}\\
	\glt `Mimi ate an apple.'}\label{GSexemple52a}
		
	\ex[]{
	\gll Mimi-ka			 sakwa-lul/ka		      mek-ko		      siph-ess-ta.\\ 
		 {Mimi-\textsc{nom}} {apple-\textsc{acc/nom}} {eat-\textsc{conn}} {wish-\textsc{pst-decl}}\\
	\glt `Mimi would like to eat an apple.'}\label{GSexemple52b}
\zl

Given that case assignment is a local phenomenon, and a verb does not influence the case of the
complement of its complement, this indicates that \emph{sakwa-} `apple' becomes the complement of
the auxiliary (see also \citealt{Yoo2003}). Moreover, in Korean, a negative polarity item such as
\emph{amwukesto} `anything' is licensed by a clause-mate negated element. (\mex{1}) provides examples. (\ref{GSexemple53a}) and (\ref{GSexemple53b}) show that the negative verb \emph{anh-} allows this negative polarity item as the argument of \emph{mek-} `to eat', the complement of the auxiliary \emph{siph-} `to want' (examples from \citealt[91]{Kim2016a-u}).

\eal
	\label{GSexemple53}
	\ex[]{
	\gll Mimi-nun			 amwukes-to	     mek-ci		         anh-ass-ta.\\ 
		 {Mimi-\textsc{top}} {anything-also} {eat-\textsc{conn}} {not-\textsc{pst-decl}}\\
	\glt `Mimi did not eat anything.'}\label{GSexemple53a}
		
	\ex[]{
	\gll Mimi-nun			 amwukes-to	     mek-ko		         siph-ci		      anh-ass-ta.\\ 
		 {Mimi-\textsc{top}} {anything-also} {eat-\textsc{conn}} {wish-\textsc{conn}} {not-\textsc{pst-decl}}\\
	\glt `Mimi did not feel like eating anything.'}\label{GSexemple53b}
	
	\ex[*]{
	\gll Mimi-lul			 amwukes-to	     mek-tolok		     seltukha-ci		      anh-ass-ta.\\ 
		 {Mimi-\textsc{acc}} {anything-also} {eat-\textsc{conn}} {persuade-\textsc{conn}} {not-\textsc{pst-decl}}\\
	\glt Intended: `(We) did not persuade Mimi to eat anything.'}\label{GSexemple53c}
	
\zl

Finally, the same argument can be levelled against an analysis which appeals to linearization, as above in German (Section~\ref{GSsection4.1.2}): so-called long passivization is possible with certain auxiliaries like \emph{chiwu-} `to do resolutely', which cannot be accounted for by appeal to linearization and domain union (examples from \citealt[164]{Chung98a-u}).\footnote{Such passives are judged unnatural by native speakers, hence the question mark.} (\ref{GSexemple54a}) exemplifies the active sentence, and (\ref{GSexemple54b}) the passive one. In (\ref{GSexemple54a}), \emph{malssengmanhun solul} `the troublesome cow' is the complement of the complement verb \emph{phal-} `to sell'. In (\ref{GSexemple54b}), \emph{malssengmanhun soka} is the subject of the passivized verb \emph{chiwe ciessta}.

\eal
	\label{GSexemple54} 
	\ex[]{
	\gll Ku  nongpwu-ka            malssengmanhun so-lul             phal-a 		      chiw-ess-ta.\\ 
		 the {farmer-\textsc{nom}} troublesome    {cow-\textsc{acc}} {sell-\textsc{conn}} {do.resolutely-\textsc{pst}-\textsc{decl}}\\
	\glt `The farmer resolutely sold the troublesome cow.'}\label{GSexemple54a}
		
	\ex[?]{
	\gll Malssengmanhun so-ka              (ku           nongpwu-eyuyhay) phal-a               chiw-e                                  ci-ess-ta.\\ 
		 Troublesome    {cow-\textsc{nom}} \spacebr{}the {farmer-by}      {sell-\textsc{conn}} {do.resolutely-\textsc{conn}} {\textsc{pass}-\textsc{pst}-\textsc{decl}}\\
	\glt `The troublesome cow was resolutely sold (by the farmer).'}\label{GSexemple54b}
\zl

Since passivization only affects the complement of the verb which is itself passivized, it follows that \emph{malssengmanhun solul} is the complement of the auxiliary in (\ref{GSexemple54a}).

The scrambling data with control verbs are very similar to those with auxiliaries, as in (\ref{GSexemple55}). There is no scrambling in (\ref{GSexemple55a}): the dative complement of the head verb is followed
by the other complement, a VP. However, in (\ref{GSexemple55b}), the subject of the head verb
(\emph{Maryka} `Mary' + nominative) occurs between the complement of the complement verb (\emph{ku
  chaykul} `the book' + accusative) and the dative complement of the head verb (\emph{Johnhanthey} `John' + dative).

\eal
	\label{GSexemple55} 
	\ex[]{
	\gll Mary-ka    	     John-hanthey        [ku           chayk-ul            ilk-ulako]       seltukha-yess-ta.\\ 
		 {Mary-\textsc{nom}} {John-\textsc{dat}} \spacebr{}the {book-\textsc{acc}} {read-\textsc{conn}} {persuade-\textsc{pst}-\textsc{decl}}\\
	\glt `Mary persuaded John to read the book.'}\label{GSexemple55a}
		
	\ex[]{
	\gll Ku  chayk-ul             Mary-ka    	     John-hanthey         ilk-ulako            seltukha-yess-ta.\\ 
		 the {book-\textsc{acc}} {Mary-\textsc{nom}} {John-\textsc{dat}}  {read-\textsc{conn}} {persuade-\textsc{pst-decl}}\\
	\glt `Mary persuaded John to read the book.'}\label{GSexemple55b}
\zl

However, we do not observe case alternation in this case, and control verbs fail to allow the negative polarity item \emph{amwukesto} `anything' as the complement of the verb complement (\citealt[91]{Kim2016a-u}).

\eal
	\label{GSexemple55added} 
	\ex[]{
	\gll Mimi-lul            amwukes-to    an mek-tolok           selkhuta-yess-ta.\\ 
		 {Mimi-\textsc{acc}} anything-also no {eat-\textsc{conn}} {persuade-\textsc{pst-decl}}\\
	\glt `(We) persuaded Mimi not to eat anything.'}\label{GSexemple55addeda}
	
	\ex[*]{
	\gll Mimi-lul            amwukes-to    mek-tolok           selkhuta-ci              anh-ass-ta.\\ 
		 {Mimi-\textsc{acc}} anything-also {eat-\textsc{conn}} {persuade-\textsc{conn}} {not-\textsc{pst-decl}}\\
	\glt Intended: `We did not persuade Mimi to eat anything.'}\label{GSexemple55addedb}
\zl

Accordingly, we follow \cite{Kim2016a-u} in not analyzing control verbs as heads of complex
predicates. They take VP complements, and scrambling in (\ref{GSexemple55}) must be due to a
different process (for instance, domain union; see \citealt{Reape94a}).


\subsubsection{Korean auxiliaries and the verbal complex}\label{GSsection4.2.2}

It has been shown in this paper that different structures could be associated with argument attraction. Korean auxiliaries are the head of a verbal complex (\citealt{Chung98a-u}; \citealt{Kim2016a-u}). The main fact is that nothing can intervene between the two verbs, for instance no parenthetical expression, such as \emph{hayekan} `anyway', as illustrated in (\ref{GSexemple56added}) (examples from \citealt[162]{Chung98a-u}). This contrasts with control verbs. In (\ref{GSexemple57added}), the adverb \emph{cengmal} `really' can occur before the embedded verb, or between the two verbs (example (\ref{GSexemple57added}) from \citealt[93]{Kim2016a-u}).


\eal
	\label{GSexemple56added} 
	\ex[]{
	\gll Mary-ka    	     hayekan sakwa-lul            mek-ko 	          iss-ta.\\ 
		 {Mary-\textsc{nom}} anyway  {apple-\textsc{acc}} {eat-\textsc{conn}} {be.in.the.process.of-\textsc{decl}}\\
	\glt `Anyway, Mary is eating an apple.'}\label{GSexemple56added-a}

	\ex[*]{
	\gll Mary-ka    	     sakwa-lul            mek-ko              hayekan iss-ta.\\ 
		 {Mary-\textsc{nom}} {apple-\textsc{acc}} {eat-\textsc{conn}} anyway  {be.in.the.process.of-\textsc{decl}}\\
	\glt Intended: `Anyway, Mary is eating an apple.'}\label{GSexemple56added-b}
\zl

\ea[]{
	\label{GSexemple57added} 	
	\gll Mimi-nun	         Haha-lul 	         (cengmal)           ttena-tolok	       (cengmal)	    seltukha-yess-ta.\\ 
		 {Mimi-\textsc{top}} {Haha-\textsc{acc}} \spacebr{}really    {leave-\textsc{conn}} \spacebr{}really {persuade-\textsc{pst-decl}}\\
	\glt `Mimi (really) persuaded Haha to (really) leave.'}
\z

In addition, there is evidence that the verb complement of an auxiliary and its complement do not form a constituent. While an NP may occur after the head verb in a so-called afterthought construction (\ref{GSexemple56a}), this is not possible for the embedded verb \emph{mek-} with its complement (\ref{GSexemple56b}) (from \citealt[162]{Chung98a-u}).

%As shown by \cite{Chung98a-u}, in Korean, the structure of a complex predicate is different depending on whether the head verb is an auxiliary or a control verb. Auxiliaries are the head of a verbal complex (as in German complex predicates), while control verbs can either take a saturated VP complement, or be the head of a flat structure (like restructuring verbs in Italian). Thus, complex predicates in Korean confirm the observation made for Romance Languages, although the complex predicate manifests itself by different properties: argument attraction is not correlated with a specific structure. 

%A characteristic property of Korean auxiliary constructions is that nothing can separate the two verbs: no parenthetical expression, such as \emph{hayekan}, can intervene (\ref{GSexemple55}). In addition, the embedded verb cannot occur by itself. Thus, while an NP may occur after the head verb in a construction called afterthought (\ref{GSexemple56a}), this is not possible for the embedded verb \emph{mekko} alone (\ref{GSexemple56b}) or with its complement (\ref{GSexemple56c}) (examples from \citealt[162]{Chung98a-u}). This behavior follows if the two verbs form a verbal complex (see Section~\ref{GSsection3.2}).

%\eal 
%	\label{GSexemple55} 
%	\ex[]{
%	\gll Mary-ka             hayekan sakwa-lul            mek-ko              iss-ta.\\ 
%		 {Mary-\textsc{nom}} anyway  {apple-\textsc{acc}} {eat-\textsc{part}} {be.in.the.process.of-\textsc{part}}\\
%	\glt `Anyway, Mary is eating an apple.'}\label{GSexemple55a}
%
 %   \ex[*]{
%	\gll Mary-ka             sakwa-lul            mek-ko              hayekan iss-ta.\\
%		 {Mary-\textsc{nom}} {apple-\textsc{acc}} {eat-\textsc{part}} anyway  {be.in.the.process.of-\textsc{part}}\\
%	\glt `Anyway, Mary is eating an apple.'}\label{GSexemple55b}
%\zl

\eal
	\label{GSexemple56} 
	\ex[]{
	\gll Mary-ka             mek-ko              iss-ta,                              sakwa-lul.\\ 
		 {Mary-\textsc{nom}} {eat-\textsc{conn}} {be.in.the.process.of-\textsc{decl}} {apple-\textsc{acc}}\\
	\glt `Mary is in the process of eating an apple.'}\label{GSexemple56a}

    \ex[*]{
	\gll Mary-ka            iss-ta,                              sakwa-lul            mek-ko.\\ 
		{Mary-\textsc{nom}} {be.in.the.process.of-\textsc{decl}} {apple-\textsc{acc}} {eat-\textsc{conn}}\\
	\glt Intended: `Mary is in the process of eating an apple.'}\label{GSexemple56b}
\zl

These data point to a verbal complex (see Section~\ref{GSsection3.2}). However, before coming to this conclusion, we must show that the two verbs do not form a compound word. \cite{no1991case} (summarized in \citealt{Chung98a-u}, \citealt{Kim2016a-u}) presents arguments to the effect that they combine in the syntax. The main one relies on the use of delimiters. A delimiter (such as \emph{-man} `only' or \emph{-to} `also') can combine with the embedded verb (\eg \emph{mekkoman issta} `to be only eating'). Delimiters are a syntactic phenomenon, not limited to verbal morphology. 
% Auxiliary constructions may involve several auxiliaries, and there is no limit to the number of auxiliaries which may combine, with the appropriate ending. It is not plausible to list all the possible combinations in the lexicon. (\ref{GSexemple57added-a}), which repeats (\ref{GSexemple51a}), exemplifies a sentence with one auxiliary, and (\ref{GSexemple57added-b}) a sequence of two auxiliaries (\citealt[172]{Chung98a-u}).
%
% \eal
% 	\label{GSexemple57added-ab}
% 	\ex[]{
% 	\gll Mary-ka            ku   chayk-ul            ilk-ko  	          iss-ta.\\ 
% 		{Mary-\textsc{nom}} the  {book-\textsc{acc}} {read-\textsc{conn}} {be.in.the.process.of-\textsc{decl}}\\
% 	\glt `Mary is in the process of reading the book.'}\label{GSexemple57added-a}
%	
% 	\ex[]{
% 	\gll Mary-ka            ku   chayk-ul           ilk-ke  	         po-ko               iss-ta.\\ 
% 		{Mary-\textsc{nom}} the {book-\textsc{acc}} {read-\textsc{conn}} {try-\textsc{conn}} {be.in.the.process.of-\textsc{decl}}\\
% 	\glt `Mary is in the process of giving the book a trial reading.'}\label{GSexemple57added-b}
% \zl
Thus, the head auxiliary and the complement verb form a verbal complex.

\subsubsection{Korean auxiliaries in HPSG}\label{GSsection4.2.3}

Given the free word order in Korean (except for the verb), there are two ways of representing the sentence: either there is a flat structure (except for the verbal complex), where all the arguments, subject and complements, are sisters of each other (see, among others, \citealt{Chung98a-u} for Korean), or there is a binary branching structure (see \citealt{Kim2016a-u} for Korean). We adopt the flat structure here since the differences between the two approaches are irrelevant for the purpose of this chapter (but see \crossrefchaptert[Section~\ref{sec-binary-flat}]{order}).


The general schema for the sentence is given in (\ref{GSexemple60added}), adapted from
\cite[178]{Chung98a-u}. 
% St. Mü. There is no / sign. I removed this sentence. 07.08.2020
%The sign `/' indicates that, by default, the value of \textsc{subj} and \textsc{comps} is the empty list.

\ea
\label{GSexemple60added}
\type{head-subject-complements-phrase} (Korean) \impl \\
\avmtmp{
[synsem|loc|cat & [ subj  & < >\\
                    comps & < >\\
                    light & $-$ ]\\
 \punk{head-dtr|synsem|loc|cat}{  [ head  & verb\\
% AUX- would not work here, since the verbal complex is AUX+. See figure. St.Mü. 07.08.2020
%[\type*{verb}\\ 
%                                      aux  & $-$]\\
                             subj  & \1\\
                             comps & \2 \\
                             light & $+$]}\\
 non-head-dtrs & \texttt{synsem2sign}(\1 \+ \2) nonempty list ]
    }
\z
This schema combines a head with its subject and its complements in one go. Since no LP constraints
are formulated, subjects and objects can be scrambled and permutations are accounted for. The \subjl and the
\compsl contains \type{synsem} elements. These lists are appended into one list, which is then
converted into a list of signs by the
relational constraint \texttt{synsem2sign}. A further constraint -- not given in (\mex{0}) -- requires that the
non-head daughters must be \light$-$.\footnote{%
  See \citet[(35)]{Mueller2005c} and \citet[Section~2.2.4]{MuellerGS} for an explicit formulation of
  such a constraint in a grammar of German.
} This ensures that arguments of auxiliaries cannot be realized
in flat structures licensed by (\ref{GSexemple60added}) since auxiliaries select for \light$+$
complements. 

The lexical item of the auxiliary \emph{issta} `be in the process of' in
(\ref{GSexemple57added-a}) is provided in (\ref{GSexemple62added}):\footnote{%
\citet[\page 94--95]{Kim2016a-u} argues that complex predicate formation in Korean results from a Head-\textsc{lex}
construction that ensures that the \compsl of the mother is identical to the \compsl of the verb
daughter that is the complement to the auxiliary. For reasons of space and to make a comparison
between Korean complex predicate formation and complex predicate formation in \ili{Romance}, \ili{German}, and
\ili{Persian} easier, we adopt a lexical analysis of complex predicate formation in Korean, as proposed in
\citew{Chung98a-u}.
}
\ea
\gll Mary-ka           ku   chayk-ul          ilk-ko  	         iss-ta.\\ 
     Mary-\textsc{nom} the  book-\textsc{acc} read-\textsc{conn} be.in.the.process.of-\textsc{decl}\\
\glt `Mary is in the process of reading the book.'\label{GSexemple57added-a}
\z
\ea
\label{GSexemple62added}
Lexical item of \emph{issta} `be in the process of': \\
\avmtmp{
[\form  < iss-ta >\\
 head & [\type*{verb}\\
           aux $+$]\\
 subj & \ibox{1}\\
 comps& \2 \+ < V[vform & ko\\
            subj  & \1\\
            comps & \2\\
            light & $+$] > ]}
\z
Auxiliaries attract both the subject (\ibox{1}) and the complements of their verbal complement (list \ibox{2}). The subject value is indicated as \ibox{1}, rather than \la\ibox{1}\ra, because the subject is not always realized in Korean. 
To indicate which ending it imposes on its complement, we use the feature \textsc{vform}, thus
allowing for the selection of the appropriate ending by the auxiliary (\citealt{Chung98a-u},
\citealt{Kim2016a-u}). So, the verb \emph{issta} selects the ending \emph{-ko} for the verbal
complement, and \emph{ilkko} `read', whose \textsc{vform} value is \emph{ko}, is appropriate. 

% \begin{exe}
%     \ex {Description of \emph{issta}: \\
%     \begin{avm}
%       {\[form \<\normalfont{iss-ta}\>\\
%       head \[\normalfont{\emph{verb}}\\
%            aux +\]\\
%       arg-st \ibox{1} $\oplus$ \< v
%                     \[vform \normalfont{\emph{ko}}\\
%                     arg-st \ibox{1} $\oplus$ \ibox{2}\\
%                     light $+$\\\]\,\> \,\,$\oplus$ \ibox{2}\]}
%     \end{avm}}\label{GSexemple62added}
% \end{exe}



The verbal complex is headed by an auxiliary verb, which is [\aux~$+$], while other verbs are [\aux~$–$].  Thus only auxiliaries can enter this structure. The schema for the verbal complex is given in (\ref{GSexemple57}). The verbal complex is [\light~$+$] and made up of two verbs, also [\light~$+$] (see Section~\ref{GSsection3.3}). 


%%%%%%%%%%%%%%%%%%%%%%%%%%%%%%%%%%%%%%%%%%%%%%%%%%%%%%%%%%%%%%%%%%%%%%%%%%%%%%%%%%%%%%%%%%%%%%%%%%%%%%%%%%%%%%%%%%%%%%%%%%%%%%%%%%%%%%%%%%%%%%%%%%%%%%%%%%%%%%%%%%%%%%%%%%%%%%%%%%%%%%%%%%%%%%%%%%%%%%%%%%%%%%%%%%%%%%%%%%%%%%%%
%%%%%%%%%%%%%%%%%%%%%%%%%%%%%%%%%%%%%%%%%%%%%%%%%%%%%%%%%%%%%%%%%%%%%%%%%%%%%%%%%%%%%%%%%%%%%%%%%%%%%%%%%%%%%%%%%%%%%%%%%%%%%%%%
%\newpage
\eas 
\label{GSexemple57}
\type{verbal-complex-phrase} (Korean) \impl \\
\avmtmp{
[synsem|loc|cat & [%head & [\type*{verb}\\   superfluous St.Mü. 03.08.2020
                   %        aux +]\\
%                   subj  & \1\\
                   comps & \1\\
                   light & $+$]\\
      \punk{head-dtr|synsem|loc|cat}{  [head  & [\type*{verb}\\
                                          aux +]\\
%                                 subj  & \1\\
                                 comps & \1 \+ < \2 >
                                 %light & $+$ redundant St.Mü. 07.08.2020
]}\\
      non-head-dtrs & < [synsem \2] %[head  & verb\\ superfluous St.Mü. 03.08.2020
                           % subj  & \1\\
                           % comps & \3\\
                           % light & $+$] 
> ]}
\zs
The verbal complex schema saturates the last element of the \compsl of the head daughter. In this
way it is parallel to the head-subject-complements phrase. The only difference is that the argument
that is combined with the auxiliary may be \light$+$ as is required by the auxiliary. The \subjl is
not mentioned in the constraints on \type{verbal-complex-phrase} but that the \subjv of the head
daughter is identical to the \subjv of the mother follows from constraints on more general types
that are inherited \crossrefchapterp[\page \pageref{prop:valence-principle}]{properties}. 



The structure of sentence (\ref{GSexemple57added-a}) is represented in Figure~\ref{GSfigure12}.


%The auxiliary \emph{iss-} in (\ref{GSexemple55a}), (\ref{GSexemple56a}) takes as its complement a light verb which is constrained to end in \emph{–ko} (different auxiliaries put different restrictions on their verbal complement, \citealt{Yoo2003}), from which it attracts all arguments. 

%We follow \cite{Chung98a-u} in assuming that the sentence in Korean has a flat structure, which is allowed by the Head-subject-complements-phrase in (\ref{GSexemple52}). Sentence (\ref{GSexemple50a}) is represented in Figure~\ref{GSfigure12}.

%%%%%%%%%%%%%%%%%%%%%%%%%%%%%%%MAS LARG%%%%%%%%%%%%%%%%%%%%%%%%%%%%%%%%%%
\begin{figure}
    \centering
\begin{forest}
sm edges
 [S [\ibox{1} NP
            [Mary-ka;Mary-\textsc{nom}]]
 [\ibox{2} NP
            [ku chayk-ul;the book-\textsc{acc}, roof]]
  [V \avmtmp{
           [aux & $+$\\
            subj  & < \1 >\\
            comps & < \2 >\\
            light & $+$]}, before computing xy={s'+=15pt} 
    [\ibox{3} V \avmtmp{
                [subj & < \1 >\\
                 comps & < \2 >\\
                 light & $+$]} [ilk-ko;read-\textsc{conn}]]
        [V \avmtmp{[aux & $+$\\
                    subj & < \1 >\\ 
                    comps & < \2, \3 >\\
                 light & $+$] }
             [iss-ta;be.in.the.process.of-\textsc{decl}]]]] 
\end{forest} \caption{Clause structure with a verbal complex in Korean}
    \label{GSfigure12}
\end{figure}{}

The structure of (\ref{GSexemple57added-b}), with a series of two auxiliaries, is represented in
Figure~\ref{GSfigure13} (adapted from \citealt[171]{Chung98a-u}). 
\ea{
 	\gll Mary-ka            ku   chayk-ul           ilk-ke  	         po-ko               iss-ta.\\ 
 		{Mary-\textsc{nom}} the {book-\textsc{acc}} {read-\textsc{conn}} {try-\textsc{conn}} {be.in.the.process.of-\textsc{decl}}\\
 	\glt `Mary is in the process of giving the book a trial reading.'}\label{GSexemple57added-b}
\z
The verb \emph{issta} `be in the process of' takes as its complement the verbal complex \emph{ilke poko} `try to read', whose head is \emph{poko} `try'. The verb \emph{poko}, being an auxiliary like \emph{issta}, takes as its complement the verb \emph{ilke}, attracting its subject and complements, which are transmitted to the verbal complex \emph{ilke poko}; \emph{ilke poko} saturates the verbal complement expected by \emph{issta}, and transmits the subject and complements to the head auxiliary (see (\ref{GSexemple62added})).

%\eal
%	\label{GSexemple58} 
%	\ex[]{
%	\gll Mary-ka            ku  chayk-ul            ilk-e                po-ko               iss-ta.\\ 
%		{Mary-\textsc{nom}} the {book-\textsc{acc}} {read-\textsc{part}} {try-\textsc{part}} {be.in.the.process.of-\textsc{decl}}\\
%	\glt `Mary is giving the book a trial reading.'}\label{GSexemple58a}
		
%	\ex[]{
%	\gll Ku  chayk-ul            Mary-ka             ilk-e                po-ko               iss-ta.\\ 
%		 the {book-\textsc{acc}} {Mary-\textsc{nom}} {read-\textsc{part}} {try-\textsc{part}} {be.in.the.process.of-\textsc{decl}}\\
%	\glt `Mary is giving the book a trial reading.'}\label{GSexemple58b}
%\zl

%%%%%%%%%%%%%%%%%%%%%%%%%%%%%%%MAS LARG%%%%%%%%%%%%%%%%%%%%%%%%%%%%%%%%%%

\begin{figure}
    \centering
    {\footnotesize
\begin{forest}
	sm edges
 [S [\ibox{1} NP [Mary-ka;Mary-\textsc{nom}]]
 [\ibox{2} NP [ku chayk-ul;the book-\textsc{acc}, roof]]
  [V \avmtmp{[aux & $+$\\
            subj & < \1 >\\
            comps & < \2 >\\
            light & $+$]}, before computing xy={s'+=15pt}
    [\ibox{4} V \avmtmp{
            [aux & $+$\\
             subj & < \1 >\\
             comps & < \2 >\\ 
             light & $+$]} [\ibox{3} V \avmtmp{
            [subj & < \1 > \\
             comps & < \2 >\\
             light & $+$]} 
            [ilk-e;read-\textsc{conn}]]
            [V \avmtmp{
            [aux & $+$\\
             subj & < \1 >\\
             comps & < \3, \2 >\\
             light & $+$]} 
            [po-ko;try-\textsc{conn}]]]
    [V \avmtmp{
        [aux & $+$\\
         subj & < \1 >\\
         comps & < \4, \2 >\\
         light & $+$]}[iss-ta;be.in.the.process.of-\textsc{decl}]]]] \end{forest}}
    \caption{Clause structure with verbal complexes in Korean}
    \label{GSfigure13}
\end{figure}

%%%%%%%%%%%%%%%%%%%%%%%%%%%%%%%MAS LARG%%%%%%%%%%%%%%%%%%%%%%%%%%%%%%%%%%

%Contrary to auxiliaries, control verbs such as \emph{seltukha-} (\ref{GSexemple51a}), (\ref{GSexemple51b}) or \emph{cisi-} (\ref{GSexemple54}) can be separated from their verbal complement, for instance by an adverb as in (\ref{GSexemple59a}). They also allow for the infinitive and its complement to form a VP constituent as in (\ref{GSexemple59b}), where it is an afterthought. Thus, control verbs are analyzed in the same way as Italian restructuring verbs: they either take a saturated VP complement (\ref{GSexemple59a}) (\ref{GSexemple59b}), or are the head of a complex predicate (\ref{GSexemple59c}) (\ref{GSexemple59d}). They contrast with Korean Raising verbs which only take a VP complement (examples (\ref{GSexemple59a}), (\ref{GSexemple59b}) from \citealt[162]{Chung98a-u}, (\ref{GSexemple59c}) from \citealt[190]{Chung98a-u}, (\ref{GSexemple59d}) from \citealt[182]{Chung98a-u}).


%
%\eal
%	\label{GSexemple59} 
%	\ex[]{
%	\gll Mary-ka            ku  mwuncey-lul            phwullye-ko            (kkuncilkikey)		   sito-hayss-ta.\\ 
%		{Mary-\textsc{nom}} the {problem-\textsc{acc}} {solve-\textsc{part}} \spacebr{}ceaselessly {try-\textsc{past}-\textsc{decl}}\\
%	\glt `Mary tried (ceaselessly) to solve the problem.'}\label{GSexemple59a}
%		
%    \ex[]{
%	\gll Mary-ka             sito-hayess-ta,          [ku           mwuncey-lul            phwullye-ko].\\ 
%		 {Mary-\textsc{nom}} {try-\textsc{part-decl}} \spacebr{}the {problem-\textsc{acc}} {solve-\textsc{part}}\\
%	\glt `Mary tried to solve the problem.'}\label{GSexemple59b}
%		
%	\ex[]{
%	\gll Ku  chayk-ul            Mary-ka             ilkulye-ko            sito-hayss-ta.\\ 
%		 the {book-\textsc{acc}} {Mary-\textsc{nom}} {read-\textsc{part}} {try-\textsc{past}-\textsc{decl}}\\
%	\glt `Mary tried to read the book.'}\label{GSexemple59c}
%		
%    \ex[]{
%	\gll Ku  chayk-ul            Mary-ka             ilkulye-ko            John-hanthey        seltukha-yss-ta.\\ 
%		 the {book-\textsc{acc}} {Mary-\textsc{nom}} {read-\textsc{part}} {John-\textsc{dat}} {persuade-\textsc{past}-\textsc{decl}}\\
%	\glt `Mary persuaded John to read the book.'}\label{GSexemple59d}
%\zl
%
%The scrambling data, together with the possibility of long passivization (\ref{GSexemple54}), show that there is a complex predicate. The subject of the head verb occurs between the complement of the infinitive and the infinitive in (\ref{GSexemple59c}), (\ref{GSexemple59d}). More precisely, there is no verbal complex in this case: the two verbs do not have to be contiguous, but can be separated, for instance by a complement as in (\ref{GSexemple59d}) \emph{(John-hanthey)}. Thus, they are the head of a flat structure as in Figure~\ref{GSfigure14} corresponding to (\ref{GSexemple59d}) (see \citealt[190]{Chung98a-u}). The head of verbal complexes in Korean is [AUX +], see (\ref{GSexemple57}). Since control verbs are [AUX –], they cannot enter verbal complexes, and the structure for complex predicates whose head is a control verb is a flat structure, corresponding to the Head-subject-complements-phrase in (\ref{GSexemple52}). 
%
%
%
%%%%%%%%%%%%%%%%%%%%%%%%%%%%%%%%MAS LARG%%%%%%%%%%%%%%%%%%%%%%%%%%%%%%%%%%
%
%\begin{figure}
%    \centering
%    {\small
%\begin{forest}
%sm edges
% [S
% [\ibox{1} NP [Ku chayk-ul;the book-\textsc{acc}, roof]]
% [NP [Mary-ka;Mary-\textsc{nom}]]
%  [\ibox{3} V \ms{
%            light & $+$\\
%            comps & \liste{ \ibox{1} }} 
%    [ilk-ulako;read-\textsc{part}]] 
%  [\ibox{2} NP[John-hanthey;John-\textsc{dat}]]
%  [\ibox{3} V \ms{
%            light & $+$\\
%            comps & \liste{ \ibox{2}, \ibox{3}, \ibox{1} }} 
%    [seltukha-yss-ta;persuade-\textsc{past}-\textsc{decl}]]] 
%    \end{forest}}
%    \caption{Clause structure with a control verb in Korean.}
%    \label{GSfigure14}
%\end{figure}{}
%
%%%%%%%%%%%%%%%%%%%%%%%%%%%%%%%%MAS LARG%%%%%%%%%%%%%%%%%%%%%%%%%%%%%%%%%%
%
%Control verbs which may be the head of a complex predicate are subject or object control verbs. They are the target of an argument attraction rule, in a way parallel to Italian Restructuring verbs. 
%
%\begin{exe}
%	\ex {Argument Attraction Rules for Korean control verbs}\label{GSexemple60} 
%	\begin{xlist}
%        \ex[]{Subject control verbs \\
%	{\small
%        \begin{avm}
%{\[head & \[{\normalfont{\emph{verb}}}\\
%		            aux-\]\\
%        subj & \<np{\normalfont{\emph{\textsubscript{i}}}}\>\\
%        comps & \<v \[subj & \<np{\normalfont{\emph{\textsubscript{i}}}}\>\\
%                    comps & \<\> \] \,\> \\
%                    light & $+$\]} \end{avm}
%          	$\Rightarrow$
%        \begin{avm}
%		{\[comps & \<v
%		            \[comps & \ibox{2}\\
%		            light & $+$\] \,\> \,$\oplus$ \ibox{2}\]}
%          	\end{avm}}}
%	
%	    \ex[]{Object control verbs \\
%    {\footnotesize
%        \begin{avm}
%		{\[head & \[{\normalfont{\emph{verb}}}\\
%		            aux-\]\\
%        subj & \<np{\normalfont{\emph{\textsubscript{i}}}}\>\\
%        comps & \<np{\normalfont{\emph{\textsubscript{j}}}}, v \[subj & \<np{\normalfont{\emph{\textsubscript{j}}}}\>\\
%                    comps & \<\> \] \,\>\\
%                    light & $+$\]}
%          	\end{avm}
%          	$\Rightarrow$
%        \begin{avm}
%		{\[comps & \<np{\normalfont{\emph{\textsubscript{j}}}}, v
%		            \[comps & \ibox{2}\\
%		            light & +\] \,\> \,$\oplus$ \ibox{2}\]}
%          	\end{avm}}} \end{xlist}
%\end{exe}
%
%An example with the verb \emph{seltukha-} is given in Figure~\ref{GSfigure14}, which represents (\ref{GSexemple59d}).
%


The head comes last in Korean, except in the afterthought construction exemplified in (\ref{GSexemple56a}), which requires an additional mechanism. Constraint (\ref{GSexemple61}) mirrors constraint (\ref{GSexemple29}) for Romance languages.

\ea
\label{GSexemple61}
    \begin{avm}
		\[synsem \ibox{1}\]~ <  \[comps\, \<\ldots{}, \ibox{1}, \ldots{}\>\,\]
	\end{avm}\jambox*{(\ili{Korean})}
\z

This constraint holds for the verbal complex, in which the head verb follows the complement verb.


%\cite{Chung98a-u} extends the possibility of argument attraction to adjuncts, as well as to constructions with a S complement, although in the latter case the data are somewhat marginal. The behavior of adjuncts is easily accounted for, if adjuncts can be treated as complements \citep{BMS2001a} by a verb. For the second case, Chung proposes a flat structure, in which the subject, the complements and the (lexical) verbs are all sisters. In this analysis, the definition of a complex predicate, which relies on a syntactic relation between words can then be maintained (but see \citealt{lee2001argument} for a different proposal based on linearization).

\section{Light verb constructions in Persian: syntax and morphology, syntax and semantics}\label{GSsection5}

Light verb constructions constitute the third guise of complex predicates. They are characterized semantically: the verb and the second predicate constitute together a semantic predicate. For instance, the French expression combining a semantically light verb and a noun \emph{faire une proposition} `to make a proposal’ is close to \emph{proposer} `to propose’. They have been studied in HPSG for Korean \citep{Ryu:93, lee2001argument, choi2001mixed, Kim2016a-u}. We focus here on Persian light verb constructions, which form a rich class and tend to replace simplex verbs.   
    
\subsection{What are complex predicates in Persian?}\label{GSsection5.1}

Persian simplex verbs constitute a small closed class of about 250 members, only around 100 of which are commonly used. Speakers resort to complex predicates, sequences of a light verb and a preverbal element belonging to various categories (adjective, noun, particle, prepositional phrase). Following \cite{bonami2010persian} and \cite{pollet2012grammaire}, such sequences are ``multi-word expressions'', that is, they are made up of several words, which, together, form a lexeme. 

Several properties show that the elements are independent syntactic units \citep{Karimi-Doostan97a, Megerdoomian2002a, pollet2012grammaire}. We concentrate on noun + verb combinations, i.e.\ complex predicates in which the preverbal elements are nouns. In what follows, we simply refer to these nominal elements in the complex predicates as ``nouns''.
All inflection is prefixed or suffixed on the verb, as is the negation in (\ref{GSexemple62}), and never on the noun, i.e.\ the nominal part of the complex predicate. The two elements can be separated by the future auxiliary, or even by clearly syntactic constituents, like the complement PP in (\ref{GSexemple62}). Both the noun and the verb can be coordinated, as shown in (\ref{GSexemple63}) and (\ref{GSexemple64}) respectively (from \citealt[3]{bonami2010persian}), where the coordinations are indicated by the brackets. 
The noun can be extracted, as in the topicalization in (\ref{GSexemple65}), where the sign -- indicates where the non-extracted noun would have occurred. 
The fact that the noun is linked to a position belonging to a verbal complement (indicated by the brackets) shows that this is extraction, and not simply variation in order.
Complex predicates can also be passivized. In this case, the nominal element of the complex predicate (\emph{tohmat} `slander' in (\ref{GSexemple66a})) becomes the subject of the passive construction (\ref{GSexemple66b}), as does the object of a transitive construction (from \citealt[251]{pollet2012grammaire}). The nominal part of the complex predicate is italicized in the examples.

\ea{
	\label{GSexemple62}
	\gll \emph{Dast} be gol-h\=a             na-zan.\\
		 hand          to {flower-\textsc{pl}} {\textsc{neg}-hit}\\
	\glt `Don't touch the flowers.'}
\z

\ea{
	\label{GSexemple63}
	\gll Mu-h\=a=y\=a\v s=r\=a    [\emph{boros} y\=a \emph{\v s\=ane}] zad.\\
		 hair-\textsc{pl=3sg=ra} \spacebr{}brush or            comb                hit\\
	\glt `He/she brushed or combed his/her hair.'}
\z

\ea{
	\label{GSexemple64}
	\gll Omid \emph{sili} [zad          va  xord].\\
		 Omid slap          \spacebr{}hit and stroke\\
	\glt `Omid gave and received slaps.'}
\z

\ea{
	\label{GSexemple65}
	\gll \emph{Dast} goft=am           [be          gol-h\=a             -- \hspace{.3em} na-zan].\\
	     hand        said=\textsc{1sg} \spacebr{}to {flower-\textsc{pl}} {} \hspace{.3em} {\textsc{neg}-hit}\\
	\glt `I told you not to touch the flowers.'}
\z

%\inlinetodostefan{There is a mistake in (\ref{GSexemple66b}) I do not know how to fix.}
\eal
	\label{GSexemple66} 
	\ex{
	\gll Maryam be Omid \emph{tohmat} zad.\\ 
		 Maryam to Omid slander         hit\\
	\glt `Maryam slandered Omid.'}\label{GSexemple66a}
		
    \ex{
	\gll Be Omid \emph{tohmat} zade \v                 sod.\\ 
		 to Omid  slander        hit.\textsc{pst.ptcp} become\\
	\glt `Omid was slandered.'}\label{GSexemple66b}
\zl

There is evidence that the verb and the nominal element in a complex predicate share one argument structure. In (\ref{GSexemple67a}), the verb \emph{d\=adan} `give' takes two complements, the noun \emph{\=ab} `water' and the PP \emph{be b\=aq\v ce} `to garden', while in (\ref{GSexemple67b}) the combination of \emph{d\=adan} and the noun \emph{\=ab} takes a direct object, which is marked with \emph{=r\=a}: in (\ref{GSexemple67b}), the noun \emph{\=ab} and the verb \emph{d\=ad} `gave' form a complex predicate.  

\eal 
	\label{GSexemple67} 
    \ex{
	\gll Maryam be b\=aq\v ce \emph{\=ab}  d\=ad.\\ 
	     Maryam to  garden      water gave\\
	\glt `Maryam watered the garden.'}\label{GSexemple67a}
		
	\ex{
	\gll Maryam b\=aq\v ce=r\=a    \emph{\=ab} d\=ad.\\ 
	     Maryam garden=\textsc{ra} water         gave\\
	\glt `Maryam watered the garden.'}\label{GSexemple67b}
\zl

Other properties show that the combination of the two elements, here a noun and a verb, behaves like a lexeme \citep{bonami2010persian}. Such combinations feed lexeme formation rules: for instance, the suffix \emph{-i} forms adjectives from verbs: \emph{xordan} `eat' > \emph{xordani} `edible', and the same is possible with complex predicates, as shown in (\ref{GSexemple68}); perfect participles can be converted into adjectives by adding the suffix \emph{-e}, and this also applies to complex predicates, as shown in (\ref{GSexemple69}) (see also Section~\ref{GSsection5.2}; from \citealt[5]{bonami2010persian}).

\eal 
	\label{GSexemple68} 
    \ex[]{
	\gll dust  d\=a\v stan  >  \hspace{.3em} dustd\=a\v stani\\ 
		friend {have (`love')} {} \hspace{.3em} lovely\\
	}\label{GSexemple68a}
		
    \ex[]{
	\gll xat     xordan                  >  \hspace{.3em} xatxordani\\ 
		 scratch {strike (`be scratched')} {} \hspace{.3em} scratchable\\
	}\label{GSexemple68b}
\zl

\eal
	\label{GSexemple69} 
	\ex[]{
	\gll dast xordan                   >  \hspace{.3em} dastxorde\\ 
		 hand {strike (`be sullied')} {} \hspace{.3em} sullied\\
	}\label{GSexemple69a}
		
	\ex[]{
	\gll xat     xordan                   > \hspace{.3em} xatxorde\\ 
		 scratch {strike (`be scratched')} {} \hspace{.3em} scratched\\
	}\label{GSexemple69b}
\zl

The meaning of the complex predicate is often a specialization of the predictable meaning of the combination: \emph{dast d\=adan} (lit.\ `hand give') means `shake hands', \emph{\v c\=aqu zadan} (lit.\ `knife hit') means `stab', \emph{\v s\=ane zadan} (lit.\ `comb hit') means `comb'. Each specialized meaning has to be learned in the same way as that of a lexeme. Analogy often plays a role in the creation of new lexemes, and this is also true of complex predicates. For instance, the family of complex predicates expressing manners of communication goes from \emph{telegr\=am zadan} `telegraph', where hitting \emph{(zadan)} is involved, to cases where hitting is not clearly involved: \emph{telefon zadan} `phone’, \emph{imeyl zadan} `email', \emph{esemes zadan} `text', etc.   

These complex predicates raise two problems: a morpho-syntactic one and a semantic one. To solve them, we rely crucially on the same property of HPSG as in the preceding syntactic cases, that is, the view of heads as sharing information with their expected complements. 

\subsection{Complex predicates and derivational processes}\label{GSsection5.2}

Although Persian complex predicates are combinations of words, they feed some derivational rules; see Section~\ref{GSsection5.1}, examples (\ref{GSexemple68}) and (\ref{GSexemple69}). We analyze here the case of agent nominalization, studied in \cite{MuellerPersian}.\footnote{Müller’s analysis adopts a slightly different approach to the issues discussed in this section.} The example he examines is especially interesting in that the nominalization does not exist independently of the complex predicate: as shown in (\ref{GSexemple70}), although no agent noun can be derived from the verb \emph{kon} `do', an agent noun can be derived from the complex predicate formed with the verb \emph{kon} and the adjective \emph{b\=az} `open'.

\eal 
	\label{GSexemple70} 
    \ex[]{
	\gll kon  \hspace{.3em} >  \hspace{.3em} * konande\\ 
		do \hspace{.3em} {} \hspace{.3em} {} {Intended: doer}\\
	}\label{GSexemple70a}
		
    \ex[]{
	\gll b\=az kon \hspace{.3em} >  \hspace{.3em} b\=az-konande\\ 
		 open do \hspace{.3em} {} \hspace{.3em} opener\\
	}\label{GSexemple70b}
\zl

The lexeme \emph{b\=az-konande} `opener' can be analyzed as a compound lexeme to which the suffix \emph{-ande} is added, indicating agent nominalization. Compound lexemes are made of two lexemes. A simple rule for noun-noun compounds is given in (\ref{GSexemple71}) \citep[178]{bonami2018lexeme}, where the elements of the compound are the value of the feature \textsc{m-dtrs} (morphological daughters):\footnote{For a similar approach, see \cite{Orgun96a}, \cite{Riehemann98a}, \cite{Koenig99a} and \cite{sag2003syntactic}; for more discussion on morphology in HPSG, see \crossrefchapteralt{morphology}.}

\begin{exe}
        \ex[]{
        \begin{avm}\label{GSexemple71}
		{\[{\normalfont{\emph{lexeme}}} \\
		phon \ibox{1} $\oplus$ \ibox{2}\\ 
		synsem | loc | cat |  head  \normalfont{\emph{noun}}\\
        m-dtrs \< \,\[{\normalfont{\emph{lexeme}}} \\
                   phon \ibox{1} \\ 
                    synsem | loc | cat | head {\normalfont{\emph{noun}}} \] \,, \[{\normalfont{\emph{lexeme}}} \\
                   phon \ibox{2} \\ 
                    synsem | loc | cat | head {\normalfont{\emph{noun}}} \] \,\>\\\]} \end{avm}}
\end{exe}

Similarly, a noun can be formed from the elements of the complex predicate, in this instance an adjective and a verb. The verb \emph{kon} in the complex predicate \emph{b\=az kon} `to open' is described in (\ref{GSexemple72}). It expects a subject NP, the agent, and two complements, an adjective and an NP, the latter being attracted from the adjective. The content of the adjective is included in the content of the verb, as the nucleus of the caused \emph{soa} (state of affairs) `make something be adj' (see \citealt[642]{MuellerPersian}).

\ea
\label{GSexemple72}
\avmtmp{
[cat|head & \type{verb}\\
 arg-st & < NP$_k$, NP$_j$, Adj[ prd & $+$  \\
			                                   arg-st & < NP$_j$ >\\
			                                   cont   & \1 [\type*{open-relation} \\
		                                                        theme & j ] ] >\\
 cont  & [\type*{soa}\\
          nucleus & [\type*{cause-relation}\\
		     causer & k\\
		     soa-arg|nucleus \1 ]]]
}
\z

The compound lexeme \emph{b\=az-konande} is made of the adjective and the verb, which are the morphological daughters, very similar to what they are in the complex predicate. The verbal element is expecting two complements, an adjective and an NP, and they have the same semantics as in the complex predicate: the verb denotes a cause relation taking as argument the adjective content, and the adjective content is a relation taking the nominal complement as its argument (\textsc{ss} abbreviates \synsem). 

\begin{exe}
        \ex[]{
            {\footnotesize
        \begin{avm}\label{GSexemple73}
		{\[\rm\emph{lexeme}\\
		phon \ibox{1} $\oplus$ \ibox{2} $\oplus$ \< \rm\emph{ande}\> \\ 
		synsem \[cat \[head \rm\emph{noun}\\
					comps \<\ibox{3} np\>\]\\
			       cont [ind \rm\emph{k}] \]\\
        m-dtrs \<\,\[phon \ibox{1} \\ 
                    ss \ibox{4} \[ cat \rm\emph{adjective}\\
                    			arg-st \< \ibox{3} np$_j$\> \\
					cont \ibox{5} \rm\emph{adj-rel (j)}\] 
                    \] \,, \[phon \ibox{2} \\ 
                    ss \[ cat \rm\emph{verb}\\
                    			arg-st \< np$_k$, \ibox{3}, \ibox{4}\>\\
					cont | nucl \[\rm\emph{cause-rel}\\
								causer \rm\emph{k}\\
								soa-arg | nucl \ibox{5}\]\]\]\,\>\\\]}\end{avm}}}
\end{exe}

As illustrated in (\ref{GSexemple74}), the agent noun keeps the NP expected by the verb (indicated by the brackets) as a complement.   

\begin{exe}
	\ex[]{
		\gll [dar-e botri] b\=az-konande\\
		     \spacebr{}lid-\textsc{ez} bottle opener\\
		\glt `a bottle opener'}\label{GSexemple74}

\end{exe}

This compound nominalization is accompanied by the appropriate changes: the noun denotes the causer, the first argument of the verb m-daughter, and a derivational suffix \emph{(-ande)} is appended to the sequence of the two elements. Nothing in the rule requires that the agent noun (\emph{*kon-ande}) exist independently of the elements of the complex predicate. Hence, the data in (\ref{GSexemple70a}) are accounted for.

\subsection{The semantics of light verb constructions}\label{GSsection5.3}

In complex predicates, the noun is not referential; rather, it participates in the meaning of the verbal combination. However, in general, these nouns may also be used as ordinary referential nouns. We assume that such nouns come in two guises: predicative, noted [PRD $+$], which occur in complex predicates, and as referential nouns, noted [PRD $–$].

These complex predicates do not have a homogeneous semantics. The general idea is that the verb serves to turn a noun into a verb \citep{bonami2010persian}, but there is a spectrum, going from a (relatively) semantically compositional combination, to idioms whose semantics is not predictable from the components. Complex predicates exploit different schemas, which can be extended to new nouns, describing new situations. We will exemplify certain common cases, drawing on the detailed study of \emph{zadan} `to hit' in \cite{pollet2012grammaire}. The uses of \emph{zadan} as a light verb are numerous and varied. We will not try to investigate them exhaustively; rather, we indicate different patterns for the combination of this verb with the noun. 

The semantics of a complex predicate is often a specialization of that of the simplex verb. For instance, \emph{lagad zadan} (lit.\ kick hit) means `kick', and \emph{sili zadan} (lit.\ slap hit) means `slap'. 

\ea[]{
	\label{GSexemple75}
	\gll Ol\=aq be Omid lagad zad.\\
		 donkey to Omid kick  hit\\
	\glt `The donkey kicked Omid.'}
\z

Within a hierarchical organization of the lexicon \crossrefchapterp{lexicon}, the content of the
simplex verb is higher and less specialized than that of the predicative noun. Thus, the content of
the complex predicate can be simply that of the noun. This is reminiscent of the way
\cite{Wechsler1995c} represents the import of a PP with a verb like \emph{talk}; the verb content
itself is represented as an \emph{soa} with one participant, the talker; the verb can take a number
of PP complements (headed by \emph{to}, \emph{about}, \ldots), which add semantic information describing the situation. The result is a description of an \emph{soa} which combines partial descriptions. Similarly here, the combination of the two contents is identical to the content of \emph{lagad} `kick', as that latter content is more specialized than that of \emph{zadan}. The complement of the complex predicate may be an NP or a PP headed by \emph{be} `to' (the preposition is optional).


\begin{exe}
        \ex[]{
        \begin{avm}\label{GSexemple76}
		{\[\rm \emph{zadan1-lexeme}\\
		  cat \rm \emph{verb}\\
		  arg-st \<np{\rm\textsubscript{\emph{k}}}, {\rm \emph{(be)}} np{\rm \textsubscript{\emph{m}}}, n 
		  	\[cat [prd $+$]\\
			  cont \ibox{1}\]\,\>\\
		  cont \[\rm \emph{soa}\\
		        nucleus \ibox{1} \[\rm \emph{kick-relation}\\
		            agent \rm \emph{k}\\
		            undergoer \rm \emph{m}\]\]\]}\end{avm}}
\end{exe}

Another case where the combination gives more information than the simplex verb is when this verb takes as its predicative complement a noun denoting an instrument crucially involved in the situation \citep{bonami2010persian}. Such are, in different domains, \emph{\v c\=aqu zadan} (lit.\ knife hit) `stab', \emph{telefon zadan} (lit.\ phone hit) `phone', \emph{pi\=ano zadan} (lit.\ piano hit) `play the piano'. We illustrate here \emph{\v s\=ane zadan} (lit.\ comb hit) `comb'.    

\ea[]{
	\label{GSexemple77}
	\gll Maryam {mu-h\=a=ya\v s=r\=a}     \v s\=ane zad.\\
		 Maryam {hair-\textsc{pl=3sg=ra}} comb      hit\\
	\glt `Maryam combed her hair.'}
\z

%\inlinetodostefan{Stefan: Avoid bold weherever possible}
\begin{exe}
        \ex[]{\begin{avm}\label{GSexemple78}
		{\[\rm \emph{zadan2-lexeme}\\
		synsem \[cat {\normalfont{\emph{verb}}}\\
		  arg-st \<np{\rm \textsubscript{\emph{k}}}, np{\rm \textsubscript{\emph{m}}}, n 
		  	\[cat [prd $+$]\\
			  cont \ibox{2}\]\,\>\\
		  cont \ibox{2} \[\rm \emph{soa}\\
		  	sit \ibox{1} \\
		        nucleus \[\rm \emph{comb-relation}\\
		            agent \rm \emph{k}\\
		            undergoer \rm \emph{m}\]\]\]\\
		 background \,\{\rm\textsf{involves (\ibox{1}, $\exists$ x} [\textsf{comb (x)} \,$\wedge$ \textsf{use (\ibox{1}, k, x)]}\}\]}
          	\end{avm}}
\end{exe}

The condition in the background can be read as follows: the situation \ibox{1} involves that there exists a comb and that the agent \emph{k} uses it in that situation. Although the complex predicate includes the content of the predicative complement, the meaning of the complex predicate does not reduce to that of its semantically more specialized member, as in the preceding case, but adds a restriction on the background: the existence of an object and the fact that, in the situation, such an object is used (see \citealt[10]{bonami2010persian}).  

Further from a compositional or recoverable meaning is the use of \emph{zadan}, or more precisely \emph{xod=r\=a zadan} (lit.\ self hit), with a series of nouns denoting illnesses, handicaps or problematic states (like stupidity, ignorance, etc.): it means `to pretend, feign' the illness or state in question (example (\ref{GSexemple79}) from \citealt[223]{pollet2012grammaire}).

\ea[]{
	\label{GSexemple79}
	\gll Maryam {xod=r\=a}         be div\=anegi zad.\\
		 Maryam {self=\textsc{ra}} to madness    hit\\
	\glt `Maryam feigned madness.'}
\z

This use of \emph{zadan} may be seen as an extension of its use with nouns denoting some sort of deceit, such as \emph{gul zadan} (lit.\ deceit hit) `to deceive’: as in (\ref{GSexemple76}), the noun imposes its content on the combination, with a metaphorical use of the verb, retaining from the physical violence meaning of \emph{zadan} `hit’ the idea of an action to the detriment of someone. Nevertheless, nothing in the actual combination in (\ref{GSexemple79}) indicates deception. Not all nouns for illnesses are acceptable, only those which cannot really be verified in the situation: a state of fatigue, but not a heart attack. We group them as \emph{internal-problematic-states}. Here the combination of the verb and the noun is standard, in that the noun is a semantic argument of the verb, but the meaning of the verb is unpredictable.

\begin{exe}
        \ex[]{{\small
        \begin{avm}\label{GSexemple80}
		{\[\rm \emph{zadan3-lexeme}\\
		cat \rm \emph{verb}\\
		  arg-st \<np\rm \textsubscript{\emph{k}}, pro\rm \textsubscript{\emph{k}}, \textsc{pp} \[cat \[pform be\\
		  prd $+$ \\ \] \\
		  cont \ibox{1} | nucleus \[\rm \emph{internal-problematic-state} \\
		            experiencer \rm \emph{k}\\\]\\\] \,\>\\
		  cont \[\rm \emph{soa}\\
		        nucleus \[\rm \emph{pretend-relation}\\
		            agent \rm \emph{k}\\
		            theme \ibox{1}\\ \]\\\]\\\]}
          	\end{avm}}}
\end{exe}

Note that, contrary to \emph{zadan1-lexeme}, with which \emph{be} `to' is optional, the \emph{zadan3-lexeme} requires the predicative complement to be in fact a PP, headed by \emph{be}. We assume that the preposition \emph{be} (frequent in the complement of a complex predicate) is contentless and shares syntactic (the [\prd $\pm$] value) and semantic information with its complement, the predicative N ([\cont \ibox{1}]); this is indicated by treating \emph{be} as the value of the feature \textsc{p(reposition) form} \citep[Chapter~3]{ps}.  

Finally, we turn to an idiom: \emph{dast zadan} (lit.\ hand hit) meaning `start'. The combination may mean, in a more recoverable way, `to touch' with PP complements denoting concrete objects (as in (\ref{GSexemple62})), or `to applaud' with a PP complement denoting a person (\ref{GSexemple81a}) (from \citealt[45]{pollet2012grammaire}). However, it means `to start' with a PP complement denoting an event as in (\ref{GSexemple81b}) (from \citealt[185]{pollet2012grammaire}).

\eal
	\label{GSexemple81} 
	\ex[]{
	\gll Bar\=a=ya\v s     xeyli dast zad-im.\\ 
		{for=\textsc{3sg}} a.lot hand {hit-\textsc{1pl}}\\
	\glt `We applauded him a lot.'}\label{GSexemple81a}
		
    \ex[]{
	\gll K\=argar-\=an       be e'tes\=ab dast zad-and.\\ 
		{worker-\textsc{pl}} to strike    hand {hit-\textsc{3pl}}\\
	\glt `The workers went on strike.'}\label{GSexemple81b}
\zl

To represent the idiom, we resort to the feature \textsc{lid} (lexical identifier) which is associated with lexemes in the lexicon, contains morpho-syntactic as well as semantic information and allows the verb to select a specific form \citep{Sag2007a, Sag2012a}. Thus, the noun \emph{dast} in the idiom has a \textsc{lid} value \emph{dast}. The preposition \emph{be}, which heads the other complement, is the same as in \emph{zadan-3}: it identifies its content with that of its complement.

The description of \emph{zadan-4}, which occurs in the idiom \emph{dast zadan} `to start' is given in (\ref{GSexemple82}). The predicative noun complement being specified with the \textsc{lid} value \emph{dast}, it is only in combination with the noun \emph{dast} that \emph{zadan} acquires this meaning.


\begin{exe}
        \ex[]{
        \begin{avm}\label{GSexemple82}
		{\[\rm \emph{zadan4-lexeme}\\
		  cat \rm \emph{verb}\\
		  arg-st \<np{\normalfont{\textsubscript{\emph{k}}}}, pp \[pform be\\
		  cont \ibox{1}\] \,, n 
		  	\[cat [prd +]\\
			  lid \rm \emph{dast}\]\,\>\\
		  cont \[\rm \emph{soa}\\
		        nucleus \[\rm \emph{start-relation}\\
		            agent \rm \emph{k}\\
		            soa-arg \ibox{1} | nucleus \rm \emph{event-relation}\]\]\]}\end{avm}}
\end{exe}

\section{Conclusion}\label{GSsection6}

Following the usual definition of complex predicates in HPSG as a series of (at least) two predicates, of which one is the head attracting the complements of the other, we have studied them in different languages: Romance languages, German, Korean and Persian. These languages illustrate three ways in which argument attraction (or composition) manifests itself: clitic climbing (and, more generally, bounded dependencies); flexible word order, mixing the arguments of the two predicates; and special semantic combinations, which build a lexeme out of the two predicates (particularly from the verb and the noun in light verb constructions). 

HPSG is well-equipped to model the complex predicates phenomenon. The feature structure associated with a predicate specifies which complements it is waiting for, and the feature structure associated with a phrase allows it to be non-saturated regarding its complements, a possibility exploited by a number of verbs which are or can be the head of a complex predicate: the phenomenon is lexically driven. Certain verbs have two entries, one which takes a saturated complement, one which is the head of a complex predicate; but a head can be itself flexible, accepting a complement which is saturated, partially saturated or not saturated at all: this is the case of the copula in Romance languages.

Crucially, the mechanism of argument attraction is not tied to a specific syntactic structure; on
the contrary, it is compatible with different structures. We have shown that the properties of a
verbal complex (where the two predicates form a syntactic unit by themselves) differ from those of a
flat structure (where the two predicates form a unit with the complements). The structures can
characterize one language as opposed to another one (Spanish restructuring verbs contrast with Italian
ones), but they can also be present in the same language (as in Romanian, for instance; see \citealt{Monachesi99b-u}). 

Similarly, the mechanism of argument attraction does not induce a specific semantic combination: it
is compatible with a compositional semantics (as in a verb + adjective combination in Persian, or
modal verb + infinitive complement in Romance languages), as well as a variety of senses specific to
the combination of the verb with a class of complements. 
The semantic description of complex predicates in HPSG can exploit different aspects of the formalism.
These include the hierarchical organization of the lexicon and the mechanism of conjunction of descriptions (informally referred to as unification, as with combinations specializing the meaning of the verb in Persian); the informational richness of feature structures which include a \textsc{background} feature that a construction can impose restrictions on (as when the noun corresponds to an instrument implied in the action); and a \textsc{lid} feature which allows a particular complex predicate to point to a specific form (for representing idioms).    

} % avmoptions

\section*{Abbreviations}

%\inlinetodostefan{Stefan: Only add stuff that is not in Leipzig Glossing rules. The word particule is not English, it seems. Add to the table below. Make sure it is complete. EZ is missing. RA as well. } % EP 25.07.2020: Taken care of.

\begin{tabularx}{.99\textwidth}{@{}lX}
\textsc{ez} & Persian suffix Ezafe \\
\textsc{conn} & connective\\
\textsc{ra} & Persian suffix \emph{rā}\\
\end{tabularx}




\section*{Acknowledgements}

We thank Anne Abeill\'e, Gabriela B\^ilb\^ie, Olivier Bonami, Caterina Donati, Han Joung-Youn, Jean-Pierre Koenig, Kil Soo Ko, Paola Monachesi, Stefan Müller, Tsuneko Nakazawa, Daniel Rojas Plata, and Stephen Wechsler. 

{\sloppy
	\printbibliography[heading=subbibliography,notkeyword=this]
}
\end{document}

%      <!-- Local IspellDict: en_US-w_accents -->
