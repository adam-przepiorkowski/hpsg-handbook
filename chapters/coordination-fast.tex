%% -*- coding:utf-8 -*-
\documentclass{scrartcl} 



\usepackage{langsci-forest-setup}
% will be overwritten in chapters when compiled standalone:
\tikzset{external/prefix={hpsg-handbook.for.dir/}}
% all the forest figures
\forestset{external/master=hpsg-handbook}
% will be overwritten in chapters when compiled standalone:
\forestset{external/master dir={./}}

% if you want externalize graphics, compile files in chapters directly
\forestset{external/readonly}
\tikzexternalize


\usepackage{langsci-lgr}

\newcommand{\MAS}{\textsc{m}\xspace} % \M is taken by somebody

%\usepackage{./langsci/styles/langsci-basic} % FIXME: contains important commands for the index
%(\isi etc.) which should be defined in the class

\newcommand{\isi}[1]{\is{#1}#1}
\newcommand{\iai}[1]{\ia{#1}#1}
\newcommand{\ili}[1]{\il{#1}#1}

\usepackage{todonotes}

\newcommand{\todostefan}[1]{{\tikzexternaldisable\todo[color=orange!80]{\footnotesize #1}}\xspace}
\newcommand{\todosatz}[1]{{\tikzexternaldisable\todo[color=red!40]{\footnotesize #1}}\xspace}

\newcommand{\inlinetodostefan}[1]{{\tikzexternaldisable\todo[color=green!40,inline]{\footnotesize #1}}\xspace}

\newcommand{\addpages}{\todostefan{add pages}}
\newcommand{\addglosses}{\todostefan{add glosses}}


\newcommand{\spacebr}{\hspaceThis{[}}


\newcommand{\crossrefchaptert}[2][]{\citet*[#1]{chapters/#2}, Chapter~\ref{chap-#2} of this volume} 
\newcommand{\crossrefchapterp}[2][]{(\citealp*[#1]{chapters/#2}, Chapter~\ref{chap-#2} of this volume)}
\newcommand{\crossrefchapteralt}[2][]{\citealt*[#1]{chapters/#2}, Chapter~\ref{chap-#2} of this volume}
\newcommand{\crossrefchapteralp}[2][]{\citealp*[#1]{chapters/#2}, Chapter~\ref{chap-#2} of this volume}
% example of optional argument:
% \crossrefchapterp[for something, see:]{name}
% gives: (for something, see: Author 2018, Chapter~X of this volume)

\let\crossrefchapterw\crossrefchaptert


\newcommand{\bibstylepath}{./langsci/}


\usepackage[
	natbib=true,
	style=\bibstylepath langsci-unified,
	citestyle=\bibstylepath langsci-unified,
	datamodel=\bibstylepath langsci   % add authauthor and autheditor as possible fields to bibtex entries
	useprefix = true, %sort von, van, de where they should appear
	%refsection=chapter,
	maxbibnames=99,
	uniquename=false,
	mincrossrefs=99,
	maxcitenames=2,
	isbn=false,
	doi=false,
	url=false,
	eprint=false,
	autolang=hyphen,
%	\iflsResetCapitals
        language=english,
	backend=biber,
	indexing=cite,
]{biblatex}

\bibliography{../Bibliographies/stmue,
../Bibliographies/coordination,
../Bibliographies/islands}

%\bibliography{macros,own,general,crossreferences,hpsg-handbook} <-from Ash  ??

% If the user provided a shortauthor in the bibtex entry, we use the authentic author (as with the
% authorindex package) if it is defined, otherwise we use the author.
% This gets F/T as shorthand right and puts the guys in the index.

\renewbibmacro*{citeindex}{%
  \ifciteindex
    {\iffieldequalstr{labelnamesource}{shortauthor} % If biblatex uses shortauthor as the label of a bibitem
      {\ifnameundef{authauthor}                     % we check whether there is something in authauthor
        {\indexnames{author}}                       % if not, we use author
        {\indexnames{authauthor}}}                  % if yes, we use authauthor
      {\iffieldequalstr{labelnamesource}{author}    % if biblatex uses author we similarly test for
                                                    % authauthor and use this field
        {\ifnameundef{authauthor}% if defined use authauthor
          {\indexnames{author}}
          {\indexnames{authauthor}}} % if defined use this field
        {\iffieldequalstr{labelnamesource}{shorteditor} % same for editor
          {\ifnameundef{autheditor}
            {\indexnames{editor}}
            {\indexnames{autheditor}}}
          {\indexnames{labelname}}}}}               % as a fallback we index on whatever biblatex used.
    {}}

\let\citet\citet

\let\orgcite=\cite
\let\cite=\citet 	% in order to prevent inconsistencies between \cite and \citet


\usepackage{./styles/makros.2020,./styles/abbrev}


\usepackage{./styles/langsci-minimal}


%\usepackage{amsmath}

\providecommand{\feat}{}
\usepackage{./styles/avm+}

\renewcommand{\feat}[1]{\textsc{#1}}

\usepackage{./styles/additional-langsci-index-shortcuts}
\usepackage{./langsci/styles/langsci-gb4e}


% Rui

\newcommand{\spc}[0]{\hspace{-1pt}\underline{\hspace{6pt}}\,}
\newcommand{\spcs}[0]{\hspace{-1pt}\underline{\hspace{6pt}}\,\,}
\newcommand{\bad}[1]{\leavevmode\llap{#1}}
\newcommand{\COMMENT}[1]{}


% Rui coordination
\newcommand{\subl}[1]{$_{\scriptstyle \textsc{#1}}$}

\usepackage{./styles/langsci-avm}
\usepackage{./styles/merkmalstruktur}
\usepackage{./styles/my-xspace}

\newcommand{\page}{}
% usefull commands for glossings:
\newlength{\stmueTmp}

% a) hspace over width of something without showing it
\newcommand*{\hspaceThis}[1]{\settowidth{\stmueTmp}{#1}\hspace*{\stmueTmp}}

\usepackage{./styles/oneline}

\usepackage{article-ex}

\usepackage{ulem}


\author{Anne Abeillé, Laboratoire de Linguistique Formelle, University of Paris and Rui P. Chaves, Linguistics Department, University at Buffalo, The State University of New York}
\title{Coordination} 


\input coordination-include.tex
