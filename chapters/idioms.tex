%% -*- coding:utf-8 -*-

\documentclass[output=paper
	        ,collection
	        ,collectionchapter
 	        ,biblatex
                ,babelshorthands
                ,newtxmath
                ,draftmode
                ,colorlinks, citecolor=brown
]{langscibook}

\IfFileExists{../localcommands.tex}{%hack to check whether this is being compiled as part of a collection or standalone
  % add all extra packages you need to load to this file 

% the ISBN assigned to the digital edition
\usepackage[ISBN=9783961102556]{ean13isbn} 

\usepackage{graphicx}
\usepackage{tabularx}
\usepackage{amsmath} 

%\usepackage{tipa}      % Davis Koenig
\usepackage{xunicode} % Provide tipa macros (BC)

\usepackage{multicol}

% Berthold morphology
\usepackage{relsize}
%\usepackage{./styles/rtrees-bc} % forbidden forest 08.12.2019


\usepackage{langsci-optional} 
% used to be in this package
\providecommand{\citegen}{}
\renewcommand{\citegen}[2][]{\citeauthor{#2}'s (\citeyear*[#1]{#2})}
\providecommand{\lsptoprule}{}
\renewcommand{\lsptoprule}{\midrule\toprule}
\providecommand{\lspbottomrule}{}
\renewcommand{\lspbottomrule}{\bottomrule\midrule}
\providecommand{\largerpage}{}
\renewcommand{\largerpage}[1][1]{\enlargethispage{#1\baselineskip}}


\usepackage{langsci-lgr}

\newcommand{\MAS}{\textsc{m}\xspace} % \M is taken by somebody

%\usepackage{./styles/forest/forest}
\usepackage{langsci-forest-setup}

\usepackage{./styles/memoize/memoize} 
\memoizeset{
  memo filename prefix={chapters/hpsg-handbook.memo.dir/},
  register=\todo{O{}+m},
  prevent=\todo,
}

\usepackage{tikz-cd}

\usepackage{./styles/tikz-grid}
\usetikzlibrary{shadows}


% removed with texlive 2020 06.05.2020
% %\usepackage{pgfplots} % for data/theory figure in minimalism.tex
% % fix some issue with Mod https://tex.stackexchange.com/a/330076
% \makeatletter
% \let\pgfmathModX=\pgfmathMod@
% \usepackage{pgfplots}%
% \let\pgfmathMod@=\pgfmathModX
% \makeatother

\usepackage{subcaption}

% Stefan Müller's styles
\usepackage{./styles/merkmalstruktur,german,./styles/makros.2020,./styles/my-xspace,./styles/article-ex,
./styles/eng-date}

\selectlanguage{USenglish}

\usepackage{./styles/abbrev}


% Has to be loaded late since otherwise footnotes will not work

%%%%%%%%%%%%%%%%%%%%%%%%%%%%%%%%%%%%%%%%%%%%%%%%%%%%
%%%                                              %%%
%%%           Examples                           %%%
%%%                                              %%%
%%%%%%%%%%%%%%%%%%%%%%%%%%%%%%%%%%%%%%%%%%%%%%%%%%%%
% remove the percentage signs in the following lines
% if your book makes use of linguistic examples
\usepackage{langsci-gb4e} 


%% St. Mü.: 03.04.2020
%% these two versions of the command can be used for series of sets of examples:
%% \eal
%% \ex
%% \ex
%% \zlcont
%% \ealcont
%% \ex
%% \ex
%% \zl

\let\zlcont\z
\def\ealcont{\exnrfont\ex\begin{xlist}[iv.]\raggedright}

% original version of \z
%\def\z{\ifnum\@xnumdepth=1\end{exe}\else\end{xlist}\fi}
% \zcont just removes \end{exe}
%\def\zcont{\ifnum\@xnumdepth=1\else\end{xlist}\fi}
\def\zcont{}

% Crossing out text
% uncomment when needed
%\usepackage{ulem}

\usepackage{./styles/additional-langsci-index-shortcuts}

% this is the completely redone avm package
\usepackage{./styles/langsci-avm}
\avmsetup{columnsep=.3ex,style=narrow}

%\let\asort\type*


\usepackage{./styles/avm+}


\renewcommand{\tpv}[1]{{\avmjvalfont\itshape #1}}

% no small caps please
\renewcommand{\phonshape}[0]{\normalfont\itshape}

\regAvmFonts

\usepackage{theorem}

\newtheorem{mydefinition}{Def.}
\newtheorem{principle}{Principle}

{\theoremstyle{break}
%\newtheorem{schema}{Schema}
\newtheorem{mydefinition-break}[mydefinition]{Def.}
\newtheorem{principle-break}[principle]{Principle}
}


%% \newcommand{schema}[2]{
%% \begin{minipage}{\textwidth}
%% {\textbf{Schema~\theschema}}]\hspace{.5em}\textbf{(#1)}\\
%% #2
%% \end{minipage}}


% This avoids linebreaks in the Schema
\newcounter{schemacounter}
\makeatletter
\newenvironment{schema}[1][]
  {%
   \refstepcounter{schemacounter}%
   \par\bigskip\noindent
   \minipage{\linewidth}%
   \textbf{Schema~\theschemacounter\hspace{.5em} \ifx&#1&\else(#1)\fi}\par
  }{\endminipage\par\bigskip\@endparenv}%
\makeatother

%\usepackage{subfig}





% Davis Koenig Lexikon

\usepackage{tikz-qtree,tikz-qtree-compat} % Davis Koenig remove

\usepackage{shadow}



\usepackage[english]{isodate} % Andy Lücking
\usepackage[autostyle]{csquotes} % Andy
%\usepackage[autolanguage]{numprint}

%\defaultfontfeatures{
%    Path = /usr/local/texlive/2017/texmf-dist/fonts/opentype/public/fontawesome/ }

%% https://tex.stackexchange.com/a/316948/18561
%\defaultfontfeatures{Extension = .otf}% adds .otf to end of path when font loaded without ext parameter e.g. \newfontfamily{\FA}{FontAwesome} > \newfontfamily{\FA}{FontAwesome.otf}
%\usepackage{fontawesome} % Andy Lücking
\usepackage{pifont} % Andy Lücking -> hand

\usetikzlibrary{decorations.pathreplacing} % Andy Lücking
\usetikzlibrary{matrix} % Andy 
\usetikzlibrary{positioning} % Andy
\usepackage{tikz-3dplot} % Andy

% pragmatics
\usepackage{eqparbox} % Andy
\usepackage{enumitem} % Andy
\usepackage{longtable} % Andy
\usepackage{tabu} % Andy              needs to be loaded before hyperref as of texlive 2020

% tabu-fix
% to make "spread 0pt" work
% -----------------------------
\RequirePackage{etoolbox}
\makeatletter
\patchcmd
	\tabu@startpboxmeasure
	{\bgroup\begin{varwidth}}%
	{\bgroup
	 \iftabu@spread\color@begingroup\fi\begin{varwidth}}%
	{}{}
\def\@tabarray{\m@th\def\tabu@currentgrouptype
    {\currentgrouptype}\@ifnextchar[\@array{\@array[c]}}
%
%%% \pdfelapsedtime bug 2019-12-15
\patchcmd
	\tabu@message@etime
	{\the\pdfelapsedtime}%
	{\pdfelapsedtime}%
	{}{}
%
%
\makeatother
% -----------------------------


% Manfred's packages

%\usepackage{shadow}

\usepackage{tabularx}
\newcolumntype{L}[1]{>{\raggedright\arraybackslash}p{#1}} % linksbündig mit Breitenangabe


% Jong-Bok

%\usepackage{xytree}

\newcommand{\xytree}[2][dummy]{Let's do the tree!}

% seems evil, get rid of it
% defines \ex is incompatible with gb4e
%\usepackage{lingmacros}

% taken from lingmacros:
\makeatletter
% \evnup is used to line up the enumsentence number and an entry along
% the top.  It can take an argument to improve lining up.
\def\evnup{\@ifnextchar[{\@evnup}{\@evnup[0pt]}}

\def\@evnup[#1]#2{\setbox1=\hbox{#2}%
\dimen1=\ht1 \advance\dimen1 by -.5\baselineskip%
\advance\dimen1 by -#1%
\leavevmode\lower\dimen1\box1}
\makeatother


% YK -- CG chapter

%\usepackage{xspace}
\usepackage{bm}
\usepackage{ebproof}


% Antonio Branco, remove this
\usepackage{epsfig}

% now unicode
%\usepackage{alphabeta}





\usepackage{pst-node}


% fmr: additional packages
%\usepackage{amsthm}


% Ash and Steve: LFG
\usepackage{./styles/lfg/dalrymple}

\RequirePackage{graphics}
%\RequirePackage{./styles/lfg/trees}
%% \RequirePackage{avm}
%% \avmoptions{active}
%% \avmfont{\sc}
%% \avmvalfont{\sc}
\RequirePackage{./styles/lfg/lfgmacrosash}

\usepackage{./styles/lfg/glue}

%%%%%%%%%%%%%%%%%%%%%%%%%%%%%%
%% Markup
%%%%%%%%%%%%%%%%%%%%%%%%%%%%%%
\usepackage[normalem]{ulem} % For thinks like strikethrough, using \sout

% \newcommand{\high}[1]{\textbf{#1}} % highlighted text
\newcommand{\high}[1]{\textit{#1}} % highlighted text
%\newcommand{\term}[1]{\textit{#1}\/} % technical term
\newcommand{\qterm}[1]{`{#1}'} % technical term, quotes
%\newcommand{\trns}[1]{\strut `#1'} % translation in glossed example
\newcommand{\trnss}[1]{\strut \phantom{\sqz{}} `#1'} % translation in ungrammatical glossed example
\newcommand{\ttrns}[1]{(`#1')} % an in-text translation of a word
%\newcommand{\feat}[1]{\mbox{\textsc{\MakeLowercase{#1}}}}     % feature name
%\newcommand{\val}[1]{\mbox{\textsc{\MakeLowercase{#1}}}}    % f-structure value
\newcommand{\featt}[1]{\mbox{\textsc{\MakeLowercase{#1}}}}     % feature name
\newcommand{\vall}[1]{\mbox{\textsc{\textup{\MakeLowercase{#1}}}}}    % f-structure value
\newcommand{\mg}[1]{\mbox{\textsc{\MakeLowercase{#1}}}}    % morphological gloss
%\newcommand{\word}[1]{\textit{#1}}       % mention of word
\providecommand{\kstar}[1]{{#1}\ensuremath{^*}}
\providecommand{\kplus}[1]{{#1}\ensuremath{^+}}
\newcommand{\template}[1]{@\textsc{\MakeLowercase{#1}}}
\newcommand{\templaten}[1]{\textsc{\MakeLowercase{#1}}}
\newcommand{\templatenn}[1]{\MakeUppercase{#1}}
\newcommand{\tempeq}{\ensuremath{=}}
\newcommand{\predval}[1]{\ensuremath{\langle}\textsc{#1}\ensuremath{\rangle}}
\newcommand{\predvall}[1]{{\rm `#1'}}
\newcommand{\lfgfst}[1]{\ensuremath{#1\,}}
\newcommand{\scare}[1]{`#1'} % scare quotes
\newcommand{\bracket}[1]{\ensuremath{\left\langle\mathit{#1}\right\rangle}}
\newcommand{\sectionw}[1][]{Section#1} % section word: for cap/non-cap
\newcommand{\tablew}[1][]{Table#1} % table word: for cap/non-cap
\newcommand{\lfgglue}{LFG+Glue}
\newcommand{\hpsgglue}{HPSG+Glue}
\newcommand{\gs}{GS}
%\newcommand{\func}[1]{\ensuremath{\mathbf{#1}}}
\newcommand{\func}[1]{\textbf{#1}}
\renewcommand{\glue}{Glue}
%\newcommand{\exr}[1]{(\ref{ex:#1}}
\newcommand{\exra}[1]{(\ref{ex:#1})}


%%%%%%%%%%%%%%%%%%%%%%%%%%%%%%
% Notation
%\newcommand{\xbar}[1]{$_{\mbox{\textsc{#1}$^{\raisebox{1ex}{}}$}}$}
\newcommand{\xprime}[2][]{\textup{\mbox{{#2}\ensuremath{^\prime_{\hspace*{-.0em}\mbox{\footnotesize\ensuremath{\mathit{#1}}}}}}}}
\providecommand{\xzero}[2][]{#2\ensuremath{^0_{\mbox{\footnotesize\ensuremath{\mathit{#1}}}}}}



\let\leftangle\langle
\let\rightangle\rangle

%\newcommand{\pslabel}[1]{}



  %add all your local new commands to this file


% Don't do this at home. I do not like the smaller font for captions.
% I just removed loading the caption packege in langscibook.cls
%% \captionsetup{%
%% font={%
%% stretch=1%.8%
%% ,normalsize%,small%
%% },%
%% width=.8\textwidth
%% }

\makeatletter
\def\blx@maxline{77}
\makeatother


\newcommand{\page}{}

\newcommand{\todostefan}[1]{\todo[color=orange!80]{\footnotesize #1}\xspace}
\newcommand{\todosatz}[1]{\todo[color=red!40]{\footnotesize #1}\xspace}

\newcommand{\inlinetodostefan}[1]{\todo[color=green!40,inline]{\footnotesize #1}\xspace}

\newcommand{\addpages}{\todostefan{add pages}}
\newcommand{\addglosses}{\todostefan{add glosses}}


\newcommand{\spacebr}{\hspaceThis{[}}

\newcommand{\danish}{\jambox{(\ili{Danish})}}
\newcommand{\english}{\jambox{(\ili{English})}}
\newcommand{\german}{\jambox{(\ili{German})}}
\newcommand{\yiddish}{\jambox{(\ili{Yiddish})}}
\newcommand{\welsh}{\jambox{(\ili{Welsh})}}

% Cite and cross-reference other chapters
\newcommand{\crossrefchaptert}[2][]{\citet*[#1]{chapters/#2}, Chapter~\ref{chap-#2} of this volume} 
\newcommand{\crossrefchapterp}[2][]{(\citealp*[#1]{chapters/#2}, Chapter~\ref{chap-#2} of this volume)}
\newcommand{\crossrefchapteralt}[2][]{\citealt*[#1]{chapters/#2}, Chapter~\ref{chap-#2} of this volume}
\newcommand{\crossrefchapteralp}[2][]{\citealp*[#1]{chapters/#2}, Chapter~\ref{chap-#2} of this volume}
% example of optional argument:
% \crossrefchapterp[for something, see:]{name}
% gives: (for something, see: Author 2018, Chapter~X of this volume)

\let\crossrefchapterw\crossrefchaptert



% Davis Koenig

\let\ig=\textsc
\let\tc=\textcolor

% evolution, Flickinger, Pollard, Wasow

\let\citeNP\citet

% Adam P

%\newcommand{\toappear}{Forthcoming}
\newcommand{\pg}[1]{p.\,#1}
\renewcommand{\implies}{\rightarrow}

\newcommand*{\rref}[1]{(\ref{#1})}
\newcommand*{\aref}[1]{(\ref{#1}a)}
\newcommand*{\bref}[1]{(\ref{#1}b)}
\newcommand*{\cref}[1]{(\ref{#1}c)}

\newcommand{\msadam}{.}
\newcommand{\morsyn}[1]{\textsc{#1}}

\newcommand{\nom}{\morsyn{nom}}
\newcommand{\acc}{\morsyn{acc}}
\newcommand{\dat}{\morsyn{dat}}
\newcommand{\gen}{\morsyn{gen}}
\newcommand{\ins}{\morsyn{ins}}
%\newcommand{\aploc}{\morsyn{loc}}
\newcommand{\voc}{\morsyn{voc}}
\newcommand{\ill}{\morsyn{ill}}
\renewcommand{\inf}{\morsyn{inf}}
\newcommand{\passprc}{\morsyn{passp}}

%\newcommand{\Nom}{\msadam\nom}
%\newcommand{\Acc}{\msadam\acc}
%\newcommand{\Dat}{\msadam\dat}
%\newcommand{\Gen}{\msadam\gen}
\newcommand{\Ins}{\msadam\ins}
\newcommand{\Loc}{\msadam\loc}
\newcommand{\Voc}{\msadam\voc}
\newcommand{\Ill}{\msadam\ill}
\newcommand{\PassP}{\msadam\passprc}

\newcommand{\Aux}{\textsc{aux}}

\newcommand{\princ}[1]{\textnormal{\textsc{#1}}} % for constraint names
\newcommand{\notion}[1]{\emph{#1}}
\renewcommand{\path}[1]{\textnormal{\textsc{#1}}}
\newcommand{\ftype}[1]{\textit{#1}}
\newcommand{\fftype}[1]{{\scriptsize\textit{#1}}}
\newcommand{\la}{$\langle$}
\newcommand{\ra}{$\rangle$}
%\newcommand{\argst}{\path{arg-st}}
\newcommand{\phtm}[1]{\setbox0=\hbox{#1}\hspace{\wd0}}
\newcommand{\prep}[1]{\setbox0=\hbox{#1}\hspace{-1\wd0}#1}

%%%%%%%%%%%%%%%%%%%%%%%%%%%%%%%%%%%%%%%%%%%%%%%%%%%%%%%%%%%%%%%%%%%%%%%%%%%

% FROM FS.STY:

%%%
%%% Feature structures
%%%

% \fs         To print a feature structure by itself, type for example
%             \fs{case:nom \\ person:P}
%             or (better, for true italics),
%             \fs{\it case:nom \\ \it person:P}
%
% \lfs        To print the same feature structure with the category
%             label N at the top, type:
%             \lfs{N}{\it case:nom \\ \it person:P}

%    Modified 1990 Dec 5 so that features are left aligned.
\newcommand{\fs}[1]%
{\mbox{\small%
$
\!
\left[
  \!\!
  \begin{tabular}{l}
    #1
  \end{tabular}
  \!\!
\right]
\!
$}}

%     Modified 1990 Dec 5 so that features are left aligned.
%\newcommand{\lfs}[2]
%   {
%     \mbox{$
%           \!\!
%           \begin{tabular}{c}
%           \it #1
%           \\
%           \mbox{\small%
%                 $
%                 \left[
%                 \!\!
%                 \it
%                 \begin{tabular}{l}
%                 #2
%                 \end{tabular}
%                 \!\!
%                 \right]
%                 $}
%           \end{tabular}
%           \!\!
%           $}
%   }

\newcommand{\ft}[2]{\path{#1}\hspace{1ex}\ftype{#2}}
\newcommand{\fsl}[2]{\fs{{\fftype{#1}} \\ #2}}

\newcommand{\fslt}[2]
 {\fst{
       {\fftype{#1}} \\
       #2 
     }
 }

\newcommand{\fsltt}[2]
 {\fstt{
       {\fftype{#1}} \\
       #2 
     }
 }

\newcommand{\fslttt}[2]
 {\fsttt{
       {\fftype{#1}} \\
       #2 
     }
 }


% jak \ft, \fs i \fsl tylko nieco ciasniejsze

\newcommand{\ftt}[2]
% {{\sc #1}\/{\rm #2}}
 {\textsc{#1}\/{\rm #2}}

\newcommand{\fst}[1]
  {
    \mbox{\small%
          $
          \left[
          \!\!\!
%          \sc
          \begin{tabular}{l} #1
          \end{tabular}
          \!\!\!\!\!\!\!
          \right]
          $
          }
   }

%\newcommand{\fslt}[2]
% {\fst{#2\\
%       {\scriptsize\it #1}
%      }
% }


% superciasne

\newcommand{\fstt}[1]
  {
    \mbox{\small%
          $
          \left[
          \!\!\!
%          \sc
          \begin{tabular}{l} #1
          \end{tabular}
          \!\!\!\!\!\!\!\!\!\!\!
          \right]
          $
          }
   }

%\newcommand{\fsltt}[2]
% {\fstt{#2\\
%       {\scriptsize\it #1}
%      }
% }

\newcommand{\fsttt}[1]
  {
    \mbox{\small%
          $
          \left[
          \!\!\!
%          \sc
          \begin{tabular}{l} #1
          \end{tabular}
          \!\!\!\!\!\!\!\!\!\!\!\!\!\!\!\!
          \right]
          $
          }
   }



% %add all your local new commands to this file

% \newcommand{\smiley}{:)}

% you are not supposed to mess with hardcore stuff, St.Mü. 22.08.2018
%% \renewbibmacro*{index:name}[5]{%
%%   \usebibmacro{index:entry}{#1}
%%     {\iffieldundef{usera}{}{\thefield{usera}\actualoperator}\mkbibindexname{#2}{#3}{#4}{#5}}}

% % \newcommand{\noop}[1]{}



% Rui

\newcommand{\spc}[0]{\hspace{-1pt}\underline{\hspace{6pt}}\,}
\newcommand{\spcs}[0]{\hspace{-1pt}\underline{\hspace{6pt}}\,\,}
\newcommand{\bad}[1]{\leavevmode\llap{#1}}
\newcommand{\COMMENT}[1]{}


% Rui coordination
\newcommand{\subl}[1]{$_{\scriptstyle \textsc{#1}}$}



% Andy Lücking gesture.tex
\newcommand{\Pointing}{\ding{43}}
% Giotto: "Meeting of Joachim and Anne at the Golden Gate" - 1305-10 
\definecolor{GoldenGate1}{rgb}{.13,.09,.13} % Dress of woman in black
\definecolor{GoldenGate2}{rgb}{.94,.94,.91} % Bridge
\definecolor{GoldenGate3}{rgb}{.06,.09,.22} % Blue sky
\definecolor{GoldenGate4}{rgb}{.94,.91,.87} % Dress of woman with shawl
\definecolor{GoldenGate5}{rgb}{.52,.26,.26} % Joachim's robe
\definecolor{GoldenGate6}{rgb}{.65,.35,.16} % Anne's robe
\definecolor{GoldenGate7}{rgb}{.91,.84,.42} % Joachim's halo

\makeatletter
\newcommand{\@Depth}{1} % x-dimension, to front
\newcommand{\@Height}{1} % z-dimension, up
\newcommand{\@Width}{1} % y-dimension, rightwards
%\GGS{<x-start>}{<y-start>}{<z-top>}{<z-bottom>}{<Farbe>}{<x-width>}{<y-depth>}{<opacity>}
\newcommand{\GGS}[9][]{%
\coordinate (O) at (#2-1,#3-1,#5);
\coordinate (A) at (#2-1,#3-1+#7,#5);
\coordinate (B) at (#2-1,#3-1+#7,#4);
\coordinate (C) at (#2-1,#3-1,#4);
\coordinate (D) at (#2-1+#8,#3-1,#5);
\coordinate (E) at (#2-1+#8,#3-1+#7,#5);
\coordinate (F) at (#2-1+#8,#3-1+#7,#4);
\coordinate (G) at (#2-1+#8,#3-1,#4);
\draw[draw=black, fill=#6, fill opacity=#9] (D) -- (E) -- (F) -- (G) -- cycle;% Front
\draw[draw=black, fill=#6, fill opacity=#9] (C) -- (B) -- (F) -- (G) -- cycle;% Top
\draw[draw=black, fill=#6, fill opacity=#9] (A) -- (B) -- (F) -- (E) -- cycle;% Right
}
\makeatother


% pragmatics
\newcommand{\speaking}[1]{\eqparbox{name}{\textsc{\lowercase{#1}\space}}}
\newcommand{\alname}[1]{\eqparbox{name}{\textsc{\lowercase{#1}}}}
\newcommand{\HPSGTTR}{HPSG$_{\text{TTR}}$\xspace}

\newcommand{\ttrtype}[1]{\textit{#1}}
\newcommand{\avmel}{\q<\quad\q>} %% shortcut for empty lists in AVM
\newcommand{\ttrmerge}{\ensuremath{\wedge_{\textit{merge}}}}
\newcommand{\Cat}[2][0.1pt]{%
  \begin{scope}[y=#1,x=#1,yscale=-1, inner sep=0pt, outer sep=0pt]
   \path[fill=#2,line join=miter,line cap=butt,even odd rule,line width=0.8pt]
  (151.3490,307.2045) -- (264.3490,307.2045) .. controls (264.3490,291.1410) and (263.2021,287.9545) .. (236.5990,287.9545) .. controls (240.8490,275.2045) and (258.1242,244.3581) .. (267.7240,244.3581) .. controls (276.2171,244.3581) and (286.3490,244.8259) .. (286.3490,264.2045) .. controls (286.3490,286.2045) and (323.3717,321.6755) .. (332.3490,307.2045) .. controls (345.7277,285.6390) and (309.3490,292.2151) .. (309.3490,240.2046) .. controls (309.3490,169.0514) and (350.8742,179.1807) .. (350.8742,139.2046) .. controls (350.8742,119.2045) and (345.3490,116.5037) .. (345.3490,102.2045) .. controls (345.3490,83.3070) and (361.9972,84.4036) .. (358.7581,68.7349) .. controls (356.5206,57.9117) and (354.7696,49.2320) .. (353.4652,36.1439) .. controls (352.5396,26.8573) and (352.2445,16.9594) .. (342.5985,17.3574) .. controls (331.2650,17.8250) and (326.9655,37.7742) .. (309.3490,39.2045) .. controls (291.7685,40.6320) and (276.7783,24.2380) .. (269.9740,26.5795) .. controls (263.2271,28.9013) and (265.3490,47.2045) .. (269.3490,60.2045) .. controls (275.6359,80.6368) and (289.3490,107.2045) .. (264.3490,111.2045) .. controls (239.3490,115.2045) and (196.3490,119.2045) .. (165.3490,160.2046) .. controls (134.3490,201.2046) and (135.4934,249.3212) .. (123.3490,264.2045) .. controls (82.5907,314.1553) and (40.8239,293.6463) .. (40.8239,335.2045) .. controls (40.8239,353.8102) and (72.3490,367.2045) .. (77.3490,361.2045) .. controls (82.3490,355.2045) and (34.8638,337.3259) .. (87.9955,316.2045) .. controls (133.3871,298.1601) and   (137.4391,294.4766) .. (151.3490,307.2045) -- cycle;
\end{scope}%
}


% KdK
\newcommand{\smiley}{:)}

\renewbibmacro*{index:name}[5]{%
  \usebibmacro{index:entry}{#1}
    {\iffieldundef{usera}{}{\thefield{usera}\actualoperator}\mkbibindexname{#2}{#3}{#4}{#5}}}

% \newcommand{\noop}[1]{}

% chngcntr.sty otherwise gives error that these are already defined
%\let\counterwithin\relax
%\let\counterwithout\relax

% the space of a left bracket for glossings
\newcommand{\LB}{\hspaceThis{[}}

\newcommand{\LF}{\mbox{$[\![$}}

\newcommand{\RF}{\mbox{$]\!]_F$}}

\newcommand{\RT}{\mbox{$]\!]_T$}}





% Manfred's

\newcommand{\kommentar}[1]{}

\newcommand{\bsp}[1]{\emph{#1}}
\newcommand{\bspT}[2]{\bsp{#1} `#2'}
\newcommand{\bspTL}[3]{\bsp{#1} (lit.: #2) `#3'}

\newcommand{\noidi}{§}

\newcommand{\refer}[1]{(\ref{#1})}

%\newcommand{\avmtype}[1]{\multicolumn{2}{l}{\type{#1}}}
\newcommand{\attr}[1]{\textsc{#1}}

\newcommand{\srdefault}{\mbox{\begin{tabular}{c}{\large <}\\[-1.5ex]$\sqcap$\end{tabular}}}

%% \newcommand{\myappcolumn}[2]{
%% \begin{minipage}[t]{#1}#2\end{minipage}
%% }

%% \newcommand{\appc}[1]{\myappcolumn{3.7cm}{#1}}


% Jong-Bok


% clean that up and do not use \def (killing other stuff defined before)
%\if 0
\newcommand\DEL{\textsc{del}}
\newcommand\del{\textsc{del}}

\newcommand\conn{\textsc{conn}}
\newcommand\CONN{\textsc{conn}}
\newcommand\CONJ{\textsc{conj}}
\newcommand\LITE{\textsc{lex}}
\newcommand\lite{\textsc{lex}}
\newcommand\HON{\textsc{hon}}

%\newcommand\CAUS{\textsc{caus}}
%\newcommand\PASS{\textsc{pass}}
\newcommand\NPST{\textsc{npst}}
%\newcommand\COND{\textsc{cond}}



\newcommand\hdlite{\textsc{head-lex construction}}
\newcommand\hdlight{\textsc{head-light} Schema}
\newcommand\NFORM{\textsc{nform}}

\newcommand\RELS{\textsc{rels}}
%\newcommand\TENSE{\textsc{tense}}


%\newcommand\ARG{\textsc{arg}}
\newcommand\ARGs{\textsc{arg0}}
\newcommand\ARGa{\textsc{arg}}
\newcommand\ARGb{\textsc{arg2}}
\newcommand\TPC{\textsc{top}}
%\newcommand\PROG{\textsc{prog}}

\newcommand\LIGHT{\textsc{light}\xspace}
\newcommand\pst{\textsc{pst}}
%\newcommand\PAST{\textsc{pst}}
%\newcommand\DAT{\textsc{dat}}
%\newcommand\CONJ{\textsc{conj}}
\newcommand\nominal{\textsc{nominal}}
\newcommand\NOMINAL{\textsc{nominal}}
\newcommand\VAL{\textsc{val}}
%\newcommand\val{\textsc{val}}
\newcommand\MODE{\textsc{mode}}
\newcommand\RESTR{\textsc{restr}}
\newcommand\SIT{\textsc{sit}}
\newcommand\ARG{\textsc{arg}}
\newcommand\RELN{\textsc{rel}}
%\newcommand\REL{\textsc{rel}}
%\newcommand\RELS{\textsc{rels}}
%\newcommand\arg-st{\textsc{arg-st}}
\newcommand\xdel{\textsc{xdel}}
\newcommand\zdel{\textsc{zdel}}
\newcommand\sug{\textsc{sug}}
%\newcommand\IMP{\textsc{imp}}
%\newcommand\conn{\textsc{conn}}
%\newcommand\CONJ{\textsc{conj}}
%\newcommand\HON{\textsc{hon}}
\newcommand\BN{\textsc{bn}}
\newcommand\bn{\textsc{bn}}
\newcommand\pres{\textsc{pres}}
\newcommand\PRES{\textsc{pres}}
\newcommand\prs{\textsc{pres}}
%\newcommand\PRS{\textsc{pres}}
\newcommand\agt{\textsc{agt}}
%\newcommand\DEL{\textsc{del}}
%\newcommand\PRED{\textsc{pred}}
\newcommand\AGENT{\textsc{agent}}
\newcommand\THEME{\textsc{theme}}
%\newcommand\AUX{\textsc{aux}}
%\newcommand\THEME{\textsc{theme}}
%\newcommand\PL{\textsc{pl}}
\newcommand\SRC{\textsc{src}}
\newcommand\src{\textsc{src}}
\newcommand{\FORMjb}{\textsc{form}}
\newcommand{\formjb}{\FORM}
\newcommand\GCASE{\textsc{gcase}}
\newcommand\gcase{\textsc{gcase}}
\newcommand\SCASE{\textsc{scase}}
\newcommand\PHON{\textsc{phon}}
%\newcommand\SS{\textsc{ss}}
\newcommand\SYN{\textsc{syn}}
%\newcommand\LOC{\textsc{loc}}
\newcommand\MOD{\textsc{mod}}
\newcommand\INV{\textsc{inv}}
%\newcommand\L{\textsc{l}}
%\newcommand\CASE{\textsc{case}}
\newcommand\SPR{\textsc{spr}}
\newcommand\COMPS{\textsc{comps}}
%\newcommand\comps{\textsc{comps}}
\newcommand\SEM{\textsc{sem}}
\newcommand\CONT{\textsc{cont}}
\newcommand\SUBCAT{\textsc{subcat}}
\newcommand\CAT{\textsc{cat}}
%\newcommand\C{\textsc{c}}
%\newcommand\SUBJ{\textsc{subj}}
\newcommand\subjjb{\textsc{subj}}
%\newcommand\SLASH{\textsc{slash}}
\newcommand\LOCAL{\textsc{local}}
%\newcommand\ARG-ST{\textsc{arg-st}}
%\newcommand\AGR{\textsc{agr}}
\newcommand\PER{\textsc{per}}
%\newcommand\NUM{\textsc{num}}
%\newcommand\IND{\textsc{ind}}
\newcommand\VFORM{\textsc{vform}}
\newcommand\PFORM{\textsc{pform}}
\newcommand\decl{\textsc{decl}}
%\newcommand\loc{\textsc{loc   }}
% \newcommand\   {\textsc{  }}

%\newcommand\NEG{\textsc{neg}}
\newcommand\FRAMES{\textsc{frames}}
%\newcommand\REFL{\textsc{refl}}

\newcommand\MKG{\textsc{mkg}}

%\newcommand\BN{\textsc{bn}}
\newcommand\HD{\textsc{hd}}
\newcommand\NP{\textsc{np}}
\newcommand\PF{\textsc{pf}}
%\newcommand\PL{\textsc{pl}}
\newcommand\PP{\textsc{pp}}
%\newcommand\SS{\textsc{ss}}
\newcommand\VF{\textsc{vf}}
\newcommand\VP{\textsc{vp}}
%\newcommand\bn{\textsc{bn}}
\newcommand\cl{\textsc{cl}}
%\newcommand\pl{\textsc{pl}}
\newcommand\Wh{\ital{Wh}}
%\newcommand\ng{\textsc{neg}}
\newcommand\wh{\ital{wh}}
%\newcommand\ACC{\textsc{acc}}
%\newcommand\AGR{\textsc{agr}}
\newcommand\AGT{\textsc{agt}}
\newcommand\ARC{\textsc{arc}}
%\newcommand\ARG{\textsc{arg}}
\newcommand\ARP{\textsc{arc}}
%\newcommand\AUX{\textsc{aux}}
%\newcommand\CAT{\textsc{cat}}
%\newcommand\COP{\textsc{cop}}
%\newcommand\DAT{\textsc{dat}}
\newcommand\NEWCOMMAND{\textsc{def}}
%\newcommand\DEL{\textsc{del}}
\newcommand\DOM{\textsc{dom}}
\newcommand\DTR{\textsc{dtr}}
%\newcommand\FUT{\textsc{fut}}
\newcommand\GAP{\textsc{gap}}
%\newcommand\GEN{\textsc{gen}}
%\newcommand\HON{\textsc{hon}}
%\newcommand\IMP{\textsc{imp}}
%\newcommand\IND{\textsc{ind}}
%\newcommand\INV{\textsc{inv}}
\newcommand\LEX{\textsc{lex}}
\newcommand\Lex{\textsc{lex}}
%\newcommand\LOC{\textsc{loc}}
%\newcommand\MOD{\textsc{mod}}
\newcommand\MRK{{\nr MRK}}
%\newcommand\NEG{\textsc{neg}}
\newcommand\NEW{\textsc{new}}
%\newcommand\NOM{\textsc{nom}}
%\newcommand\NUM{\textsc{num}}
%\newcommand\PER{\textsc{per}}
%\newcommand\PST{\textsc{pst}}
\newcommand\QUE{\textsc{que}}
%\newcommand\REL{\textsc{rel}}
\newcommand\SEL{\textsc{sel}}
%\newcommand\SEM{\textsc{sem}}
%\newcommand\SIT{\textsc{arg0}}
%\newcommand\SPR{\textsc{spr}}
%\newcommand\SRC{\textsc{src}}
\newcommand\SUG{\textsc{sug}}
%\newcommand\SYN{\textsc{syn}}
%\newcommand\TPC{\textsc{top}}
%\newcommand\VAL{\textsc{val}}
%\newcommand\acc{\textsc{acc}}
%\newcommand\agt{\textsc{agt}}
\newcommand\cop{\textsc{cop}}
%\newcommand\dat{\textsc{dat}}
\newcommand\foc{\textsc{focus}}
%\newcommand\FOC{\textsc{focus}}
\newcommand\fut{\textsc{fut}}
\newcommand\hon{\textsc{hon}}
\newcommand\imp{\textsc{imp}}
\newcommand\kes{\textsc{kes}}
%\newcommand\lex{\textsc{lex}}
%\newcommand\loc{\textsc{loc}}
\newcommand\mrk{{\nr MRK}}
%\newcommand\nom{\textsc{nom}}
%\newcommand\num{\textsc{num}}
\newcommand\plu{\textsc{plu}}
\newcommand\pne{\textsc{pne}}
%\newcommand\pst{\textsc{pst}}
\newcommand\pur{\textsc{pur}}
%\newcommand\que{\textsc{que}}
%\newcommand\src{\textsc{src}}
%\newcommand\sug{\textsc{sug}}
\newcommand\tpc{\textsc{top}}
%\newcommand\utt{\textsc{utt}}
%\newcommand\val{\textsc{val}}
%% \newcommand\LITE{\textsc{lex}}
%% \newcommand\PAST{\textsc{pst}}
%% \newcommand\POSP{\textsc{pos}}
%% \newcommand\PRS{\textsc{pres}}
%% \newcommand\mod{\textsc{mod}}%
%% \newcommand\newuse{{`kes'}}
%% \newcommand\posp{\textsc{pos}}
%% \newcommand\prs{\textsc{pres}}
%% \newcommand\psp{{\it en\/}}
%% \newcommand\skes{\textsc{kes}}
%% \newcommand\CASE{\textsc{case}}
%% \newcommand\CASE{\textsc{case}}
%% \newcommand\COMP{\textsc{comp}}
%% \newcommand\CONJ{\textsc{conj}}
%% \newcommand\CONN{\textsc{conn}}
%% \newcommand\CONT{\textsc{cont}}
%% \newcommand\DECL{\textsc{decl}}
%% \newcommand\FOCUS{\textsc{focus}}
%% %\newcommand\FORM{\textsc{form}} duplicate
%% \newcommand\FREL{\textsc{frel}}
%% \newcommand\GOAL{\textsc{goal}}
\newcommand\HEAD{\textsc{head}}
%% \newcommand\INDEX{\textsc{ind}}
%% \newcommand\INST{\textsc{inst}}
%% \newcommand\MODE{\textsc{mode}}
%% \newcommand\MOOD{\textsc{mood}}
%% \newcommand\NMLZ{\textsc{nmlz}}
%% \newcommand\PHON{\textsc{phon}}
%% \newcommand\PRED{\textsc{pred}}
%% %\newcommand\PRES{\textsc{pres}}
%% \newcommand\PROM{\textsc{prom}}
%% \newcommand\RELN{\textsc{pred}}
%% \newcommand\RELS{\textsc{rels}}
%% \newcommand\STEM{\textsc{stem}}
%% \newcommand\SUBJ{\textsc{subj}}
%% \newcommand\XARG{\textsc{xarg}}
%% \newcommand\bse{{\it bse\/}}
%% \newcommand\case{\textsc{case}}
%% \newcommand\caus{\textsc{caus}}
%% \newcommand\comp{\textsc{comp}}
%% \newcommand\conj{\textsc{conj}}
%% \newcommand\conn{\textsc{conn}}
%% \newcommand\decl{\textsc{decl}}
%% \newcommand\fin{{\it fin\/}}
%% %\newcommand\form{\textsc{form}}
%% \newcommand\gend{\textsc{gend}}
%% \newcommand\inf{{\it inf\/}}
%% \newcommand\mood{\textsc{mood}}
%% \newcommand\nmlz{\textsc{nmlz}}
%% \newcommand\pass{\textsc{pass}}
%% \newcommand\past{\textsc{past}}
%% \newcommand\perf{\textsc{perf}}
%% \newcommand\pln{{\it pln\/}}
%% \newcommand\pred{\textsc{pred}}


%% %\newcommand\pres{\textsc{pres}}
%% \newcommand\proc{\textsc{proc}}
%% \newcommand\nonfin{{\it nonfin\/}}
%% \newcommand\AGENT{\textsc{agent}}
%% \newcommand\CFORM{\textsc{cform}}
%% %\newcommand\COMPS{\textsc{comps}}
%% \newcommand\COORD{\textsc{coord}}
%% \newcommand\COUNT{\textsc{count}}
%% \newcommand\EXTRA{\textsc{extra}}
%% \newcommand\GCASE{\textsc{gcase}}
%% \newcommand\GIVEN{\textsc{given}}
%% \newcommand\LOCAL{\textsc{local}}
%% \newcommand\NFORM{\textsc{nform}}
%% \newcommand\PFORM{\textsc{pform}}
%% \newcommand\SCASE{\textsc{scase}}
%% \newcommand\SLASH{\textsc{slash}}
%% \newcommand\SLASH{\textsc{slash}}
%% \newcommand\THEME{\textsc{theme}}
%% \newcommand\TOPIC{\textsc{topic}}
%% \newcommand\VFORM{\textsc{vform}}
%% \newcommand\cause{\textsc{cause}}
%% %\newcommand\comps{\textsc{comps}}
%% \newcommand\gcase{\textsc{gcase}}
%% \newcommand\itkes{{\it kes\/}}
%% \newcommand\pass{{\it pass\/}}
%% \newcommand\vform{\textsc{vform}}
%% \newcommand\CCONT{\textsc{c-cont}}
%% \newcommand\GN{\textsc{given-new}}
%% \newcommand\INFO{\textsc{info-st}}
%% \newcommand\ARG-ST{\textsc{arg-st}}
%% \newcommand\SUBCAT{\textsc{subcat}}
%% \newcommand\SYNSEM{\textsc{synsem}}
%% \newcommand\VERBAL{\textsc{verbal}}
%% \newcommand\arg-st{\textsc{arg-st}}
%% \newcommand\plain{{\it plain}\/}
%% \newcommand\propos{\textsc{propos}}
%% \newcommand\ADVERBIAL{\textsc{advl}}
%% \newcommand\HIGHLIGHT{\textsc{prom}}
%% \newcommand\NOMINAL{\textsc{nominal}}

\newenvironment{myavm}{\begingroup\avmvskip{.1ex}
  \selectfont\begin{avm}}%
{\end{avm}\endgroup\medskip}
\newcommand\pfix{\vspace{-5pt}}


\newcommand{\jbsub}[1]{\lower4pt\hbox{\small #1}}
\newcommand{\jbssub}[1]{\lower4pt\hbox{\small #1}}
\newcommand\jbtr{\underbar{\ \ \ }\ }


%\fi

% cl

\newcommand{\delphin}{\textsc{delph-in}}


% YK -- CG chapter

\newcommand{\grey}[1]{\colorbox{mycolor}{#1}}
\definecolor{mycolor}{gray}{0.8}

\newcommand{\GQU}[2]{\raisebox{1.6ex}{\ensuremath{\rotatebox{180}{\textbf{#1}}_{\scalebox{.7}{\textbf{#2}}}}}}

\newcommand{\SetInfLen}{\setpremisesend{0pt}\setpremisesspace{10pt}\setnamespace{0pt}}

\newcommand{\pt}[1]{\ensuremath{\mathsf{#1}}}
\newcommand{\ptv}[1]{\ensuremath{\textsf{\textsl{#1}}}}

\newcommand{\sv}[1]{\ensuremath{\bm{\mathcal{#1}}}}
\newcommand{\sX}{\sv{X}}
\newcommand{\sF}{\sv{F}}
\newcommand{\sG}{\sv{G}}

\newcommand{\syncat}[1]{\textrm{#1}}
\newcommand{\syncatVar}[1]{\ensuremath{\mathit{#1}}}

\newcommand{\RuleName}[1]{\textrm{#1}}

\newcommand{\SemTyp}{\textsf{Sem}}

\newcommand{\E}{\ensuremath{\bm{\epsilon}}\xspace}

\newcommand{\greeka}{\upalpha}
\newcommand{\greekb}{\upbeta}
\newcommand{\greekd}{\updelta}
\newcommand{\greekp}{\upvarphi}
\newcommand{\greekr}{\uprho}
\newcommand{\greeks}{\upsigma}
\newcommand{\greekt}{\uptau}
\newcommand{\greeko}{\upomega}
\newcommand{\greekz}{\upzeta}

\newcommand{\Lemma}{\ensuremath{\hskip.5em\vdots\hskip.5em}\noLine}
\newcommand{\LemmaAlt}{\ensuremath{\hskip.5em\vdots\hskip.5em}}

\newcommand{\I}{\iota}

\newcommand{\sem}{\ensuremath}

\newcommand{\NoSem}{%
\renewcommand{\LexEnt}[3]{##1; \syncat{##3}}
\renewcommand{\LexEntTwoLine}[3]{\renewcommand{\arraystretch}{.8}%
\begin{array}[b]{l} ##1;  \\ \syncat{##3} \end{array}}
\renewcommand{\LexEntThreeLine}[3]{\renewcommand{\arraystretch}{.8}%
\begin{array}[b]{l} ##1; \\ \syncat{##3} \end{array}}}

\newcommand{\hypml}[2]{\left[\!\!#1\!\!\right]^{#2}}

%%%%for bussproof
\def\defaultHypSeparation{\hskip0.1in}
\def\ScoreOverhang{0pt}

\newcommand{\MultiLine}[1]{\renewcommand{\arraystretch}{.8}%
\ensuremath{\begin{array}[b]{l} #1 \end{array}}}

\newcommand{\MultiLineMod}[1]{%
\ensuremath{\begin{array}[t]{l} #1 \end{array}}}

\newcommand{\hypothesis}[2]{[ #1 ]^{#2}}

\newcommand{\LexEnt}[3]{#1; \ensuremath{#2}; \syncat{#3}}

\newcommand{\LexEntTwoLine}[3]{\renewcommand{\arraystretch}{.8}%
\begin{array}[b]{l} #1; \\ \ensuremath{#2};  \syncat{#3} \end{array}}

\newcommand{\LexEntThreeLine}[3]{\renewcommand{\arraystretch}{.8}%
\begin{array}[b]{l} #1; \\ \ensuremath{#2}; \\ \syncat{#3} \end{array}}

\newcommand{\LexEntFiveLine}[5]{\renewcommand{\arraystretch}{.8}%
\begin{array}{l} #1 \\ #2; \\ \ensuremath{#3} \\ \ensuremath{#4}; \\ \syncat{#5} \end{array}}

\newcommand{\LexEntFourLine}[4]{\renewcommand{\arraystretch}{.8}%
\begin{array}{l} \pt{#1} \\ \pt{#2}; \\ \syncat{#4} \end{array}}

\newcommand{\ManySomething}{\renewcommand{\arraystretch}{.8}%
\raisebox{-3mm}{\begin{array}[b]{c} \vdots \,\,\,\,\,\, \vdots \\
\vdots \end{array}}}

\newcommand{\lemma}[1]{\renewcommand{\arraystretch}{.8}%
\begin{array}[b]{c} \vdots \\ #1 \end{array}}

\newcommand{\lemmarev}[1]{\renewcommand{\arraystretch}{.8}%
\begin{array}[b]{c} #1 \\ \vdots \end{array}}

\newcommand{\p}{\ensuremath{\upvarphi}}

% clashes with soul package
\newcommand{\yusukest}{\textbf{\textsf{st}}}

\newcommand{\shortarrow}{\xspace\hskip-1.2ex\scalebox{.5}[1]{\ensuremath{\bm{\rightarrow}}}\hskip-.5ex\xspace}

\newcommand{\SemInt}[1]{\mbox{$[\![ \textrm{#1} ]\!]$}}

\newcommand{\HypSpace}{\hskip-.8ex}
\newcommand{\RaiseHeight}{\raisebox{2.2ex}}
\newcommand{\RaiseHeightLess}{\raisebox{1ex}}

\newcommand{\ThreeColHyp}[1]{\RaiseHeight{\Bigg[}\HypSpace#1\HypSpace\RaiseHeight{\Bigg]}}
\newcommand{\TwoColHyp}[1]{\RaiseHeightLess{\Big[}\HypSpace#1\HypSpace\RaiseHeightLess{\Big]}}

\newcommand{\LemmaShort}{\ensuremath{ \ \vdots} \ \noLine}
\newcommand{\LemmaShortAlt}{\ensuremath{ \ \vdots} \ }

\newcommand{\fail}{**}
\newcommand{\vs}{\raisebox{.05em}{\ensuremath{\upharpoonright}}}
\newcommand{\DerivSize}{\small}

\def\maru#1{{\ooalign{\hfil
  \ifnum#1>999 \resizebox{.25\width}{\height}{#1}\else%
  \ifnum#1>99 \resizebox{.33\width}{\height}{#1}\else%
  \ifnum#1>9 \resizebox{.5\width}{\height}{#1}\else #1%
  \fi\fi\fi%
\/\hfil\crcr%
\raise.167ex\hbox{\mathhexbox20D}}}}

\newenvironment{samepage2}%
 {\begin{flushleft}\begin{minipage}{\linewidth}}
 {\end{minipage}\end{flushleft}}

\newcommand{\cmt}[1]{\textsl{\textbf{[#1]}}}
\newcommand{\trns}[1]{\textbf{#1}\xspace}
\newcommand{\ptfont}{}
\newcommand{\gp}{\underline{\phantom{oo}}}
\newcommand{\mgcmt}{\marginnote}

\newcommand{\term}[1]{\emph{#1}}

\newcommand{\citeposs}[1]{\citeauthor{#1}'s \citeyearpar{#1}}

% for standalone compilations Felix: This is in the class already
%\let\thetitle\@title
%\let\theauthor\@author 
\makeatletter
\newcommand{\togglepaper}[1][0]{ 
\bibliography{../Bibliographies/stmue,../localbibliography,
../Bibliographies/properties,
../Bibliographies/np,
../Bibliographies/negation,
../Bibliographies/ellipsis,
../Bibliographies/binding,
../Bibliographies/complex-predicates,
../Bibliographies/control-raising,
../Bibliographies/coordination,
../Bibliographies/morphology,
../Bibliographies/lfg,
collection.bib}
  %% hyphenation points for line breaks
%% Normally, automatic hyphenation in LaTeX is very good
%% If a word is mis-hyphenated, add it to this file
%%
%% add information to TeX file before \begin{document} with:
%% %% hyphenation points for line breaks
%% Normally, automatic hyphenation in LaTeX is very good
%% If a word is mis-hyphenated, add it to this file
%%
%% add information to TeX file before \begin{document} with:
%% \include{localhyphenation}
\hyphenation{
A-la-hver-dzhie-va
anaph-o-ra
ana-phor
ana-phors
an-te-ced-ent
an-te-ced-ents
affri-ca-te
affri-ca-tes
ap-proach-es
Atha-bas-kan
Athe-nä-um
Bona-mi
Chi-che-ŵa
com-ple-ments
con-straints
Cope-sta-ke
Da-ge-stan
Dor-drecht
er-klä-ren-de
Ginz-burg
Gro-ning-en
Jap-a-nese
Jon-a-than
Ka-tho-lie-ke
Ko-bon
krie-gen
Le-Sourd
moth-er
Mül-ler
Nie-mey-er
Par-a-digm
Prze-piór-kow-ski
phe-nom-e-non
re-nowned
Rie-he-mann
un-bound-ed
with-in
}

% listing within here does not have any effect for lfg.tex % 2020-05-14

% why has "erklärende" be listed here? I specified langid in bibtex item. Something is still not working with hyphenation.


% to do: check
%  Alahverdzhieva


% biblatex:

% This is a LaTeX frontend to TeX’s \hyphenation command which defines hy- phenation exceptions. The ⟨language⟩ must be a language name known to the babel/polyglossia packages. The ⟨text ⟩ is a whitespace-separated list of words. Hyphenation points are marked with a dash:

% \DefineHyphenationExceptions{american}{%
% hy-phen-ation ex-cep-tion }

\hyphenation{
A-la-hver-dzhie-va
anaph-o-ra
ana-phor
ana-phors
an-te-ced-ent
an-te-ced-ents
affri-ca-te
affri-ca-tes
ap-proach-es
Atha-bas-kan
Athe-nä-um
Bona-mi
Chi-che-ŵa
com-ple-ments
con-straints
Cope-sta-ke
Da-ge-stan
Dor-drecht
er-klä-ren-de
Ginz-burg
Gro-ning-en
Jap-a-nese
Jon-a-than
Ka-tho-lie-ke
Ko-bon
krie-gen
Le-Sourd
moth-er
Mül-ler
Nie-mey-er
Par-a-digm
Prze-piór-kow-ski
phe-nom-e-non
re-nowned
Rie-he-mann
un-bound-ed
with-in
}

% listing within here does not have any effect for lfg.tex % 2020-05-14

% why has "erklärende" be listed here? I specified langid in bibtex item. Something is still not working with hyphenation.


% to do: check
%  Alahverdzhieva


% biblatex:

% This is a LaTeX frontend to TeX’s \hyphenation command which defines hy- phenation exceptions. The ⟨language⟩ must be a language name known to the babel/polyglossia packages. The ⟨text ⟩ is a whitespace-separated list of words. Hyphenation points are marked with a dash:

% \DefineHyphenationExceptions{american}{%
% hy-phen-ation ex-cep-tion }

  \memoizeset{
    memo filename prefix={hpsg-handbook.memo.dir/},
    % readonly
  }
  \papernote{\scriptsize\normalfont
    \@author.
    \@title. 
    To appear in: 
    Stefan Müller, Anne Abeillé, Robert D. Borsley \& Jean-Pierre Koenig (eds.)
    HPSG Handbook
    Berlin: Language Science Press. [preliminary page numbering]
  }
  \pagenumbering{roman}
  \setcounter{chapter}{#1}
  \addtocounter{chapter}{-1}
}
\makeatother

\makeatletter
\newcommand{\togglepaperminimal}[1][0]{ 
  \bibliography{../Bibliographies/stmue,
                ../localbibliography,
  ../Bibliographies/coordination,
collection.bib}
  %% hyphenation points for line breaks
%% Normally, automatic hyphenation in LaTeX is very good
%% If a word is mis-hyphenated, add it to this file
%%
%% add information to TeX file before \begin{document} with:
%% %% hyphenation points for line breaks
%% Normally, automatic hyphenation in LaTeX is very good
%% If a word is mis-hyphenated, add it to this file
%%
%% add information to TeX file before \begin{document} with:
%% \include{localhyphenation}
\hyphenation{
A-la-hver-dzhie-va
anaph-o-ra
ana-phor
ana-phors
an-te-ced-ent
an-te-ced-ents
affri-ca-te
affri-ca-tes
ap-proach-es
Atha-bas-kan
Athe-nä-um
Bona-mi
Chi-che-ŵa
com-ple-ments
con-straints
Cope-sta-ke
Da-ge-stan
Dor-drecht
er-klä-ren-de
Ginz-burg
Gro-ning-en
Jap-a-nese
Jon-a-than
Ka-tho-lie-ke
Ko-bon
krie-gen
Le-Sourd
moth-er
Mül-ler
Nie-mey-er
Par-a-digm
Prze-piór-kow-ski
phe-nom-e-non
re-nowned
Rie-he-mann
un-bound-ed
with-in
}

% listing within here does not have any effect for lfg.tex % 2020-05-14

% why has "erklärende" be listed here? I specified langid in bibtex item. Something is still not working with hyphenation.


% to do: check
%  Alahverdzhieva


% biblatex:

% This is a LaTeX frontend to TeX’s \hyphenation command which defines hy- phenation exceptions. The ⟨language⟩ must be a language name known to the babel/polyglossia packages. The ⟨text ⟩ is a whitespace-separated list of words. Hyphenation points are marked with a dash:

% \DefineHyphenationExceptions{american}{%
% hy-phen-ation ex-cep-tion }

\hyphenation{
A-la-hver-dzhie-va
anaph-o-ra
ana-phor
ana-phors
an-te-ced-ent
an-te-ced-ents
affri-ca-te
affri-ca-tes
ap-proach-es
Atha-bas-kan
Athe-nä-um
Bona-mi
Chi-che-ŵa
com-ple-ments
con-straints
Cope-sta-ke
Da-ge-stan
Dor-drecht
er-klä-ren-de
Ginz-burg
Gro-ning-en
Jap-a-nese
Jon-a-than
Ka-tho-lie-ke
Ko-bon
krie-gen
Le-Sourd
moth-er
Mül-ler
Nie-mey-er
Par-a-digm
Prze-piór-kow-ski
phe-nom-e-non
re-nowned
Rie-he-mann
un-bound-ed
with-in
}

% listing within here does not have any effect for lfg.tex % 2020-05-14

% why has "erklärende" be listed here? I specified langid in bibtex item. Something is still not working with hyphenation.


% to do: check
%  Alahverdzhieva


% biblatex:

% This is a LaTeX frontend to TeX’s \hyphenation command which defines hy- phenation exceptions. The ⟨language⟩ must be a language name known to the babel/polyglossia packages. The ⟨text ⟩ is a whitespace-separated list of words. Hyphenation points are marked with a dash:

% \DefineHyphenationExceptions{american}{%
% hy-phen-ation ex-cep-tion }

  \memoizeset{
    memo filename prefix={hpsg-handbook.memo.dir/},
    % readonly
  }
  \papernote{\scriptsize\normalfont
    \@author.
    \@title. 
    To appear in: 
    Stefan Müller, Anne Abeillé, Robert D. Borsley \& Jean-Pierre Koenig (eds.)
    HPSG Handbook
    Berlin: Language Science Press. [preliminary page numbering]
  }
  \pagenumbering{roman}
  \setcounter{chapter}{#1}
  \addtocounter{chapter}{-1}
}
\makeatother




% In case that year is not given, but pubstate. This mainly occurs for titles that are forthcoming, in press, etc.
\renewbibmacro*{addendum+pubstate}{% Thanks to https://tex.stackexchange.com/a/154367 for the idea
  \printfield{addendum}%
  \iffieldequalstr{labeldatesource}{pubstate}{}
  {\newunit\newblock\printfield{pubstate}}
}

\DeclareLabeldate{%
    \field{date}
    \field{year}
    \field{eventdate}
    \field{origdate}
    \field{urldate}
    \field{pubstate}
    \literal{nodate}
}

%\defbibheading{diachrony-sources}{\section*{Sources}} 

% if no langid is set, it is English:
% https://tex.stackexchange.com/a/279302
\DeclareSourcemap{
  \maps[datatype=bibtex]{
    \map{
      \step[fieldset=langid, fieldvalue={english}]
    }
  }
}


% for bibliographies
% biber/biblatex could use sortname field rather than messing around this way.
\newcommand{\SortNoop}[1]{}


% Doug Ball

\newcommand{\elist}{\q<\ \ \q>}

\newcommand{\esetDB}{\q\{\ \ \q\}}


\makeatletter

\newcommand{\nolistbreak}{%

  \let\oldpar\par\def\par{\oldpar\nobreak}% Any \par issues a \nobreak

  \@nobreaktrue% Don't break with first \item

}

\makeatother


% intermediate before Frank's trees are fixed
% This will be removed!!!!!
%\newcommand{\tree}[1]{} % ignore them blody trees
%\usepackage{tree-dvips}


\newcommand{\nodeconnect}[2]{}
\newcommand{\nodetriangle}[2]{}



% Doug relative clauses
%% I've compiled out almost all my private LaTeX command, but there are some
%% I found hard to get rid of. They are defined here.
%% There are few others which defined in places in the document where they have only
%% local effect (e.g. within figures); their names all end in DA, e.g. \MotherDA
%% There are a lot of \labels -- they are all of the form \label{sec:rc-...} or
%% \label{x:rc-...} or similar, so there should be no clashes.

% Subscripts -- scriptsize italic shape lowered by .25ex 
\newcommand{\subscr}[1]{\raisebox{-.5ex}{\protect{\scriptsize{\itshape #1\/}}}}
% A boxed subscript, for avm tags in normal text
\newcommand{\subtag}[1]{\subscr{\idx{#1}}}

%% Sets and tuples: I use \setof{} to get brackets that are upright, not slanted
%\newcommand{\setof}[1]{\ensuremath{\lbrace\,\mathit{#1}\,\rbrace}}
% 11.10.2019 EP: Doug requested replacement of existing \setof definition with the following:
%\newcommand{\setof}[1]{\begin{avm}\{\textcolor{red}{#1}\}\end{avm}}
% 31.1.2019 EP: Doug requested re-replacement of the above \textcolour version with the following:
\newcommand{\setof}[1]{\begin{avm}\{#1\}\end{avm}}

\newcommand{\tuple}[1]{\ensuremath{\left\langle\,\mbox{\textit{#1}}\,\right\rangle}}

% Single pile of stuff, optional arugment is psn (e.g. t or b)
% e.g. to put a over b over c in a centered column, top aligned, do:
%   \cPile[t]{a\\b\\c} 
\newcommand{\cPile}[2][]{%
  \begingroup%
  \renewcommand{\arraystretch}{.5}\begin{tabular}[#1]{c}#2\end{tabular}%
  \endgroup%
}

%% for linguistic examples in running text (`linguistic citation'):
\newcommand{\lic}[1]{\textit{#1}}

%% A gap marked by an underline, raised slightly
%% Default argument indicates how long the line should be:
\newcommand{\uGap}[1][3ex]{\raisebox{.25em}{\underline{\hspace{#1}}}\xspace}

%% \TnodeDA{XP}{avmcontents} -- in a Tree, put a node label next to an AVM
\newcommand{\TnodeDA}[2]{#1~\begin{avm}{#2}\end{avm}}

%% This allows tipa stuff to be put in \emph -- we need to change to cmr first.
%% It is used in the discussion of Arabic.
\newcommand{\emphtipa}[1]{{\fontfamily{cmr}\emph{\tipaencoding #1}}} 



 
 
\definecolor{lsDOIGray}{cmyk}{0,0,0,0.45}


% morphology.tex:
% Berthold

\newcommand{\dnode}[1]{\rnode{#1}{\fbox{#1}}}
\newcommand{\tnode}[1]{\rnode{#1}{\textit{#1}}}

\newcommand{\tl}[2]{#2}

\newcommand{\rrr}[3]{%
  \psframebox[linestyle=none]{%
    \avmoptions{center}
    \begin{avm}
      \[mud & \{ #1 \}\\
      ms & \{ #2 \}\\
      mph & \<  #3 \> \]
    \end{avm}
  }
}
\newcommand{\rr}[2]{%
  \psframebox[linestyle=none]{%
    \avmoptions{center}
    \begin{avm}
      \[mud & \{ #1 \}\\
      mph & \<  #2 \> \]
    \end{avm}
  }
}
 

% Frank Richter
\newtheorem{mydef}{Definition}

\long\def\set[#1\set=#2\set]%
{%
\left\{%
\tabcolsep 1pt%
\begin{tabular}{l}%
#1%
\end{tabular}%
\left|%
\tabcolsep 1pt%
\begin{tabular}{l}%
#2%
\end{tabular}%
\right.%
\right\}%
}

\newcommand{\einruck}{\\ \hspace*{1em}}


%\newcommand{\NatNum}{\mathrm{I\hspace{-.17em}N}}
\newcommand{\NatNum}{\mathbb{N}}
\newcommand{\Aug}[1]{\widehat{#1}}
%\newcommand{\its}{\mathrm{:}}
% Felix 14.02.2020
\DeclareMathOperator{\its}{:}

\newcommand{\sequence}[1]{\langle#1\rangle}

\newcommand{\INTERPRETATION}[2]{\sequence{#1\mathsf{U}#2,#1\mathsf{S}#2,#1\mathsf{A}#2,#1\mathsf{R}#2}}
\newcommand{\Interpretation}{\INTERPRETATION{}{}}

\newcommand{\Inte}{\mathsf{I}}
\newcommand{\Unive}{\mathsf{U}}
\newcommand{\Speci}{\mathsf{S}}
\newcommand{\Atti}{\mathsf{A}}
\newcommand{\Reli}{\mathsf{R}}
\newcommand{\ReliT}{\mathsf{RT}}

\newcommand{\VarInt}{\mathsf{G}}
\newcommand{\CInt}{\mathsf{C}}
\newcommand{\Tinte}{\mathsf{T}}
\newcommand{\Dinte}{\mathsf{D}}

% this was missing from ash's stuff.

%% \def \optrulenode#1{
%%   \setbox1\hbox{$\left(\hbox{\begin{tabular}{@{\strut}c@{\strut}}#1\end{tabular}}\right)$}
%%   \raisebox{1.9ex}{\raisebox{-\ht1}{\copy1}}}



\newcommand{\pslabel}[1]{}

\newcommand{\addpagesunless}{\todostefan{add pages unless you cite the
 work as such}}

% dg.tex
% framed boxes as used in dg.tex
% original idea from stackexchange, but modified by Saso
% http://tex.stackexchange.com/questions/230300/doing-something-like-psframebox-in-tikz#230306
\tikzset{
  frbox/.style={
    rounded corners,
    draw,
    thick,
    inner sep=5pt,
    anchor=base,
  },
}

% get rid of these morewrite messages:
% https://tex.stackexchange.com/questions/419489/suppressing-messages-to-standard-output-from-package-morewrites/419494#419494
\ExplSyntaxOn
\cs_set_protected:Npn \__morewrites_shipout_ii:
  {
    \__morewrites_before_shipout:
    \__morewrites_tex_shipout:w \tex_box:D \g__morewrites_shipout_box
    \edef\tmp{\interactionmode\the\interactionmode\space}\batchmode\__morewrites_after_shipout:\tmp
  }
\ExplSyntaxOff


% This is for places where authors used bold. I replace them by \emph
% but have the information where the bold was. St. Mü. 09.05.2020
\newcommand{\textbfemph}[1]{\emph{#1}}



% Felix 09.06.2020: copy code from the third line into localcommands.tex: https://github.com/langsci/langscibook#defined-environments-commands-etc
\patchcmd{\mkbibindexname}{\ifdefvoid{#3}{}{\MakeCapital{#3} }}{\ifdefvoid{#3}{}{#3 }}{}{\AtEndDocument{\typeout{mkbibindexname could not be patched.}}}

  %% -*- coding:utf-8 -*-

%%%%%%%%%%%%%%%%%%%%%%%%%%%%%%%%%%%%%%%%%%%%%%%%%%%%%%%%%%%%
%
% gb4e

% fixes problem with to much vertical space between \zl and \eal due to the \nopagebreak
% command.
\makeatletter
\def\@exe[#1]{\ifnum \@xnumdepth >0%
                 \if@xrec\@exrecwarn\fi%
                 \if@noftnote\@exrecwarn\fi%
                 \@xnumdepth0\@listdepth0\@xrectrue%
                 \save@counters%
              \fi%
                 \advance\@xnumdepth \@ne \@@xsi%
                 \if@noftnote%
                        \begin{list}{(\thexnumi)}%
                        {\usecounter{xnumi}\@subex{#1}{\@gblabelsep}{0em}%
                        \setcounter{xnumi}{\value{equation}}}
% this is commented out here since it causes additional space between \zl and \eal 06.06.2020
%                        \nopagebreak}%
                 \else%
                        \begin{list}{(\roman{xnumi})}%
                        {\usecounter{xnumi}\@subex{(iiv)}{\@gblabelsep}{\footexindent}%
                        \setcounter{xnumi}{\value{fnx}}}%
                 \fi}
\makeatother

% the texlive 2020 langsci-gb4e adds a newline after \eas, the texlive 2017 version was OK.
\makeatletter
\def\eas{\ifnum\@xnumdepth=0\begin{exe}[(34)]\else\begin{xlist}[iv.]\fi\ex\begin{tabular}[t]{@{}b{.99\linewidth}@{}}}
\makeatother


%%%%%%%%%%%%%%%%%%%%%%%%%%%%%%%%%%%%%%%%%%%%%%%%%%%%%%%%%%
%
% biblatex

% biblatex sets the option autolang=hyphens
%
% This disables language shorthands. To avoid this, the hyphens code can be redefined
%
% https://tex.stackexchange.com/a/548047/18561

\makeatletter
\def\hyphenrules#1{%
  \edef\bbl@tempf{#1}%
  \bbl@fixname\bbl@tempf
  \bbl@iflanguage\bbl@tempf{%
    \expandafter\bbl@patterns\expandafter{\bbl@tempf}%
    \expandafter\ifx\csname\bbl@tempf hyphenmins\endcsname\relax
      \set@hyphenmins\tw@\thr@@\relax
    \else
      \expandafter\expandafter\expandafter\set@hyphenmins
      \csname\bbl@tempf hyphenmins\endcsname\relax
    \fi}}
\makeatother


% the package defined \attop in a way that produced a box that has textwidth
%
\def\attop#1{\leavevmode\begin{minipage}[t]{.995\linewidth}\strut\vskip-\baselineskip\begin{minipage}[t]{.995\linewidth}#1\end{minipage}\end{minipage}}


%%%%%%%%%%%%%%%%%%%%%%%%%%%%%%%%%%%%%%%%%%%%%%%%%%%%%%%%%%%%%%%%%%%%


% Don't do this at home. I do not like the smaller font for captions.
% This does not work. Throw out package caption in langscibook
% \captionsetup{%
% font={%
% stretch=1%.8%
% ,normalsize%,small%
% },%
% width=\textwidth%.8\textwidth
% }
% \setcaphanging


  \togglepaper[18]
}{}

\author{Manfred Sailer\affiliation{Goethe-Unversität Frankfurt}}
\title{Idioms}

% \chapterDOI{} %will be filled in at production

%\epigram{Change epigram in chapters/03.tex or remove it there }
\abstract{
This chapter first sketches basic empirical properties of idioms. The state of the art before the emergence of HPSG is presented, followed by a discussion of four types of HPSG approaches to idioms. A section on future research closes the discussion.
}

\begin{document}
\maketitle
\label{chap-idioms}


%%% Index cross references
%\is{MWE|see{multiword expression}}
%%%


\kommentar{
\eal
\ex 
\gll dass er dem Mann das Buch gab\\
     that he the man the book gave\\
\glt `that he gave the man the book'
\ex
\gll dass er versucht, [dem Mann das Buch zu geben]\\
     that he tried     \spacebr{}the man the book to give\\
\glt `that he tried to give the man the book'
\zl

\citet{RS2009a}

In a previous paper,
\citet{Sailer:12}, I gave an overview over the development of idiom research in HPSG. 
Since that paper was in German and published in a Festschrift, its accessibility to a wider audience is rather limited. The planned chapter will follow the structure of this earlier paper, but take into consideration important recent developments in HPSG idiom research and relate it to the development in other formal frameworks, TAG \citep{Abeille:95,Lichte:Kallmeyer:16} and Minimalism \citep{vCraenenbroeck:al:16draft}.

The basic analytic challenge of idioms for HPSG is the fact that idioms are perceived as multiword, phrasal units, while the HPSG is strongly lexical, i.e., within the standard formalization of the framework \citep{King89,Richter:04}, it is impossible to implement an \emph{en bloc}-insertion theory of idioms. I will show in this paper that HPSG-accounts moved from trying to simulate a phrasal analysis to reaching the conviction that a lexical analysis is, indeed, empirically motivated.

}

\section{Introduction}
\label{Sec-Intro}
%\section{Introduction}

In this chapter, I will use the term \emph{idiom} interchangeably with the broader terms such as \emph{phraseme}\is{phraseme|see{idiom}}, \emph{phraseologism}\is{phraseologism|see{idiom}}, \emph{phraseological unit}\is{phraseological unit|see{idiom}}, or \emph{multiword expression}\is{MWE|see{idiom}}. This means, that I will subsume under this notion expressions such as prototypical idioms (\bspT{kick the bucket}{die}), support verb constructions (\bsp{take advantage}), formulaic expressions (\bsp{Good morning!}), and many more.%
\footnote{I will provide a paraphrase for all idioms at their first mention. They are also listed in the appendix, together with their paraphrase and a remark on which aspects of the idiom are discussed in the text.}
The main focus of the discussion will, however, be on prototypical idioms, as these have been in the center of the theoretical development.

%In the rest of this section, 
I will sketch some empirical aspects of idioms in Section~\ref{Sec-EmpiricalDomain}.
%and, then, characterize theoretical issues that arise in the formal modelling of idioms, both in general and with respect to HPSG in particular. 
In Section~\ref{Sec-Predecessors}, I will present the theoretical context within which idiom analyses arose in HPSG.
An overview of the development within HPSG will be given in Section~\ref{Sec-Analyses}. 
%Section~\ref{Sec-RecentOtherFrameworks} contains a brief sketch of the theoretical development outside HPSG. 
Desiderata for future research are mentioned in Section~\ref{Sec-WhereToGo}, before I close with a short conclusion.

%\subsection
\section{Empirical domain}
\label{Sec-EmpiricalDomain}
%Defining the empirical domain of idioms and phraseology. I will aim at a very inclusive definition, i.e., more in lines with ``phraseology'' than with ``idioms'' in the strict sense.

%I will assume the basic characterization of a phraseological unit from \citet{Fleischer97a-u} and \citet{Burger:98} as \emph{complex} units that show at least one of \emph{fixedness}, \emph{(semantic) idiomaticity}, and \emph{lexicalization}.

\is{idiomaticity|(}
In the context of the present handbook, the most useful characterization of idioms might be the definition of \emph{multiword expression}\is{multiword expression} from  \citet{Baldwin:Kim:10}.
%\footnote{See also \citet{Sailer:18SemComp} for a more detailed summary of }
For them, any combination of words counts as a multiword expression if it is syntactically complex and shows some degree of \emph{idiomaticity} (i.e., irregularity), be it lexical, syntactic, semantic, pragmatic, or statistical.%
\footnote{In the phraseological tradition, the aspect of \emph{lexicalization} is added \citep{Fleischer97a-u,Burger:98}. This means that an expression is stored in the lexicon. This criterion might have the same coverage as \emph{conventionalization} used in \citet{NSW94a}. 
These criteria are addressing the mental representation of idioms as a unit and are, thus, rather psycholinguistic in nature.}
%
I speak of a ``combination of words'' in the sense of a \emph{substantive} or \emph{lexically filled idiom}, which 
contrasts with \emph{formal} or \emph{lexically open idioms} \citep[505]{FKoC88a}. 

\citeauthor{Baldwin:Kim:10}'s criteria can help us structure the data presentation in this section, expanding their criteria where it seems suitable.
My expansions concern the aspect known as \emph{fixedness} in the phraseological tradition as in \citet{Fleischer97a-u}.%
\footnote{\citet{Baldwin:Kim:10} describe idioms in terms of syntactic fixedness, but they seem to consider fixedness a derived notion.}


%\medskip%
\is{idiomaticity!lexical|(}
For \citet{Baldwin:Kim:10}, \emph{lexical idiosyncrasy} concerns expressions with words that only occur in an idiom, so-called \emph{phraseologically bound words}\is{bound word}, or \emph{cranberry words}\is{cranberry word|see{bound word}} \citep{Aronoff76a-u}. Examples include \bspT{make headway}{make progress}, 
\bspT{take umbrage}{take offence},
%\bspT{at first blush}{at first sight}, 
\bspT{in a trice}{in a moment/""very quickly}.%
\footnote{
See https://www.english-linguistics.de/codii/, accessed 2019-09-03, for a list of bound words\is{bound word} in \ili{English} and \ili{German} \citep{Trawinski:al:08lrec}.}
For such expressions, the grammar has to make sure that the bound word does not occur outside the idiom, i.e., we need to prevent combinations such as \refer{trice-ko}.%
\footnote{Tom Wasow (p.c.) points out that there are attested uses of many alleged bound words\is{bound word} outside their canonical idiom, as in \pref{not-a-trice}. Such uses are, however, rare and restricted.
\ea
[]{Not a trice later, the sounds of gunplay were to be heard echoing from Bad Man's Rock. (COCA)\label{not-a-trice}}
\z 

}

\eal \label{trice}
\ex []{They fixed the problem in a trice.}
\ex [*]{It just took them a trice to fix the problem.\label{trice-ko}}
\zl 

We can expand this type of idiosyncrasy to include  a second important property of idioms. 
Most idioms have a fixed inventory of words. In their summary of this aspect of idioms, \cite[\page 827--828]{Gibbs:Colston:07} include the following examples: \bsp{kick the bucket} means `die', but \bsp{kick the pail}, \bsp{punt the bucket}, or \bsp{punt the pail} do not have this meaning. However, some degree of lexical variation seems to be allowed, as
the idiom \bspT{break the ice}{relieve tension in a strained situation} can be varied into \bsp{shatter the ice}.%
\footnote{\label{fn-semmeln}While \citet{Gibbs:Colston:07}, following \citet{Gibbs:al:89}, present this example as a lexical variation, \cite[\page 85]{Glucksberg:01}, from which it is taken, characterizes it as having a somewhat different aspect of an ``abrupt change in the social climate''. Clear cases of synonymy under lexical substitution are found with \ili{German} \bspTL{wie warme Semmeln/Brötchen/Schrippen weggehen}{like warm rolls vanish}{sell like hotcakes} in which some regional terms for rolls can be used in  the idiom.}
%For example, \cite[\page 85]{Glucksberg:01} mentions the variation \bsp{shatter the ice} of the idiom \bspT{break the ice}{XXX}.
 So, a challenge for idiom theories is to guarantee that the 
right lexical elements are used in the right constellation.
\is{idiomaticity!lexical|)}

%\medskip%
\is{idiomaticity!syntactic|(}
\emph{Syntactic idiomaticity} is used in \citet{Baldwin:Kim:10} to describe expressions that are not formed according to the productive rules of English syntax, following \citet{FKoC88a}, such as \bspT{by and large}{on the whole/""everything considered}, \bspT{trip the light fantastic}{dance}.  

In my expanded use of this notion, this also subsumes irregularities/""restrictions in the syntactic flexibility of an idiom, i.e., the question whether an idiom can occur in the same syntactic constructions as an analogous non"=idiomatic combination. In Transformational Grammar\is{Transformational Grammar}, such as \citet{Weinreich:69} and \citet{Fraser:70}, lists of different syntactic transformations were compiled and it was observed that some idioms allow for certain transformations but not for others. This method has been pursued systematically in the framework of \emph{Lexicon-Grammar}\is{Lexicon-Grammar} \citep{Gross:82}.%
\footnote{See \citet{Laporte:18} for a recent discussion of applying this method for a classification of idioms.}
%
\citet{SBBCF2002a-ausgedruckt} distinguish three levels of fixedness: \emph{fixed}, \emph{semi-fixed}, and \emph{flexible}. 
Completely fixed idioms include \bsp{of course}, \bsp{ad hoc} and are often called \emph{words with spaces}\is{word with spaces}.
Semi-fixed idioms allow for morphosyntactic variation such as inflection. These include some prototypical idioms (\bsp{trip the light fantastic}, \bsp{kick the bucket}) and complex proper names. In English, semi-fixed idioms show inflection, but cannot easily be passivized, nor do they allow for parts of the idiom being topicalized, see \refer{kick-ex}.

\eal \label{kick-ex} 
\ex []{Alex kicked / might kick the bucket.}
\ex [*]{The bucket was kicked by Alex.}
\ex [*]{The bucket, Alex kicked.}
\zl 

%For other languages, such as \ili{Dutch} and \ili{German}, \citet{Schenk:95} claims that other automatic/meaningless syntactic processes should be included as well, such as verb-second movement\is{verb second} and some types of fronting.


Flexible idioms pattern with free combinations. For them, we do not only find inflection, but also passivization, topicalization, pronominalization of parts, etc. Free combinations include some prototypical idioms (\bspT{spill the beans}{reveal a secret}, \bspT{pull strings}{exert influence/""use one's connections}), but also collocations (\bsp{brush one's teeth}) and light verbs (\bsp{make a mistake}).

The assumption of two flexibility classes is not uncontroversial: 
\citet{Horn:03} distinguishes two types among what \citet{SBBCF2002a-ausgedruckt} consider flexible idioms. 
\citet{Fraser:70} assumes six flexibility classes, looking at a wide range of syntactic operations.
\citet{Ruwet:91} takes issue with the cross-linguistic applicability of the classification of syntactic operations. Similarly, \citet{Schenk:95} claims that for languages such as \ili{Dutch} and \ili{German}, automatic/meaningless syntactic processes other than just inflection are possible for semi-fixed idioms, such as verb-second movement\is{verb second} and some types of fronting.

The analytic challenge of syntactic idiomaticity is to capture the difference in flexibility in a
non"=ad hoc way. It is this aspect of idioms that has received particular attention in Mainstream
Generative\is{Generative Grammar} Grammar (MGG),\footnote{%
  I follow \citet[\page 3]{CJ2005a} in using the term \emph{Mainstream Generative Grammar} to
  refer to work in Minimalism and the earlier Government \& Binding framework.}$^,$\footnote{See the
  references in \citet{Corver:al:19} for a brief up-to-date overview of MGG work.}
but also in the HPSG approaches sketched in Section~\ref{Sec-Analyses}.
\is{idiomaticity!syntactic|)}

%\medskip%
\is{idiomaticity!semantic|(}
\emph{Semantic idiomaticity} may sound pleonastic, as, traditionally, an expression is called idiomatic if it has a conventional meaning that is different from its literal meaning. 
Since I use the terms idiom and idiomaticity in their broad senses of phraseological unit and irregularity, respectively, the qualification \emph{semantic} idiom(aticity) is needed. 

One challenge of the modelling of idioms is capturing the relation between the literal and the idiomatic meaning of an expression.
\citet{Gibbs:Colston:07} give an overview of 
psycholinguistic research on idioms. Whereas it was first assumed that speakers would compute the literal meaning of an expression and then derive the idiomatic meaning, evidence has been accumulated that the idiomatic meaning is accessed directly.

\citet{WSN84a-u} and \citet{NSW94a} explore various semantic relations for idioms, in particular \emph{decomposability} \is{decomposability} and \emph{transparency}\is{transparency}.
An idiom is \emph{decomposable} if its idiomatic meaning can be distributed over its component parts in such a way that we would arrive at the idiomatic meaning of the overall expression if we interpreted the syntactic structure on the basis of such a meaning assignment. 
The idiomatic meaning of the expression \bsp{pull strings} can be decomposed by interpreting \bsp{pull} as \bsp{exploit/use} and \bsp{strings} as \bsp{connections}. 
%To make this criterion non"=arbitrary, it is now common to require that there be empirical support for such a decomposition. In English, we can often insert an adjective 
The expressions \bsp{kick the bucket} and \bspT{saw logs}{snore} are not decomposable.

An idiom is \emph{transparent} if there is a synchronically accessible relation between the literal and the idiomatic meaning of an idiom. 
For some speakers, \bsp{saw logs} is transparent in this sense, as the noise produced by this activity is similar to a snoring noise. 
For \bsp{pull strings}, there is an analogy to a puppeteer controlling the puppets' behavior by pulling strings. A non"=transparent idiom is called \emph{opaque}. 

Some idioms do not show semantic idiomaticity at all, such as collocations\is{collocation} (\bsp{brush one's teeth}) or support verb constructions (\bsp{take a shower}). 
Many body-part expressions such as \bspT{shake hands}{greet} or \bspT{shake one's head}{decline/""negate} constitute a more complex case: They describe a conventionalized activity and denote the social meaning of this activity.% 
\footnote{The basic reference for the phraseological properties of body-part expressions is \cite{Burger:76}.}

In addition, we might need to assume a \emph{figurative} interpretation. For some expressions, in particular proverbs\is{proverb} or cases like 
\bspT{take the bull by the horns}{approach a problem directly}) we might get a figurative reading rather than an idiomatic reading. 
%
\citet{Glucksberg:01} explicitly distinguishes between idiomatic and figurative interpretations. In his view, the above-mentioned case of \bsp{shatter the ice} would be a figurative use of the idiom \bsp{break the ice}. 
While there has been a considerable amount of work on figurativity in psycholinguistics, the integration of its results into formal linguistics is still a desideratum.%
\is{idiomaticity!semantic|)}



%\medskip%
\is{idiomaticity!pragmatic|(}
\emph{Pragmatic idiomaticity} covers expressions that have a \emph{pragmatic point} in the terminology of \citet{FKoC88a}. These include complex formulaic expressions (\bsp{Good morning!}). There has been little work on this aspect of idiomaticity in formal phraseology.
\is{idiomaticity!pragmatic|)}

%\medskip%
\is{idiomaticity!statistical|(}
The final type of idiomaticity is \emph{statistical idiomaticity}. 
Contrary to the other idiomaticity criteria, this is a usage-based aspect. If we find a high degree of co-occurrence of a particular combination of words that is idiosyncratic for this combination, we can speak of a statistical idiomaticity. This category includes \emph{collocations}\is{collocation}. \citet{Baldwin:Kim:10} mention \bsp{immaculate performance} as an example. Collocations are important in computational linguistics and in foreign-language learning, but their status for theoretical linguistics and for a competence-oriented\is{competence orientation} framework such as HPSG is unclear. 
\is{idiomaticity!statistical|)}

%\bigskip%
This discussion of the various types of idiomaticity shows that idioms do not form a homogeneous empirical domain but rather are defined negatively. 
This leads to the basic analytical challenge of idioms: while the empirical domain is defined by  absence of regularity in at least one aspect, idioms largely obey the principles of grammar. 
In other words, there is a lot of regularity in the domain of idioms, while any approach still needs to be able to model the irregular properties. 
%I have tried to introduce the notions that are most commonly used in HPSG research and to identify the analytical problems related to them. 


\is{idiomaticity|)}

%\subsection{Language-theoretical interest in idioms}
%\label{Sec-TheoreticalInterest}

%\begin{itemize}
%\item between rule-based and idiosyncratic
%\item compositional challenge and collocational challenge
%\end{itemize}


\section{Predecessors to HPSG analyses of idioms}
\label{Sec-Predecessors}


In this section, I will sketch the theoretical environment within which HPSG and HPSG analyses of idioms have emerged.

The general assumption about idioms in MGG\is{Generative Grammar} is that they
must be represented as a complex phrasal form-meaning unit. 
Such units are inserted \emph{en bloc}\is{en bloc insertion} into the structure rather than built by syntactic operations.
This view goes back to \cite[\page 190]{Chomsky:65}. 
With this unquestioned assumption, arguments for or against particular analyses can be constructed. 
To give just one classical example, \citet{Chomsky81a} uses the passivizabilty of some idioms as an argument for the existence of Deep Structure, i.e., a structure on which the idiom is inserted holistically. 
%
\citet{Ruwet:91} and \citet{NSW94a} go through a number of such lines of argumentation showing their basic problems. 

The holistic view on idioms is most plausible for idioms that show many types of idiomaticity at the same time, though it becomes more and more problematic if only one or a few types of idiomaticity are attested.
HPSG is less driven by analytical pre-decisions than other frameworks; see \crossrefchaptert{minimalism}. Nonetheless, idioms have been used to motivate assumptions about the architecture of linguistic signs in HPSG as well.

\citet{WSN84a-u} and \citet{NSW94a} are probably the two most influential papers in formal phraseology in the last decades. 
%These papers have also shaped the analysis of idioms in \emph{Generalized Phrase Structure Grammar} \citep{GKPS85a}\is{Generalized Phrase Structure Grammar} and, consequently in HPSG. 
While there are many aspects of \citet{NSW94a} that have not been integrated into the formal modelling of idioms, 
there are at least two insights that have been widely adapted in HPSG.
First, not all idioms should be represented holistically. 
Second, the syntactic flexibility of an idiom is related to its semantic decomposability. 
In fact, \citet{NSW94a} state this last insight even more generally:%
\footnote{Aspects of this approach are already present in \citet{Higgins:74} and \citet{Newmeyer:74}.}

%\ea \label{NSW-quote} \citet[\page 531]{NSW94a}:
\begin{quote}
We predict that the syntactic flexibility of a particular idiom will ultimately be explained in terms of the compatibility of its semantics with the semantics and pragmatics of various constructions.
\citep[\page 531]{NSW94a}\label{NSW-quote}
\end{quote}
%\z 

%One of the theoretical predecessors of HPSG is \emph{Generalized Phrase Structure Grammar} \citep{GKPS85a}\is{Generalized Phrase Structure Grammar}. 


\citet{WSN84a-u} and \citet{NSW94a} propose a simplified first approach to a theory that would be in line with this quote. They argue that, for English, there is a correlation between syntactic flexibility and semantic decomposability in that non"=decomposable idioms are only semi-fixed, whereas decomposable idioms are flexible, to use our terminology from Section~\ref{Sec-EmpiricalDomain}. 
\is{Generalized Phrase Structure Grammar|(}
This idea has been directly encoded formally in the idiom theory of
\citet[Chapter~7]{GKPS85a}, who define the framework of 
\emph{Generalized Phrase Structure Grammar}\is{Generalized Phrase Structure Grammar} (GPSG)\is{GPSG|see{Generalized Phrase Structure Grammar}}.

\citet{GKPS85a} assume that  non"=decomposable idioms are inserted into sentences \emph{en bloc}\is{en bloc insertion}, i.e., as fully specified syntactic trees which are assigned the idiomatic meaning holistically. This means that the otherwise strictly context-free grammar of GPSG needs to be expanded by adding a (small) set of larger trees. 
Since non"=decomposable idioms are inserted as units, their parts cannot be accessed for syntactic operations such as passivization or movement. Consequently, the generalization about semantic non"=decomposability and syntactic fixedness of English idioms from \citet{WSN84a-u} is implemented directly.

Decomposable idioms are analyzed as free combinations in syntax. The idiomaticity of such expressions is achieved by two assumptions: First, there is lexical ambiguity, i.e., for an idiom like \bsp{pull strings}, the verb \bsp{pull} has both a literal meaning and an idiomatic meaning. Similarly for \bsp{strings}.
Second, \citet{GKPS85a} assume that lexical items are not necessarily translated into total functions but can be partial functions. Whereas the literal meaning of \bsp{pull} might be a total function, the idiomatic meaning of the word would be a partial function that is only defined on elements that are in the denotation of the idiomatic meaning of \bsp{strings}. This analysis predicts syntactic flexibility for decomposable idioms, just as proposed in \citet{WSN84a-u}.

\is{Generalized Phrase Structure Grammar|)}



%Related to this general dilemma are two more concrete analytical challenges, which \cite[\page 12]{Bargmann:Sailer:18} call the \emph{compositional challenge} and the \emph{collocational challenge}. The compositional challenge consists in associating a sequence of words with a non"=literal, idiomatic, meaning. The collocational challenge consists in making sure that the components of an idiom all occur together in the right constellation.


%\section{Predecessors to HPSG analyses of idioms}
%\label{Sec-Predecessors}

%\begin{itemize}
%\item Generalized Phrase Structure Grammar \citep{GKPS85a}
%\item Semi-formal, influential papers \citep{WSN84a-u,NSW94a}
%\item Construction Grammar \citep{FKoC88a}
%\end{itemize}


\citet[511--514]{NSW94a} show that the connection between semantic decomposability and syntactic flexibility is not as straightforward as suggested. They say that, in German and Dutch, ``noncompositional\is{idiom!non-decomposable} idioms  are syntactically versatile'' \citep[514]{NSW94a}. Similar observations have been brought forward for French in \citet{Ruwet:91}. 
\citet{Bargmann:Sailer:18} and \cite{Fellbaum:19} argue that even for English, passive examples are attested for non"=decomposable idioms such as \pref{ex-kick-fellbaum}.

\ea 
Live life to the fullest, you never know when the bucket will be kicked. \citep[756]{Fellbaum:19}\label{ex-kick-fellbaum}
\z 

The current state of our knowledge of the relation between syntactic and semantic idiosyncrasy is that the semantic idiomaticity of an idiom does have an effect on its syntactic flexibility, though the relation is less direct than assumed in the literature based on \cite{WSN84a-u} and \cite{NSW94a}.


\section{HPSG analyses of idioms}
\label{Sec-Analyses}

%\subsection{HPSG-specific research questions}

\kommentar{
\begin{itemize}
\item Phrasal entities in a lexical framework? i.e., basic problem of phrasal vs.\@ lexical analyses in a framework like HPSG.
\end{itemize}
}

HPSG does not make a core-periphery\is{core-periphery distinction} distinction; see \citet{MuellerKernigkeit}. Consequently, idioms belong to the empirical domain to be covered by an HPSG grammar.
Nonetheless, idioms are not discussed in \citet{ps2} and their architecture of grammar does not have a direct place for an analysis of idioms.%
\footnote{This section follows the basic structure and argument of \citet{Sailer:12} and \citet{Richter:Sailer:14}.} 
They situate all idiosyncrasy in the lexicon, which consists of lexical entries for basic words. 
Every word has to satisfy a lexical entry and all principles of grammar; see \crossrefchaptert{lexicon}.%
\footnote{I refer to the lexicon\is{lexicon} in the technical sense as the collection of lexical entries, i.e., as \emph{descriptions}, rather than as a collection of lexical items, i.e., linguistic signs. 
Since \citet{ps2} do not discuss morphological processes, their lexical entries describe full forms. 
If there is a finite number of such lexical entries, the
lexicon can be expressed by a \emph{Word Principle}\is{Word Principle}, a constraint on words that contains a disjunction of all such lexical entries. 
Once we include morphology, lexical rules, and phrasal lexical entries in the picture, we need to refine this simplified view.
%
}
%
All properties of a phrase can be inferred from the properties of the lexical items occurring in the phrase and the constraints of grammar. 

In their grammar, \citet{ps2} adhere to the \emph{Strong Locality Hypothesis} (SLH),\is{locality!Strong Locality Hypothesis}\is{locality|(} i.e., all lexical entries describe leaf nodes in a syntactic structure and all phrases are constrained by principles that only refer to local (i.e., \type{synsem}) properties of the phrase and to local properties of its immediate daughters. This hypothesis is summarized in \refer{slh}.

\vbox{
\ea Strong Locality Hyphothesis\label{slh} (SLH)\\
The rules and principles of grammar are statements on a single node of a linguistic structure or on nodes that are immediately dominated by that node.\is{SLH|see{locality!Strong Locality Hypothesis}}
\z 

}

This precludes any purely phrasal approaches to idioms. 
Following the heritage of GPSG\is{Generalized Phrase Structure Grammar}, we would assume that all regular aspects of linguistic expressions can be handled by mechanisms that follow the SLH, 
whereas idiomaticity would be a range of phenomena that may violate it. 
It is, therefore, remarkable that a grammar framework that denies a core-periphery distinction would start with a strong assumption of locality, and, consequently, of regularity. 

This is in sharp contrast to the basic motivation of Construction Grammar\is{Construction Grammar}, which assumes that constructions can be of arbitrary depth and of an arbitrary degree of idiosyncrasy. 
\citet{FKoC88a} use idiom data and the various types of idiosyncrasy discussed in Section~\ref{Sec-EmpiricalDomain}
as an important motivation for this assumption. 
To contrast this position clearly with the one taken in \citet{ps2}, I will state the \emph{Strong Non"=locality Hypothesis}\is{locality!Strong Non"=locality Hypothesis} (SNH) in \refer{snh}. 

\vbox{
\ea Strong Non"=locality Hypothesis\is{locality!Strong Non-locality Hypothese} (SNH)\is{SNH|see{locality!Strong Non-locality Hypothesis}}\label{snh}\\
The internal structure of a construction can be arbitrarily deep and show an arbitrary degree of irregularity at any substructure.
\z

}


The actual formalism used in \citet{ps2} and \citet{King89} -- see \crossrefchaptert{formal-background} -- does not require the strong versions of the locality and the non"=locality hypotheses, but is compatible with weaker versions. I will call these the \emph{Weak Locality Hypothesis} (WLH),\is{locality!Weak Locality Hypothesis} and the 
\emph{Weak Non"=locality Hypothesis} (WNH); see \refer{wlh} and \refer{wnh} respectively.

\vbox{
\ea  Weak Locality Hypothesis (WLH)\label{wlh}\\
At most the highest node in a structure is licensed by a rule of grammar or a lexical entry.\is{WLH|see{locality!Weak Locality Hypothesis}}
%The rules and principles of grammar can constrain the internal structure of a linguistic sign at arbitrary depth, but each sign needs to be licensed independently.
\z 

}

According to the WLH, just as in the SLH, each sign needs to be licensed by the lexicon and/""or the grammar. 
This precludes any \emph{en bloc}-insertion analyses, which would be compatible with the SNH.
%
According to the WNH, in line with the SLH, a sign can, however, impose further constraints on its component parts, that may go beyond local (i.e., \type{synsem}) properties of its immediate daughters.%


\ea Weak Non"=locality Hypothesis\is{locality!Weak Non-locality Hypothesis} (WNH)\is{WNH|see{locality!Weak Non-locality Hypothesis}}\label{wnh}\\
The rules and principles of grammar can constrain -- though not license -- the internal structure of a linguistic sign at arbitrary depth.
\z 

This means that all substructures of a syntactic node need to be licensed by the grammar, but the node may impose idiosyncratic constraints on which particular well-formed substructures it may contain.



\kommentar{
For this reason, HPSG approaches have attempted to argue that there is more regularity than meets the eye and, basically, try to treat idioms very much like free combinations. 
This is also the perspective in which idioms have been used to argue for or against a particular organization of linguistic signs: attributes and sorts can be motivated just by their necessity for an analysis of idioms. 
}

\bigskip%
In this section, I will review four types of analyses developed within HPSG in a mildly chronological order:
First, I will discuss a conservative extension of \citet{ps2} for idioms \citep{KE94a} that sticks to the SLH. 
Then, I will look at attempts to incorporate constructional ideas more directly, i.e., ways to include a version of the SNH. 
The third type of approach will exploit the WLH. Finally, I will summarize recent approaches, which are, again, emphasizing the locality of idioms.



\is{locality|)}

\subsection{Early lexical approaches}
\label{Sec-EarlyLexical}

\kommentar{
\begin{itemize}
\item \citet{KE94a}
%\citet{Krenn:Erbach:94}
\end{itemize}

\citet{SS2003a}
}


\citet{KE94a}, based on \citet{Erbach92a}, present the first comprehensive HPSG account of idioms. 
They look at a wide variety of different types of German idioms, including support verb constructions. 
They only modify the architecture of \citet{ps2} marginally and stick to the Strong Locality Hypothesis. 
They base their analysis on the apparent correlation between syntactic flexibility and semantic decomposability from \citet{WSN84a-u} and \citet{NSW94a}. 
Their analysis is a representational variant of the analysis in \citet{GKPS85a}.

To maintain the SLH, \citet{KE94a} assume that the information available in syntactic selection is slightly richer than what has been assumed in \citet{ps2}:
First, they use a lexeme-identification feature, \attrib{lexeme}, which is located inside the \attrib{index} value and whose value is the semantic constant associated with a lexeme. 
Second, they include a feature \attrib{theta-role}, whose value indicates which thematic role a sign is assigned in a structure. In addition to standard thematic roles, they include a dummy value \type{nil}.
Third, as the paper was written in the transition phase between \citet{ps} and \citet{ps2}, they assume that the selectional attributes contain complete \type{sign} objects rather than just \type{synsem} objects. 
Consequently, selection for phonological properties and internal constituent structure is possible, which we could consider a violation of the SLH. 

The effect of these changes in the analysis of idioms can be seen in \refer{ke-spill} and \refer{ke-kick}. In \refer{ke-spill}, I sketch the analysis of the syntactically flexible, decomposable idiom \emph{spill the beans}.
There are individual lexical items for the idiomatic words.%
\footnote{We do not need to specify the \attrib{rel} value for the noun \emph{beans}, as the \attrib{listeme} and the \attrib{rel} value are usually identical.}


\eal % Analysis of \emph{spill the beans} in the spirit of \citet{KE94a}
\label{ke-spill}
\ex 
\ms{phon & \phonliste{spill}\\
synsem & \ms{cat & \ms{subcat & \liste{NP, NP\ms{lexeme & beans\_i}}}\\
            cont & \ms{rel & spill\_i}
}}
\ex 
\ms{phon & \phonliste{beans}\\
synsem & \ms{content & \ms{index & \ms{lexeme & beans\_i}}}}
\zl 

The \attrib{lexeme} values of these words can be used to distinguish them from their ordinary, non"=idiomatic homonyms. 
Each idiomatic word comes with its idiomatic meaning, which models the decomposability of the expression. 
The lexical items satisfying the entries in \refer{ke-spill} can undergo lexical rules such as passivization. 

The idiomatic verb \emph{spill} selects an NP complement with the \attrib{lexeme} value \type{beans\_i}. 
The lexicon is built in such a way that no other word selects for this \attrib{lexeme} value. 
This models the lexical fixedness of the idiom.

The choice of putting the lexical identifier into the \attrib{index} guarantees that it is shared between a lexical head and its phrase, which allows for syntactic flexibility inside the NP. 
Similarly, the information shared between a trace and its antecedent contains the \attrib{index} value. Consequently, participation in unbounded dependency constructions is equally accounted for.
Finally, since a pronoun has the same \attrib{index} value as its antecedent, pronominalization is also possible. 
%


I sketch the analysis of a non"=decomposable, fixed idiom, \emph{kick the bucket}, in \refer{ke-kick}. 
In this case, there is only a lexical entry of the syntactic head of the idiom, the verb \emph{kick}. 
It selects the full phonology of its complement. This blocks any syntactic processes inside this NP. It also follows that the complement cannot be realized as a trace, which blocks extraction.%
\footnote{See \crossrefchaptert{udc} for details on the treatment of extraction in HPSG.}
The special \attrib{theta-role} value \type{nil} will be used to 
restrict the lexical rules that can be applied. 
The passive lexical rule, for example, would be specified in such a way that it cannot apply if the NP complement in its input has this theta-role.


\ea % Analysis of \emph{spill the beans} in the spirit of \citet{KE94a} 
\label{ke-kick}
\ms{phon & \phonliste{kick}\\
synsem & \ms{cat & \ms{subcat & \liste{NP, 
                                 NP\ms{phon & \phonliste{the, bucket}\\
                                       theta-role & nil
                        }}}\\
             cont & \ms{rel & die}}
}
\z 


With this analysis, \citet{KE94a} capture both the idiosyncratic aspects and the regularity of idioms. 
They show how it generalizes to a wide range of idiom types. 
I will briefly mention some problems of the approach, though.
%There are, however, a number of problems. I will just mention few of them here.

There are two problems for the analysis of non"=decomposable idioms. 
First, the approach is too restrictive with respect to the syntactic flexibility of \emph{kick the bucket}, as it excludes cases such as \emph{kick the social/figurative bucket}, which are discussed in  \citet{Ernst:81}. 
Second, it is built on equating the class of non"=decomposable idioms with that of semi-fixed idioms. As shown in my discussion around example \pref{ex-kick-fellbaum}, this cannot be maintained. 

There are also some undesired properties of the \textsc{lexeme} value selection. The index identity between a pronoun and its antecedent would require that the subject of the relative clause\is{relative clause} in \refer{strings-relcl} has the same \attrib{index} value as the head noun \emph{strings}. However, the account of the lexical fixedness of idioms is built on the assumption that no verb except for the idiomatic \emph{pull} selects for an argument with \attrib{lexeme} value \type{strings\_i}.%
\footnote{\citet{Pulman:93} discusses the analogous problem for the denotational theory of \citet{GKPS85a}.}

\ea \label{strings-relcl}
Parky pulled the strings that got me the job.
\citep[137]{McCawley:81}
\z 

Notwithstanding these problems, 
the analytic ingredients of \citet{KE94a} constitute the basis of later HPSG analyses. In particular, a mechanism for lexeme-specific selection has been widely assumed in most approaches. The attribute \attrib{theta-role} can be seen as a simple form of an \emph{inside-out} mechanism\is{inside-out constraint}, i.e., as a mechanism of encoding information about the larger structure within which a sign appears. 

%\citet{SS2003a}: inside out, 
%\citet{Sag2012a}

\subsection{Phrasal approach}
\label{Sec-Phrasal}

\is{constructional HPSG|(}
With the advent of constructional analyses within HPSG, starting with \citet{Sag97a}, it is natural to expect phrasal accounts of idioms to emerge as well, as idiomaticity is a central empirical domain for Construction Grammar\is{Construction Grammar}; see \crossrefchaptert{cxg}. 
In this version of HPSG, there is an elaborate type hierarchy below \type{phrase}. 
\citet{Sag97a} also introduces \emph{defaults}\is{default} into HPSG, which play an important role in the treatment of idioms in constructional HPSG.
The clearest phrasal approach to idioms can be found in \citet{Riehemann2001a}, which incorporates insights from earlier publications such as \citet{Riehemann97a} and \citet{RB99a}.
%
The overall framework of \citet{Riehemann2001a} is constructional HPSG with \emph{Minimal Recursion Semantics}\is{Minimal Recursion Semantics}  \citep{CFMRS95a-u,CFPS2005a}; see also \crossrefchaptert{semantics}.

For Riehemann, idioms are phrasal units. 
Consequently, she assumes a subtype of \type{phrase} for each idiom, such as \type{spill-beans-idiomatic-phrase} or \type{kick-bucket-idiomatic-phrase}.
The proposal in \citet{Riehemann2001a} simultaneously is phrasal and obeys the SLH. To achieve this, \citet{Riehemann2001a} assumes an attribute \attrib{words}, whose value contains all words dominated by a phrase. This makes it possible to say that a phrase of type \type{spill-beans-idiomatic-phrase} dominates the words \emph{spill} and \emph{beans}. This is shown in the relevant type constraint for the idiom \emph{spill the beans} in \refer{sr-spillbeans}.%
\footnote{The percolation mechanism for the feature \attrib{words} is rather complex. In fact, in \cite[Section 5.2.1]{Riehemann2001a} the idiom-specific words appear within a \attrib{c-words} value and the other words are dominated by the idiomatic phrase in the value of an attribute \attrib{other-words}, both of which together form the value of \attrib{words}. While all the values of these features are subject to local percolation principles, the fact that entire words are percolated undermines the locality intuition behind the SLH.}


\vbox{
\ea Constraint on the type \type{spill-beans-idiomatic-phrase} from \citet[185]{Riehemann2001a}:\label{sr-spillbeans}\\
\ms[spill-beans-ip]{
words & \ensuremath{\left\{
        \begin{tabular}{@{}l@{}}
        \ms[i\_spill]{\ldots{} liszt & \liste{
                               \ms[i\_spill\_rel]{undergoer & \ibox{1}}}
                     } 
        \srdefault 
        \ms{\ldots{} liszt & \liste{\type{\_spill\_rel}}},\\
        \ms[i\_beans]{\ldots{} liszt & \liste{
                                       \ms[i\_beans\_rel]{inst& \ibox{1}}}
                     }
        \srdefault 
        \ms{\ldots{} liszt & \liste{\type{\_beans\_rel}}}, \ldots 
        \end{tabular}
        \right\}}
}
\z 
}

The \attrib{words} value of the idiomatic phrase contains at least two elements, the idiomatic words of type \type{i\_spill} and \type{i\_beans}. 
The special symbol \srdefault\ used in this constraint expresses a default\is{default}. It says that the idiomatic version of the word \emph{spill} is just like its non"=idiomatic homonym, except for the parts specified in the left-hand side of the default. 
In this case, the type of the words and the type of the semantic predicate contributed by the words are changed. 
\citet{Riehemann2001a} only has to introduce the types for the idiomatic words in the type hierarchy but need not specify type constraints on the individual idiomatic words, as these are constrained by the default statement within the constraints on the idioms containing them.


As in the account of \citet{KE94a}, the syntactic flexibilty of the idiom follows from its free syntactic combination and the fact that all parts of the idiom are assigned an independent semantic contribution. The lexical fixedness is a consequence of the requirement that particular words are dominated by the phrase, namely the idiomatic versions of \emph{spill} and \emph{beans}.

The appeal of the account is particularly clear in its application to non"=de\-com\-posable, semi-fixed idioms such as \emph{kick the bucket} \citep[\page 212]{Riehemann2001a}. 
For such expressions, the idiomatic words that constitute them are assumed to have an empty semantics and the meaning of the idiom is contributed as a constructional semantic contribution only by the idiomatic phrase. 
Since the \attrib{words} list contains entire words, it is also possible to require that the idiomatic word \emph{kick} be in active voice and/""or that it take a complement compatible with the description of the idiomatic word \emph{bucket}.
This analysis captures the syntactically regular internal structure of this type of idioms and is compatible with the occurrence of modifiers such as \emph{proverbial}. At the same time, it prevents passivization and excludes extraction of the complement.

Riehemann's approach clearly captures the intuition of idioms as phrasal units much better than any other approach in HPSG. 
However, it faces a number of problems.
First, the integration of the approach with constructional HPSG is done in such a way that the phrasal types for idioms are cross-classified in complex type hierarchies with the various syntactic constructions in which the idiom can appear. 
This allows Riehemann to account for idiosyncratic differences in the syntactic flexibility of idioms, but the question is whether such an explicit encoding misses generalizations that should follow from independent properties of the components of an idiom and/""or of the syntactic construction -- in line with the quote from \citet{NSW94a} on page \pageref{NSW-quote}.


Second, the mechanism of percolating dominated words to each phrase is not compatible with the intuitions of most HPSG researchers. 
Since no empirical motivation for such a mechanism aside from idioms is provided in \citet{Riehemann2001a}, this idea has not been pursued in other papers. 

Third, the question of how to block the free occurrence of idiomatic words, i.e., the occurrence of an idiomatic word without the rest of the idiom, is not solved in \citet{Riehemann2001a}. While the idiom requires the presence of particular idiomatic words, the occurrence of these words is not restricted.%
\footnote{Since the problem of free occurrences of idiomatic words is not an issue for parsing, versions of Riehemann's approach have been integrated into practical parsing systems \citep{Villavicencio:Copestake:02}; see \crossrefchaptert{cl}. 
Similarly, the approach to idioms sketched in \citet{Flickinger:15Slides2} 
is part of a system for parsing and machine translation\is{machine translation}. Idioms in the source language are identified by bits of semantic representation -- analogous to the elements in the \attrib{words} set. This approach, however, does not constitute a theoretical modelling of idioms; it does not exclude ill-formed uses of idioms but identifies potential occurrences of an idiom in the output of a parser.}
Note that idiomatic words may sometimes be found without the other elements of the idiom -- evidenced by expressions such as in \emph{bucket list} `list of things to do before one dies'.
Such data may be considered as support of Riehemann's approach; however, the extent to which we find such free occurrences of idiomatic words is extremely small.%
\footnote{See the discussion around \pref{trice} for a parallel situation with bound words.}


%\bigskip%
Before closing this subsection, I would like to point out that 
\citet{Riehemann2001a} and \citet{RB99a} are the only HPSG papers on idioms that address the question of statistical idiomaticity\is{statistical idiomaticity}, based on the variationist study in \citet{Bender2000a}. 
In particular, \citet[\page 297--301]{Riehemann2001a} proposes phrasal constructions for collocations even if these do not show any lexical, syntactic, semantic, or pragmatic idiosyncrasy but just a statistical co-occurrence preference. 
She extends this into a larger plea for an \emph{experience-based HPSG}\is{experience-based HPSG}. 
%
\citet{Bender2000a} discusses the same idea under the notions of \emph{minimal} versus \emph{maximal} grammars, i.e., grammars that are as free of redundancy as possible to capture the grammatical sentences of a language with their correct meaning versus grammars that might be open to a connection with usage-based approaches\is{usage-based grammar} to language modelling.
\citet[\page 292]{Bender2000a} sketches a version of HPSG with frequencies/probabilities attached to lexical and phrasal types.%
\footnote{An as-yet unexplored solution to the problem of free occurrence of idiomatic words within an experience-based version of HPSG could be to assign the type \type{idiomatic-word} an extremely low probability of occurring. This might have the effect that such a word can only be used if it is explicitly required in a construction. However, note that neither defaults\is{default} nor probabilities are well-defined part of the formal foundations of theoretical work on HPSG; see \crossrefchaptert{formal-background}.}
\is{constructional HPSG|)}

\subsection{Mixed lexical and phrasal approaches}
\label{Sec-Mixed}

While \citet{Riehemann2001a} proposes a parallel treatment of decomposable and non"=decomposable idioms -- and of flexible and semi-fixed idioms -- the division between fixed and non"=fixed expressions is at the core of another approach, the \emph{two-dimensional theory of idioms}\is{two-dimensional theory of idioms}. This approach was first outlined in \citet{Sailer2000a} and referred to under this label in \citet{Richter:Sailer:09,Richter:Sailer:14}. It is intended to combine constructional and collocational approaches to grammar.

The basic intuition behind this approach is that signs have internal and external properties. 
All properties that are part of the feature structure of a sign are called \emph{internal}. 
Properties that relate to larger feature structures containing this sign are called its \emph{external} properties. 
The approach assumes that there is a notion of \emph{regularity} and that anything diverging from it is \emph{idiosyncratic} -- or idiomatic, in the terminology of this chapter. 

This approach is another attempt to reify the GPSG\is{Generalized Phrase Structure Grammar} analysis within HPSG.
\citet{Sailer2000a} follows the distinction of \citet{NSW94a} into non"=decomposable and non"=flexible idioms on the one hand and decomposable and flexible idioms on the other. The first group is considered internally irregular and receives a constructional analysis in terms of a \emph{phrasal lexical entry}\is{phrasal lexical entry}. The second group is considered to consist of independent, smaller lexical units that show an external irregularity in being constrained to co-occur within a larger structure. 
Idioms of the second group receive a collocational analysis. The two types of irregularity are connected by the  \emph{Predictability Hypothesis}, given in \refer{PredHypo}.

\ea Predictability Hypothesis\is{Predictability Hypothesis} \citep[\page 366]{Sailer2000a}:\label{PredHypo}\\
For every sign whose internal properties are fully predictable, the distributional
behavior of this sign is fully predictable as well.
\z 


In the most recent version of this approach, \citet{Richter:Sailer:09,Richter:Sailer:14}, there is a feature \attrib{coll} defined on all signs. 
The value of this feature specifies the type of internal irregularity. 
The authors assume a cross-classification of regularity and irregularity with respect to syntax, semantics, and phonology -- ignoring pragmatic and statistical (ir)regularity in their paper. 
Every basic lexical entry is defined as completely irregular, as its properties are not predictable. 
Fully regular phrases such as \emph{read a book} have a trivial value of \attrib{coll}. 
A syntactically internally regular but fixed idiom such as \emph{kick the bucket} is classified as having only semantic irregularity, whereas a syntactically irregular expression such as \emph{trip the light fantastic} is of an irregularity type that is a subsort of syntactic and semantic irregularity, but not of phonological irregularity.
Following the terminology of \citet{FKoC88a}, this type is called \type{extra-grammatical-idiom}.
%
The phrasal lexical entry for \emph{trip the light fantastic} is sketched in  \refer{rs-trip}, adjusted to the feature geometry of \cite{Sag97a}.

\inlinetodostefan{Sag97 does not have \val}
\vbox{
\ea Phrasal lexical entry for the idioms \emph{trip the light fantastic}:\label{rs-trip}\\
\ms[headed-phrase]{
phon & \ibox{1} $\oplus$ \phonliste{the, light, fantastic}\\
synsem &  \ms{head & \ibox{4}\\
                        listeme & trip-the-light-fantastic\\
                        val & \ibox{5} \ms{subj & \liste{\ibox{2} \ms{\ldots index & \ibox{3}}}}\\
                     cont & \ms[trip-light-fant]{dancer & \ibox{3}}
                     }\\
head-dtr &
        \ms{phon & \ibox{1}\\
            synsem & \ms{head & \ibox{4} verb\\
                             listeme & trip\\
                            val & \ibox{5} \ms{subj & \liste{\ibox{2}}\\
                            comps & \eliste}
                            }
            }\\
coll & extra-grammatical-idiom
%\ms[extra-grammatical-idiom]{req & \eliste}
}
\z 

}

\kommentar{
\ea Phrasal lexical entry for the idioms \emph{trip the light fantastic}:\label{rs-trip}\\
\ms[phrase]{
phon & \ibox{1} $\oplus$ \phonliste{the, light, fantastic}\\
syns & \ms{loc  \ms{cat  \ms{head & \ibox{4}\\
                        listeme & trip-the-light-fantastic\\
                        val & \ibox{5} \ms{subj & \liste{\ibox{2} \ms{\ldots index & \ibox{3}}}}}\\
                     cont \ms[trip-light-fant]{danser & \ibox{3}}
                     }}\\
dtrs & \ms[headed-structure]{head-dtr &
        \ms{phon & \ibox{1}\\
            \ldots cat & \ms{head & \ibox{4} verb\\
                             listeme & trip\\
                            val & \ibox{5} \ms{subj & \liste{\ibox{2}}\\
                            comps & \eliste}
                            }}}\\
coll & extra-grammatical-idiom
%\ms[extra-grammatical-idiom]{req & \eliste}
}
\z 

}

In \refer{rs-trip}, the constituent structure of the phrase is not specified, but the phonology is fixed, with the exception of the head daughter's phonological contribution. This accounts for the syntactic irregularity of the idiom. The semantics of the idiom is not related to the semantic contributions of its components, which accounts for the semantic idiomaticity.

\citet{Soehn2006a} applies this theory to German\il{German}. He solves the problem of the relatively large degree of flexibility of non"=decomposable idioms in German
by using underspecified descriptions of the constituent structure dominated by the idiomatic phrase.

For decomposable idioms, the two-dimensional theory assumes a collocational component. This component is integrated into the value of an attribute \attrib{req}, which is only defined on \type{coll} objects of one of the irregularity types. 
This encodes the Predictability Hypothesis.
%
The most comprehensive version of this collocational theory is given in \citet{Soehn:09}, summarizing and extending ideas from \citet{Soehn2006a} and \citet{richter-soehn:2006}. 
Soehn assumes that collocational requirements can be of various types: 
a lexical item can be constrained to co-occur with particular \emph{licensers} (or collocates). These can be other lexemes, semantic operators, or phonological units. In addition, the domain within which this licensing has to be satisfied is specified in terms of syntactic barriers, i.e., syntactic nodes dominating the externally irregular item.

To give an example, the idiom \emph{spill the beans} would be analyzed as consisting of two  idiomatic words \emph{spill} and \emph{beans} with special \attrib{listeme} values \type{spill-i} and \type{beans-i}. The idiomatic verb \emph{spill} imposes a lexeme selection on its complement. The idiomatic noun \emph{beans} has a non"=empty \attrib{req} value, which specifies that it must be selected by a word with \attrib{listeme} value \type{spill-i} within the smallest complete clause dominating it.

%\citet{Richter:Sailer:09,Richter:Sailer:14} look at idioms with 

%\bigskip
The two-dimensional approach suffers from a number of weaknesses. 
First, it presupposes a notion of regularity. This assumption is not shared by all linguists.
Second, the criteria for whether an expression should be treated constructionally or collocationally are not always clear. 
Idioms with irregular syntactic structure need to be analyzed constructionally, but this is less clear for non"=decomposable idioms with regular syntactic structure such as \emph{kick the bucket}.


%\begin{itemize}
%\item \citet{Sailer2000a}, \citet{Soehn2006a}, \citet{Richter:Sailer:09}
%\end{itemize}

\subsection{Recent lexical approaches}
\label{Sec-RecentLexical}

\citet{KSF2015a} marks an important re-orientation in the analysis of idioms: the lexical analysis is extended to all syntactically regular idioms, i.e., to both decomposable (\emph{spill the beans}) and non"=decomposable idioms (\emph{kick the bucket}).% 
\footnote{This idea has been previously expressed within a  Minimalist perspective in \citet{Everaert:10}. 
}
%\citet{KSF2015a} use Sign-based Construction Grammar\is{Sign-based Construction Grammar}
%
\citet{KSF2015a} achieve a lexical analysis of non"=decomposable idioms by two means: (i), an extension of the HPSG selection mechanism, and (ii), the assumption of semantically empty idiomatic words. 

As in previous accounts, the relation among idiom parts is established through lexeme-specific selection, using a feature \attrib{lid} (for \emph{lexical identifier}). 
The authors assume that there is a difference between idiomatic and non"=idiomatic \attrib{lid} values. 
Only heads that are part of idioms themselves can select for idiomatic words. 
%Quote: Ordinary, non"=idiom predicators are lexically specified as requiring all members of their VAL list to be nonidiomatic.

For the idiom \emph{kick the bucket}, \citet{KSF2015a} assume that all meaning is carried by the lexical head, an idiomatic version of \emph{kick}, whereas the other two words, \emph{the} and \emph{bucket} are meaningless. 
This meaninglessness allows Kay et al.\@ to block the idiom from occurring in constructions which require meaningful constituents, such as questions, \emph{it}-clefts, middle voice, and others. 
To exclude passivization, the authors assume that the English passive cannot apply to verbs selecting a semantically empty direct object.

The approach in \citet{KSF2015a} is a recent attempt to maintain the SLH as much as possible. 
Since the SLH has been a major conceptual motivation for Sign-based Construction Grammar\is{Sign-based Construction Grammar}, \citeauthor{KSF2015a}'s paper is an important contribution showing the empirical robustness of this assumption.

%\medskip%
\citet{Bargmann:Sailer:18} propose a similar lexical approach to non"=de\-com\-pos\-able idioms. 
They take as their starting point the syntactic flexibility of semantically non"=decomposable idioms in English and, in particular, in German.
There are two main differences between \citeauthor{KSF2015a}'s paper and \citeauthor{Bargmann:Sailer:18}'s: (i), \citeauthor{Bargmann:Sailer:18} assume a collocational rather than a purely selectional mechanism to capture lexeme restrictions of idioms, and (ii), they propose a redundant semantics rather than an empty semantics for idiom parts in non"=decomposable idioms. In other words, \citet{Bargmann:Sailer:18} propose that both \emph{kick} and \emph{bucket} contribute the semantics of the idiom \emph{kick the bucket}. 
\citeauthor{Bargmann:Sailer:18} argue that the semantic contributions of parts of non"=decomposable, syntactically regular idioms are the same across languages, whereas the differences in syntactic flexibility are related to the different syntactic, semantic, and pragmatic constraints imposed on various constructions. 
To give just one example, while there are barely any restrictions on passive subjects in German, there are strong discourse-structural constraints on passive subjects in English.

Both \citet{KSF2015a} and \citet{Bargmann:Sailer:18} attempt to derive the (partial) syntactic inflexibility of non"=decomposable idioms from independent properties of the relevant constructions. 
As such, they subscribe to the programmatic statement of \citet{NSW94a} quoted on page \pageref{NSW-quote}.  
In this respect, the extension of the lexical approach from decomposable idioms to all syntactically regular expressions has been a clear step forward. 

%\bigskip%
\citet{Findlay:17} provides a recent discussion and criticism of lexical approaches to idioms in general, which applies in particular to non"=decomposable expressions. 
His reservations comprise the following points. 
First, there is a massive proliferation of lexical entries for otherwise homophonous words. 
%Is is unclear, for example, if a separate definite article is needed for each idiom which contains one, i.e., it might turn out that we need different lexical entries for the word \emph{the} in \emph{kick the bucket}, \emph{shoot the breeze}, and \emph{shit hits the fan}.%
%
Second, the lexical analysis does not represent idioms as units, which might make it difficult to connect their theoretical treatment with processing evidence. 
Findlay refers to psycholinguistic studies, such as \citet{Swinney:Cutler:79}, that point to a faster processing of idioms than of free combinations.
While the relevance of processing arguments for an HPSG analysis are not clear, I share the basic intuition that idioms, decomposable or not, are a unit and that this should be part of their linguistic representation.




\kommentar{
\begin{itemize}
\item \citet{KSF2015a}, \citet{Sag2012a}
\item \citet{Bargmann:Sailer:18}
\end{itemize}
}

\kommentar{
\section{Recent developments in other frameworks}
\label{Sec-RecentOtherFrameworks}

The interest in idioms varies 

\begin{itemize}
\item TAG: \citet{Lichte:Kallmeyer:16}
\item Minimalism: \citet{vCraenenbroeck:al:16draft}, \citet{Everaert:10}
\item LFG: \citet{Findlay:17}
\end{itemize}
}

\section{Where to go from here?}
\label{Sec-WhereToGo}

The final section of this article contains short overviews of research that has been done in areas of phraseology that are outside the main thread of this chapter. I will also identify desiderata. 


%In this final section, I would like to point to directions for future research. 
\subsection{Neglected phenomena}
\label{Sec-Neglected}

Not all types of idioms or idiomaticity mentioned in Section~\ref{Sec-EmpiricalDomain} have received an adequate treatment in the (HPSG) literature.
I will briefly look at three empirical areas that deserve more attention: neglected types of idiom variation, phraseological patterns, and the literal and non"=literal meaning components of idioms.

%\bigskip%Variation
Most studies on idiom variation have looked at verb- and sentence-related syntactic constructions, such as passive and topicalization. 
However, not much attention has been paid to lexical variation in idioms. This variation is illustrated by the following examples from \citet[\page 184, 191]{Richards:01}. 

\eal  \label{creeps}
\ex The Count gives everyone the creeps.
\ex You get the creeps (just looking at him).
\ex I have the creeps.
\zl 

In~\refer{creeps}, the alternation of the verb seems to be very systematic -- and has been used by \citet{Richards:01} to motivate a lexical decomposition of these verbs.
A similar argument has been made in \citet{Mateu:Espinal:07} for similar idioms in Catalan. 
We are lacking systematic, larger empirical studies of this type of substitution, and it would be important to see how it can be modelled in HPSG. 
One option would be to capture the \emph{give}--\emph{get}--\emph{have} alternation(s) with lexical rules. Such lexical rules would be different from the standard cases, however, as they would change the lexeme itself rather than just alternating its morpho-syntactic properties or its semantic contribution.

In the case mentioned in footnote \ref{fn-semmeln}, the alternation consists of substituting a word with a (near) synonym and keeping the meaning of the idiom intact. Again, HPSG seems to have all the required tools to model this phemonenon -- for example, by means of hierarchies of \textsc{lexical-id} values. 
However, the extent of this phenomenon across the set of idioms is not known empirically. 

%\medskip%
Concerning syntactic variation, the nominal domain has not yet received the attention it might deserve. 
There is a well-known variation with respect to the marking of possession within idioms. 
This has been documented for English\il{English} in \citet{Ho:15}, for Modern Hebrew\il{Hebrew} in \citet{Almog:12}, and for Modern Greek and German\il{German} in \citet{Markantonatou:Sailer:16}. 
In German, we find a relatively free alternation between a plain definite and a possessive; see \refer{ex-verstand}. This is, however, not possible with all idioms; see  \refer{ex-frieden}.

\eal \label{ex-verstand-herz}
\ex 
\gll Alex hat den / seinen Verstand verloren.\\
Alex has the {} his mind lost\\
\glt `Alex lost his mind.'\label{ex-verstand}
\ex 
\gll Alex hat *den / ihren Frieden mit der Situation gemacht.\\
     Alex has \hspaceThis{*}the {} her peace with the situation made\\
\glt `Alex made her peace with the situation.'\label{ex-frieden}
\zl 


We can also find a free dative in some cases, expressing the possessor. 
In \refer{ex-herz}, a dative possessor may co-occur with a plain definite or a coreferential possessive determiner; in \refer{ex-augen}, only the definite article but not the possessive determiner is possible.  


\eal \label{ex-herz-augen}
\ex 
\gll Alex hat mir das / mein Herz gebrochen.\\
Alex has me.\textsc{dat} the {} my heart broken\\
\glt `Alex broke my heart.'\label{ex-herz}
\ex 
\gll Alex sollte mir lieber aus den / *meinen Augen gehen.\\
Alex should me.\textsc{dat} rather {out of} the {} \hspaceThis{*}my eyes go\\
\glt `Alex should rather disappear from my sight.'\label{ex-augen}
\zl 

While they do not offer a formal encoding, \citet{Markantonatou:Sailer:16} observe that a particular encoding of possession in idioms is only possible if it would also be possible in a free combination. However, an idiom may be idiosyncratically restricted to a subset of the realizations that would be possible in a corresponding free combination. A formalization in HPSG might consist of a treatment of possessively used definite determiners, combined with an analysis of free datives as an extension of a verb's argument structure.

 

%\bigskip%Patterns
Related to the question of lexical variation are \is{phraseological patterns}\emph{phraseological patterns}, i.e., very schematic idioms 
in which the lexical material is largely free. Some examples 
of phraseological patterns are
  the \emph{Incredulity Response Construction} as in \emph{What, me worry?} \citep{Akmajian:84,Lambrecht:90}, 
or the \emph{What's X doing Y?} construction \citep{KF99a}.
Such patterns are of theoretical importance as they typically involve a non"=canonical syntactic pattern. 
The different locality and non"=locality hypotheses introduced above make different predictions. 
\citet{FKoC88a} have presented such constructions as a motivation for the non"=locality of constructions, i.e., as support of a SNH. However, \citet{KF99a} show that a lexical analysis might be possible for some cases at least, which they illustrate with the \emph{What's X doing Y?} construction. 

\citet{Borsley2004a} looks at another phraseological pattern, the \emph{the X-er the Y-er} construction, or \emph{comparative correlative construction}.
Borsley analyzes this construction by means of two special (local) phrase structure types: one for the comparative \emph{the}-clauses, and one for the overall construction. He shows that (i), the idiosyncrasy of the construction concerns two levels of embedding and is, therefore, non"=local; however, (ii),
a local analysis is still possible. This approach raises the question of whether the WNH is empirically vacuous since we can always encode a non"=local construction in terms of a series of idiosyncratic local constructions. 
Clearly, work on more phraseological patterns is needed to assess the various analytical options and their consequences for the architecture of grammar.


%\bigskip%Literal
A major charge for the conceptual and semantic analysis of idioms is the interaction between the literal and the idiomatic meaning. 
I presented the basic empirical facts in Section~\ref{Sec-EmpiricalDomain}. 
All HPSG approaches to idioms so far basically ignore the literal meaning.
This position might be justified, as  an HPSG grammar should just model the structure and meaning of an utterance and need not worry about the meta-linguistic relations among different lexical items or among different readings of the same (or a homophonous) expression.
Nonetheless, this issue touches on an important conceptual point. 
Addressing it might immediately provide possibilities to connect HPSG research to other disciplines and/""or frameworks like cognitive linguistics, such as in \citet{Dobrovolskij:Piirainen:05}, and psycholinguistics.



\kommentar{
\begin{itemize}
\item Variation in nominal parts/possessive idioms \citep{Bond:al:15}
\item Literal and non"=literal meaning components (Bargmann, Gehrke \& Richter on \emph{conjunction modification}, \citet{Hoeksema:Sailer:12}, \ldots).
\item Phraseological patterns
\end{itemize}
}

\subsection{Challenges from other languages}
\label{Sec-OtherLanguages}

The majority of work on idioms in HPSG has been done on English and German. 
%This led to a limitation of the possible phenomena that can be studied on idioms. 
As discussed in Section~\ref{Sec-RecentLexical}, the recent trend in HPSG idiom research necessitates a detailed study of individual syntactic structures. 
Consequently, the restriction to two closely related languages limits the possible phenomena that can be studied concerning idioms. 
It would be essential to expand the empirical coverage of idiom analyses in HPSG to as many different languages as possible. 
The larger degree of syntactic flexibility of French, German, and Dutch idioms \citep{Ruwet:91,NSW94a,Schenk:95} has led to important refinements of the analysis in \citet{NSW94a} and, ultimately, to the lexical analyses of all syntactically regular idioms. 

Similarly, the above-mentioned data on possessive alternations only become prominent when languages beyond English are taken into account. Modern Greek\il{Greek}, German\il{German}, and many others
%mentioned above \refer{ex-herz-augen} 
show the type of external possessor classified as a European areal phenomenon in \citet{Haspelmath:99}. 
It would be important to look at idioms in languages with other types of external possessors.


In a recent paper, \citet{Sheinfux:al:19} provide data from Modern Hebrew\il{Hebrew} that show that opacity and figurativity of an idiom are decisive for its syntactic flexibility, rather than decomposability.
This result stresses the importance of the literal reading for an adequate account of the syntactic behavior of idioms. 
%
It shows that the inclusion of other languages can cause a shift of focus to other types of idioms or other types of idiomaticity. 

To add just one more example, HPSG(-related) work on Persian such as \citet{MuellerPersian} and \citet{Samvelian:Faghiri:16} establishes a clear connection between complex predicates and idioms. 
Their insights might also lead to a reconsideration of the similarities between light verbs and idioms, as already set out in \citet{KE94a}.



\kommentar{
The majority of work on idioms in HPSG has been done on English and German, with a new line of research on Modern Hebrew\il{Hebrew} \citep{Sheinfux:al:15,HMW2016a-u}. 
Work on complex predicates in Persian\il{Persian} \citep{MuellerPersian,Samvelian:Faghiri:16}, establishes a clear connection between the idioms and complex predicates.
}


As far as I can see, the following empirical phenomena have not been addressed in HPSG approaches to idioms, as they do not occur in the main object languages for which we have idiom analyses, i.e., English and German. They are, however, common in other languages: the occurrence of clitics\is{clitic} in idioms (found in Romance\il{Romance} and Greek\il{Greek}); aspectual\is{aspect} alternations in verbs (Slavic\il{Slavic} and Greek); argument alternations other than passive and dative alternation (such as anti-passive\is{anti-passive}, causative\is{causative}, inchoative\is{inchoative}, etc.\
(in part found in Hebrew and addressed in \citealt{Sheinfux:al:19}); and 
displacement of idiom parts into special syntactic positions (focus position in Hungarian\is{Hungarian}). 

Finally, so far, idioms have usually been considered as either offering irregular structures or as being more restricted in their structures than free combinations. In some languages, however, we find archaic syntactic structures and function words in idioms that do not easily fit these two analytic options. To name just a few, \citet{Lodrup:09} argues that Norwegian\il{Norwegian} used to have an external possessor construction similar to that of other European languages, which is only conserved in some idioms. Similarly, Dutch\il{Dutch} has a number of archaic case inflections in multiword expressions \citep[\page 129]{Kuiper:18}, and there are archaic forms in Modern Greek\il{Greek} multiword expressions. It is far from clear what the best way would be to integrate such cases into an HPSG grammar. 


\kommentar{
In other formal frameworks, other languages have received more systematic attention. 

To mention just a few, there are many analyses within versions of minimalism.  
\citet{Mateu:Espinal:07} and other work of these authors on Catalan. \citet{vCraenenbroeck:al:16draft} discuss idioms from Dutch\il{Dutch} and Dutch dialects in a minimalist framework. 

For many other languages, there are computational treatments of various types of phraseological units, which have, however, not necessarily had a
}

\kommentar{
\begin{itemize}
\item Work by Espinal on Spanish and Catalan in comparison to English
\item Hebrew \citep{HMW2016a-u}
\item German (if not already discussed enough in earlier sections)
\end{itemize}
}

\kommentar{
\subsection{Methodological considerations}
\label{Sec-Methods}

Brief overview of some corpuslinguistic and psycholinguistic results and the question what they could contribute to the question of modeling idioms in HPSG.
}

\section{Conclusion}
\label{Sec-Summary}

Idioms are among the topics in linguistics for which HPSG-related publications have had a clear impact on the field and have been widely quoted across frameworks.
This handbook article aimed at providing an overview over the development of idiom analyses in HPSG. 
There seems to be a development towards ever more lexical analyses, starting from the holistic approach for all idioms in Chomsky's work, to a lexical account for all syntactically regular expressions. 
Notwithstanding the advantages of the lexical analyses, 
I consider it a basic problem of such approaches that the unit status of idioms is lost. Consequently, I think that the right balance between phrasal and lexical aspects in the analysis of idioms has not yet been fully achieved.

The sign-based character of HPSG seems to be particularly suited for a theory of idioms as it allows one to take into consideration syntactic, semantic, and pragmatic aspects and to use them to constrain the occurrence of idioms appropriately.



\kommentar{
\section{Where we came from} 
Phasellus maximus erat ligula, accumsan rutrum augue facilisis in. Proin sit amet pharetra nunc, sed maximus erat. Duis egestas mi eget purus venenatis vulputate vel quis nunc. Nullam volutpat facilisis tortor, vitae semper ligula dapibus sit amet. Suspendisse fringilla, quam sed laoreet maximus, ex ex placerat ipsum, porta ultrices mi risus et lectus. Maecenas vitae mauris condimentum justo fringilla sollicitudin. Fusce nec interdum ante. Curabitur tempus dui et orci convallis molestie \citep{Chomsky:57}.

\begin{table}
\caption{Frequencies of word classes}
\label{tab:1:frequencies}
 \begin{tabular}{lllll} 
  \lsptoprule
            & nouns & verbs & adjectives & adverbs\\ 
  \midrule
  absolute  &   12 &    34  &    23     & 13\\
  relative  &   3.1 &   8.9 &    5.7    & 3.2\\
  \lspbottomrule
 \end{tabular}
\end{table}

Sed nisi urna, dignissim sit amet posuere ut, luctus ac lectus. Fusce vel ornare nibh. Nullam non sapien in tortor hendrerit suscipit. Etiam sollicitudin nibh ligula. Praesent dictum gravida est eget maximus. Integer in felis id diam sodales accumsan at at turpis. Maecenas dignissim purus non libero scelerisque porttitor. Integer porttitor mauris ac nisi iaculis molestie. Sed nec imperdiet orci. Suspendisse sed fringilla elit, non varius elit. Sed varius nisi magna, at efficitur orci consectetur a. Cras consequat mi dui, et cursus lacus vehicula vitae. Pellentesque sit amet justo sed lectus luctus vehicula. Suspendisse placerat augue eget felis sagittis placerat. 

\ea
\gll cogito                           ergo      sum\\  
     think.\textsc{1sg}.\textsc{pres} therefore \textsc{cop}.\textsc{1sg}.\textsc{pres}\\ 
\glt `I think therefore I am.'
\z

Sed cursus eros condimentum mi consectetur, ac consectetur sapien pulvinar. Sed consequat, magna eu scelerisque laoreet, ante erat tristique justo, nec cursus eros diam eu nisl. Vestibulum non arcu tellus. Nunc dignissim tristique massa ut gravida. Nullam auctor orci gravida tellus egestas, vitae pharetra nisl porttitor. Pellentesque turpis nulla, venenatis id porttitor non, volutpat ut leo. Etiam hendrerit scelerisque luctus. Nam sed egestas est. Suspendisse potenti. Nunc vestibulum nec odio non laoreet. Proin lacinia nulla lectus, eu vehicula erat vehicula sed. 

}

%\appendix
\section*{Appendix: List of used idioms}

Some idioms do not show semantic idiomaticity at all, such as collocations\is{collocation} (\bsp{brush one's teeth}) or support verb constructions (\bsp{take a shower}). 
Many body-part expressions such as \bsp{shake hands} or \bsp{shake one's head} constitute a more complex case. They describe a conventionalized activity and denote the social meaning of this activity \citep{Burger:76}.

\subsection*{English}

\begin{tabular}{@{}lL{3.8cm}L{3.8cm}@{}}
idiom & paraphrase & comment\\\hline
break the ice      & relieve tension in a strained situation  & non"=decomposable\\
brush one's teeth  & clean one's teeth with a tooth brush     & collocation, no idiomaticity\\
give s.o.\ the creeps & make s.o.\ feel uncomfortable               & systematic lexical variation\\
Good morning!      & (morning greeting)                       & formulaic expression\\
immaculate performance & perfect performance & statistical idiomaticity\\
in a trice & in a moment & bound word: \emph{trice}\\
kick the bucket & die & non"=decomposable\\
make headway & make progress & bound word: \emph{headway}\\
pull strings & exert influence/""use one's connections & flexible\\
saw logs & snore & {transparent, non"=decomposable, semi"=flexible}\\
shake hands & greet & body-part expression\\
shake one's head & decline/""negate & {body-part expression, possessive idiom}\\
shit hit the fan & there is trouble & {subject as idiom component, transparent/""figurative, non"=decomposable}\\
shoot the breeze & chat & non"=decomposable\\
spill the beans & reveal a secret & flexible\\
take a shower & clean oneself using a shower & collocation, light verb construction\\
take the bull by the horns & {approach a problem directly} & figurative expression\\
take umbrage & take offence & bound word: \emph{umbrage}\\
trip the light fantastic & dance & syntactically irregular\\
\end{tabular}


\subsection*{German}

\begin{sideways}
\begin{tabular}{@{}L{4cm}lL{4.0cm}L{4.0cm}}
idiom & gloss & translation & comment\\\hline
%
 {den/seinen Verstand verlieren}
 & {the/""one's mind lose}
 & {lose one's mind}
 & {alternation of possessor marking}
 \\
 %
 {jdm.\ das Herz brechen} & s.o.\ the heart break & break s.o.'s heart
 & {dative possessor and possessor alternation}\\
 %
 {jdm.\ aus den Augen gehen} & s.o.\ out of the eyes go
 & {disappear from s.o.'s sight} &
 {dative possessor, restricted possessor alternation}\\
 %
 {seinen Frieden machen mit}
 & {one's peace make with}
 & {make one's peace with}
 & {no possessor alternation possible}\\
 %
 {wie warme Semmeln/""Brötchen/""Schrippen weggehen}
 & {like warm rolls vanish} & sell like hotcakes & 
 {parts can be exchanged by synonyms}\\
\end{tabular}
\end{sideways}

\section*{Abbreviations}

\begin{tabular}{@{}ll}
GPSG & Generalized Phrase Structure Grammar \citep{GKPS85a}\\
%MRS & Minimal Recursion Semantics \citep{CFPS2005a}\\
%MWE & multiword expression\\
MGG & Mainstream Generative Grammar\\
SLH & Strong Locality Hypothesis, see page \pageref{slh}\\
SNH & Strong Non"=locality Hypothesis, see page \pageref{snh}\\
WLH & Weak Locality Hypothesis, see page \pageref{wlh}\\
WNH & Weak Non"=locality Hypothesis, see page \pageref{wnh}\\
\end{tabular}

\section*{Acknowledgements}

I have perceived Ivan A.\@ Sag\ia{Sag, Ivan A.} and his work with various colleagues as a major inspiration for a lot of my own work on idioms and multiword expressions. 
This is clearly reflected in the structure of this paper, too. 
I apologize for this bias, but I think it is legitimate within an HPSG handbook.
%
I am grateful to Stefan Müller and Tom Wasow for comments on the outline and the first version of this chapter. 
%
I would not have been able to format this chapter without the support of Sebastian Nordhoff.
%
I would like to thank Elizabeth Pankratz for comments and proofreading.

{\sloppy
\printbibliography[heading=subbibliography,notkeyword=this] }
\end{document}
