\documentclass[output=paper
	        ,collection
	        ,collectionchapter
 	        ,biblatex
                ,babelshorthands
                ,newtxmath
                ,draftmode
                ,colorlinks, citecolor=brown
]{langscibook}

\IfFileExists{../localcommands.tex}{%hack to check whether this is being compiled as part of a collection or standalone
  % add all extra packages you need to load to this file 

% the ISBN assigned to the digital edition
\usepackage[ISBN=9783961102556]{ean13isbn} 

\usepackage{graphicx}
\usepackage{tabularx}
\usepackage{amsmath} 

%\usepackage{tipa}      % Davis Koenig
\usepackage{xunicode} % Provide tipa macros (BC)

\usepackage{multicol}

% Berthold morphology
\usepackage{relsize}
%\usepackage{./styles/rtrees-bc} % forbidden forest 08.12.2019


\usepackage{langsci-optional} 
% used to be in this package
\providecommand{\citegen}{}
\renewcommand{\citegen}[2][]{\citeauthor{#2}'s (\citeyear*[#1]{#2})}
\providecommand{\lsptoprule}{}
\renewcommand{\lsptoprule}{\midrule\toprule}
\providecommand{\lspbottomrule}{}
\renewcommand{\lspbottomrule}{\bottomrule\midrule}
\providecommand{\largerpage}{}
\renewcommand{\largerpage}[1][1]{\enlargethispage{#1\baselineskip}}


\usepackage{langsci-lgr}

\newcommand{\MAS}{\textsc{m}\xspace} % \M is taken by somebody

%\usepackage{./styles/forest/forest}
\usepackage{langsci-forest-setup}

\usepackage{./styles/memoize/memoize} 
\memoizeset{
  memo filename prefix={chapters/hpsg-handbook.memo.dir/},
  register=\todo{O{}+m},
  prevent=\todo,
}

\usepackage{tikz-cd}

\usepackage{./styles/tikz-grid}
\usetikzlibrary{shadows}


% removed with texlive 2020 06.05.2020
% %\usepackage{pgfplots} % for data/theory figure in minimalism.tex
% % fix some issue with Mod https://tex.stackexchange.com/a/330076
% \makeatletter
% \let\pgfmathModX=\pgfmathMod@
% \usepackage{pgfplots}%
% \let\pgfmathMod@=\pgfmathModX
% \makeatother

\usepackage{subcaption}

% Stefan Müller's styles
\usepackage{./styles/merkmalstruktur,german,./styles/makros.2020,./styles/my-xspace,./styles/article-ex,
./styles/eng-date}

\selectlanguage{USenglish}

\usepackage{./styles/abbrev}


% Has to be loaded late since otherwise footnotes will not work

%%%%%%%%%%%%%%%%%%%%%%%%%%%%%%%%%%%%%%%%%%%%%%%%%%%%
%%%                                              %%%
%%%           Examples                           %%%
%%%                                              %%%
%%%%%%%%%%%%%%%%%%%%%%%%%%%%%%%%%%%%%%%%%%%%%%%%%%%%
% remove the percentage signs in the following lines
% if your book makes use of linguistic examples
\usepackage{langsci-gb4e} 


%% St. Mü.: 03.04.2020
%% these two versions of the command can be used for series of sets of examples:
%% \eal
%% \ex
%% \ex
%% \zlcont
%% \ealcont
%% \ex
%% \ex
%% \zl

\let\zlcont\z
\def\ealcont{\exnrfont\ex\begin{xlist}[iv.]\raggedright}

% original version of \z
%\def\z{\ifnum\@xnumdepth=1\end{exe}\else\end{xlist}\fi}
% \zcont just removes \end{exe}
%\def\zcont{\ifnum\@xnumdepth=1\else\end{xlist}\fi}
\def\zcont{}

% Crossing out text
% uncomment when needed
%\usepackage{ulem}

\usepackage{./styles/additional-langsci-index-shortcuts}

% this is the completely redone avm package
\usepackage{./styles/langsci-avm}
\avmsetup{columnsep=.3ex,style=narrow}

%\let\asort\type*


\usepackage{./styles/avm+}


\renewcommand{\tpv}[1]{{\avmjvalfont\itshape #1}}

% no small caps please
\renewcommand{\phonshape}[0]{\normalfont\itshape}

\regAvmFonts

\usepackage{theorem}

\newtheorem{mydefinition}{Def.}
\newtheorem{principle}{Principle}

{\theoremstyle{break}
%\newtheorem{schema}{Schema}
\newtheorem{mydefinition-break}[mydefinition]{Def.}
\newtheorem{principle-break}[principle]{Principle}
}


%% \newcommand{schema}[2]{
%% \begin{minipage}{\textwidth}
%% {\textbf{Schema~\theschema}}]\hspace{.5em}\textbf{(#1)}\\
%% #2
%% \end{minipage}}


% This avoids linebreaks in the Schema
\newcounter{schemacounter}
\makeatletter
\newenvironment{schema}[1][]
  {%
   \refstepcounter{schemacounter}%
   \par\bigskip\noindent
   \minipage{\linewidth}%
   \textbf{Schema~\theschemacounter\hspace{.5em} \ifx&#1&\else(#1)\fi}\par
  }{\endminipage\par\bigskip\@endparenv}%
\makeatother

%\usepackage{subfig}





% Davis Koenig Lexikon

\usepackage{tikz-qtree,tikz-qtree-compat} % Davis Koenig remove

\usepackage{shadow}



\usepackage[english]{isodate} % Andy Lücking
\usepackage[autostyle]{csquotes} % Andy
%\usepackage[autolanguage]{numprint}

%\defaultfontfeatures{
%    Path = /usr/local/texlive/2017/texmf-dist/fonts/opentype/public/fontawesome/ }

%% https://tex.stackexchange.com/a/316948/18561
%\defaultfontfeatures{Extension = .otf}% adds .otf to end of path when font loaded without ext parameter e.g. \newfontfamily{\FA}{FontAwesome} > \newfontfamily{\FA}{FontAwesome.otf}
%\usepackage{fontawesome} % Andy Lücking
\usepackage{pifont} % Andy Lücking -> hand

\usetikzlibrary{decorations.pathreplacing} % Andy Lücking
\usetikzlibrary{matrix} % Andy 
\usetikzlibrary{positioning} % Andy
\usepackage{tikz-3dplot} % Andy

% pragmatics
\usepackage{eqparbox} % Andy
\usepackage{enumitem} % Andy
\usepackage{longtable} % Andy
\usepackage{tabu} % Andy              needs to be loaded before hyperref as of texlive 2020

% tabu-fix
% to make "spread 0pt" work
% -----------------------------
\RequirePackage{etoolbox}
\makeatletter
\patchcmd
	\tabu@startpboxmeasure
	{\bgroup\begin{varwidth}}%
	{\bgroup
	 \iftabu@spread\color@begingroup\fi\begin{varwidth}}%
	{}{}
\def\@tabarray{\m@th\def\tabu@currentgrouptype
    {\currentgrouptype}\@ifnextchar[\@array{\@array[c]}}
%
%%% \pdfelapsedtime bug 2019-12-15
\patchcmd
	\tabu@message@etime
	{\the\pdfelapsedtime}%
	{\pdfelapsedtime}%
	{}{}
%
%
\makeatother
% -----------------------------


% Manfred's packages

%\usepackage{shadow}

\usepackage{tabularx}
\newcolumntype{L}[1]{>{\raggedright\arraybackslash}p{#1}} % linksbündig mit Breitenangabe


% Jong-Bok

%\usepackage{xytree}

\newcommand{\xytree}[2][dummy]{Let's do the tree!}

% seems evil, get rid of it
% defines \ex is incompatible with gb4e
%\usepackage{lingmacros}

% taken from lingmacros:
\makeatletter
% \evnup is used to line up the enumsentence number and an entry along
% the top.  It can take an argument to improve lining up.
\def\evnup{\@ifnextchar[{\@evnup}{\@evnup[0pt]}}

\def\@evnup[#1]#2{\setbox1=\hbox{#2}%
\dimen1=\ht1 \advance\dimen1 by -.5\baselineskip%
\advance\dimen1 by -#1%
\leavevmode\lower\dimen1\box1}
\makeatother


% YK -- CG chapter

%\usepackage{xspace}
\usepackage{bm}
\usepackage{ebproof}


% Antonio Branco, remove this
\usepackage{epsfig}

% now unicode
%\usepackage{alphabeta}





\usepackage{pst-node}


% fmr: additional packages
%\usepackage{amsthm}


% Ash and Steve: LFG
\usepackage{./styles/lfg/dalrymple}

\RequirePackage{graphics}
%\RequirePackage{./styles/lfg/trees}
%% \RequirePackage{avm}
%% \avmoptions{active}
%% \avmfont{\sc}
%% \avmvalfont{\sc}
\RequirePackage{./styles/lfg/lfgmacrosash}

\usepackage{./styles/lfg/glue}

%%%%%%%%%%%%%%%%%%%%%%%%%%%%%%
%% Markup
%%%%%%%%%%%%%%%%%%%%%%%%%%%%%%
\usepackage[normalem]{ulem} % For thinks like strikethrough, using \sout

% \newcommand{\high}[1]{\textbf{#1}} % highlighted text
\newcommand{\high}[1]{\textit{#1}} % highlighted text
%\newcommand{\term}[1]{\textit{#1}\/} % technical term
\newcommand{\qterm}[1]{`{#1}'} % technical term, quotes
%\newcommand{\trns}[1]{\strut `#1'} % translation in glossed example
\newcommand{\trnss}[1]{\strut \phantom{\sqz{}} `#1'} % translation in ungrammatical glossed example
\newcommand{\ttrns}[1]{(`#1')} % an in-text translation of a word
%\newcommand{\feat}[1]{\mbox{\textsc{\MakeLowercase{#1}}}}     % feature name
%\newcommand{\val}[1]{\mbox{\textsc{\MakeLowercase{#1}}}}    % f-structure value
\newcommand{\featt}[1]{\mbox{\textsc{\MakeLowercase{#1}}}}     % feature name
\newcommand{\vall}[1]{\mbox{\textsc{\textup{\MakeLowercase{#1}}}}}    % f-structure value
\newcommand{\mg}[1]{\mbox{\textsc{\MakeLowercase{#1}}}}    % morphological gloss
%\newcommand{\word}[1]{\textit{#1}}       % mention of word
\providecommand{\kstar}[1]{{#1}\ensuremath{^*}}
\providecommand{\kplus}[1]{{#1}\ensuremath{^+}}
\newcommand{\template}[1]{@\textsc{\MakeLowercase{#1}}}
\newcommand{\templaten}[1]{\textsc{\MakeLowercase{#1}}}
\newcommand{\templatenn}[1]{\MakeUppercase{#1}}
\newcommand{\tempeq}{\ensuremath{=}}
\newcommand{\predval}[1]{\ensuremath{\langle}\textsc{#1}\ensuremath{\rangle}}
\newcommand{\predvall}[1]{{\rm `#1'}}
\newcommand{\lfgfst}[1]{\ensuremath{#1\,}}
\newcommand{\scare}[1]{`#1'} % scare quotes
\newcommand{\bracket}[1]{\ensuremath{\left\langle\mathit{#1}\right\rangle}}
\newcommand{\sectionw}[1][]{Section#1} % section word: for cap/non-cap
\newcommand{\tablew}[1][]{Table#1} % table word: for cap/non-cap
\newcommand{\lfgglue}{LFG+Glue}
\newcommand{\hpsgglue}{HPSG+Glue}
\newcommand{\gs}{GS}
%\newcommand{\func}[1]{\ensuremath{\mathbf{#1}}}
\newcommand{\func}[1]{\textbf{#1}}
\renewcommand{\glue}{Glue}
%\newcommand{\exr}[1]{(\ref{ex:#1}}
\newcommand{\exra}[1]{(\ref{ex:#1})}


%%%%%%%%%%%%%%%%%%%%%%%%%%%%%%
% Notation
%\newcommand{\xbar}[1]{$_{\mbox{\textsc{#1}$^{\raisebox{1ex}{}}$}}$}
\newcommand{\xprime}[2][]{\textup{\mbox{{#2}\ensuremath{^\prime_{\hspace*{-.0em}\mbox{\footnotesize\ensuremath{\mathit{#1}}}}}}}}
\providecommand{\xzero}[2][]{#2\ensuremath{^0_{\mbox{\footnotesize\ensuremath{\mathit{#1}}}}}}



\let\leftangle\langle
\let\rightangle\rangle

%\newcommand{\pslabel}[1]{}



  %add all your local new commands to this file


% Don't do this at home. I do not like the smaller font for captions.
% I just removed loading the caption packege in langscibook.cls
%% \captionsetup{%
%% font={%
%% stretch=1%.8%
%% ,normalsize%,small%
%% },%
%% width=.8\textwidth
%% }

\makeatletter
\def\blx@maxline{77}
\makeatother


\newcommand{\page}{}

\newcommand{\todostefan}[1]{\todo[color=orange!80]{\footnotesize #1}\xspace}
\newcommand{\todosatz}[1]{\todo[color=red!40]{\footnotesize #1}\xspace}

\newcommand{\inlinetodostefan}[1]{\todo[color=green!40,inline]{\footnotesize #1}\xspace}

\newcommand{\addpages}{\todostefan{add pages}}
\newcommand{\addglosses}{\todostefan{add glosses}}


\newcommand{\spacebr}{\hspaceThis{[}}

\newcommand{\danish}{\jambox{(\ili{Danish})}}
\newcommand{\english}{\jambox{(\ili{English})}}
\newcommand{\german}{\jambox{(\ili{German})}}
\newcommand{\yiddish}{\jambox{(\ili{Yiddish})}}
\newcommand{\welsh}{\jambox{(\ili{Welsh})}}

% Cite and cross-reference other chapters
\newcommand{\crossrefchaptert}[2][]{\citet*[#1]{chapters/#2}, Chapter~\ref{chap-#2} of this volume} 
\newcommand{\crossrefchapterp}[2][]{(\citealp*[#1]{chapters/#2}, Chapter~\ref{chap-#2} of this volume)}
\newcommand{\crossrefchapteralt}[2][]{\citealt*[#1]{chapters/#2}, Chapter~\ref{chap-#2} of this volume}
\newcommand{\crossrefchapteralp}[2][]{\citealp*[#1]{chapters/#2}, Chapter~\ref{chap-#2} of this volume}
% example of optional argument:
% \crossrefchapterp[for something, see:]{name}
% gives: (for something, see: Author 2018, Chapter~X of this volume)

\let\crossrefchapterw\crossrefchaptert



% Davis Koenig

\let\ig=\textsc
\let\tc=\textcolor

% evolution, Flickinger, Pollard, Wasow

\let\citeNP\citet

% Adam P

%\newcommand{\toappear}{Forthcoming}
\newcommand{\pg}[1]{p.\,#1}
\renewcommand{\implies}{\rightarrow}

\newcommand*{\rref}[1]{(\ref{#1})}
\newcommand*{\aref}[1]{(\ref{#1}a)}
\newcommand*{\bref}[1]{(\ref{#1}b)}
\newcommand*{\cref}[1]{(\ref{#1}c)}

\newcommand{\msadam}{.}
\newcommand{\morsyn}[1]{\textsc{#1}}

\newcommand{\nom}{\morsyn{nom}}
\newcommand{\acc}{\morsyn{acc}}
\newcommand{\dat}{\morsyn{dat}}
\newcommand{\gen}{\morsyn{gen}}
\newcommand{\ins}{\morsyn{ins}}
%\newcommand{\aploc}{\morsyn{loc}}
\newcommand{\voc}{\morsyn{voc}}
\newcommand{\ill}{\morsyn{ill}}
\renewcommand{\inf}{\morsyn{inf}}
\newcommand{\passprc}{\morsyn{passp}}

%\newcommand{\Nom}{\msadam\nom}
%\newcommand{\Acc}{\msadam\acc}
%\newcommand{\Dat}{\msadam\dat}
%\newcommand{\Gen}{\msadam\gen}
\newcommand{\Ins}{\msadam\ins}
\newcommand{\Loc}{\msadam\loc}
\newcommand{\Voc}{\msadam\voc}
\newcommand{\Ill}{\msadam\ill}
\newcommand{\PassP}{\msadam\passprc}

\newcommand{\Aux}{\textsc{aux}}

\newcommand{\princ}[1]{\textnormal{\textsc{#1}}} % for constraint names
\newcommand{\notion}[1]{\emph{#1}}
\renewcommand{\path}[1]{\textnormal{\textsc{#1}}}
\newcommand{\ftype}[1]{\textit{#1}}
\newcommand{\fftype}[1]{{\scriptsize\textit{#1}}}
\newcommand{\la}{$\langle$}
\newcommand{\ra}{$\rangle$}
%\newcommand{\argst}{\path{arg-st}}
\newcommand{\phtm}[1]{\setbox0=\hbox{#1}\hspace{\wd0}}
\newcommand{\prep}[1]{\setbox0=\hbox{#1}\hspace{-1\wd0}#1}

%%%%%%%%%%%%%%%%%%%%%%%%%%%%%%%%%%%%%%%%%%%%%%%%%%%%%%%%%%%%%%%%%%%%%%%%%%%

% FROM FS.STY:

%%%
%%% Feature structures
%%%

% \fs         To print a feature structure by itself, type for example
%             \fs{case:nom \\ person:P}
%             or (better, for true italics),
%             \fs{\it case:nom \\ \it person:P}
%
% \lfs        To print the same feature structure with the category
%             label N at the top, type:
%             \lfs{N}{\it case:nom \\ \it person:P}

%    Modified 1990 Dec 5 so that features are left aligned.
\newcommand{\fs}[1]%
{\mbox{\small%
$
\!
\left[
  \!\!
  \begin{tabular}{l}
    #1
  \end{tabular}
  \!\!
\right]
\!
$}}

%     Modified 1990 Dec 5 so that features are left aligned.
%\newcommand{\lfs}[2]
%   {
%     \mbox{$
%           \!\!
%           \begin{tabular}{c}
%           \it #1
%           \\
%           \mbox{\small%
%                 $
%                 \left[
%                 \!\!
%                 \it
%                 \begin{tabular}{l}
%                 #2
%                 \end{tabular}
%                 \!\!
%                 \right]
%                 $}
%           \end{tabular}
%           \!\!
%           $}
%   }

\newcommand{\ft}[2]{\path{#1}\hspace{1ex}\ftype{#2}}
\newcommand{\fsl}[2]{\fs{{\fftype{#1}} \\ #2}}

\newcommand{\fslt}[2]
 {\fst{
       {\fftype{#1}} \\
       #2 
     }
 }

\newcommand{\fsltt}[2]
 {\fstt{
       {\fftype{#1}} \\
       #2 
     }
 }

\newcommand{\fslttt}[2]
 {\fsttt{
       {\fftype{#1}} \\
       #2 
     }
 }


% jak \ft, \fs i \fsl tylko nieco ciasniejsze

\newcommand{\ftt}[2]
% {{\sc #1}\/{\rm #2}}
 {\textsc{#1}\/{\rm #2}}

\newcommand{\fst}[1]
  {
    \mbox{\small%
          $
          \left[
          \!\!\!
%          \sc
          \begin{tabular}{l} #1
          \end{tabular}
          \!\!\!\!\!\!\!
          \right]
          $
          }
   }

%\newcommand{\fslt}[2]
% {\fst{#2\\
%       {\scriptsize\it #1}
%      }
% }


% superciasne

\newcommand{\fstt}[1]
  {
    \mbox{\small%
          $
          \left[
          \!\!\!
%          \sc
          \begin{tabular}{l} #1
          \end{tabular}
          \!\!\!\!\!\!\!\!\!\!\!
          \right]
          $
          }
   }

%\newcommand{\fsltt}[2]
% {\fstt{#2\\
%       {\scriptsize\it #1}
%      }
% }

\newcommand{\fsttt}[1]
  {
    \mbox{\small%
          $
          \left[
          \!\!\!
%          \sc
          \begin{tabular}{l} #1
          \end{tabular}
          \!\!\!\!\!\!\!\!\!\!\!\!\!\!\!\!
          \right]
          $
          }
   }



% %add all your local new commands to this file

% \newcommand{\smiley}{:)}

% you are not supposed to mess with hardcore stuff, St.Mü. 22.08.2018
%% \renewbibmacro*{index:name}[5]{%
%%   \usebibmacro{index:entry}{#1}
%%     {\iffieldundef{usera}{}{\thefield{usera}\actualoperator}\mkbibindexname{#2}{#3}{#4}{#5}}}

% % \newcommand{\noop}[1]{}



% Rui

\newcommand{\spc}[0]{\hspace{-1pt}\underline{\hspace{6pt}}\,}
\newcommand{\spcs}[0]{\hspace{-1pt}\underline{\hspace{6pt}}\,\,}
\newcommand{\bad}[1]{\leavevmode\llap{#1}}
\newcommand{\COMMENT}[1]{}


% Rui coordination
\newcommand{\subl}[1]{$_{\scriptstyle \textsc{#1}}$}



% Andy Lücking gesture.tex
\newcommand{\Pointing}{\ding{43}}
% Giotto: "Meeting of Joachim and Anne at the Golden Gate" - 1305-10 
\definecolor{GoldenGate1}{rgb}{.13,.09,.13} % Dress of woman in black
\definecolor{GoldenGate2}{rgb}{.94,.94,.91} % Bridge
\definecolor{GoldenGate3}{rgb}{.06,.09,.22} % Blue sky
\definecolor{GoldenGate4}{rgb}{.94,.91,.87} % Dress of woman with shawl
\definecolor{GoldenGate5}{rgb}{.52,.26,.26} % Joachim's robe
\definecolor{GoldenGate6}{rgb}{.65,.35,.16} % Anne's robe
\definecolor{GoldenGate7}{rgb}{.91,.84,.42} % Joachim's halo

\makeatletter
\newcommand{\@Depth}{1} % x-dimension, to front
\newcommand{\@Height}{1} % z-dimension, up
\newcommand{\@Width}{1} % y-dimension, rightwards
%\GGS{<x-start>}{<y-start>}{<z-top>}{<z-bottom>}{<Farbe>}{<x-width>}{<y-depth>}{<opacity>}
\newcommand{\GGS}[9][]{%
\coordinate (O) at (#2-1,#3-1,#5);
\coordinate (A) at (#2-1,#3-1+#7,#5);
\coordinate (B) at (#2-1,#3-1+#7,#4);
\coordinate (C) at (#2-1,#3-1,#4);
\coordinate (D) at (#2-1+#8,#3-1,#5);
\coordinate (E) at (#2-1+#8,#3-1+#7,#5);
\coordinate (F) at (#2-1+#8,#3-1+#7,#4);
\coordinate (G) at (#2-1+#8,#3-1,#4);
\draw[draw=black, fill=#6, fill opacity=#9] (D) -- (E) -- (F) -- (G) -- cycle;% Front
\draw[draw=black, fill=#6, fill opacity=#9] (C) -- (B) -- (F) -- (G) -- cycle;% Top
\draw[draw=black, fill=#6, fill opacity=#9] (A) -- (B) -- (F) -- (E) -- cycle;% Right
}
\makeatother


% pragmatics
\newcommand{\speaking}[1]{\eqparbox{name}{\textsc{\lowercase{#1}\space}}}
\newcommand{\alname}[1]{\eqparbox{name}{\textsc{\lowercase{#1}}}}
\newcommand{\HPSGTTR}{HPSG$_{\text{TTR}}$\xspace}

\newcommand{\ttrtype}[1]{\textit{#1}}
\newcommand{\avmel}{\q<\quad\q>} %% shortcut for empty lists in AVM
\newcommand{\ttrmerge}{\ensuremath{\wedge_{\textit{merge}}}}
\newcommand{\Cat}[2][0.1pt]{%
  \begin{scope}[y=#1,x=#1,yscale=-1, inner sep=0pt, outer sep=0pt]
   \path[fill=#2,line join=miter,line cap=butt,even odd rule,line width=0.8pt]
  (151.3490,307.2045) -- (264.3490,307.2045) .. controls (264.3490,291.1410) and (263.2021,287.9545) .. (236.5990,287.9545) .. controls (240.8490,275.2045) and (258.1242,244.3581) .. (267.7240,244.3581) .. controls (276.2171,244.3581) and (286.3490,244.8259) .. (286.3490,264.2045) .. controls (286.3490,286.2045) and (323.3717,321.6755) .. (332.3490,307.2045) .. controls (345.7277,285.6390) and (309.3490,292.2151) .. (309.3490,240.2046) .. controls (309.3490,169.0514) and (350.8742,179.1807) .. (350.8742,139.2046) .. controls (350.8742,119.2045) and (345.3490,116.5037) .. (345.3490,102.2045) .. controls (345.3490,83.3070) and (361.9972,84.4036) .. (358.7581,68.7349) .. controls (356.5206,57.9117) and (354.7696,49.2320) .. (353.4652,36.1439) .. controls (352.5396,26.8573) and (352.2445,16.9594) .. (342.5985,17.3574) .. controls (331.2650,17.8250) and (326.9655,37.7742) .. (309.3490,39.2045) .. controls (291.7685,40.6320) and (276.7783,24.2380) .. (269.9740,26.5795) .. controls (263.2271,28.9013) and (265.3490,47.2045) .. (269.3490,60.2045) .. controls (275.6359,80.6368) and (289.3490,107.2045) .. (264.3490,111.2045) .. controls (239.3490,115.2045) and (196.3490,119.2045) .. (165.3490,160.2046) .. controls (134.3490,201.2046) and (135.4934,249.3212) .. (123.3490,264.2045) .. controls (82.5907,314.1553) and (40.8239,293.6463) .. (40.8239,335.2045) .. controls (40.8239,353.8102) and (72.3490,367.2045) .. (77.3490,361.2045) .. controls (82.3490,355.2045) and (34.8638,337.3259) .. (87.9955,316.2045) .. controls (133.3871,298.1601) and   (137.4391,294.4766) .. (151.3490,307.2045) -- cycle;
\end{scope}%
}


% KdK
\newcommand{\smiley}{:)}

\renewbibmacro*{index:name}[5]{%
  \usebibmacro{index:entry}{#1}
    {\iffieldundef{usera}{}{\thefield{usera}\actualoperator}\mkbibindexname{#2}{#3}{#4}{#5}}}

% \newcommand{\noop}[1]{}

% chngcntr.sty otherwise gives error that these are already defined
%\let\counterwithin\relax
%\let\counterwithout\relax

% the space of a left bracket for glossings
\newcommand{\LB}{\hspaceThis{[}}

\newcommand{\LF}{\mbox{$[\![$}}

\newcommand{\RF}{\mbox{$]\!]_F$}}

\newcommand{\RT}{\mbox{$]\!]_T$}}





% Manfred's

\newcommand{\kommentar}[1]{}

\newcommand{\bsp}[1]{\emph{#1}}
\newcommand{\bspT}[2]{\bsp{#1} `#2'}
\newcommand{\bspTL}[3]{\bsp{#1} (lit.: #2) `#3'}

\newcommand{\noidi}{§}

\newcommand{\refer}[1]{(\ref{#1})}

%\newcommand{\avmtype}[1]{\multicolumn{2}{l}{\type{#1}}}
\newcommand{\attr}[1]{\textsc{#1}}

\newcommand{\srdefault}{\mbox{\begin{tabular}{c}{\large <}\\[-1.5ex]$\sqcap$\end{tabular}}}

%% \newcommand{\myappcolumn}[2]{
%% \begin{minipage}[t]{#1}#2\end{minipage}
%% }

%% \newcommand{\appc}[1]{\myappcolumn{3.7cm}{#1}}


% Jong-Bok


% clean that up and do not use \def (killing other stuff defined before)
%\if 0
\newcommand\DEL{\textsc{del}}
\newcommand\del{\textsc{del}}

\newcommand\conn{\textsc{conn}}
\newcommand\CONN{\textsc{conn}}
\newcommand\CONJ{\textsc{conj}}
\newcommand\LITE{\textsc{lex}}
\newcommand\lite{\textsc{lex}}
\newcommand\HON{\textsc{hon}}

%\newcommand\CAUS{\textsc{caus}}
%\newcommand\PASS{\textsc{pass}}
\newcommand\NPST{\textsc{npst}}
%\newcommand\COND{\textsc{cond}}



\newcommand\hdlite{\textsc{head-lex construction}}
\newcommand\hdlight{\textsc{head-light} Schema}
\newcommand\NFORM{\textsc{nform}}

\newcommand\RELS{\textsc{rels}}
%\newcommand\TENSE{\textsc{tense}}


%\newcommand\ARG{\textsc{arg}}
\newcommand\ARGs{\textsc{arg0}}
\newcommand\ARGa{\textsc{arg}}
\newcommand\ARGb{\textsc{arg2}}
\newcommand\TPC{\textsc{top}}
%\newcommand\PROG{\textsc{prog}}

\newcommand\LIGHT{\textsc{light}\xspace}
\newcommand\pst{\textsc{pst}}
%\newcommand\PAST{\textsc{pst}}
%\newcommand\DAT{\textsc{dat}}
%\newcommand\CONJ{\textsc{conj}}
\newcommand\nominal{\textsc{nominal}}
\newcommand\NOMINAL{\textsc{nominal}}
\newcommand\VAL{\textsc{val}}
%\newcommand\val{\textsc{val}}
\newcommand\MODE{\textsc{mode}}
\newcommand\RESTR{\textsc{restr}}
\newcommand\SIT{\textsc{sit}}
\newcommand\ARG{\textsc{arg}}
\newcommand\RELN{\textsc{rel}}
%\newcommand\REL{\textsc{rel}}
%\newcommand\RELS{\textsc{rels}}
%\newcommand\arg-st{\textsc{arg-st}}
\newcommand\xdel{\textsc{xdel}}
\newcommand\zdel{\textsc{zdel}}
\newcommand\sug{\textsc{sug}}
%\newcommand\IMP{\textsc{imp}}
%\newcommand\conn{\textsc{conn}}
%\newcommand\CONJ{\textsc{conj}}
%\newcommand\HON{\textsc{hon}}
\newcommand\BN{\textsc{bn}}
\newcommand\bn{\textsc{bn}}
\newcommand\pres{\textsc{pres}}
\newcommand\PRES{\textsc{pres}}
\newcommand\prs{\textsc{pres}}
%\newcommand\PRS{\textsc{pres}}
\newcommand\agt{\textsc{agt}}
%\newcommand\DEL{\textsc{del}}
%\newcommand\PRED{\textsc{pred}}
\newcommand\AGENT{\textsc{agent}}
\newcommand\THEME{\textsc{theme}}
%\newcommand\AUX{\textsc{aux}}
%\newcommand\THEME{\textsc{theme}}
%\newcommand\PL{\textsc{pl}}
\newcommand\SRC{\textsc{src}}
\newcommand\src{\textsc{src}}
\newcommand{\FORMjb}{\textsc{form}}
\newcommand{\formjb}{\FORM}
\newcommand\GCASE{\textsc{gcase}}
\newcommand\gcase{\textsc{gcase}}
\newcommand\SCASE{\textsc{scase}}
\newcommand\PHON{\textsc{phon}}
%\newcommand\SS{\textsc{ss}}
\newcommand\SYN{\textsc{syn}}
%\newcommand\LOC{\textsc{loc}}
\newcommand\MOD{\textsc{mod}}
\newcommand\INV{\textsc{inv}}
%\newcommand\L{\textsc{l}}
%\newcommand\CASE{\textsc{case}}
\newcommand\SPR{\textsc{spr}}
\newcommand\COMPS{\textsc{comps}}
%\newcommand\comps{\textsc{comps}}
\newcommand\SEM{\textsc{sem}}
\newcommand\CONT{\textsc{cont}}
\newcommand\SUBCAT{\textsc{subcat}}
\newcommand\CAT{\textsc{cat}}
%\newcommand\C{\textsc{c}}
%\newcommand\SUBJ{\textsc{subj}}
\newcommand\subjjb{\textsc{subj}}
%\newcommand\SLASH{\textsc{slash}}
\newcommand\LOCAL{\textsc{local}}
%\newcommand\ARG-ST{\textsc{arg-st}}
%\newcommand\AGR{\textsc{agr}}
\newcommand\PER{\textsc{per}}
%\newcommand\NUM{\textsc{num}}
%\newcommand\IND{\textsc{ind}}
\newcommand\VFORM{\textsc{vform}}
\newcommand\PFORM{\textsc{pform}}
\newcommand\decl{\textsc{decl}}
%\newcommand\loc{\textsc{loc   }}
% \newcommand\   {\textsc{  }}

%\newcommand\NEG{\textsc{neg}}
\newcommand\FRAMES{\textsc{frames}}
%\newcommand\REFL{\textsc{refl}}

\newcommand\MKG{\textsc{mkg}}

%\newcommand\BN{\textsc{bn}}
\newcommand\HD{\textsc{hd}}
\newcommand\NP{\textsc{np}}
\newcommand\PF{\textsc{pf}}
%\newcommand\PL{\textsc{pl}}
\newcommand\PP{\textsc{pp}}
%\newcommand\SS{\textsc{ss}}
\newcommand\VF{\textsc{vf}}
\newcommand\VP{\textsc{vp}}
%\newcommand\bn{\textsc{bn}}
\newcommand\cl{\textsc{cl}}
%\newcommand\pl{\textsc{pl}}
\newcommand\Wh{\ital{Wh}}
%\newcommand\ng{\textsc{neg}}
\newcommand\wh{\ital{wh}}
%\newcommand\ACC{\textsc{acc}}
%\newcommand\AGR{\textsc{agr}}
\newcommand\AGT{\textsc{agt}}
\newcommand\ARC{\textsc{arc}}
%\newcommand\ARG{\textsc{arg}}
\newcommand\ARP{\textsc{arc}}
%\newcommand\AUX{\textsc{aux}}
%\newcommand\CAT{\textsc{cat}}
%\newcommand\COP{\textsc{cop}}
%\newcommand\DAT{\textsc{dat}}
\newcommand\NEWCOMMAND{\textsc{def}}
%\newcommand\DEL{\textsc{del}}
\newcommand\DOM{\textsc{dom}}
\newcommand\DTR{\textsc{dtr}}
%\newcommand\FUT{\textsc{fut}}
\newcommand\GAP{\textsc{gap}}
%\newcommand\GEN{\textsc{gen}}
%\newcommand\HON{\textsc{hon}}
%\newcommand\IMP{\textsc{imp}}
%\newcommand\IND{\textsc{ind}}
%\newcommand\INV{\textsc{inv}}
\newcommand\LEX{\textsc{lex}}
\newcommand\Lex{\textsc{lex}}
%\newcommand\LOC{\textsc{loc}}
%\newcommand\MOD{\textsc{mod}}
\newcommand\MRK{{\nr MRK}}
%\newcommand\NEG{\textsc{neg}}
\newcommand\NEW{\textsc{new}}
%\newcommand\NOM{\textsc{nom}}
%\newcommand\NUM{\textsc{num}}
%\newcommand\PER{\textsc{per}}
%\newcommand\PST{\textsc{pst}}
\newcommand\QUE{\textsc{que}}
%\newcommand\REL{\textsc{rel}}
\newcommand\SEL{\textsc{sel}}
%\newcommand\SEM{\textsc{sem}}
%\newcommand\SIT{\textsc{arg0}}
%\newcommand\SPR{\textsc{spr}}
%\newcommand\SRC{\textsc{src}}
\newcommand\SUG{\textsc{sug}}
%\newcommand\SYN{\textsc{syn}}
%\newcommand\TPC{\textsc{top}}
%\newcommand\VAL{\textsc{val}}
%\newcommand\acc{\textsc{acc}}
%\newcommand\agt{\textsc{agt}}
\newcommand\cop{\textsc{cop}}
%\newcommand\dat{\textsc{dat}}
\newcommand\foc{\textsc{focus}}
%\newcommand\FOC{\textsc{focus}}
\newcommand\fut{\textsc{fut}}
\newcommand\hon{\textsc{hon}}
\newcommand\imp{\textsc{imp}}
\newcommand\kes{\textsc{kes}}
%\newcommand\lex{\textsc{lex}}
%\newcommand\loc{\textsc{loc}}
\newcommand\mrk{{\nr MRK}}
%\newcommand\nom{\textsc{nom}}
%\newcommand\num{\textsc{num}}
\newcommand\plu{\textsc{plu}}
\newcommand\pne{\textsc{pne}}
%\newcommand\pst{\textsc{pst}}
\newcommand\pur{\textsc{pur}}
%\newcommand\que{\textsc{que}}
%\newcommand\src{\textsc{src}}
%\newcommand\sug{\textsc{sug}}
\newcommand\tpc{\textsc{top}}
%\newcommand\utt{\textsc{utt}}
%\newcommand\val{\textsc{val}}
%% \newcommand\LITE{\textsc{lex}}
%% \newcommand\PAST{\textsc{pst}}
%% \newcommand\POSP{\textsc{pos}}
%% \newcommand\PRS{\textsc{pres}}
%% \newcommand\mod{\textsc{mod}}%
%% \newcommand\newuse{{`kes'}}
%% \newcommand\posp{\textsc{pos}}
%% \newcommand\prs{\textsc{pres}}
%% \newcommand\psp{{\it en\/}}
%% \newcommand\skes{\textsc{kes}}
%% \newcommand\CASE{\textsc{case}}
%% \newcommand\CASE{\textsc{case}}
%% \newcommand\COMP{\textsc{comp}}
%% \newcommand\CONJ{\textsc{conj}}
%% \newcommand\CONN{\textsc{conn}}
%% \newcommand\CONT{\textsc{cont}}
%% \newcommand\DECL{\textsc{decl}}
%% \newcommand\FOCUS{\textsc{focus}}
%% %\newcommand\FORM{\textsc{form}} duplicate
%% \newcommand\FREL{\textsc{frel}}
%% \newcommand\GOAL{\textsc{goal}}
\newcommand\HEAD{\textsc{head}}
%% \newcommand\INDEX{\textsc{ind}}
%% \newcommand\INST{\textsc{inst}}
%% \newcommand\MODE{\textsc{mode}}
%% \newcommand\MOOD{\textsc{mood}}
%% \newcommand\NMLZ{\textsc{nmlz}}
%% \newcommand\PHON{\textsc{phon}}
%% \newcommand\PRED{\textsc{pred}}
%% %\newcommand\PRES{\textsc{pres}}
%% \newcommand\PROM{\textsc{prom}}
%% \newcommand\RELN{\textsc{pred}}
%% \newcommand\RELS{\textsc{rels}}
%% \newcommand\STEM{\textsc{stem}}
%% \newcommand\SUBJ{\textsc{subj}}
%% \newcommand\XARG{\textsc{xarg}}
%% \newcommand\bse{{\it bse\/}}
%% \newcommand\case{\textsc{case}}
%% \newcommand\caus{\textsc{caus}}
%% \newcommand\comp{\textsc{comp}}
%% \newcommand\conj{\textsc{conj}}
%% \newcommand\conn{\textsc{conn}}
%% \newcommand\decl{\textsc{decl}}
%% \newcommand\fin{{\it fin\/}}
%% %\newcommand\form{\textsc{form}}
%% \newcommand\gend{\textsc{gend}}
%% \newcommand\inf{{\it inf\/}}
%% \newcommand\mood{\textsc{mood}}
%% \newcommand\nmlz{\textsc{nmlz}}
%% \newcommand\pass{\textsc{pass}}
%% \newcommand\past{\textsc{past}}
%% \newcommand\perf{\textsc{perf}}
%% \newcommand\pln{{\it pln\/}}
%% \newcommand\pred{\textsc{pred}}


%% %\newcommand\pres{\textsc{pres}}
%% \newcommand\proc{\textsc{proc}}
%% \newcommand\nonfin{{\it nonfin\/}}
%% \newcommand\AGENT{\textsc{agent}}
%% \newcommand\CFORM{\textsc{cform}}
%% %\newcommand\COMPS{\textsc{comps}}
%% \newcommand\COORD{\textsc{coord}}
%% \newcommand\COUNT{\textsc{count}}
%% \newcommand\EXTRA{\textsc{extra}}
%% \newcommand\GCASE{\textsc{gcase}}
%% \newcommand\GIVEN{\textsc{given}}
%% \newcommand\LOCAL{\textsc{local}}
%% \newcommand\NFORM{\textsc{nform}}
%% \newcommand\PFORM{\textsc{pform}}
%% \newcommand\SCASE{\textsc{scase}}
%% \newcommand\SLASH{\textsc{slash}}
%% \newcommand\SLASH{\textsc{slash}}
%% \newcommand\THEME{\textsc{theme}}
%% \newcommand\TOPIC{\textsc{topic}}
%% \newcommand\VFORM{\textsc{vform}}
%% \newcommand\cause{\textsc{cause}}
%% %\newcommand\comps{\textsc{comps}}
%% \newcommand\gcase{\textsc{gcase}}
%% \newcommand\itkes{{\it kes\/}}
%% \newcommand\pass{{\it pass\/}}
%% \newcommand\vform{\textsc{vform}}
%% \newcommand\CCONT{\textsc{c-cont}}
%% \newcommand\GN{\textsc{given-new}}
%% \newcommand\INFO{\textsc{info-st}}
%% \newcommand\ARG-ST{\textsc{arg-st}}
%% \newcommand\SUBCAT{\textsc{subcat}}
%% \newcommand\SYNSEM{\textsc{synsem}}
%% \newcommand\VERBAL{\textsc{verbal}}
%% \newcommand\arg-st{\textsc{arg-st}}
%% \newcommand\plain{{\it plain}\/}
%% \newcommand\propos{\textsc{propos}}
%% \newcommand\ADVERBIAL{\textsc{advl}}
%% \newcommand\HIGHLIGHT{\textsc{prom}}
%% \newcommand\NOMINAL{\textsc{nominal}}

\newenvironment{myavm}{\begingroup\avmvskip{.1ex}
  \selectfont\begin{avm}}%
{\end{avm}\endgroup\medskip}
\newcommand\pfix{\vspace{-5pt}}


\newcommand{\jbsub}[1]{\lower4pt\hbox{\small #1}}
\newcommand{\jbssub}[1]{\lower4pt\hbox{\small #1}}
\newcommand\jbtr{\underbar{\ \ \ }\ }


%\fi

% cl

\newcommand{\delphin}{\textsc{delph-in}}


% YK -- CG chapter

\newcommand{\grey}[1]{\colorbox{mycolor}{#1}}
\definecolor{mycolor}{gray}{0.8}

\newcommand{\GQU}[2]{\raisebox{1.6ex}{\ensuremath{\rotatebox{180}{\textbf{#1}}_{\scalebox{.7}{\textbf{#2}}}}}}

\newcommand{\SetInfLen}{\setpremisesend{0pt}\setpremisesspace{10pt}\setnamespace{0pt}}

\newcommand{\pt}[1]{\ensuremath{\mathsf{#1}}}
\newcommand{\ptv}[1]{\ensuremath{\textsf{\textsl{#1}}}}

\newcommand{\sv}[1]{\ensuremath{\bm{\mathcal{#1}}}}
\newcommand{\sX}{\sv{X}}
\newcommand{\sF}{\sv{F}}
\newcommand{\sG}{\sv{G}}

\newcommand{\syncat}[1]{\textrm{#1}}
\newcommand{\syncatVar}[1]{\ensuremath{\mathit{#1}}}

\newcommand{\RuleName}[1]{\textrm{#1}}

\newcommand{\SemTyp}{\textsf{Sem}}

\newcommand{\E}{\ensuremath{\bm{\epsilon}}\xspace}

\newcommand{\greeka}{\upalpha}
\newcommand{\greekb}{\upbeta}
\newcommand{\greekd}{\updelta}
\newcommand{\greekp}{\upvarphi}
\newcommand{\greekr}{\uprho}
\newcommand{\greeks}{\upsigma}
\newcommand{\greekt}{\uptau}
\newcommand{\greeko}{\upomega}
\newcommand{\greekz}{\upzeta}

\newcommand{\Lemma}{\ensuremath{\hskip.5em\vdots\hskip.5em}\noLine}
\newcommand{\LemmaAlt}{\ensuremath{\hskip.5em\vdots\hskip.5em}}

\newcommand{\I}{\iota}

\newcommand{\sem}{\ensuremath}

\newcommand{\NoSem}{%
\renewcommand{\LexEnt}[3]{##1; \syncat{##3}}
\renewcommand{\LexEntTwoLine}[3]{\renewcommand{\arraystretch}{.8}%
\begin{array}[b]{l} ##1;  \\ \syncat{##3} \end{array}}
\renewcommand{\LexEntThreeLine}[3]{\renewcommand{\arraystretch}{.8}%
\begin{array}[b]{l} ##1; \\ \syncat{##3} \end{array}}}

\newcommand{\hypml}[2]{\left[\!\!#1\!\!\right]^{#2}}

%%%%for bussproof
\def\defaultHypSeparation{\hskip0.1in}
\def\ScoreOverhang{0pt}

\newcommand{\MultiLine}[1]{\renewcommand{\arraystretch}{.8}%
\ensuremath{\begin{array}[b]{l} #1 \end{array}}}

\newcommand{\MultiLineMod}[1]{%
\ensuremath{\begin{array}[t]{l} #1 \end{array}}}

\newcommand{\hypothesis}[2]{[ #1 ]^{#2}}

\newcommand{\LexEnt}[3]{#1; \ensuremath{#2}; \syncat{#3}}

\newcommand{\LexEntTwoLine}[3]{\renewcommand{\arraystretch}{.8}%
\begin{array}[b]{l} #1; \\ \ensuremath{#2};  \syncat{#3} \end{array}}

\newcommand{\LexEntThreeLine}[3]{\renewcommand{\arraystretch}{.8}%
\begin{array}[b]{l} #1; \\ \ensuremath{#2}; \\ \syncat{#3} \end{array}}

\newcommand{\LexEntFiveLine}[5]{\renewcommand{\arraystretch}{.8}%
\begin{array}{l} #1 \\ #2; \\ \ensuremath{#3} \\ \ensuremath{#4}; \\ \syncat{#5} \end{array}}

\newcommand{\LexEntFourLine}[4]{\renewcommand{\arraystretch}{.8}%
\begin{array}{l} \pt{#1} \\ \pt{#2}; \\ \syncat{#4} \end{array}}

\newcommand{\ManySomething}{\renewcommand{\arraystretch}{.8}%
\raisebox{-3mm}{\begin{array}[b]{c} \vdots \,\,\,\,\,\, \vdots \\
\vdots \end{array}}}

\newcommand{\lemma}[1]{\renewcommand{\arraystretch}{.8}%
\begin{array}[b]{c} \vdots \\ #1 \end{array}}

\newcommand{\lemmarev}[1]{\renewcommand{\arraystretch}{.8}%
\begin{array}[b]{c} #1 \\ \vdots \end{array}}

\newcommand{\p}{\ensuremath{\upvarphi}}

% clashes with soul package
\newcommand{\yusukest}{\textbf{\textsf{st}}}

\newcommand{\shortarrow}{\xspace\hskip-1.2ex\scalebox{.5}[1]{\ensuremath{\bm{\rightarrow}}}\hskip-.5ex\xspace}

\newcommand{\SemInt}[1]{\mbox{$[\![ \textrm{#1} ]\!]$}}

\newcommand{\HypSpace}{\hskip-.8ex}
\newcommand{\RaiseHeight}{\raisebox{2.2ex}}
\newcommand{\RaiseHeightLess}{\raisebox{1ex}}

\newcommand{\ThreeColHyp}[1]{\RaiseHeight{\Bigg[}\HypSpace#1\HypSpace\RaiseHeight{\Bigg]}}
\newcommand{\TwoColHyp}[1]{\RaiseHeightLess{\Big[}\HypSpace#1\HypSpace\RaiseHeightLess{\Big]}}

\newcommand{\LemmaShort}{\ensuremath{ \ \vdots} \ \noLine}
\newcommand{\LemmaShortAlt}{\ensuremath{ \ \vdots} \ }

\newcommand{\fail}{**}
\newcommand{\vs}{\raisebox{.05em}{\ensuremath{\upharpoonright}}}
\newcommand{\DerivSize}{\small}

\def\maru#1{{\ooalign{\hfil
  \ifnum#1>999 \resizebox{.25\width}{\height}{#1}\else%
  \ifnum#1>99 \resizebox{.33\width}{\height}{#1}\else%
  \ifnum#1>9 \resizebox{.5\width}{\height}{#1}\else #1%
  \fi\fi\fi%
\/\hfil\crcr%
\raise.167ex\hbox{\mathhexbox20D}}}}

\newenvironment{samepage2}%
 {\begin{flushleft}\begin{minipage}{\linewidth}}
 {\end{minipage}\end{flushleft}}

\newcommand{\cmt}[1]{\textsl{\textbf{[#1]}}}
\newcommand{\trns}[1]{\textbf{#1}\xspace}
\newcommand{\ptfont}{}
\newcommand{\gp}{\underline{\phantom{oo}}}
\newcommand{\mgcmt}{\marginnote}

\newcommand{\term}[1]{\emph{#1}}

\newcommand{\citeposs}[1]{\citeauthor{#1}'s \citeyearpar{#1}}

% for standalone compilations Felix: This is in the class already
%\let\thetitle\@title
%\let\theauthor\@author 
\makeatletter
\newcommand{\togglepaper}[1][0]{ 
\bibliography{../Bibliographies/stmue,../localbibliography,
../Bibliographies/properties,
../Bibliographies/np,
../Bibliographies/negation,
../Bibliographies/ellipsis,
../Bibliographies/binding,
../Bibliographies/complex-predicates,
../Bibliographies/control-raising,
../Bibliographies/coordination,
../Bibliographies/morphology,
../Bibliographies/lfg,
collection.bib}
  %% hyphenation points for line breaks
%% Normally, automatic hyphenation in LaTeX is very good
%% If a word is mis-hyphenated, add it to this file
%%
%% add information to TeX file before \begin{document} with:
%% %% hyphenation points for line breaks
%% Normally, automatic hyphenation in LaTeX is very good
%% If a word is mis-hyphenated, add it to this file
%%
%% add information to TeX file before \begin{document} with:
%% \include{localhyphenation}
\hyphenation{
A-la-hver-dzhie-va
anaph-o-ra
ana-phor
ana-phors
an-te-ced-ent
an-te-ced-ents
affri-ca-te
affri-ca-tes
ap-proach-es
Atha-bas-kan
Athe-nä-um
Bona-mi
Chi-che-ŵa
com-ple-ments
con-straints
Cope-sta-ke
Da-ge-stan
Dor-drecht
er-klä-ren-de
Ginz-burg
Gro-ning-en
Jap-a-nese
Jon-a-than
Ka-tho-lie-ke
Ko-bon
krie-gen
Le-Sourd
moth-er
Mül-ler
Nie-mey-er
Par-a-digm
Prze-piór-kow-ski
phe-nom-e-non
re-nowned
Rie-he-mann
un-bound-ed
with-in
}

% listing within here does not have any effect for lfg.tex % 2020-05-14

% why has "erklärende" be listed here? I specified langid in bibtex item. Something is still not working with hyphenation.


% to do: check
%  Alahverdzhieva


% biblatex:

% This is a LaTeX frontend to TeX’s \hyphenation command which defines hy- phenation exceptions. The ⟨language⟩ must be a language name known to the babel/polyglossia packages. The ⟨text ⟩ is a whitespace-separated list of words. Hyphenation points are marked with a dash:

% \DefineHyphenationExceptions{american}{%
% hy-phen-ation ex-cep-tion }

\hyphenation{
A-la-hver-dzhie-va
anaph-o-ra
ana-phor
ana-phors
an-te-ced-ent
an-te-ced-ents
affri-ca-te
affri-ca-tes
ap-proach-es
Atha-bas-kan
Athe-nä-um
Bona-mi
Chi-che-ŵa
com-ple-ments
con-straints
Cope-sta-ke
Da-ge-stan
Dor-drecht
er-klä-ren-de
Ginz-burg
Gro-ning-en
Jap-a-nese
Jon-a-than
Ka-tho-lie-ke
Ko-bon
krie-gen
Le-Sourd
moth-er
Mül-ler
Nie-mey-er
Par-a-digm
Prze-piór-kow-ski
phe-nom-e-non
re-nowned
Rie-he-mann
un-bound-ed
with-in
}

% listing within here does not have any effect for lfg.tex % 2020-05-14

% why has "erklärende" be listed here? I specified langid in bibtex item. Something is still not working with hyphenation.


% to do: check
%  Alahverdzhieva


% biblatex:

% This is a LaTeX frontend to TeX’s \hyphenation command which defines hy- phenation exceptions. The ⟨language⟩ must be a language name known to the babel/polyglossia packages. The ⟨text ⟩ is a whitespace-separated list of words. Hyphenation points are marked with a dash:

% \DefineHyphenationExceptions{american}{%
% hy-phen-ation ex-cep-tion }

  \memoizeset{
    memo filename prefix={hpsg-handbook.memo.dir/},
    % readonly
  }
  \papernote{\scriptsize\normalfont
    \@author.
    \@title. 
    To appear in: 
    Stefan Müller, Anne Abeillé, Robert D. Borsley \& Jean-Pierre Koenig (eds.)
    HPSG Handbook
    Berlin: Language Science Press. [preliminary page numbering]
  }
  \pagenumbering{roman}
  \setcounter{chapter}{#1}
  \addtocounter{chapter}{-1}
}
\makeatother

\makeatletter
\newcommand{\togglepaperminimal}[1][0]{ 
  \bibliography{../Bibliographies/stmue,
                ../localbibliography,
  ../Bibliographies/coordination,
collection.bib}
  %% hyphenation points for line breaks
%% Normally, automatic hyphenation in LaTeX is very good
%% If a word is mis-hyphenated, add it to this file
%%
%% add information to TeX file before \begin{document} with:
%% %% hyphenation points for line breaks
%% Normally, automatic hyphenation in LaTeX is very good
%% If a word is mis-hyphenated, add it to this file
%%
%% add information to TeX file before \begin{document} with:
%% \include{localhyphenation}
\hyphenation{
A-la-hver-dzhie-va
anaph-o-ra
ana-phor
ana-phors
an-te-ced-ent
an-te-ced-ents
affri-ca-te
affri-ca-tes
ap-proach-es
Atha-bas-kan
Athe-nä-um
Bona-mi
Chi-che-ŵa
com-ple-ments
con-straints
Cope-sta-ke
Da-ge-stan
Dor-drecht
er-klä-ren-de
Ginz-burg
Gro-ning-en
Jap-a-nese
Jon-a-than
Ka-tho-lie-ke
Ko-bon
krie-gen
Le-Sourd
moth-er
Mül-ler
Nie-mey-er
Par-a-digm
Prze-piór-kow-ski
phe-nom-e-non
re-nowned
Rie-he-mann
un-bound-ed
with-in
}

% listing within here does not have any effect for lfg.tex % 2020-05-14

% why has "erklärende" be listed here? I specified langid in bibtex item. Something is still not working with hyphenation.


% to do: check
%  Alahverdzhieva


% biblatex:

% This is a LaTeX frontend to TeX’s \hyphenation command which defines hy- phenation exceptions. The ⟨language⟩ must be a language name known to the babel/polyglossia packages. The ⟨text ⟩ is a whitespace-separated list of words. Hyphenation points are marked with a dash:

% \DefineHyphenationExceptions{american}{%
% hy-phen-ation ex-cep-tion }

\hyphenation{
A-la-hver-dzhie-va
anaph-o-ra
ana-phor
ana-phors
an-te-ced-ent
an-te-ced-ents
affri-ca-te
affri-ca-tes
ap-proach-es
Atha-bas-kan
Athe-nä-um
Bona-mi
Chi-che-ŵa
com-ple-ments
con-straints
Cope-sta-ke
Da-ge-stan
Dor-drecht
er-klä-ren-de
Ginz-burg
Gro-ning-en
Jap-a-nese
Jon-a-than
Ka-tho-lie-ke
Ko-bon
krie-gen
Le-Sourd
moth-er
Mül-ler
Nie-mey-er
Par-a-digm
Prze-piór-kow-ski
phe-nom-e-non
re-nowned
Rie-he-mann
un-bound-ed
with-in
}

% listing within here does not have any effect for lfg.tex % 2020-05-14

% why has "erklärende" be listed here? I specified langid in bibtex item. Something is still not working with hyphenation.


% to do: check
%  Alahverdzhieva


% biblatex:

% This is a LaTeX frontend to TeX’s \hyphenation command which defines hy- phenation exceptions. The ⟨language⟩ must be a language name known to the babel/polyglossia packages. The ⟨text ⟩ is a whitespace-separated list of words. Hyphenation points are marked with a dash:

% \DefineHyphenationExceptions{american}{%
% hy-phen-ation ex-cep-tion }

  \memoizeset{
    memo filename prefix={hpsg-handbook.memo.dir/},
    % readonly
  }
  \papernote{\scriptsize\normalfont
    \@author.
    \@title. 
    To appear in: 
    Stefan Müller, Anne Abeillé, Robert D. Borsley \& Jean-Pierre Koenig (eds.)
    HPSG Handbook
    Berlin: Language Science Press. [preliminary page numbering]
  }
  \pagenumbering{roman}
  \setcounter{chapter}{#1}
  \addtocounter{chapter}{-1}
}
\makeatother




% In case that year is not given, but pubstate. This mainly occurs for titles that are forthcoming, in press, etc.
\renewbibmacro*{addendum+pubstate}{% Thanks to https://tex.stackexchange.com/a/154367 for the idea
  \printfield{addendum}%
  \iffieldequalstr{labeldatesource}{pubstate}{}
  {\newunit\newblock\printfield{pubstate}}
}

\DeclareLabeldate{%
    \field{date}
    \field{year}
    \field{eventdate}
    \field{origdate}
    \field{urldate}
    \field{pubstate}
    \literal{nodate}
}

%\defbibheading{diachrony-sources}{\section*{Sources}} 

% if no langid is set, it is English:
% https://tex.stackexchange.com/a/279302
\DeclareSourcemap{
  \maps[datatype=bibtex]{
    \map{
      \step[fieldset=langid, fieldvalue={english}]
    }
  }
}


% for bibliographies
% biber/biblatex could use sortname field rather than messing around this way.
\newcommand{\SortNoop}[1]{}


% Doug Ball

\newcommand{\elist}{\q<\ \ \q>}

\newcommand{\esetDB}{\q\{\ \ \q\}}


\makeatletter

\newcommand{\nolistbreak}{%

  \let\oldpar\par\def\par{\oldpar\nobreak}% Any \par issues a \nobreak

  \@nobreaktrue% Don't break with first \item

}

\makeatother


% intermediate before Frank's trees are fixed
% This will be removed!!!!!
%\newcommand{\tree}[1]{} % ignore them blody trees
%\usepackage{tree-dvips}


\newcommand{\nodeconnect}[2]{}
\newcommand{\nodetriangle}[2]{}



% Doug relative clauses
%% I've compiled out almost all my private LaTeX command, but there are some
%% I found hard to get rid of. They are defined here.
%% There are few others which defined in places in the document where they have only
%% local effect (e.g. within figures); their names all end in DA, e.g. \MotherDA
%% There are a lot of \labels -- they are all of the form \label{sec:rc-...} or
%% \label{x:rc-...} or similar, so there should be no clashes.

% Subscripts -- scriptsize italic shape lowered by .25ex 
\newcommand{\subscr}[1]{\raisebox{-.5ex}{\protect{\scriptsize{\itshape #1\/}}}}
% A boxed subscript, for avm tags in normal text
\newcommand{\subtag}[1]{\subscr{\idx{#1}}}

%% Sets and tuples: I use \setof{} to get brackets that are upright, not slanted
%\newcommand{\setof}[1]{\ensuremath{\lbrace\,\mathit{#1}\,\rbrace}}
% 11.10.2019 EP: Doug requested replacement of existing \setof definition with the following:
%\newcommand{\setof}[1]{\begin{avm}\{\textcolor{red}{#1}\}\end{avm}}
% 31.1.2019 EP: Doug requested re-replacement of the above \textcolour version with the following:
\newcommand{\setof}[1]{\begin{avm}\{#1\}\end{avm}}

\newcommand{\tuple}[1]{\ensuremath{\left\langle\,\mbox{\textit{#1}}\,\right\rangle}}

% Single pile of stuff, optional arugment is psn (e.g. t or b)
% e.g. to put a over b over c in a centered column, top aligned, do:
%   \cPile[t]{a\\b\\c} 
\newcommand{\cPile}[2][]{%
  \begingroup%
  \renewcommand{\arraystretch}{.5}\begin{tabular}[#1]{c}#2\end{tabular}%
  \endgroup%
}

%% for linguistic examples in running text (`linguistic citation'):
\newcommand{\lic}[1]{\textit{#1}}

%% A gap marked by an underline, raised slightly
%% Default argument indicates how long the line should be:
\newcommand{\uGap}[1][3ex]{\raisebox{.25em}{\underline{\hspace{#1}}}\xspace}

%% \TnodeDA{XP}{avmcontents} -- in a Tree, put a node label next to an AVM
\newcommand{\TnodeDA}[2]{#1~\begin{avm}{#2}\end{avm}}

%% This allows tipa stuff to be put in \emph -- we need to change to cmr first.
%% It is used in the discussion of Arabic.
\newcommand{\emphtipa}[1]{{\fontfamily{cmr}\emph{\tipaencoding #1}}} 



 
 
\definecolor{lsDOIGray}{cmyk}{0,0,0,0.45}


% morphology.tex:
% Berthold

\newcommand{\dnode}[1]{\rnode{#1}{\fbox{#1}}}
\newcommand{\tnode}[1]{\rnode{#1}{\textit{#1}}}

\newcommand{\tl}[2]{#2}

\newcommand{\rrr}[3]{%
  \psframebox[linestyle=none]{%
    \avmoptions{center}
    \begin{avm}
      \[mud & \{ #1 \}\\
      ms & \{ #2 \}\\
      mph & \<  #3 \> \]
    \end{avm}
  }
}
\newcommand{\rr}[2]{%
  \psframebox[linestyle=none]{%
    \avmoptions{center}
    \begin{avm}
      \[mud & \{ #1 \}\\
      mph & \<  #2 \> \]
    \end{avm}
  }
}
 

% Frank Richter
\newtheorem{mydef}{Definition}

\long\def\set[#1\set=#2\set]%
{%
\left\{%
\tabcolsep 1pt%
\begin{tabular}{l}%
#1%
\end{tabular}%
\left|%
\tabcolsep 1pt%
\begin{tabular}{l}%
#2%
\end{tabular}%
\right.%
\right\}%
}

\newcommand{\einruck}{\\ \hspace*{1em}}


%\newcommand{\NatNum}{\mathrm{I\hspace{-.17em}N}}
\newcommand{\NatNum}{\mathbb{N}}
\newcommand{\Aug}[1]{\widehat{#1}}
%\newcommand{\its}{\mathrm{:}}
% Felix 14.02.2020
\DeclareMathOperator{\its}{:}

\newcommand{\sequence}[1]{\langle#1\rangle}

\newcommand{\INTERPRETATION}[2]{\sequence{#1\mathsf{U}#2,#1\mathsf{S}#2,#1\mathsf{A}#2,#1\mathsf{R}#2}}
\newcommand{\Interpretation}{\INTERPRETATION{}{}}

\newcommand{\Inte}{\mathsf{I}}
\newcommand{\Unive}{\mathsf{U}}
\newcommand{\Speci}{\mathsf{S}}
\newcommand{\Atti}{\mathsf{A}}
\newcommand{\Reli}{\mathsf{R}}
\newcommand{\ReliT}{\mathsf{RT}}

\newcommand{\VarInt}{\mathsf{G}}
\newcommand{\CInt}{\mathsf{C}}
\newcommand{\Tinte}{\mathsf{T}}
\newcommand{\Dinte}{\mathsf{D}}

% this was missing from ash's stuff.

%% \def \optrulenode#1{
%%   \setbox1\hbox{$\left(\hbox{\begin{tabular}{@{\strut}c@{\strut}}#1\end{tabular}}\right)$}
%%   \raisebox{1.9ex}{\raisebox{-\ht1}{\copy1}}}



\newcommand{\pslabel}[1]{}

\newcommand{\addpagesunless}{\todostefan{add pages unless you cite the
 work as such}}

% dg.tex
% framed boxes as used in dg.tex
% original idea from stackexchange, but modified by Saso
% http://tex.stackexchange.com/questions/230300/doing-something-like-psframebox-in-tikz#230306
\tikzset{
  frbox/.style={
    rounded corners,
    draw,
    thick,
    inner sep=5pt,
    anchor=base,
  },
}

% get rid of these morewrite messages:
% https://tex.stackexchange.com/questions/419489/suppressing-messages-to-standard-output-from-package-morewrites/419494#419494
\ExplSyntaxOn
\cs_set_protected:Npn \__morewrites_shipout_ii:
  {
    \__morewrites_before_shipout:
    \__morewrites_tex_shipout:w \tex_box:D \g__morewrites_shipout_box
    \edef\tmp{\interactionmode\the\interactionmode\space}\batchmode\__morewrites_after_shipout:\tmp
  }
\ExplSyntaxOff


% This is for places where authors used bold. I replace them by \emph
% but have the information where the bold was. St. Mü. 09.05.2020
\newcommand{\textbfemph}[1]{\emph{#1}}



% Felix 09.06.2020: copy code from the third line into localcommands.tex: https://github.com/langsci/langscibook#defined-environments-commands-etc
\patchcmd{\mkbibindexname}{\ifdefvoid{#3}{}{\MakeCapital{#3} }}{\ifdefvoid{#3}{}{#3 }}{}{\AtEndDocument{\typeout{mkbibindexname could not be patched.}}}

  %% -*- coding:utf-8 -*-

%%%%%%%%%%%%%%%%%%%%%%%%%%%%%%%%%%%%%%%%%%%%%%%%%%%%%%%%%%%%
%
% gb4e

% fixes problem with to much vertical space between \zl and \eal due to the \nopagebreak
% command.
\makeatletter
\def\@exe[#1]{\ifnum \@xnumdepth >0%
                 \if@xrec\@exrecwarn\fi%
                 \if@noftnote\@exrecwarn\fi%
                 \@xnumdepth0\@listdepth0\@xrectrue%
                 \save@counters%
              \fi%
                 \advance\@xnumdepth \@ne \@@xsi%
                 \if@noftnote%
                        \begin{list}{(\thexnumi)}%
                        {\usecounter{xnumi}\@subex{#1}{\@gblabelsep}{0em}%
                        \setcounter{xnumi}{\value{equation}}}
% this is commented out here since it causes additional space between \zl and \eal 06.06.2020
%                        \nopagebreak}%
                 \else%
                        \begin{list}{(\roman{xnumi})}%
                        {\usecounter{xnumi}\@subex{(iiv)}{\@gblabelsep}{\footexindent}%
                        \setcounter{xnumi}{\value{fnx}}}%
                 \fi}
\makeatother

% the texlive 2020 langsci-gb4e adds a newline after \eas, the texlive 2017 version was OK.
\makeatletter
\def\eas{\ifnum\@xnumdepth=0\begin{exe}[(34)]\else\begin{xlist}[iv.]\fi\ex\begin{tabular}[t]{@{}b{.99\linewidth}@{}}}
\makeatother


%%%%%%%%%%%%%%%%%%%%%%%%%%%%%%%%%%%%%%%%%%%%%%%%%%%%%%%%%%
%
% biblatex

% biblatex sets the option autolang=hyphens
%
% This disables language shorthands. To avoid this, the hyphens code can be redefined
%
% https://tex.stackexchange.com/a/548047/18561

\makeatletter
\def\hyphenrules#1{%
  \edef\bbl@tempf{#1}%
  \bbl@fixname\bbl@tempf
  \bbl@iflanguage\bbl@tempf{%
    \expandafter\bbl@patterns\expandafter{\bbl@tempf}%
    \expandafter\ifx\csname\bbl@tempf hyphenmins\endcsname\relax
      \set@hyphenmins\tw@\thr@@\relax
    \else
      \expandafter\expandafter\expandafter\set@hyphenmins
      \csname\bbl@tempf hyphenmins\endcsname\relax
    \fi}}
\makeatother


% the package defined \attop in a way that produced a box that has textwidth
%
\def\attop#1{\leavevmode\begin{minipage}[t]{.995\linewidth}\strut\vskip-\baselineskip\begin{minipage}[t]{.995\linewidth}#1\end{minipage}\end{minipage}}


%%%%%%%%%%%%%%%%%%%%%%%%%%%%%%%%%%%%%%%%%%%%%%%%%%%%%%%%%%%%%%%%%%%%


% Don't do this at home. I do not like the smaller font for captions.
% This does not work. Throw out package caption in langscibook
% \captionsetup{%
% font={%
% stretch=1%.8%
% ,normalsize%,small%
% },%
% width=\textwidth%.8\textwidth
% }
% \setcaphanging


  \togglepaper[3]
}{}

% you may switch off externalization of changed files here:
%\forestset{external/readonly}


\author{%
 Frank Richter\affiliation{Goethe Universität Frankfurt}%
}
\title{Formal background}  

% \chapterDOI{} %will be filled in at production
% \epigram{}
\abstract{
  This chapter provides a very condensed introduction to a formalism for
  \cite{PollardSag1994} and explains its fundamental concepts. It pays
  special attention to the model-theoretic meaning of HPSG grammars. In
  addition, it points out some links to other, related formalisms, such as
  feature logics of partial information, and to related
  terminology in the context of grammar implementation platforms.
}

\begin{document}
\maketitle
\label{chap-formal-background}

%--------------------------------------------------------------
\section{Introduction}
\label{sec-introduction}

The two HPSG books by Pollard and Sag
\citep{PollardSag1987,PollardSag1994} do not present grammar formalisms
with the intention to provide precise definitions.
Instead they refer to various inspirations in the
logics of typed feature structures or in predicate logic, informally
characterize the intended formalisms, and explain them as they are
used in concrete grammars of \ili{English}. \citet{PollardSag1994} further clarify
their intentions in an appendix which lists most (but not all) of the components
of their grammar of \ili{English} explicitly, and summarizes most of their core
assumptions. With this strategy, both books leave room for
interpretation.

There are a number of challenges with reviewing the formal background
of HPSG. Some of them have to do with the long publication history of relevant papers and books, some
with the considerable influence of grammar implementation platforms, which have
their own formalisms and shape the way in which
linguists think and talk about grammars with their platform-specific terminology
and notational conventions. % (see \crossrefchaptert{cl}). % book chapter? 
Salient examples include convenient
notations for phrase structure rules, the treatment of lexical
representations or the lexicon, mechanisms for lexical rules, and
notations for default values, among many other devices.  Many of these
notations are well-known in the HPSG community; they are convenient,
compact and arguably even necessary to write readable grammars. At
the same time, they are
a meta-notation in the sense that they
do not (directly) belong to the syntax of the assumed feature logics.
However, even if they are outside a declarative, logical formalism for HPSG,
there is usually a way to
interpret them in HPSG-compatible formalisms, but the necessary
re-interpretation can deviate to a larger or lesser degree from
what their users have in mind when they write their grammars. For example,
a phrase structure rule in the sense of a context-free or context-sensitive
rewrite system is not the same as an ID Schema written in a feature logic, which
might matter in some cases but not in others. To name one difference,
an ID Schema may easily leave the number of a phrase's daughters unspecified (and
thus potentially infinite).
The differences may be sometimes subtle and sometimes significant, but
they entail that the meaning of the notations seen through the lens
of logic is not what their users might assume either
based on their meaning in other contexts or on what is gleaned 
from the behavior
of a given implementation platform for parsing or generation which
employs that kind of syntax.
Similarly, terminology that belongs to the computational environment
of implementations is often transferred to grammar theory, and
again, when checking the technical specifics, a re-interpretation in terms of a feature-logical HPSG
formalism can sometimes be trivial and sometimes nearly impossible,
and different available re-interpretation choices
lead to significantly different results.

Reviewing\label{formal:page-unification-start} HPSG's formal background,
it is not only the multi-purpose character and flexibility of the ubiquitous informal attribute-value matrix (AVM) notation and its
practical notational enhancements (for lexical rules, decorated sort
hierarchies, phrase structure trees, etc.) that one needs to be
aware of, but also early changes in
foundational assumptions and terminology.
When first presented in a book in 1987, HPSG was conceived of as a
\emph{unification-based} grammar theory, a name, the authors
explain, which ``arises from the algebra that governs partial
information structures'' \citep[7]{PollardSag1987}. This algebra was
populated by partial feature structures with unification as a
fundamental algebraic operation. In the framework envisioned seven
years later in \citet{PollardSag1994},
that algebra did not exist anymore, feature structures were no longer
partial but total objects in models of a logical theory, and
unification was no longer defined in the new setting (as the relevant
algebra was gone). However, most of the notation and considerable
portions of the terminology of 1987 remain with us to this day, such as the
\emph{types} of feature structures (replaced by
\emph{sorts} in 1994,
when the term \emph{type} was used for a different concept, to be
discussed below), the pieces of information (for 1987-style feature
structures) or even the word \emph{unification}, which took on a
casual life of its own without the original algebra in
which it had been defined. Occasionally these words still have a
precise technical interpretation in the language of grammar
implementation environments or in their run-time system, which may
reinforce their use in the community despite their lack of meaning in
the standard formalism of HPSG. Implementation platforms also often add
their own technical and notational devices, thereby
inviting linguists to import them as useful tools into their theoretical grammar
writing.\label{formal:page-unification-end}

%Another aspect that is frequently overlooked or ignored concerns
%feature structures themselves: In the earlier version of HPSG they
%were so fundamental to the framework and their properties were

%\#\# further explanation, including references to the literature where
%the classical HPSG formalism was discussed and what these papers focused
%on  --> if at all, this will be in Sections~\ref{sec-alt-gr-meaning} and
%\ref{sec-alt-formalisms}

This handbook article cannot disentangle the history of and relationships
between the various formalisms leading to an explication of the
1994 version of HPSG, nor of those that existed and still exist in
parallel. It sets out to clarify the terminology and
structure of a formalism for \cite{PollardSag1994} and
presents a canonical formalism of the final version of
HPSG in \cite{PollardSag1994}. Only occasionally will it point out some of
the differences to its 1987 precursor where
the older terminology is still present in current HPSG papers and
may be confusing to an audience unaware of the different
usages of terms. Similarly, it does not cover SBCG (Sign-Based
Construction Grammar, \crossrefchaptert{cxg}).

%I will clarify parallel terminological traditions and

The main sources of the present summary are the construction of a
model theory for HPSG by \citet{King99a-u} and \citet{Pollard99a}, and
their synoptic reconstruction on the basis of a comprehensive logical
language for HPSG, \emph{Relational Speciate Re-entrant Language}
(RSRL) by \citet{Richter2004a-u}, including the critique and
extensions sketched in
\citet{Richter2007a}. Section~\ref{sec-essentials} gives a largely
non-technical introductory overview which should provide sufficient
background to follow all linguistic chapters of the present
handbook. The subsequent sections
(\ref{sec-signatures}--\ref{sec-alt-gr-meaning}) are for readers
keen on obtaining a deeper understanding or looking for clarification
of what might
remain vague and imprecise in an initial broad overview. Those sections
might be more challenging for the casual reader, but in return offer a
fairly self-contained and comprehensive summary, omitting only
the mathematical groundwork and
definitions needed to spell out alternative
model-theories, as this goes beyond what can reasonably be compressed to
handbook format.


\section{Essentials: An informal overview}
\label{sec-essentials}

%informal first pass through the architecture of the formalism, introducing
%terminology and main concepts

This section presents an informal summary of the essentials of an HPSG
formalism in the sense of \cite{PollardSag1994} as it emerged from their
original outline and its subsequent elaboration. From here on, the term
``HPSG formalism'' always refers to this tradition, unless explicitly
stated otherwise.  All later sections in this chapter will flesh out
the basic ideas introduced here with a precise technical treatment of
the relevant notions. Readers who are already familiar with feature
logics and are specifically interested in technical details may want
to skip ahead to Section~\ref{sec-signatures}.

At the heart of HPSG is a fundamental distinction between descriptions
and described objects: a grammar avails itself of descriptions with
the purpose of describing linguistic objects.  \cite[17--18, 396]{PollardSag1994}
commit to the ontological assumption that linguistic objects only
exist as complete objects. Partial linguistic objects do not
exist. Descriptions of linguistic objects, however, are typically
\emph{partial}, i.e.\ they do not mention many, or even most, properties of the
objects in their denotation. They are \emph{underspecified}.
A word can be described as being nominal
and plural, leaving all its other properties (gender, case, number and
category of its arguments, etc.) unspecified. But any
concrete word being so described will have all other properties that
a plural noun can have, with none of them missing.
A single underspecified description can therefore
describe many distinct linguistic objects. Grammatical descriptions
often describe an infinity of objects. Again considering plural nouns,
\ili{English} can be thought of as having a very large number or an infinity
of them due to morphological processes such as compounding, depending on
the choice of morphological analysis.

Descriptions are couched in a (language of a) feature logic rather
than in \ili{English} for precision. Linguistic objects as the subject of linguistic study
are sharply distinguished from their logical descriptions and are
entities in the denotation of the grammatical descriptions.  The
feature logic of HPSG can be seen as a particularly expressive variant
of description logics. With this architecture, HPSG is a
\emph{model-theoretic} grammar framework as opposed to
\emph{generative-enumerative} grammar frameworks, which
have rewrite systems that generate expressions
from some start symbol(s) \citep{PS2001a}.

A small digression might be in order to prevent confusion arising from
the co-existence of different versions of feature logics.  Varieties
of HPSG more closely related to the tradition of \cite{PollardSag1987}
do not make the same distinction between descriptions and described
objects. Instead they employ a notion of \emph{feature structures} as
entities carrying \emph{partial information}. These partial feature
structures are, or correspond to, logical expressions in a certain
normal form and are ordered in an algebra of partial information
according to the amount of information they carry. In informal
notation, they are written as AVMs just like the descriptions of the
formalism we are presently concerned with, and this notational
similarity contributes to obscuring substantial differences.  When two
partial feature structures carry compatible information, they are said
to be unifiable. Their unification returns a unique third feature
structure in the given algebra that carries the more specific
information that is obtained when combining the previous two pieces of
information (supposing they were not the same to begin with). These ideas and the
properties of algebras employed by feature logics of partial information are
still essential for all current HPSG implementation platforms (see \crossrefchaptert{cl}), which
is presumably one of the reasons why the terminology of unification
and unification-based grammars is still popular in the HPSG
community. Returning to \cite{PollardSag1994}, in a certain
informal and casual sense, combining two non-contradictory descriptions
into one single bigger description by logical conjunction could be
called -- and often is called -- their unification. However, since the
logical descriptions of HPSG in the tradition of \cite{PollardSag1994}
can no longer be arranged in an appropriate algebra, there is no
technical interpretation of the term in this context.\footnote{This
  state of affairs is also responsible for the fact that
  implementation platforms often provide only a restricted syntax of
  descriptions and may also supply additional syntactic constructs which
  extend their logic of partial information toward the expressiveness of
  a feature logic with classical interpretation of negation and
  relational expressions.}

HPSG employs partial descriptions in all areas of grammar, comprising
at least syntax, semantics \crossrefchapterp{semantics}, pragmatics \crossrefchapterp{pragmatics} and morphology \crossrefchapterp{morphology}. The
descriptions are normally notated as AVMs and contain sort symbols (by
convention in italics with lower case letters) and attribute symbols
(in small caps). These are augmented by the standard
logical connectives (conjunction, disjunction, negation and
implication) and relation symbols. So-called tags, boxed
numbers, function as variables. (\ref{ex-noun}) shows a typical example
in which \type{word}, \type{noun} and \type{plural} are sorts and
\feat{synsem}, \feat{local}, \feat{category}, etc.\ are attributes.\footnote{Tags, relations and logical connectives in descriptions will be illustrated later, in (\ref{ex-more-avms}).}
The AVM is a description of plural nouns.

% \begin{exe}
%   \ex\label{ex-noun}
%   \begin{avm}
%     \[\asort{word}
%     synsem$|$local & \[category & \hspace{-1.9cm}\[head & noun\]\\
%                      content$|$index$|$number & plural\]
%     \]
%   \end{avm}
% \end{exe}
% todo avm: type should be unnecessary
\ea
\label{ex-noun}
\avmtmp{
[\type*{word}
 synsem|local & [category [ head & noun ]\\
                 content|index|number & plural ] ]
}
\z

A description such as (\ref{ex-noun}) presupposes a declaration of the
admissible nonlogical symbols: as in any formal logical theory, the
vocabulary of the formal language in which the logical theory is
written must be explicitly introduced as the \emph{alphabet} of the
language, together with a set of logical symbols. This means that the
sorts, attributes and relation symbols must be listed. HPSG
goes beyond merely stating the nonlogical vocabulary as sets of
symbols by imposing additional structure on the set of sorts and on
the relationship between sorts and attributes. This additional
structure is known as the \emph{sort hierarchy} and the \emph{feature}
  (\emph{appropriateness}) \emph{declarations}.

The sort hierarchy and the feature declarations essentially provide the
space of possible structures of the linguistic universe that an HPSG
grammar talks about with its grammar principles. Metaphorically
speaking, they generate a space of possible structures which is then
constrained to the actual, well-formed structures which a
linguist deems the grammatical structures of a language. The
interaction between sort hierarchy and feature declarations is
regulated by assumptions about feature inheritance and feature value
inheritance. This can best be explained with a small example, using
the tiny (and slightly modified) fragment from the sort hierarchy and
feature appropriateness of \cite{PollardSag1994} shown in
Figure~\ref{ex-hier-decl}.

\begin{figure}
% \begin{forest}
% [{\footnotesize\textit{object}}
% [{{\avmoptions{center}\begin{avm}\[\type*{substantive}
% 						prd & boolean\]}}}
%   [{{\avmoptions{center}\begin{avm}\[\type*{verb} vform & vform\\
%                                                   prd & plus\]}}}]
%   [{{\avmoptions{center}\begin{avm}\[\type*{noun} case & case\]}}}] ]
% [{\footnotesize\textit{case}}]
% [{\footnotesize\textit{vform}}]
% [{\footnotesize\textit{boolean}}
%   [{\footnotesize\textit{plus}}]
%   [{\footnotesize\textit{minus}}] ]
% ]
% \end{forest}
% todo alignment of AVMs and type north
\begin{forest} type hierarchy
  [object,
    calign=midpoint, calign children={2}{3},
    [\avmtmp{[\type*{substantive} 
           prd & boolean]}
      [\avmtmp{[\type*{verb}
             vform & vform\\
             prd & plus]}]
      [\avmtmp{[\type*{noun} 
             case & case]}]
    ]
    [case]
    [vform]
    [boolean
      [plus]
      [minus]
    ]
  ]
\end{forest}
\caption{\label{ex-hier-decl}Example of sort hierarchy with feature declarations}
\end{figure}

%\begin{exe}
% \ex\label{ex-subst}
%   substantive hierarchy
% \ex\label{ex-bool}
%   boolean hierarchy
% \ex\label{ex-sub-decl}
%   \type{substantive}:
% \ex\label{ex-decl-verb}
%   \type{verb}:
% \ex\label{ex-decl-noun}
%   \type{noun}:
%\end{exe}

According to Figure~\ref{ex-hier-decl}, a top sort \type{object} is
the highest sort with immediate subsorts \type{substantive}, \type{case},
\type{vform} and \type{boolean}. The two sorts \type{substantive} and
\type{boolean} have their own immediate subsorts: \type{verb} and \type{noun},
and \type{plus} and \type{minus}, respectively. All are subsorts of
\type{object}. The six sorts \type{verb}, \type{noun}, \type{case},
\type{vform}, \type{plus} and \type{minus} are \emph{maximally specific} in
this hierarchy, because they do not have proper subsorts. Such sorts are
called \emph{species}. The four sorts \type{case}, \type{vform},
\type{plus} and \type{minus} are also called \emph{atomic}, because they are species
and they do not have attributes appropriate to them.

Figure~\ref{ex-hier-decl} contains nontrivial feature declarations for the sorts
\type{substantive}, \type{verb} and \type{noun}, and it also illustrates
the idea behind feature inheritance. First of all, \type{verb} and
\type{noun} have attributes which are only appropriate to them but to
no other sort: \feat{vform} is only appropriate to \type{verb},
and \feat{case} is only appropriate to \type{noun}. But there is
one more attribute appropriate to both due to feature inheritance:
the attribute \feat{prd} is declared appropriate to \type{substantive},
and appropriateness declarations are inherited by subsorts, so \feat{prd}
is also appropriate to \type{verb} and \type{noun}. The sort \emph{noun}
inherits the declaration unchanged from \type{substantive}.

Finally, we have to consider attribute values and their inheritance
mechanism. Whereas attributes are called appropriate \emph{to} a sort,
I call a sort appropriate \emph{for} an attribute at a given sort when
talking about attribute values. For example, the non-maximal sort
\type{boolean} is declared appropriate for the attribute \feat{prd}
at \type{substantive}. This value declaration is also inherited by the
subsorts, with a slight twist to it: at any subsort, the value for an
attribute can become \emph{more specific} (but not less specific) than
at its supersort(s), and this is what happens here at the subsort
\type{verb} of \type{substantive}.  At \type{verb} the value of
\feat{prd} must be one particular subsort of \type{boolean}, namely
\type{plus}.\footnote{The \type{plus} value for \feat{prd} at verbs
  is introduced here to create a useful example; it is not usually
found in grammars.}

A further crucial aspect of the sort hierarchy and the feature
declarations is their significance for the meaning of
grammars. Structures in the denotation of a grammar must fulfill all
their combined restrictions plus the constraints imposed by all
grammar principles. Every denoted object must be of a maximally
specific sort, i.e.\ Figure~\ref{ex-hier-decl} allows only objects of
the six species in the hierarchy. In addition, all attributes
declared appropriate for a species (possibly by inheritance) must be
present on objects of that species, with the values of course also
obeying the feature declarations and being maximally specific. For
example, an object of sort \type{noun} has \feat{case} and
\feat{prd} properties. The object that is the \feat{case} value
must be of sort \type{case} (because \type{case} is a species in the
present example, unlike in real grammars where \type{case} has
subsorts), and the sort of the \feat{prd} value must be either
\type{plus} or \type{minus}, one of the two species which are
maximally specific subsorts of \type{boolean}.  With these
restrictions, specifications like in Figure~\ref{ex-hier-decl} determine
the ontology of possible structures in the denotation of a
grammar. The possible structures are further narrowed down by the
grammar principles, leaving the well-formed structures as the
predictions of a grammar.

This is a good opportunity to reconsider underspecified descriptions.
With the sort hierarchy and feature declarations of Figure~\ref{ex-hier-decl},
there are a number of ways to underspecify the description of structures
of sort \type{noun}. All following AVMs describe the same structures
but differ in their degree of explicitness:

% \begin{exe}\label{ex-desc-n}
%   \ex
%   \begin{xlist}
%          \ex\label{ex-n-full} \begin{avm}
%         \[\type*{noun}
%         case & case\\
%         prd & plus $\vee$ minus]
%         }
%     \ex\label{ex-n-n} \begin{avm}
%         \[\type*{noun}]
%         }
%     \ex\label{ex-n-o}  \begin{avm}
%       \[\type*{noun}
%         case & object]
%     }
%     \ex\label{ex-o-o} \begin{avm}
%       \[\type*{object}
%         case & object]
%         }
%          \ex\label{ex-n-sfull} \begin{avm}
%         \[\type*{noun}
%         case & case\\
%         prd & plus]
%          }  $\vee$ \begin{avm}
%         \[\type*{noun}
%         case & case\\
%         prd & minus]
%          } 
         
%   \end{xlist}
% \end{exe}
\eal
\label{ex-desc-n}
\ex\label{ex-n-full} 
\avmtmp{
       [\type*{noun}
        case & case\\
        prd & plus $\lor$ minus]
    }
\ex\label{ex-n-n}
 \avmtmp{
        [\type*{noun}]
    }
\ex\label{ex-n-o}
  \avmtmp{
      [\type*{noun}
       case & object]
    }
\ex\label{ex-o-o} 
  \avmtmp{
      [\type*{object}
        case & object]
        }
\ex\label{ex-n-sfull} 
   \avmtmp{
   [\type*{noun}
        case & case\\
        prd & plus]
   }  $\lor$ \avmtmp{
        [\type*{noun}
         case & case\\
         prd & minus]
        } 
\zl

All AVMs in (\ref{ex-desc-n}) denote the same two configurations as
the fully specific AVM description in (\ref{ex-n-full}): two
\type{noun} structures with the \feat{case} property \type{case} and
the \feat{prd} property \type{plus} or the \feat{prd} value
\type{minus}. But a description of these structures can be
underspecified in many different ways. For \type{noun} structures in
general (\ref{ex-n-n}), the two just described are the only two
structural choices, as can be verified by inspecting
Figure~\ref{ex-hier-decl}. The description could mention in addition
to what (\ref{ex-n-n}) says that
the structures have a \feat{case} property, leaving its value
underspecified (\ref{ex-n-o}), but that does not make a difference
with respect to the shape of the structures satisfying the
description. Moreover, the only \type{object}s with \feat{case}
(\ref{ex-o-o}) are nouns, but since that leaves exactly the two possible
\feat{prd} values \type{plus} and \type{minus}, (\ref{ex-o-o}) is
yet another way to underspecify the two structures which
({\ref{ex-n-full}) describes exhaustively. Omitting the sort
  symbol \type{object} in the upper left-hand corner of (\ref{ex-o-o})
  would in fact be one more way to describe all nouns to the exclusion of
  everything else,
  because saying that something is an \type{object} does not restrict
  the range of choices. Finally, the disjunction
  embedded in the AVM in (\ref{ex-n-full}) can be lifted to
the top level of the description, yielding (\ref{ex-n-sfull}).


Grammar principles are descriptions which every structure is supposed
to obey, together with all its substructures. The Head Feature
Principle, shown in (\ref{ex-hfp}), is a frequent example. Every
phrase whose syntax is a headed phrase (\type{hd-ph}) is such that its
\feat{head} value equals the \feat{head} value of its head
daughter, indicated by the repeated occurrence of tag \idx{1} as the
value of the two \feat{head} features.  Every structure which is
described by the AVM to the left of the implication symbol (in this
case simply a sort, but see (\ref{ex-complex-ante})) must also
fulfill the requirements in the AVM to its right. If something is not
a \type{hd-ph}, it is
not restricted by the Head Feature Principle because it is not described by the
antecedent of the principle. At the same time, a structure which
is not described by \type{hd-ph} still satisfies
the Head Feature Principle as an implicational statement. For example,
the \feat{synsem} value of each phrase is usually assumed to be an
object of sort \type{synsem}, i.e.\ it is not a phrase of sort
\type{hd-ph}. As a
\type{synsem} object, it is not described by the antecedent of
(\ref{ex-hfp}), thereby still fulfilling the principle. In classical
logic, $A \rightarrow B$ is equivalent to $\neg A \vee B$, so something
that is not an $A$ satisfies $\neg A \vee B$. This is highly relevant for the
ultimate idea that a structure is only licensed by an HPSG grammar when
it is well-formed in all its components with respect to all the grammar
principles: every component of each structure that
is described by the antecedent of a grammar principle also obeys what
the consequent of the principle requires, or a given component of the
structure is licensed by not being described by the antecedent of the
given principle.

The tag \idx{1} signals the identity of the value found at the end of
the two distinct attribute paths leading to its occurrences. This state
of affairs is often referred to as \emph{token identity}. In
the Head Feature Principle, the tag notation
could be an informal notation for a \emph{path equation}, or it could mean
that \idx{1} plays the role of a variable. The description language of
Sections~\ref{sec-signatures}--\ref{sec-meaning} offers both options for
rendering such occurrences of tags in the syntax of RSRL.

\begin{exe}
  \ex\label{ex-more-avms}
  \begin{xlist}
    \ex\label{ex-hfp}
    \type{hd-ph}
%    \avmtmp{
%      [\type*{phrase}
%        dtrs & headed-struc]
%    }
    $\Rightarrow$
    \avmtmp{
      [synsem$|$local$|$category$|$head & \1\\
       head-dtr$|$synsem$|$local$|$category$|$head & \1
      ]
    }
    
    \ex\label{ex-word-principle}
    \type{word} $\Rightarrow$ $\left(\mbox{LE$_1$} \vee \mbox{LE$_2$} \vee \ldots  \vee \mbox{LE$_n$} \right)$
    
\ex\label{ex-semantics}
\type{sign} $\Rightarrow$ \avmtmp{
			[synsem$|$loc & [qstore & \@1 \\
			 		 pool & \@2]\\
					 retrieved & \@3] } \\
\hspace*{1.1cm}	$\wedge$ \texttt{set-of-elements} (\avmbox{3},\avmbox{4})
			$\wedge$ \avmbox{4} $\subseteq$ \avmbox{2}
$\wedge$ \avmbox{1} $=$ \avmbox{2} $-$ \avmbox{4}

\ex\label{ex-complex-ante}
\avmtmp{
  [retrieved & nelist]
}
$\Rightarrow$
\avmtmp{
  [synsem$|$local$|$content & psoa]
}

  \end{xlist}
\end{exe}

The licensing of words by the grammar can also be
understood as a consequence of a grammar principle with the
shape of an implication. (\ref{ex-word-principle}) is known as the
Word Principle \citep*[500]{HoehleSpurenloseExtraktion}. LE$_{1}$ to LE$_{n}$
in (\ref{ex-word-principle}) are the
\emph{lexical entries} of the grammar, descriptions of words.  If an
object is a word, it must be described by (at least) one of the
disjuncts in the consequent of the Word Principle.

The semantic principle in (\ref{ex-semantics}), taken
from \citew[420]{PY98a-u},\footnote{This principle is also discussed
in the semantics chapter, \crossrefchapterw[example (\ref{sign-cons})]{semantics}.}
illustrates one more syntactic
construct of HPSG's description language, relations. The consequent of
the principle consists of an AVM description conjoined with three
relational expressions.  Relations in HPSG often occur in
connection with lists and sets, and so do the relations here: the
binary relation \texttt{set-of-elements} relates the \feat{retrieved} value
(a list) to a set \idx{4} containing the elements on list \idx{3} such
that the set value \idx{2} of \feat{pool} is a superset of \idx{4}
(using the subset relation), and the set value \idx{1} of \feat{qstore}
contains those elements of \idx{2} which are not on the
\feat{retrieved} list (using set difference). In other words,
each element of \feat{pool} is either in \mbox{\feat{qstore}} or
a list element on \feat{retrieved}, and nothing else is in \feat{qstore}
or on the \feat{retrieved} list.\footnote{One
  additional interesting property of this principle concerns the set
  designated by tag \idx{4}. Structures described by the consequent of
  the principle do not necessarily contain an attribute with the set
  value \idx{4}. However, the list \idx{3} and the sets \idx{1} and \idx{2} are
  all attribute values which are restricted in (\ref{ex-semantics}) by reference to set
  \idx{4}. Such constellations motivate the introduction of
  \emph{chains} in the description language. Chains model lists (or
  sets) of objects that are not themselves attribute values, but whose
  members are (see Section~\ref{sec-signatures} for the syntax and
  Section~\ref{sec-meaning} for the semantics of chains). \idx{4} is best
  described as a chain.}

The grammar principle (\ref{ex-complex-ante}), which is also from
\cite[421]{PY98a-u}, is a case of a principle
with a complex description in the antecedent,
unlike (\ref{ex-hfp})--(\ref{ex-semantics}), in which the antecedent
consists of a sort symbol. Any kind of description
may serve as antecedent of a grammar principle.

An HPSG grammar is a signature consisting of a sort hierarchy, feature
appropriateness declarations and relation symbols, together with a set
of grammar principles. The meaning of the grammar is given by a class
of structures (linguistic objects) which obey the structural
restrictions of the signature and are completely well-formed with
respect to the grammar principles. The nature of the \emph{linguistic
  objects} and how the relevant models of an HPSG grammar should be
conceived of has been subject to intense
discussion. \cite[8--9]{PollardSag1994} think of them as \emph{types} and
want to construct them as a set of totally well-typed and
sort-resolved abstract feature structures. Each such type is supposed
to correspond to the set of token occurrences of the same utterance. For
example, in this view, the \ili{English} utterance \emph{Breakfast is ready},
which may occur as a concrete utterance token
at different places and at different times, always belongs to the
unique type \emph{Breakfast is ready}, rendered as an abstract feature
structure licensed by an HPSG grammar of \ili{English}.

All HPSG
model theories after \cite{PollardSag1994} give up the idea of
postulating types as objects in the intended
grammar model and do not construct models which are populated with feature
structures.\footnote{\cite[294]{Pollard99a} still uses the term \emph{feature
    structure}, but it is applied to a special kind of \emph{interpretation}
  in the sense of Definition~\ref{def-initial-int}. See the
  more detailed characterization of these structures in Section~6 below.} \citet{King99a-u} suggests \emph{exhaustive models},
collections of possible language tokens. Whereas two types are always
distinct, linguistic tokens in
exhaustive models can be isomorphic when they are different token
occurrences of the same utterance. \citet{Pollard99a} rejects the idea
that models contain possible tokens and essentially uses a variant of
King's exhaustive models for constructing sets of unique mathematical
idealizations of linguistic utterances: any well-formed utterance
finds its structurally isomorphic unique counterpart in this model,
called the \emph{strong generative capacity} of the grammar. The
relationship between the elements of the strong generative capacity
and empirical linguistic events is much tighter than it is for
Pollard and Sag's object types: for the former, it is a relationship
of structural isomorphism, for the latter it is only a conventional
notion of correspondence. Moreover, Pollard's models avoid an
ontological commitment to the reality of types. \citet{Richter2007a}
points out shortcomings with the postulated one-to-one correspondence
between linguistic types \citep{PollardSag1994} or mathematical
idealizations \citep{Pollard99a} and the groups of
linguistically indistinguishable utterances they
are supposed to represent (e.g.\ the group of realizations of
\emph{Breakfast is ready}). The failure of achieving the intended
one-to-one correspondence
is due to technical properties of the structure of the respective models
and to imprecisions of actual HPSG grammar
specifications, and the two factors are partially
independent. \citet{Richter2007a} suggests schematic amendments to
grammars (by a small set of axioms and an extended signature),
leading to \emph{normal form grammars} whose \emph{minimal exhaustive
  models} exhibit the intended one-to-one correspondence
between structural configurations in the model and (groups of linguistically
indistinguishable) empirically
observable utterance events. Despite being a certain kind of
exhaustive model, minimal exhaustive models are not token models and
do not suffer from the problematic concept of \emph{potential token} models
which is characteristic of King's approach.

HPSG as a model-theoretic grammar framework provides linguists with an
expressive class of logical description languages. Their semantics
makes it possible to investigate closely the predictions of a given
set of grammar principles and the internal and mutual consistency
of different modules of grammar. At a more foundational level, HPSG is
exceptional with its alternative characterizations of the meaning of
grammars based on one and the same set of core definitions of the
syntax and semantics of its descriptive devices. This common core in
the service of
philosophically different approaches to the scientific description of
human languages makes their respective advantages and disadvantages
comparable within one single framework, and it renders the discussion
of very abstract concepts from the philosophy of science unusually concrete.  Alternative approaches
to grammatical meaning based on different views of the nature of
scientific description of an empirical domain can be investigated and
compared with a degree of detail that is hardly achieved elsewhere in
linguistics.


%origin of HPSG in work on feature logics, especially the work culminating in Carpenter (1990?): logic of typed feature structures, logics of partial information; this work was particularly interested in inferencing and used feature structures and their properties to investigate various logics, with unification as a central notion in appropriate algebras of partial feature structures. HPSG went beyond that approach

%debate on the meaning of grammars discusses the ontological status of the structures in the denotation of an HPSG grammar. special for HPSG: different proposals were made mathematical precise with the same class of formal languages, i.e.\ the proposals can be directly compared and mappings between them can be defined - its not only an abstract philosophical discussion, they can be compared in unprecedented level of detail.

The structure of the remainder of this chapter is as follows:
Section~\ref{sec-signatures} turns to the syntax of RSRL, defines
signatures with sort hierarchies and feature appropriateness for the
non-logical vocabulary, and introduces terms and formul\ae\ as
expressions. A subclass of formul\ae\ is called
descriptions and corresponds to the informal
AVMs augmented with logical connectives and relational
expressions which we saw above in (\ref{ex-noun})--(\ref{ex-more-avms}). Section~\ref{sec-meaning} furnishes the syntactic expressions
with a semantics similar to what is familiar from classical logic,
except that formul\ae\ and descriptions denote sets of objects rather than truth
values. Section~\ref{sec-grammar-meaning} turns to the meaning of
grammars, taking King's exhaustive models as a concrete example of the
four explications outlined above, since it is technically the easiest
to define.  The final section (Section~\ref{sec-alt-gr-meaning})
outlines how the other three approaches to the meaning of HPSG grammars
differ from King's possible token models without fully defining all constructs
they involve.

The function of Sections~\ref{sec-signatures}--\ref{sec-alt-gr-meaning}
is thus to spell out in more depth what the present section summarized
in much broader strokes. Readers who do not wish to pursue HPSG's
formal foundations further
can stop here without missing anything fundamentally new.


\section{Signatures and descriptions}
\label{sec-signatures}

As logical theories of entities in a domain of objects, HPSG grammars
consist of two main components. First, a logical signature, which
provides the symbols for describing the domain of interest, in this
case a natural language. And second, an exact delineation of all and
only the legitimate entities in the denotation of the grammar, written
as a collection of statements about their configuration. These statements are
descriptions within a logical language and are composed from logical
constants, variables, quantifiers, brackets and the symbols provided
by the signature. They are variously known to linguists as principles
of grammar, constraints, or rules. In the following, I will use the
term \emph{principles} to designate these statements.
%As we will see, they must be of a certain form in order to do the job
%they are meant to perform, and
Linguists often use abbreviatory
conventions for conceptually distinguished groups of principles, such
as grammar rules, lexical entries, or lexical rules. From a logical
perspective,
then, a grammar is a pair consisting of a signature and a collection
of principles.  The appendix of \citet{PollardSag1994} provides an
early example in HPSG of this conception.
%\#\# say more?\#\#; July 2019: no, later

Signatures in HPSG go beyond supplying non-logical symbols for
descriptions, they impose additional restrictions on the
organization of the non-logical symbols. These restrictions ultimately have an effect on
how the domain of described objects is structured. Let us first
investigate the two most prominent sets of non-logical symbols: sorts
and attributes. The set of \emph{sort} symbols is arranged in a
\emph{sort hierarchy}, and that sort hierarchy is in turn connected to
the set of \emph{attribute} symbols (also known as
\emph{features}). The sort hierarchy is a partial order,\footnote{A partial order is given by a set whose elements stand in a reflexive,
antisymmetric and transitive ordering relation.} and
attributes are declared \emph{appropriate to} sorts
in the sort hierarchy. This appropriateness declaration must not be
entirely random: if an attribute is declared appropriate to some sort,
it must also be declared appropriate to all its subsorts. This requirement
is known as \emph{feature inheritance}.\footnote{See Figure~\ref{ex-hier-decl} and its explanation in Section~\ref{sec-essentials} for an example which also points out the subtle distinction between the use of the term \emph{appropriate to} (feature to sort) vs.\ the term \emph{appropriate for} (sort value for a feature at a given sort).}  Moreover, for each
sort $\sigma$ and attribute $\phi$ such that $\phi$ is appropriate to
$\sigma$, some other sort $\sigma'$ is \emph{appropriate for} $\phi$
at $\sigma$. In other words, a certain attribute value ($\sigma'$) is declared
appropriate for $\phi$ at $\sigma$. These attribute values must not be
completely random either: for any subsort of $\sigma$, the value of an
appropriate feature $\phi$ of $\sigma$ is of course also appropriate
to that subsort (by feature inheritance), but in addition, the value of $\phi$ at that subsort
must be at least as specific as it is at $\sigma$. This means the value is either
$\sigma'$ or a subsort thereof. It may not be less specific, or, to put
it differently, it may not be a
supersort of $\sigma'$.

Some sorts in the sort hierarchy enjoy a special status by being
\emph{maximally specific}. They are called \emph{species}. Species
are sorts without proper subsorts.
Sorts that are maximally specific and lack any appropriate attribute receive
a special name and are called \emph{atomic} sorts or simply \emph{atoms}.

In addition to sorts and attributes, a signature provides relation
symbols.  Well-known examples are a ternary \texttt{append} relation
and a binary \texttt{member} relation, but grammars may also require
relations such as (often ternary) \texttt{shuffle} and binary
\texttt{o-command}. Each relation symbol comes with
a positive natural number for the number of arguments, its \emph{arity}.

Putting all of this together, we obtain a definition of signatures
as a septuple with sort hierarchy $\left<S,\sqsubseteq\right>$, species
$S_{max}$, attributes $A$, and relation symbols $R$; the function $F$ handles
the feature appropriateness and function $Ar$ is for the number of arguments
of each relation.

\begin{mydef}\label{def-signature}
  $\Sigma$ is a \emph{signature} iff\\
  $\Sigma$ is a septuple $\left<S,\sqsubseteq,S_{max},A,F,R,Ar\right>$,\\
  $\left<S,\sqsubseteq\right>$ is a partial order,\\
  $S_{max} = \left\{\sigma\in S\ |\ \mbox{for each}\ \sigma' \in S, \mbox{if}\ \sigma'\sqsubseteq\sigma \mbox{then}\ \sigma=\sigma'\right\}$,\\
  $A$ is a set,\\
  $F$ is a partial function from $S\times A$ to $S$,\\
  for each $\sigma_1\in S$, for each $\sigma_2\in S$, for each $\phi\in A$,\\
  \hspace*{.5cm} if $F(\sigma_1,\phi)$ is defined and $\sigma_2\sqsubseteq\sigma_1$\\
  \hspace*{.5cm} then $F(\sigma_2,\phi)$ is defined and
             $F(\sigma_2,\phi)\sqsubseteq F(\sigma_1,\phi)$,\\
  $R$ is a finite set, and\\
  $Ar$ is a total function from $R$ to the positive integers.
\end{mydef}

The partial order $\left<S,\sqsubseteq\right>$ is the sort hierarchy,
and the set of sorts $S$, just like the set of attributes $A$, can in principle
be infinite. In actual grammars it is finite, and in HPSG grammars it is
also assumed that $S$ contains a top element, which is a sort that subsumes
all other sorts in the sort hierarchy. $S_{max}$ is the set of maximally
specific sorts, which will play a prominent role in the semantics of
descriptions. $F$ is a function for fixing the appropriateness conditions
on attributes and attribute values, and the conditions on that function
reflect HPSG's restrictions on feature declarations. $F$ is called the
\emph{(feature) appropriateness function}. The last two lines of
the definition provide the set of relation symbols, $R$, with their arity, $Ar$.
We assume that relations are at least unary.

Relations in HPSG often express relationships between
lists (\texttt{append}, \texttt{shuffle}) or sets (\texttt{union},
\texttt{intersection}). Lists are usually encoded\label{page-list-encoding}
in HPSG with attributes \feat{first} and
\feat{rest}, and sorts \type{list}, \type{elist} (for empty list) and
\type{nelist} (for non-empty list), but of course the exact naming does not matter.
A fragment of the sort hierarchy which declares the sorts and attributes
for regular lists is shown in Figure~\ref{ex-list-decl}.

\begin{figure}
\begin{forest}
type hierarchy
[list
  [\avmtmp{
    [\type*{nelist} first & object\\
     rest & list]}]
  [elist ]]
\end{forest}
\caption{\label{ex-list-decl}Fragment of a sort hierarchy for encoding lists}
\end{figure}

An AVM description of a list with two \type{synsem} objects can then be notated
as in example (\ref{list-w2-eles}).
Of course, grammar writers usually abbreviate list descriptions in AVMs by
a syntax with angled brackets for superior readability, as shown in
(\ref{list-w2-brackets}), a more transparent rendering of
(\ref{list-w2-eles}), but that is just a convention that presupposes
the existence of a sort hierarchy fragment like in Figure~\ref{ex-list-decl}.


\begin{exe}
  \ex
  \begin{xlist}
  \ex\label{list-w2-eles}
  \avmtmp{
    [\type*{nelist}
    first & synsem\\
    rest & [\type*{nelist}
             first & synsem\\
             rest & elist]
    ]
  }
  \ex\label{list-w2-brackets}
  $\left<\mbox{\avmtmp{[\type*{synsem}]}, \avmtmp{[\type*{synsem}]}}\right>$

  \end{xlist}
\end{exe}

In combination with relations, grammarians occasionally require a more
generalized use of lists (and sets) than their basic encoding above
supports. Starting already with \citet{PollardSag1994}, we find structures in
arguments of relations which behave like regular lists or sets, except
that they do not occur as attribute values anywhere in the structures
in which the relations are supposed to hold.\footnote{See
  (\ref{ex-semantics}) above for an example in the second argument of
  a binary relation \texttt{set-of-elements}.}  In order to account
for these applications of lists and sets in arguments of relations, RSRL
introduces \emph{chains}. Chains are handled with dedicated sorts and
attributes with a fixed interpretation that extend every signature.
They can be thought of as a more flexible treatment of lists alongside
their regular explicit encoding in HPSG.

RSRL adds chains to all signatures.
Informally, the extra symbols act very much like sorts and attributes for
lists: \type{chain} for \type{list}, \type{echain} and \type{nechain} for
\type{elist} and \type{nelist}, respectively, and the reserved symbols
$\dagger$ and $\triangleright$ for \feat{first} and \feat{rest}.
In order to integrate the reserved new sort symbols with any
signature a linguist might specify, a distinguished sort \textit{metatop} serves as unique
top element of the extended sort hierarchy. The extensions are defined for
any signature by adding reserved \emph{pseudo-sorts} and \emph{pseudo-attributes}
and structuring the expanded sort hierarchy in the desired way:

\begin{mydef}\label{def-sig-chains}
For each signature $\Sigma=\left<S,\sqsubseteq,S_{max},A,F,R,Ar\right>$,\\
$\Aug{S}= S \cup \left\{\mathit{chain, echain, nechain, metatop}\right\}$,\\
$\Aug\sqsubseteq=\ \sqsubseteq\cup\left\{\left<\mathit{echain},\mathit{chain}\right>,\left<\mathit{nechain},\mathit{chain}\right>\right\}\cup\left\{\left<\sigma,\sigma\right>|\sigma\in\Aug{S}\backslash S\right\}$\\
\hspace*{.65cm}       $\cup\left\{\left<\sigma,\mathit{metatop}\right>|\sigma\in\Aug{S}\right\}$,\\
$\Aug{S}_{max}=S_{max}\cup\left\{\mathit{echain},\mathit{nechain}\right\}$, and\\
$\Aug{A}=A\cup\left\{\dagger,\triangleright\right\}$.
\end{mydef}

%For simplicity, the reserved additional symbols in $\Aug{S}$ and
%$\Aug{A}$ will occasionally be referred to as \emph{pseudo-sorts} and
%\emph{pseudo-attributes}, respectively.\marginpar{delete?}
The extended sort hierarchy relation, $\Aug\sqsubseteq$, simply integrates the
new pseudo-sorts into the given relation by ordering \type{echain} and
\type{nechain} under \type{chain}, keeping the reflexive closure intact
and ordering every sort and pseudo-sort under the new top element of the
partial order, \type{metatop}. Corresponding to \type{elist} and \type{nelist}
above, \type{echain} and \type{nechain} are treated as maximally specific
by including them in the extension of $S_{max}$, designated as $\Aug{S}_{max}$.\footnote{Extending the appropriateness function, $F$, is unnecessary since the
  relevant effects follow immediately from the semantics of the new reserved symbols in
Definition~\ref{def-expanded-interpretation}.}
An AVM describing a chain with two \type{synsem} objects corresponding to the description
of a list with two \type{synsem} objects in (\ref{list-w2-eles}) now appears as
follows:

\begin{exe}
  \ex
  \avmtmp{
    [\type*{nechain}
    $\dagger$ & synsem\\
    $\triangleright$ & [\type*{nechain}
             $\dagger$ & synsem\\
             $\triangleright$ & echain]
    ]
  }
\end{exe}

Apart from the non-logical constants from (expanded) signatures and
some logical symbols, we also need a countably infinite set of
variables, which will be symbolized by $V$. We use lower-case letters
from the Latin alphabet as variable symbols, typically $x$.


For expository reasons, the syntax of descriptions, to be introduced
next, does not employ AVMs, the common
lingua franca of con\-straint-based grammar formalisms. The reasons are
twofold: most importantly, although AVMs provide an extremely readable
and flexible notation, they are quite cumbersome to define as a
rigorous logical language which meets all the expressive needs of HPSG. Some
of this awkwardness in explicit definitions derives from the very
flexibility and redundancy in notation that makes AVMs perfect for
everyday linguistic practice. Second, the original syntax of RSRL
is, by contrast, easy to define, and, as long as it is not used
for descriptions as complex as they occur in real grammars, its expressions are
still transparent for everyone who is familiar with AVMs. Readers who
want to explore how our description syntax relates to a formal syntax
of AVMs are referred to \cite{Richter2004a-u} for details and a
correspondence proof.

The definition of the syntax of descriptions proceeds in two steps,
quite similar to first-order predicate logic. We will first introduce
terms and then build formul\ae\ and descriptions from terms. Terms are essentially
what is known to linguists as \emph{paths}, sequences of attributes:

\begin{mydef}\label{def-terms}
  For each signature $\Sigma=\left<S,\sqsubseteq,S_{max},A,F,R,Ar\right>$,
  $T^{\Sigma}$ is the smallest set such that\\
  $\its\in T^{\Sigma}$,\\
  for each $x\in V$, $x\in T^{\Sigma}$,\\
  for each $\phi \in \Aug{A}$ and each $\tau\in T^{\Sigma}$, $\tau\phi\in T^{\Sigma}$.
\end{mydef}

Simply put, sequences of attributes (including the two pseudo-attributes
$\dagger$ and $\triangleright$)
starting either with the colon or a single variable are $\Sigma$ terms.
Equipped with terms, we can immediately proceed to formul\ae, the penultimate
step on the way to descriptions.
%Descriptions will be defined as formul\ae\ in which no occurrence of a
%variable is free.
There are three kinds of simple formul\ae: formul\ae\ that assign a sort
to the value of a path, formul\ae\ which state that two paths have the
same value (\emph{structure sharing}, in linguistic terminology), and
relational formul\ae. Complex formul\ae\ can be built from these by existential
and universal quantification, negation, and the classical binary logical
connectives.

\begin{mydef}
  For each signature $\Sigma=\left<S,\sqsubseteq,S_{max},A,F,R,Ar\right>$,
  $D^{\Sigma}$ is the smallest set such that\\
  for each $\sigma\in\Aug{S}$, for each $\tau\in T^{\Sigma}$,
  $\tau\sim\sigma\in D^{\Sigma}$,\\
  for each $\tau_1, \tau_2\in T^{\Sigma}$, $\tau_1 \approx \tau_2 \in D^{\Sigma}$,\\
  for each $\rho\in R$, for each $x_1, \ldots, x_{Ar(\rho)}\in V$,
  $\rho(x_1,\ldots,x_{Ar(\rho)})\in D^{\Sigma}$,\\
  for each $x\in V$, for each $\delta\in D^{\Sigma}$,
  $\exists x\delta\in D^{\Sigma}$, \hfill (analogous for $\forall$)\\
%  for each $x\in V$, for each $\delta\in D^{\Sigma}$,
%  $\forall x\delta\in D^{\Sigma}$,\\
  for each $\delta\in D^{\Sigma}$, $\neg\delta\in D^{\Sigma}$,\\
  for each $\delta_1,\delta_2\in D^{\Sigma}$, and
  $\left(\delta_1\land\delta_2\right) \in D^{\Sigma}$.
  \hfill (analogous for $\lor,\rightarrow,\leftrightarrow$)
\end{mydef}

In this syntax, the Head Feature Principle of (\ref{ex-hfp}) can be
rendered as in (\ref{ex-hfp-rsrl-a}) or, equivalently, as in
(\ref{ex-hfp-rsrl-b}).\footnote{The brackets in the antecedent are
for readability.}

\begin{exe}
  \ex
  \begin{xlist}
  \ex\label{ex-hfp-rsrl-a}
  $\left(\its\sim\mbox{\type{hd-ph}}\right)\rightarrow$\\ % \wedge \its\feat{dtrs}\sim\mbox{\type{headed-struc}}
  $\left(\its\feat{synsem local category head}\approx\right.$\\
 \hspace*{.1em} $\left.\its\feat{head-dtr synsem local category head}\right)$
  \ex\label{ex-hfp-rsrl-b}
   $\left(\its\sim\mbox{\type{hd-ph}}\right)\rightarrow$\\ % \wedge \its\feat{dtrs}\sim\mbox{\type{headed-struc}}
  $\exists x\left(\its\feat{synsem local category head}~\approx x\ \wedge\right.$\\
\hspace*{1.2em}  $\left.\its\feat{head-dtr synsem local category head}\approx x\right)$
  \end{xlist}
\end{exe}

Finally, $FV$ is a function that determines for every $\Sigma$ term and
$\Sigma$ formula the set of variables that occur free in them.

\begin{mydef}
  For each signature $\Sigma=\left<S,\sqsubseteq,S_{max},A,F,R,Ar\right>$,\\
  $FV(\its)=\left\{\right\}$,\\
  for each $x\in V$, $FV(x)=\left\{x\right\}$,\\
  for each $\tau\in T^{\Sigma}$, for each $\phi\in\Aug{A}, FV(\tau\phi)=FV(\tau)$,\\
  for each $\tau\in T^{\Sigma}$, for each $\sigma\in\Aug{S}, FV(\tau\sim\sigma)=FV(\tau)$,\\
  for each $\tau_1, \tau_2\in T^{\Sigma}$, $FV(\tau_1\approx\tau_2)=FV(\tau_1)\cup FV(\tau_2)$,\\
  for each $\rho\in R$, for each $x_1,\ldots, x_{Ar(\rho)}\in V$,\\
  \hspace*{.5cm} $FV(\rho(x_1,\ldots, x_{Ar(\rho)}))=\left\{x_1,\ldots, x_{Ar(\rho)}\right\}$,\\
  for each $\delta\in D^{\Sigma}$, for each $x\in V$,
  $FV(\exists x\delta)=FV(\delta)\backslash\hspace{-.25em}\left\{x \right\}$,\hfill(analogous for $\forall$)\\
  for each $\delta\in D^{\Sigma}$, $FV(\neg\delta)=FV(\delta)$,\\
  for each $\delta_1,\delta_2\in D^{\Sigma}$,
  $FV((\delta_1\land\delta_2))=FV(\delta_1)\cup FV(\delta_2)$.
  \hfill (analogous for $\lor,\rightarrow,\leftrightarrow$)
\end{mydef}

Informally, an occurrence of a variable is free in a $\Sigma$ term or a
$\Sigma$ formula if it is not bound by a quantifier. We single out
$\Sigma$ formul\ae\ without free occurrences of variables as a kind of formula
of special interest and reserve the term $\Sigma$ \emph{description} for them:

\begin{mydef}
  For each signature $\Sigma$,
  $D_0^{\Sigma}=\left\{\delta\in D^{\Sigma} | FV(\delta)=\left\{\right\}\right\}$.
\end{mydef}

$D_0^{\Sigma}$ is the set of $\Sigma$ descriptions.
When a signature is fixed by the context, or when the exact signature is
irrelevant in the discussion, we can simply speak of \emph{descriptions}
instead of $\Sigma$ descriptions. Descriptions are the syntactic units
that linguists use in grammar writing. (\ref{ex-hfp-rsrl-a}) and (\ref{ex-hfp-rsrl-b}) are descriptions. Grammars, as we will see in
Section~\ref{sec-grammar-meaning}, are written by declaring a signature
and stating a set of descriptions. But before we can investigate grammars and
what they mean, we
have to explain the meaning of signatures and of descriptions.


  
\section{Meaning of signatures and descriptions}
\label{sec-meaning}

Descriptions of RSRL are interpreted similarly to expressions of
classical logics such as first order logic, except that they are not
evaluated as true or false in a given structure; instead, they denote
collections of structures.

Defining the meaning of descriptions begins with delineating the
structures which interpret signatures. In particular, species and
attributes must receive a meaning, which should be tied to the
HPSG-specific intentions behind sort hierarchies and feature
declarations; and so must relation symbols, whose interpretation
should heed their arity. Due to some extra restrictions which we will
ultimately want to put on the interpretation of relation symbols (to meet
intuitions of grammarians) and whose
formulation presupposes a notion of term interpretation, we start with
\emph{initial interpretations}. They will be refined in a second
step to full interpretations (Definition~\ref{def-full-interpretation}).

Some additional notation is convenient in the upcoming definition of
initial interpretations. If $S$ is a
set, $S^{*}$ is the set of all finite sequences (or $n$-tuples) of elements
of $S$. $S^{+}$ is the same set without the empty sequence. $\overline{S}$
is short for the set $S\cup S^{*}$. Initial interpretations employ a set
$\Unive$ of entities which form the domain of grammars. The functions
$\Speci$, $\Atti$ and $\Reli$ interpret sort symbols, attribute symbols
and relation symbols in that domain, respecting certain general restrictions which come with
HPSG's ontological assumptions about languages. In particular, the behavior of
attribute interpretation is tied to the feature appropriateness conditions,
i.e.\ feature inheritance in the sort hierarchy.

\begin{mydef}\label{def-initial-int}
  For each signature $\Sigma=\left<S,\sqsubseteq,S_{max},A,F,R,Ar\right>$,
  $\Inte$ is an \emph{initial $\Sigma$ interpretation} iff\\
  $\Inte = \Interpretation$,\\
  $\Unive$ is a set,\\
  $\Speci$ is a total function from $\Unive$ to $S_{max}$,\\
  $\Atti$ is a total function from $A$ to the set of partial functions from
  $\Unive$ to $\Unive$,\\
  for each $\phi\in A$ and each $u\in\Unive$\\
  \hspace*{.5cm} if $\Atti(\phi)(u)$ is defined\\
  \hspace*{.5cm} then $F(\Speci(u),\phi)$ is defined, and
  $\Speci(\Atti(\phi)(u))\sqsubseteq F(\Speci(u),\phi)$, and\\
  for each $\phi\in A$ and each $u\in\Unive$,\\
  \hspace*{.5cm} if $F(\Speci(u),\phi)$ is defined
  then $\Atti(\phi)(u)$ is defined,\\
  $\Reli$ is a total function from $R$ to the power set of
  $\bigcup\limits_{n\in\NatNum}{\overline{\Unive}}^n$, and\\
  \hspace*{.5cm} for each $\rho\in R$,
  $\Reli(\rho)\subseteq\overline{\Unive}^{Ar(\rho)}$.
\end{mydef}

Initial $\Sigma$ interpretations are quadruples consisting of four
components. The first three of them will remain unchanged in full
$\Sigma$ interpretations (Definition~\ref{def-full-interpretation}). The elements of $\Unive$
are entities which populate the universe of structures. Their ontological
status has been debated fiercely in HPSG, and will be discussed in
Sections~\ref{sec-grammar-meaning} and~\ref{sec-alt-gr-meaning}. For the
moment, assume that they are either linguistic objects or appropriate
abstractions thereof. $\Speci$ assigns each object in the universe
a species, which is another way of saying that each object is of exactly
one maximally specific sort. This is what is known as the property of being
\emph{sort-resolved}. The attribute interpretation function $\Atti$
interprets each attribute symbol as a (partial) function that assigns an
object of the universe to an object of the universe, and as such it
obeys the restrictions of the feature declarations of the signature,
embodied in the function $F$: attributes are defined on all and only
those objects $u_1$ which have a species to which the attributes are appropriate
according to $F$; and the object which $u_1$ is mapped to by the attribute
must in turn be of a species which is appropriate for the attribute
(at the species of $u_1$). %\footnote{See Section~\ref{sec-essentials} for the terminological
%  distinction between an attribute being appropriate \emph{to} a sort and a sort
%  being appropriate \emph{for} an attribute at a given sort.}
This is what is known as the property of
interpreting structures as being \emph{totally well-typed}. Originally both
of these properties of interpreting structures were formulated with
respect to so-called \emph{feature structures}, but, as we will see below,
this conception of interpreting structures for grammars
was soon given up for philosophical reasons.\footnote{Of course, the
  informal term \emph{feature structure} is still alive among
  linguists, and in a technical sense, feature structures are essential constructs
  for implementation platforms.} The relation interpretation function
$\Reli$ finally interprets $n$-ary
relation symbols as sets of $n$-tuples of objects. However, there is
an additional option, which makes the definition look more complex: an
object in an $n$-tuple may in fact not be an atomic object; it can
alternatively be a tuple of objects itself. These tuples in
argument positions of relations will be described as \emph{chains}
with the pseudo-sorts and pseudo-attributes, which were
added to signatures in
Definition~\ref{def-sig-chains} above.  As pointed out there, chains
are a construct which gives grammarians the flexibility to
use (finite) lists in all the ways in which they are put in relations
in actual HPSG grammars (see (\ref{ex-semantics}) for an example). %\footnote{The advantage is that objects can be treated as being ordered on a chain without actually occurring on a list. See \cite{Richter2004-u} for discussion.}

Since chains are provided by an extension of the set of sort symbols
and attributes (Definition~\ref{def-sig-chains}), the interpretation of the additional symbols must be
defined separately. This is very simple, since these
symbols behave essentially analogously to the conventional sort and attribute
symbols of HPSG's list encoding.

\begin{mydef}\label{def-expanded-interpretation}
  For each signature $\Sigma=\left<S,\sqsubseteq,S_{max},A,F,R,Ar\right>$,
  for each initial $\Sigma$ interpretation $\Inte=\Interpretation$,\\
  $\Aug\Speci$ is the total function from $\overline\Unive$ to $\Aug{S}$
  such that\\
  \hspace*{.5cm} for each $u\in\Unive$, $\Aug{\Speci}\left(u\right)=\Speci\left(u\right)$,\\
  \hspace*{.5cm} for each $u_1, \ldots, u_n\in\Unive$,
            \(\Aug{\Speci}\left(\left<u_1,\ldots,u_n\right>\right)=\left\{
\begin{tabular}{ll}
$\mathit{echain}$
&
if $n = 0$,\\
$\mathit{nechain}$
&
if $n > 0$
\end{tabular}
\right.\), and\\
$\Aug{\Atti}$ is the total function from $\Aug{A}$ to the set of partial
functions from $\overline{\Unive}$ to $\overline{\Unive}$ such that
\hspace*{.5cm} for each $\phi\in A$, $\Aug{\Atti}\left(\phi\right)=\Atti\left(\phi\right)$,\\
\hspace*{.5cm} $\Aug{\Atti}\left(\dagger\right)$ is the total function from
$\Unive^{+}$ to $\Unive$ such that for each
$\left<u_0,\ldots,u_n\right>\in\Unive^{+}$,\\
\hspace*{1cm} $\Aug{\Atti}\left(\dagger\right)\left(\left<u_0,\ldots,u_n\right>\right)=u_0$, and\\
\hspace*{.5cm} $\Aug{\Atti}\left(\triangleright\right)$ is the total function from
$\Unive^{+}$ to $\Unive^{*}$ such that for each
$\left<u_0,\ldots,u_n\right>\in\Unive^{+}$,\\
\hspace*{1cm} $\Aug{\Atti}\left(\triangleright\right)\left(\left<u_0,\ldots,u_n\right>\right)=\left<u_1,\ldots,u_n\right>$.
\end{mydef}

$\Aug{\Speci}$ is the \emph{expanded species assignment function}, and
$\Aug{\Atti}$ is the  \emph{expanded attribute interpretation function}.
The pseudo-species symbols \textit{echain} and \textit{nechain} label
empty chains and non-empty chains, respectively. Given a non-empty
chain, the pseudo-attribute $\dagger$ picks out its first member,
corresponding to the function of the \textsc{first} attribute on non-empty
lists. Conversely, $\triangleright$ cuts off the first element
of a non-empty chain and returns the remainder of the chain, as does the
standard attribute \textsc{rest} for lists.

In addition to attributes, terms may also contain variables (Definition~\ref{def-terms}). Term
interpretation thus requires a notion of \emph{variable assignments} in
(initial) interpretations.

\begin{mydef}
  For each signature $\Sigma$,
  for each initial $\Sigma$ interpretation $\Inte=\Interpretation$,\\
  $\VarInt_{\Inte}=\overline{\Unive}^{\mbox{$V$}}$ is the set of variable
  assignments in $\Inte$.
\end{mydef}

An element of $\VarInt_{\Inte}$ (the set of total functions from the
set of variables to the set of objects and chains of objects of
$\Unive$) will be notated as $g$, following a convention frequently
observed in predicate logic.  With variable assignments in (initial)
interpretations, variables denote objects in the universe $\Unive$
and chains of objects of the universe.

Terms map objects of the universe
to objects (or chains of objects) of the universe as
determined by a term interpretation function $\Tinte^{g}_{\Inte}$:

\begin{mydef}
For each signature $\Sigma=\left<S,\sqsubseteq,S_{max},A,F,R,Ar\right>$,
for each initial $\Sigma$ interpretation $\Inte=\Interpretation$,
for each $g\in\VarInt_{\Inte}$,
$\Tinte^{g}_{\Inte}$ is the total function from $T^{\Sigma}$ to the
set of partial functions from $\Unive$ to $\overline{\Unive}$ such that
for each $u\in\Unive$,\\
\hspace*{.5cm} $\Tinte^{g}_{\Inte}(\its)(u)$ is defined and
$\Tinte^{g}_{\Inte}(\its)(u)=u$,\\
\hspace*{.5cm}for each $x\in V$, $\Tinte^{g}_{\Inte}(x)(u)$ is defined and
$\Tinte^{g}_{\Inte}(x)(u)=g(x)$,\\
\hspace*{.5cm}for each $\tau\in T^{\Sigma}$, for each $\phi\in\Aug{A}$,\\
\hspace*{1cm}$\Tinte^{g}_{\Inte}(\tau\phi)(u)$ is defined %\\ \hspace*{1cm}
             iff $\Tinte^{g}_{\Inte}(\tau)(u)$ is defined and 
             $\Aug{\Atti}(\phi)(\Tinte^{g}_{\Inte}(\tau)(u))$ is defined, and\\
\hspace*{1cm}if $\Tinte^{g}_{\Inte}(\tau\phi)(u)$ is defined then
             $\Tinte^{g}_{\Inte}(\tau\phi)(u)=\Aug{\Atti}(\phi)(\Tinte^{g}_{\Inte}(\tau)(u))$.
\end{mydef}
$\Tinte^{g}_{\Inte}$ is called the \emph{term interpretation function
  under $\Inte$ under $g$}. $\Sigma$ terms either start with a
variable or with the special symbol colon (`$\its$'). The colon
denotes the identity function. Interpreted on any object, it returns
that object.  If a term $\tau$ starts with the colon, its term
interpretation starts, so to speak, at the object $u$ to which it is
applied ($\Tinte^{g}_{\Inte}(\tau)(u)$) and, if each subsequent
attribute in $\tau$ is defined on the object to which the
interpretation of the earlier attribute(s) took us, the term
interpretation will yield the object reached by the last
attribute. When a $\Sigma$ term starts with a variable $x$, the given
variable assignment $g$ will determine the starting point of
interpreting the sequence of attributes ($g(x)$).  Of course,
variables may be assigned chains of objects, in which case the symbols
of the expanded attribute set can be used to navigate the elements of
the chain.

The set of objects which are reachable from a single given object in
an interpretation by following sequences of attribute interpretations
is important for the way in which quantification is conceived of by
grammarians. It also plays a role in thinking about which objects can in
principle stand in a relation, and it is crucial for explicating
different notions of the meaning of grammars. Definition~\ref{def-components-of-u}
captures this notion, which we call the set of components of an object
in an (initial) interpretation. Note that all terms in
Definition~\ref{def-components-of-u} start with the
colon.
  
\begin{mydef}\label{def-components-of-u}
For each signature $\Sigma=\left<S,\sqsubseteq,S_{max},A,F,R,Ar\right>$,
for each initial $\Sigma$ interpretation $\Inte=\Interpretation$,
for each $u\in\Unive$,\\
\hspace*{.5cm}  $\CInt_{\Inte}^u
   =\set[$u'\in\Unive$
    \set=for some $g\in\VarInt_{\Inte}$,\\
         for some \(\pi \in A^*\),\einruck
         $\Tinte^{g}_{\Inte}(\its\pi)(u)$ is defined, and\einruck
         $u'=\Tinte^{g}_{\Inte}(\its\pi)(u)$
    \set]$.

\end{mydef}

$\CInt_{\Inte}^u$ is the set of components of $u$ in $\Inte$.
The purpose of $\CInt_{\Inte}^u$ is to capture the set of all objects
that are reachable from some object $u$ in the universe by following a
path of interpreted attributes. Thinking of these configurations as
directed
graphs, the set of components of $u$ in $\Inte$ is the set of nodes
that can be reached by following any sequence of vertices
(in the direction of attribute interpretation) starting from $u$. This
corresponds to how linguists normally conceive of the substructures of
some structured object.\footnote{Phrasing this more carefully, the
  object itself is not structured, but there is a structure generated by
  the object by following the vertices, or more technically,
  by the composition of functions which interpret attribute symbols.}
The set of components of objects is used in two ways in the definitions
of full interpretations and description denotations: it restricts
the set of objects that are permitted in relations, and it provides
the domain of quantification in quantificational expressions of the
logical language.

According to Definition~\ref{def-initial-int} of initial
interpretations, relation symbols are simply interpreted as tuples of
objects (and chains of objects) in the universe of interpretation.
However, HPSGians have a slightly more restricted notion of relations:
for them, relations hold between objects that occur within a sign (or
a similar kind of larger linguistic structure); they are not relations
between objects that occur in separate (unconnected) signs. The
following notion of \emph{possible relation tuples in an
  interpretation} captures this intuition.

\begin{mydef}
For each signature $\Sigma=\left<S,\sqsubseteq,S_{max},A,F,R,Ar\right>$,
for each initial $\Sigma$ interpretation $\Inte=\Interpretation$,\\
\hspace*{.5cm}\(\ReliT_{\Inte}
=\bigcup\limits_{n\in\NatNum}
\set[$\left<u_1,\ldots,u_n\right>\in\overline{\Unive}^n$
  \set= for some $u\in\Unive$,\\
        for each $i\in\NatNum$, $1\leq i\leq n$\einruck
            $u_i\in\overline{\CInt_{\Inte}^u}$
\set]
\).
\end{mydef}

$\ReliT_{\Inte}$ is the set of possible relation tuples in $\Inte$.
Possible relation tuples in an initial interpretation are characterized
by the existence of some object in the interpretation from which each
object in a relation tuple can be reached by a sequence of attribute
interpretations. In case an argument in a tuple is a chain, then the
objects on the chain are thus restricted.

The notion of \emph{full interpretations} integrates the restriction on
possible relations, keeping everything else unchanged from initial
interpretations:

\begin{mydef}\label{def-full-interpretation}
For each signature $\Sigma=\left<S,\sqsubseteq,S_{max},A,F,R,Ar\right>$,
for each initial $\Sigma$ interpretation $\Inte'=\left<\Unive',\Speci',\Atti',
\Reli'\right>$, for the set of possible relation tuples in
$\Inte'$, $\ReliT_{\Inte'}$,
$\Inte=\Interpretation$ is a full $\Sigma$ interpretation iff\\
$\Unive=\Unive'$, $\Speci=\Speci'$, $\Atti=\Atti'$, and
$\Reli$ is a total function from $R$ to the power set of
$\ReliT_{\Inte'}$, and %\\ \hspace*{.5cm}
  for each $\rho\in R$,
  $\Reli(\rho)\subseteq\left(\ReliT_{\Inte'}\cap\overline{\Unive}^{Ar(\rho)}\right)$.
\end{mydef}


It can be checked that variable assignments in initial interpretations
and sets of components of objects in initial interpretations are the
same as in corresponding full interpretations with the same universe,
species interpretation and attribute interpretation functions, since
variable assignments and sets of components of objects do not depend on the
interpretation of relations. From now on, all of the above will be used with
respect to full interpretations, and full interpretations will simply
be called interpretations.

We are now ready to define the meaning of formul\ae\ in
interpretations as sets of objects in an interpretation. A
sort assignment formula constructed from a term, a reserved assignment
symbol and a sort symbol such as
$\its$\feat{case}$\sim$\type{nominative} denotes the set
of objects in the interpretation on which the \feat{case} attribute
is defined and, when interpreted on them, leads to an object of sort
\type{nominative}; and the path equation $\its$\feat{synsem local
  category head} $\approx\its$\feat{head-dtr synsem local category
  head} denotes those objects on which the two given paths are
defined and lead to the same object. Relational formul\ae, the
third kind of atomic formula, also denote sets of objects and will
be discussed in more detail below. Existential quantification and universal
quantification are restricted to components of objects; and the logical
connectives are treated with the familiar operations of set union
(disjunction), set intersection (conjunction) and set complement (negation),
or with combinations thereof (implication, bi-implication).
The definition of $\Sigma$ formula denotation for quantificational
expressions needs a
notation for modifying variable assignments with respect to the
value of designated variables. For any variable assignment $g\in\VarInt_{\Inte}$,
for $g'=g[x \mapsto u]$, $g'$ is just like $g$ except that
$g'$ maps variable $x$ to object $u$ (possibly a tuple).

%\begin{mydef}
%notation for modifying assignment functions
%\end{mydef}

\begin{mydef}\label{def-formula-denotation}
For each signature $\Sigma=\left<S,\sqsubseteq,S_{max},A,F,R,Ar\right>$,
for each (full) $\Sigma$ interpretation $\Inte=\Interpretation$,
for each $g\in\VarInt_{\Inte}$,
$\Dinte_{\Inte}^{g}$ is the total function from $D^{\Sigma}$ to the power
set of $\Unive$ such that\\
\hspace*{.5cm} for each $\tau\in T^{\Sigma}$, for each $\sigma\in \Aug{S}$,\\
\hspace*{1cm}$\Dinte_{\Inte}^{g}(\tau\sim\sigma)=
\set[ $u\in\Unive$
  \set= $\Tinte_{\Inte}^{g}(\tau)(u)$ is defined, and\\
  $\Aug{\Speci}\left(\Tinte_{\Inte}^{g}(\tau)(u)\right) \Aug{\sqsubseteq}\ \sigma$
\set]$,\\
\hspace*{.5cm} for each $\tau_1, \tau_2 \in T^{\Sigma}$,\\
\hspace*{1cm}$\Dinte_{\Inte}^{g}(\tau_1 \approx \tau_2)=
\set[ $u\in\Unive$
  \set= $\Tinte_{\Inte}^{g}(\tau_1)(u)$ is defined,\\
        $\Tinte_{\Inte}^{g}(\tau_2)(u)$ is defined, and\\
        $\Tinte_{\Inte}^{g}(\tau_1)(u)=\Tinte_{\Inte}^{g}(\tau_2)(u)$
  \set]$,\\
\hspace*{.5cm} for each $\rho\in R$, for each $x_1, \ldots, x_{Ar(\rho)}\in V$,\\
\hspace*{.75cm} $\Dinte_{\Inte}^{g}\left(\rho(x_1,\ldots,x_{Ar(\rho)})\right)=
\set[$u\in\Unive$
  \set= $\left<g(x_1),\ldots,g(x_{Ar(\rho)})\right>\in\Reli(\rho)$
  \set]$,\\
\hspace*{.5cm} for each $x\in V$, for each $\delta\in D^{\Sigma}$,
$\Dinte_{\Inte}^{g}\left(\exists x\delta\right)=
\set[$u\in\Unive$
  \set= for some $u'\in\overline{\CInt_{\Inte}^u}$\\
        $u\in\Dinte_{\Inte}^{g[x \mapsto u']}(\delta)$
\set]$,\\
\hspace*{.5cm} for each $x\in V$, for each $\delta\in D^{\Sigma}$,
$\Dinte_{\Inte}^{g}\left(\forall x\delta\right)=
\set[$u\in\Unive$
  \set= for each $u'\in\overline{\CInt_{\Inte}^u}$\\
        $u\in\Dinte_{\Inte}^{g[x \mapsto u']}(\delta)$
\set]$,\\
\hspace*{.5cm} for each $\delta\in D^{\Sigma}$,
$\Dinte_{\Inte}^g(\neg\delta)=\Unive\backslash\Dinte_{\Inte}^g(\delta)$,\\
\hspace*{.5cm} for each $\delta_1, \delta_2\in D^{\Sigma}$,
$\Dinte_{\Inte}^g\left((\delta_1\land\delta_2)\right)=\Dinte_{\Inte}^g(\delta_1)\cap\Dinte_{\Inte}^g(\delta_2)$\\
\hspace*{.5cm} for each $\delta_1, \delta_2\in D^{\Sigma}$,
$\Dinte_{\Inte}^g\left((\delta_1\lor\delta_2)\right)=\Dinte_{\Inte}^g(\delta_1)\cup\Dinte_{\Inte}^g(\delta_2)$\\
\hspace*{.5cm} for each $\delta_1, \delta_2\in D^{\Sigma}$,
$\Dinte_{\Inte}^g\left((\delta_1\rightarrow\delta_2)\right)=\left(\Unive\backslash\Dinte_{\Inte}^g(\delta_1)\right)\cup\Dinte_{\Inte}^g(\delta_2)$, 
and\\
\hspace*{.5cm} for each $\delta_1, \delta_2\in D^{\Sigma}$,\\
\hspace*{.75cm}$\Dinte_{\Inte}^g\left((\delta_1\leftrightarrow\delta_2)\right)=
((\Unive\backslash\Dinte_{\Inte}^g(\delta_1))\cap
(\Unive\backslash\Dinte_{\Inte}^g(\delta_2)))\cup
(\Dinte_{\Inte}^g(\delta_1)\cap\Dinte_{\Inte}^g(\delta_2))$.
\end{mydef}

$\Dinte_{\Inte}^g$ is the \emph{$\Sigma$ formula interpretation function with
  respect to $\Inte$ under a variable assignment, $g$, in $\Inte$}.
%The logical
%connectives are interpreted classically, and negation corresponds to set
%complement in the universe of objects.
Sort assignment formul\ae, $\tau\sim\sigma$, denote sets of objects on
which the attribute path $\tau$ is defined and leads to an object $u'$
of sort $\sigma$. If $\sigma$ is not a species, the object $u'$ must
be of a maximally specific subsort of $\sigma$. Path equations of the
form $\tau_1 \approx \tau_2$ hold of an object $u$ when path
$\tau_1$ and path $\tau_2$ lead to the same object $u'$. And an
$n$-ary relational formula $\rho(x_1,\ldots,x_n)$ denotes a set of
objects such that the $n$-tuples of objects (or chains of objects)
assigned to the variables $x_1$ to $x_n$ are in the denotation of the
relation $\rho$. This means that a relational formula either denotes
the entire universe $\Unive$ or the empty set, depending on the
variable assignment $g$ in $\Inte$. For example, according to
Definition~\ref{def-formula-denotation}, the formula
\texttt{append}$(x_1, x_2, x_3)$ denotes the universe of objects
if the triple $\left<g(x_1), g(x_2), g(x_3)\right>$ is in
$\Reli(\texttt{append})$, or else the empty set. We will return to the
meaning of relational formul\ae\ after defining the meaning of grammars
to confirm that this is a useful way to determine their denotation.

Negation is interpreted as set complement of the denotation of a
formula, conjunction and disjunction of formul\ae\ as set intersection
and set union of the denotations of two formul\ae, respectively. The
meaning of implication and bi-implication follows the pattern of
classical logic and could alternatively be defined on the basis of
negation and disjunction (or conjunction) alone. Quantificational
expressions are special in that they implement the idea of restricted
quantification by referring to the set of components of objects in
$\Inte$.  An existentially quantified formula, $\exists x\delta$,
denotes the set of objects $u$ such that there is at least one
component (or chain of components) $u'$ of $u$, and interpreting $x$
as $u'$ leads to $\delta$ describing $u$.  With universal
quantification, the corresponding condition must hold for \emph{all}
components (or chains of components) of the objects $u$ in the denotation
of the quantified formula. Again turning to the
application of these definitions of formula denotations in grammar
writing, the intuition is that linguists quantify over the components
of grammatical structures (sentences, phrases), and not over a
universe of objects that may include unrelated sentences and
grammatical structures, or components thereof: a certain kind of
object exists within a given structure, or all objects in a certain
structure fulfill certain conditions.


A standard proof shows that the denotation of $\Sigma$ formul\ae\
without free occurrences of variables, i.e.\ the denotation of
$\Sigma$ descriptions, is independent of the initial choice of
variable assignment. For $\Sigma$ descriptions, we can thus define a
simpler \emph{$\Sigma$ description denotation function with respect to
  an interpretation $\Inte$}, $\Dinte_{\Inte}$:

\begin{mydef}
For each signature $\Sigma=\left<S,\sqsubseteq,S_{max},A,F,R,Ar\right>$,
for each (full) $\Sigma$ interpretation $\Inte=\Interpretation$,
$\Dinte_{\Inte}$ is the total function from $D_0^{\Sigma}$ to the power
set of $\Unive$ such that
$\Dinte_{\Inte}(\delta)=
\set[$u\in\Unive$
  \set= for each $g\in\VarInt_{\Inte}$, $u\in\Dinte_{\Inte}^g$
\set]$.
\end{mydef}

For each description $\delta$, $\Dinte_{\Inte}$ returns the set of
objects in the universe of $\Inte$ that are described by
$\delta$. With $\Sigma$ descriptions and their denotation as sets of
objects, we have everything in place to symbolize all grammar
principles of a grammar such as the one presented by
\citet{PollardSag1994} in logical notation, and the grammar principles
receive an interpretation along the lines
informally characterized by Pollard and Sag. A comprehensive logical
rendering of their grammar of \ili{English} can be found in Appendix~C of
\citet{Richter2004a-u}. It includes the treatments of (finite) sets
and of parametric sorts\is{sort!parametric} (such as \type{list(synsem)}), which are not
specifically addressed -- but implicitly covered -- in
the preceding presentation. Moreover, as shown there, all syntactic
constructs of the logical languages above are necessary to achieve
that goal without reformulating the grammar.



\section{Meaning of grammars}
\label{sec-grammar-meaning}

Grammars comprise sets of descriptions, the principles of grammar. These sets of
principles are often called \emph{theories} in the context of
logical languages for HPSG, although this terminology can occasionally
be confusing.\footnote{The problem with this term is that
  it can be argued that theories, defined this way, do not constitute what would
  traditionally be called a \emph{theory of a language}, since many central
  aspects of a theory in the latter sense are not embodied in that kind
  of formalized theory.} Theories, i.e.\ sets of descriptions, are
symbolized with $\theta$.  A grammar is simply a theory together with a
signature:

\begin{mydef}\label{def-grammar}
  $\Gamma$ is a \emph{grammar} iff
  
$\Gamma$ is a pair
\( \left<\Sigma, \theta \right>\), where
$\Sigma$ is a signature, and
$\theta \subseteq D_0^{\Sigma}$.
\end{mydef}

Essentially, the denotation of a theory can be thought of as
the denotation of the conjunction of the descriptions in the theory.
The difference is
that theories can, in principle (and contrary to deliberate linguistic
convention), be infinite in the sense of containing infinitely many
descriptions. Conjunctions of descriptions are finite, since conjunctive
formul\ae\ are finite.

\begin{mydef}
For each signature $\Sigma$,
for each $\Sigma$ interpretation $\Inte=\Interpretation$,
$\Theta_{\Inte}$ is the total function from the power set of $D_0^{\Sigma}$ to the
power set of $\Unive$ such that
for each $\theta\subseteq D_0^{\Sigma}$,

$\Theta_{\Inte}(\theta)= %\bigcap
\set[
$u\in \Unive$
\set= for each $\delta \in \theta$,
      $u\in\Dinte_{\Inte}(\delta)$
\set]$.
%\{u\in U\mid$ for each $\delta\in\theta$, $u\in D_{\Inte}(\delta)\}$.
\end{mydef}

$\Theta_{\Inte}$ is the \emph{theory denotation function with respect
  to $\Inte$}. A theory consisting of a set of descriptions holds of
every object $u$ in the universe exactly if every description in the theory
holds of $u$. In short, a theory denotes the set of objects that are described
by everything in the theory. These objects do not violate any
restriction that the theory
expresses in one of its descriptions.


A first approximation to the meaning of grammars is provided by the notion
of a $\Gamma$ model, a model of a grammar $\Gamma$:

\begin{mydef}\label{def-rsrl-model}
For each grammar $\Gamma = \left< \Sigma, \theta \right>$,
for each $\Sigma$ interpretation $\Inte=$\linebreak $\Interpretation$,
$\Inte$
is a \emph{$\Gamma$ model} iff
\(\Theta_{\Inte}(\theta) =\Unive\).
\end{mydef}

A $\Gamma$ model is an interpretation $\Inte=\Interpretation$ in which
every description in the theory of grammar $\Gamma$ describes every object
in the interpretation's universe $\Unive$. In other words, each object
in the interpretation fulfills all conditions which are imposed by
the grammar principles. There is no object in a $\Gamma$ model that
violates any principle.

Models of grammars are an appropriate starting point for revisiting the
denotation of relational formul\ae. Assume we want to define a unary
relation \texttt{synsem-rel} which contains all objects of sort
\type{synsem} of a typical HPSG grammar. To achieve this, we declare the
relation symbol \texttt{synsem-rel} in the signature and we add the
description in (\ref{synsem-rel}) to the theory of the grammar $\Gamma$:

\begin{exe}
  \ex\label{synsem-rel}
  $\forall x \left(\texttt{synsem-rel}(x)\leftrightarrow x\sim\mbox{\type{synsem}}\right)$
\end{exe}

Consider a non-empty $\Gamma$ model $\Inte$ containing words and
phrases. Since by assumption (\ref{synsem-rel}) is in the theory of
$\Gamma$, and we consider a \emph{model}, (\ref{synsem-rel}) describes
every object in the universe of $\Inte$.  By the bi-implication, every
component object of every object $u$ in $\Inte$ which is of sort
\type{synsem} is in $\Reli$(\texttt{synsem-rel}) (right to left), and
every element of $\Reli$(\texttt{synsem-rel}) is a \type{synsem}
object (left to right). But if the bi-implication in
(\ref{synsem-rel}) holds in both directions in $\Inte$, it follows
that the expression $\exists x\,\texttt{synsem-rel}(x)$ describes
every object $u$ in $\Inte$ which has a component that is in the
\texttt{synsem-rel} relation. The expression $\forall
x\,\texttt{synsem-rel}(x)$ describes every object in $\Inte$ all of
whose components are in the \texttt{synsem-rel} relation.\footnote{Of
  which there are none, given the usual structure of signs where
  \type{synsem} objects always have components of other sorts.}

Now assume we have a description much like (\ref{synsem-rel})
in our grammar theory, but instead of defining the meaning of
\texttt{synsem-rel}, it defines the meaning of \texttt{append}:
the new description says that for every object in a grammar
model which contains three lists as components, the lists are in the
ternary \texttt{append} relation as triple $\left<l_1, l_2,
l_3\right>$ iff $l_3$ is the concatenation of $l_1$ and $l_2$ (in that
order). Then we can use this \texttt{append} relation in yet another
grammar principle as follows:%
\footnote{In RSRL syntax, (\ref{ex-word-order}) can be written as\\
  $\its\sim$\type{head-fill-ph} $\rightarrow$\\
  $\exists x_{1}\; \exists x_{2}\; \exists x_{3}$\\
  $\left(\its\mbox{\feat{phon}}\approx x_3\wedge\its\mbox{\feat{head-dtr phon}}\approx x_2\wedge\its\mbox{\feat{non-hd-dtrs first phon}}\approx x_1\wedge\mbox{\texttt{append}}(x_1,x_2,x_3)\right)$
}

\ea
% todo avm remove &
\label{ex-word-order}
  \type{head-fill-ph}  $\Rightarrow$ $\exists \idx{1}\; \exists \idx{2}\; \exists \idx{3}$ $\left(\mbox{
  \avmtmp{
    [phon \3\\
     non-hd-dtrs|first|phon \1 &\\
     head-dtr|phon \2 ]
  }
  } \wedge \mbox{\texttt{append}}(\idx{1},\idx{2},\idx{3})\right)$
\z

The filler daughter is the only non-head daughter in a head-filler phrase.
In \ili{English}, the phonology of the filler daughter precedes the phonology
of the head daughter. According to (\ref{ex-word-order}), a head-filler
phrase has three components, \idx{1}, \idx{2}, and \idx{3} such that
they are the list values of the \feat{phon} attributes of the
non-head daughter, the head daughter and the phrase as a whole, and they
are in the \texttt{append} relation (in the given order). But being in the
\texttt{append} relation means that list \idx{3} is the concatenation
of list \idx{1} and list \idx{2}. Obviously, the denotation of relational
formul\ae\ works as intended in grammar models.

%Applying
%the idea behind this definition of the interpretation of relational
%formul\ae\ to a typical HPSG grammar, each well formed phrase described
%by an HPSG grammar is such that \texttt{append} holds of exactly those
%triples of lists (and chains) in that phrase of which it is supposed
%to hold according to the grammarian's definition of
%\texttt{append}. This guarantees that the use of \texttt{append}
%in grammar principles has the effect it is supposed to have. How exactly
%it is achieved will become clear when we introduce the meaning of grammars.


Linguists use grammars to make predictions about the grammatical
structures of languages. In classical generative terminology, a
grammar undergenerates if there are grammatical structures it does not
capture. It overgenerates if it permits structures that are deemed
ungrammatical. It is uncontroversial that an appropriate notion of the
meaning of a grammar should support linguists in making such
predictions with their grammars. However, the notion of $\Gamma$
models in Definition~\ref{def-rsrl-model} is not strong enough for
this purpose. To see this, suppose there is a signature $\Sigma$ which
is fit to describe the entire \ili{English} language, and there is a theory
$\theta$ which expresses correctly all and only what there is to say
about \ili{English}.  Interestingly, a $\left< \Sigma, \theta \right>$ model
$\Inte$ of this perfect grammar of \ili{English} can be arbitrarily small, as
long as every object in the $\Sigma$ interpretation $\Inte$ is
described by every grammar principle in $\theta$, as this is a condition
on models of a grammar. Therefore a $\left< \Sigma, \theta \right>$
model of our perfect grammar may consist of nothing but a structure of
the single sentence \emph{Elon is going to Mars}. This follows from
the definition of $\Gamma$ models, because any appropriate grammar of
\ili{English} must describe all objects that together make up this
well-formed sentence.  But this one-sentence-model of the grammar of
\ili{English} is obviously too small to count as a good candidate for
the \ili{English} language, because \ili{English} contains much more than this
single sentence. We conclude that in arbitrarily chosen models, it
cannot be detected if a grammar undergenerates or overgenerates.


\citeauthor{King99a-u}'s (\citeyear{King99a-u}) \emph{exhaustive models} are a possibility to
define the meaning of grammars in such a way that the models reflect the
basic expectations of generative linguists. The underlying intuition
is to choose a maximal model which contains a congruent copy of any
configuration of objects which can be found in some model of the grammar.
This way, the model chosen for the meaning of a grammar is in a
relevant sense big enough so that all
the consequences of the grammar can be observed in it. If the grammar
overgenerates, the model will contain ill-formed structures. If the grammar
undergenerates, expected well-formed structures will be absent.

The simplest way to spell this out is by considering each and every
alternative model $\Inte'$ of a grammar and observing that whenever
you can describe something in an alternative model $\Inte'$ with an
arbitrary set of descriptions, that set of descriptions also picks out
something in the targeted, sufficiently large model $\Inte$:

\begin{mydef}\label{def-exhaustive-models}
For each grammar $\Gamma = \left< \Sigma, \theta \right>$,
for each $\Sigma$ interpretation $\Inte$,

$\Inte$ is an \emph{exhaustive $\Gamma$ model} iff

$\Inte$ is a $\Gamma$ model, and

for each \(\theta' \subseteq D_0^{\Sigma}\),
for each $\Sigma$ interpretation $\Inte'$,

if $\Inte'$ is a $\Gamma$ model and \(\Theta_{\Inte'}(\theta') \not= \emptyset\)
then \(\Theta_{\Inte}(\theta') \not= \emptyset\).
\end{mydef}

Any grammar with a non-empty model also has a non-empty exhaustive
mod\-el.
In addition to being a model of a given grammar $\Gamma = \left<
\Sigma, \theta \right>$, an exhaustive $\Gamma$ model $\Inte$ has the
property that each arbitrarily chosen set of descriptions $\theta'$
which denotes anything at all in any $\Gamma$ model also denotes
something in $\Inte$. An alternative algebraic way to characterize
this requirement is to say that any configuration of objects in any
$\Gamma$ model has a congruent counterpart in an exhaustive $\Gamma$
model. At the same time, since an exhaustive model is from a special class
of \emph{models}, if a description in $\theta$ does not describe some object in
a $\Gamma$ interpretation $\Inte'$, then this object in $\Inte'$ cannot
have a counterpart in an exhaustive $\Gamma$ model.

This is sufficient to capture relevant grammar-theoretic notions of
linguistics: a grammar $\Gamma$ of a language $\mathcal{L}$
overgenerates iff an exhaustive $\Gamma$ model contains configurations
that are not (congruent to) grammatical expressions in $\mathcal{L}$;
it undergenerates iff an exhaustive $\Gamma$ model does not contain
configurations which are (congruent to) grammatical expressions in
$\mathcal{L}$.

% don't discuss the problems (conceptual controversies) of this here yet
% -- postpone to the next section



\section{Alternative conceptions of the meaning of grammars}
\label{sec-alt-gr-meaning}

Section~\ref{sec-essentials} gave an informal overview of four
different ways to conceive of models which explain the meaning of HPSG
grammars: Theory T1 of \cite{PollardSag1994} views the adequate model
as a collection of the object types of the expressions of the language
$\mathcal{L}$ that a given grammar describes. T2 by \citet{King99a-u}
takes the intended model to be one from a class of models which contains
all possible linguistic tokens of $\mathcal{L}$.  T3
\citep{Pollard99a} constructs the model $\Gamma$ of language $\mathcal{L}$ as a
collection of mathematical idealizations such that each grammatical
structure of $\mathcal{L}$ should find a structurally isomorphic
counterpart in the model. This model is called the \emph{strong generative
capacity} of grammar $\Gamma$. And T4 by \citet{Richter2007a} defines a
schematic extension to grammars called their normal form which
guarantees the existence of a model (a minimal exhaustive model) in
which all and only the grammatical utterances of $\mathcal{L}$ find
exactly one structurally matching configuration each, without
commiting to the ontological status of the configurations in the
model.

All four share the common core of aiming at capturing the predictions
of a grammar in the sense of directly reflecting possible
overgeneration or undergeneration (Section~\ref{sec-grammar-meaning}):
all and only the grammatical structures of $\mathcal{L}$ are supposed
to be in the intended model or to find a corresponding counterpart in
it. The significant differences between T1, T2, T3 and T4 reside in
their assumptions about the nature of the model. The decision of what
kind of entities populate the model determines the ontological and
structural properties of the entities in the model, which in turn leads
to substantial technical differences in the construction of the
models.
The four theories T1--T4 are numbered chronologically in the order in
which they were developed.

%The present section elaborates T2 by explaining the role that the
%exhaustive models of Definition~\ref{def-exhaustive-models} play in
%it, and it elaborates the other three model theories of HPSG grammars
%by informally explaining their structure to exhaustive grammar models.

Deviating from chronological order, we begin with T2, the theory of
exhaustive models (Definition~\ref{def-exhaustive-models}). T2 has the
distinguished property of insisting on a \emph{token} model of the
language $\mathcal{L}$ of a given grammar, $\left< \Sigma,
\theta\right>$.  According to T2, actual well-formed linguistic tokens
are the immediate object of grammatical description. They are the
objects $u$ in the intended exhaustive model $\Inte =
\Interpretation$.  For any occurrence of an utterance of $\mathcal{L}$
in the real world, the intended exhaustive model contains the actual
utterance itself.  Since linguists cannot know how often an utterance
of a concrete token in $\mathcal{L}$ did occur and will occur in the
world, exhaustive models are a class of models. For T2 it does not
matter how often the token utterance {\em Elon is going to Mars} is
encountered at a concrete place and time in the world, because among
the class of exhaustive models of \ili{English} there is one with the
correct number of occurrences for this utterance and all other actual
utterances, and that exhaustive model is the intended one. However,
there is a crucial complication: it is clear that most conceivable
well-formed expressions of any given human language were never
produced and never will be. Since, by construction, an exhaustive
model must contain all potential well-formed expressions of a language
which obey the principles of grammar, in addition to actual utterance
tokens, the theory of exhaustive models must admit \emph{potential
  tokens} in the intended exhaustive model for those utterances which
never occur in the real world. If token models are already suspicious
(or unacceptable) to many linguists, models comprising non-actual
tokens are even more contentious.


T2 is designed in deliberate opposition to the chronologically
preceding theory T1 of \citet{PollardSag1994}, the only one which
employs feature structures. T1 proposes that a grammar $\Gamma=\left<
\Sigma, \theta\right>$ denotes a set of mathematical representations
of \emph{types} of linguistic events. The main idea is that the object
types abstract away from individual circumstances of token
occurrences, because for T1 a grammar of a language is assumed not to
be concerned with individual linguistic events or tokens.  The object
types capture individual linguistic token events in the sense that an
object type conventionally corresponds to ``those imaginable linguistic
objects that are actually predicted to be possible ones''
\citep[7]{PollardSag1994} in
the language $\mathcal{L}$ that $\left< \Sigma, \theta\right>$
describes.  The postulated intuitive correspondence is not explicated
further, but it is expected that a trained linguist will recognize
which object type a linguistic token encountered in the real world
corresponds to. When observing a token expression of \ili{English} in the
world, for example in a situation in which someone exclaims \emph{Elon
  is going to Mars!}, the linguist recognizes the corresponding object
type. The informality of the relationship between the denotation of a
grammar (mathematical objects serving as object types) and the domain
of empirically measurable events (utterances of grammatical expressions
of a language) is one of the reasons to reject T1.  In addition to the
weak connection between the object types and the domain of empirically
accessible data, object types have been criticized for being
ontologically dubious and in any case superfluous and thus falling victim
to Occam's razor. A theory of meaning without such an additional
ontological postulate is deemed to be stronger.


%strongly criticized by
%\citet{King99a-u}, who argues that it is far from clear how a linguist
%would recognize the correspondence and if two linguists would reliably
%agree on it. Falsification of the predictions
%of a grammar would therefore become unnecessarily hard: the proponents
%of a grammar could argue that their grammar is correct because the
%correspondence between observed utterances and the object types
%admitted by the grammar was not the one assumed by their
%detractors. An utterance supposedly not predicted by the grammar could
%be argued to correspond to an object type which another linguist did not
%think it corresponded to, and an object type
%that one linguist says corresponds to an ungrammatical token utterance
%(thus claiming that the grammar overgenerates) could be claimed
%to correspond  to a grammatical token utterance instead.

T1 is implemented by constructing linguistic object types as abstract
feature structures. In first approximation -- to be refined presently
-- these can be thought of as rooted directed graphs, or, in terms of
our previous grammar models, as configurations of objects under a root
node. Definition~\ref{def-components-of-u} introduced $\CInt_{\Inte}^u$
as the set of components of an object $u$ in an
interpretation $\Inte$.
The root node of  the directed graph corresponds to the distinguished object $u$ in a set
$\CInt_{\Inte}^u$.  The abstract feature structures
used as mathematical representations of object types, however, are not
graph-like objects, as two distinct graphs could be isomorphic, in
violation of the core idea of proposing unique object types for
classes of linguistic events.  Abstract feature structures are
therefore defined as (tuples of) sets, representing each node $\nu$ in
the graph as an equivalence class of paths that lead to $\nu$ from the
root node. A labeling function assigns sorts to these abstract
nodes in accordance with the feature appropriateness function of the
signature, and relations are basically tuples of abstract nodes. A
satisfaction function determines what it means for a feature structure
to satisfy a description, which is then elaborated in the notion of
grammars admitting sets of abstract feature structures. In terms of
the exhaustive models of T2, the abstract feature structures admitted
by a grammar $\Gamma$ can be imagined as a normal form representation with
the abstract feature structures (the linguistic types) serving as
the objects $u$ in a canonical exhaustive model $\Inte$ of $\Gamma$.\footnote{This characterization is slightly simplistic; see \cite[Appendix A, Definition 80]{Richter2004a-u} for details. Abstract feature structures are in fact extended
  to \emph{canonical entities} to obtain canonical interpretations/models/exhaustive models.} The earlier ontological criticism of T1 amounts
to rejecting the insinuation that linguists consider
(abstract) feature structures the subject of their grammars and
affirming that their real interest lies in the description of languages.
Assuming the existence of abstract feature structures is then a
superfluous detour in the linguistic enterprise.


Meaning theory T3 is positioned against the theory T1 of object types
for classes of theoretically indistinguishable linguistic tokens, and
against the theory T2 of perceiving the meaning of a grammar in an
intended exhaustive model populated with actual and non-actual
linguistic tokens.  With T3, \cite{Pollard99a} is firmly opposed to
token models and sees mathematical idealizations as fundamental to
grammatical meaning.  The concept of non-actual tokens is deemed
unacceptable and self-contradictory. However, \cite{Pollard99a} also
rejects T1's ontological commitment to object types and wants to
strengthen the relationship between the structures in the denotation
of a grammar and empirically observable token expressions.  According
to T3, no two structures in the \emph{strong generative capacity}, the
collection denoted by a grammar $\left< \Sigma, \theta\right>$ of
language $\mathcal{L}$, are
structurally isomorphic, and each utterance token of language
$\mathcal{L}$ which is judged grammatical finds a structurally
isomorphic counterpart in the grammar's strong generative capacity.
An occurrence of the question \emph{Is Elon really going to Mars?}, just
like the occurrence of any other grammatical token of \ili{English}, must
find a unique structurally isomorphic mathematical idealization in the
strong generative capacity of an
adequate grammar of \ili{English}. With this requirement, T3 tightens the
connection between observables and the mathematical model, cutting out
the types and establishing a much stricter link between the
predictions of a grammar and the domain of empirical phenomena than
the abstract feature structure models of \cite{PollardSag1994} offer
with their appeal to conventional correspondence.

T3 is spelled out on the basis of models
(Definition~\ref{def-rsrl-model}),\footnote{\cite{Pollard99a} is in
  fact based on Speciate Re-entrant Logic (SRL), King's precursor of
  RSRL, but a straightforward extension to full RSRL is provided in
  \cite{Richter2004a-u}.} offering three alternative ways of
characterizing the strong generative capacity of a grammar. The
structures in Pollard's models can be understood as pairs of
interpretations $\Inte=\Interpretation$ and a root node $u$ whose set
of components ($\CInt_{\Inte}^u$) constitute $\Inte$'s universe
$\Unive$. The objects in $\CInt_{\Inte}^u$ are all defined as
canonical representations by a construction employing equivalence
classes of attribute paths originating at the root node: given a
grammar $\Gamma$, its strong generative capacity is the set of all
such canonical representations whose interpretations are $\Gamma$
models. By construction, they are all pairwise non-isomorphic, and
with their internal (set-theoretic) structure, they can be assumed to
be structurally isomorphic to grammatical utterance tokens of a
language, in contrast to the abstract feature structures of
\cite{PollardSag1994}. The canonical representations in the strong
generative capacity can be abstracted from each exhaustive model.



A central tenet of theories T1 and T3 of the meaning of grammars as
sets of abstract feature structures and as mathematical idealizations
in the strong generative capacity is the one-to-one
correspondence either of object types or of mathematical idealizations
to (linguistically indistinguishable groups of) grammatical utterances
in a language. \citet{Richter2007a}
investigates the models of existing HPSG grammars, such as the fragment
of \ili{English} developed in \cite{PollardSag1994}, and notes that
T1 and T3 necessarily trigger an unintended one-to-many
relationship between grammatical utterances and structures in the
denotation of typical HPSG grammars: one token utterance leads to more than one
structure in the grammar denotation. The main reason
is that, in both theories, each structure which corresponds
to a grammatical utterance entails the presence of a large
number of further structures. For the strong generative capacity,
the additional structures come from the substructural nodes in the
mathematical idealization of an utterance which, by design, must
in turn function as root nodes of admissible structures. But these additional
structures are not mathematical idealizations of empirically
observable grammatical utterances. In fact, many of the structures
present in the strong generative capacity do not correspond to
structures which can occur in grammatical utterances at all.\footnote{See
  \cite[Section 4]{Richter2007a} for extensive discussion and 
  examples.} While the abstract feature structures
of T1 do not have substructures, the abstract feature structure admission
relation relies on a mechanism with exactly the same effect: admitting
the unique type of \emph{Elon must be on his way to Mars} entails the existence
of many other types, so-called reducts of the intended
type, and these reducts do not have empirical counterparts in linguistic
utterance tokens.

%\#\#\#\#
%For
%example, there are many structures under \textit{synsem} nodes in the
%denotation of grammars that cannot occur as the \textsc{synsem} value
%of signs, because the grammars impose structural restrictions on signs
%which are incompatible with the shape of these configurations under a
%\textit{synsem} object. Not only are they metempirical, they are even
%explicitly excluded from empirical events. As argued by
%\citet{Richter2007a}, this problem even extends to expressions which
%have the form of finite sentences containing extraction sites of long
%distance dependencies. These structures may contain configurations
%that are impossible in any structure which also contains a filler
%linked to the extraction site. In other words, there are no overt
%fillers to complete these idealizations in a token utterance, but
%these structures are contained
%in the grammar denotation, including their potentially observable
%phonetic string and a meaning representation, they are predicted to occur
%as observable linguistic data.
%\#\#\#\#


In response to these problems, T4 proposes \emph{normal form
  grammars}, schematic signature and theory extensions applicable to
any HPSG grammar. The core idea behind the canonical grammar extension
is to partition the denotation of grammars into utterances and to
guarantee by construction that every \emph{connected configuration of
  objects} in a grammar's denotation is isomorphic to an utterance
token in a language. For T1 and T3, this extension is insufficient to
establish the intended one-to-one correspondence between observable
utterances and object types or mathematical idealizations, because the
structures predicted by T1 and T3 still generate additional linguistic
types or mathematical idealizations corresponding to each feature
structure reduct or substructure, respectively. However, normal form
grammars allow the definition of \emph{minimal exhaustive models},
because normal form grammars can be shown to have exhaustive models
which contain non-isomorphic connected configurations of objects with
the special property that each of these configurations corresponds to
a grammatical utterance. According to T4, \emph{Elon must be on his way
  to Mars} corresponds to exactly one connected configuration in the
minimal exhaustive model of a perfect grammar of \ili{English}, and so does
any other well-formed \ili{English} utterance. Proposal T4
is not forced to make any assumptions about the ontological status of
the inhabitants of minimal exhaustive models of normal form grammars,
since they do not have to be defined as a particular kind of
mathematical structure (nor is this option excluded if it is
desired).\footnote{The techniques enlisted in the construction of
mathematical idealizations in T3 can easily be adapted to this end.} T4 shares with T3 the commitment to providing an isomorphic
structure to each grammatical utterance of a given language rather
than just a corresponding linguistic type. With King's theory T2, it
shares the avoidance of mathematical entities representing linguistic
facts.

HPSG is among a small group of grammar formalisms with a very precise
outline of their formal foundations. This high degree of precision
extends up to different but closely related ways of characterizing the
meaning of grammars. The differences are in part of a very technical nature,
but under the technical surface, they are due to different opinions of
what grammars ought to describe. It is an advantage of HPSG as a
grammar framework that all these approaches are built on the same
explicit logical foundations. As a consequence, their relationships can
be studied with the rigorous tools of mathematical logic. The
philosophical debate regarding the adequacy of each interpretation of
the nature and purpose of grammars is thus grounded in concrete
mathematical structures. Finally, independent of 
philosophical arguments and preferences, proposal T1, enlisting typed feature
structures as canonical structures in models, provides a bridge to the
literature on feature logics, connecting linguistic theory to an
interesting set of efficient computational methods, pursued in other
chapters of the present handbook \crossrefchapterp{cl}. This connection
to computation and the rich literature on feature structures is untouched
by whether these models are deemed adequate for linguistic theory.



%HPSG is among a small group of grammar formalisms with a very precise
%outline of its formal foundations. It is exceptional with its
%alternative characterizations of the meaning of grammars based on one
%and the same set of core definitions of the syntax and semantics of
%its descriptive devices. This common core of philosophically different
%approaches to the scientific description of human languages makes their
%respective advantages and disadvantages comparable within one single
%framework, and it renders the discussion of very abstract concepts
%unusually concrete.  Alternative approaches to grammatical meaning
%based on different views of the nature of scientific description of an
%empirical domain can be investigated and compared with a degree of
%detail that is hardly achieved elsewhere in linguistics.

}

%\medskip
%\#\# mention closed world semantics of feature structures?

%advantage: the logical framework introduced above is capable of making
%all alternatives equally precise and supporting the study of their
%relationships


%\section{Alternative formalisms}
%\label{sec-alt-formalisms}



%The abstract feature structures of the previous section are a degenerated
%version of feature structures as carriers of partial information in
%unification-based grammar frameworks.

%very short remarks on HPSG87 (and related)



%say somewhere, that (technically) HPSG has never been conceived of as
%a PSG (formal language, type 2)



%--------------------------------------------------------------
%\section*{Abbreviations}


%--------------------------------------------------------------
\section*{Acknowledgements}

I would like to thank Adam Przepi\'orkowski and the reviewers
Jean-Pierre Koenig, Stefan M\"uller and Manfred Sailer for their
helpful suggestions which helped improve this chapter considerably.


{\sloppy
\printbibliography[heading=subbibliography,notkeyword=this]
}
\end{document}


%------------------------------------------------------------

%\begin{table}
%\caption{Frequencies of word classes}
%\label{tab:1:frequencies}
% \begin{tabular}{lllll} 
%  \lsptoprule
%            & nouns & verbs & adjectives & adverbs\\ 
%  \midrule
%  absolute  &   12 &    34  &    23     & 13\\
%  relative  &   3.1 &   8.9 &    5.7    & 3.2\\
%  \lspbottomrule
% \end{tabular}
%\end{table}



%\ea
%\gll cogito                           ergo      sum\\  
%     think.\textsc{1sg}.\textsc{pres} therefore \textsc{cop}.\textsc{1sg}.\textsc{pres}\\ 
%\glt `I think therefore I am.'
%\z



