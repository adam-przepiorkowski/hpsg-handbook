%% -*- coding:utf-8 -*-

\documentclass[output=paper
	        ,collection
	        ,collectionchapter
 	        ,biblatex
                ,babelshorthands
                ,newtxmath
                ,draftmode
                ,colorlinks, citecolor=brown
]{langscibook}

\IfFileExists{../localcommands.tex}{%hack to check whether this is being compiled as part of a collection or standalone
  % add all extra packages you need to load to this file 

% the ISBN assigned to the digital edition
\usepackage[ISBN=9783961102556]{ean13isbn} 

\usepackage{graphicx}
\usepackage{tabularx}
\usepackage{amsmath} 

%\usepackage{tipa}      % Davis Koenig
\usepackage{xunicode} % Provide tipa macros (BC)

\usepackage{multicol}

% Berthold morphology
\usepackage{relsize}
%\usepackage{./styles/rtrees-bc} % forbidden forest 08.12.2019


\usepackage{langsci-optional} 
% used to be in this package
\providecommand{\citegen}{}
\renewcommand{\citegen}[2][]{\citeauthor{#2}'s (\citeyear*[#1]{#2})}
\providecommand{\lsptoprule}{}
\renewcommand{\lsptoprule}{\midrule\toprule}
\providecommand{\lspbottomrule}{}
\renewcommand{\lspbottomrule}{\bottomrule\midrule}
\providecommand{\largerpage}{}
\renewcommand{\largerpage}[1][1]{\enlargethispage{#1\baselineskip}}


\usepackage{langsci-lgr}

\newcommand{\MAS}{\textsc{m}\xspace} % \M is taken by somebody

%\usepackage{./styles/forest/forest}
\usepackage{langsci-forest-setup}

\usepackage{./styles/memoize/memoize} 
\memoizeset{
  memo filename prefix={chapters/hpsg-handbook.memo.dir/},
  register=\todo{O{}+m},
  prevent=\todo,
}

\usepackage{tikz-cd}

\usepackage{./styles/tikz-grid}
\usetikzlibrary{shadows}


% removed with texlive 2020 06.05.2020
% %\usepackage{pgfplots} % for data/theory figure in minimalism.tex
% % fix some issue with Mod https://tex.stackexchange.com/a/330076
% \makeatletter
% \let\pgfmathModX=\pgfmathMod@
% \usepackage{pgfplots}%
% \let\pgfmathMod@=\pgfmathModX
% \makeatother

\usepackage{subcaption}

% Stefan Müller's styles
\usepackage{./styles/merkmalstruktur,german,./styles/makros.2020,./styles/my-xspace,./styles/article-ex,
./styles/eng-date}

\selectlanguage{USenglish}

\usepackage{./styles/abbrev}


% Has to be loaded late since otherwise footnotes will not work

%%%%%%%%%%%%%%%%%%%%%%%%%%%%%%%%%%%%%%%%%%%%%%%%%%%%
%%%                                              %%%
%%%           Examples                           %%%
%%%                                              %%%
%%%%%%%%%%%%%%%%%%%%%%%%%%%%%%%%%%%%%%%%%%%%%%%%%%%%
% remove the percentage signs in the following lines
% if your book makes use of linguistic examples
\usepackage{langsci-gb4e} 


%% St. Mü.: 03.04.2020
%% these two versions of the command can be used for series of sets of examples:
%% \eal
%% \ex
%% \ex
%% \zlcont
%% \ealcont
%% \ex
%% \ex
%% \zl

\let\zlcont\z
\def\ealcont{\exnrfont\ex\begin{xlist}[iv.]\raggedright}

% original version of \z
%\def\z{\ifnum\@xnumdepth=1\end{exe}\else\end{xlist}\fi}
% \zcont just removes \end{exe}
%\def\zcont{\ifnum\@xnumdepth=1\else\end{xlist}\fi}
\def\zcont{}

% Crossing out text
% uncomment when needed
%\usepackage{ulem}

\usepackage{./styles/additional-langsci-index-shortcuts}

% this is the completely redone avm package
\usepackage{./styles/langsci-avm}
\avmsetup{columnsep=.3ex,style=narrow}

%\let\asort\type*


\usepackage{./styles/avm+}


\renewcommand{\tpv}[1]{{\avmjvalfont\itshape #1}}

% no small caps please
\renewcommand{\phonshape}[0]{\normalfont\itshape}

\regAvmFonts

\usepackage{theorem}

\newtheorem{mydefinition}{Def.}
\newtheorem{principle}{Principle}

{\theoremstyle{break}
%\newtheorem{schema}{Schema}
\newtheorem{mydefinition-break}[mydefinition]{Def.}
\newtheorem{principle-break}[principle]{Principle}
}


%% \newcommand{schema}[2]{
%% \begin{minipage}{\textwidth}
%% {\textbf{Schema~\theschema}}]\hspace{.5em}\textbf{(#1)}\\
%% #2
%% \end{minipage}}


% This avoids linebreaks in the Schema
\newcounter{schemacounter}
\makeatletter
\newenvironment{schema}[1][]
  {%
   \refstepcounter{schemacounter}%
   \par\bigskip\noindent
   \minipage{\linewidth}%
   \textbf{Schema~\theschemacounter\hspace{.5em} \ifx&#1&\else(#1)\fi}\par
  }{\endminipage\par\bigskip\@endparenv}%
\makeatother

%\usepackage{subfig}





% Davis Koenig Lexikon

\usepackage{tikz-qtree,tikz-qtree-compat} % Davis Koenig remove

\usepackage{shadow}



\usepackage[english]{isodate} % Andy Lücking
\usepackage[autostyle]{csquotes} % Andy
%\usepackage[autolanguage]{numprint}

%\defaultfontfeatures{
%    Path = /usr/local/texlive/2017/texmf-dist/fonts/opentype/public/fontawesome/ }

%% https://tex.stackexchange.com/a/316948/18561
%\defaultfontfeatures{Extension = .otf}% adds .otf to end of path when font loaded without ext parameter e.g. \newfontfamily{\FA}{FontAwesome} > \newfontfamily{\FA}{FontAwesome.otf}
%\usepackage{fontawesome} % Andy Lücking
\usepackage{pifont} % Andy Lücking -> hand

\usetikzlibrary{decorations.pathreplacing} % Andy Lücking
\usetikzlibrary{matrix} % Andy 
\usetikzlibrary{positioning} % Andy
\usepackage{tikz-3dplot} % Andy

% pragmatics
\usepackage{eqparbox} % Andy
\usepackage{enumitem} % Andy
\usepackage{longtable} % Andy
\usepackage{tabu} % Andy              needs to be loaded before hyperref as of texlive 2020

% tabu-fix
% to make "spread 0pt" work
% -----------------------------
\RequirePackage{etoolbox}
\makeatletter
\patchcmd
	\tabu@startpboxmeasure
	{\bgroup\begin{varwidth}}%
	{\bgroup
	 \iftabu@spread\color@begingroup\fi\begin{varwidth}}%
	{}{}
\def\@tabarray{\m@th\def\tabu@currentgrouptype
    {\currentgrouptype}\@ifnextchar[\@array{\@array[c]}}
%
%%% \pdfelapsedtime bug 2019-12-15
\patchcmd
	\tabu@message@etime
	{\the\pdfelapsedtime}%
	{\pdfelapsedtime}%
	{}{}
%
%
\makeatother
% -----------------------------


% Manfred's packages

%\usepackage{shadow}

\usepackage{tabularx}
\newcolumntype{L}[1]{>{\raggedright\arraybackslash}p{#1}} % linksbündig mit Breitenangabe


% Jong-Bok

%\usepackage{xytree}

\newcommand{\xytree}[2][dummy]{Let's do the tree!}

% seems evil, get rid of it
% defines \ex is incompatible with gb4e
%\usepackage{lingmacros}

% taken from lingmacros:
\makeatletter
% \evnup is used to line up the enumsentence number and an entry along
% the top.  It can take an argument to improve lining up.
\def\evnup{\@ifnextchar[{\@evnup}{\@evnup[0pt]}}

\def\@evnup[#1]#2{\setbox1=\hbox{#2}%
\dimen1=\ht1 \advance\dimen1 by -.5\baselineskip%
\advance\dimen1 by -#1%
\leavevmode\lower\dimen1\box1}
\makeatother


% YK -- CG chapter

%\usepackage{xspace}
\usepackage{bm}
\usepackage{ebproof}


% Antonio Branco, remove this
\usepackage{epsfig}

% now unicode
%\usepackage{alphabeta}





\usepackage{pst-node}


% fmr: additional packages
%\usepackage{amsthm}


% Ash and Steve: LFG
\usepackage{./styles/lfg/dalrymple}

\RequirePackage{graphics}
%\RequirePackage{./styles/lfg/trees}
%% \RequirePackage{avm}
%% \avmoptions{active}
%% \avmfont{\sc}
%% \avmvalfont{\sc}
\RequirePackage{./styles/lfg/lfgmacrosash}

\usepackage{./styles/lfg/glue}

%%%%%%%%%%%%%%%%%%%%%%%%%%%%%%
%% Markup
%%%%%%%%%%%%%%%%%%%%%%%%%%%%%%
\usepackage[normalem]{ulem} % For thinks like strikethrough, using \sout

% \newcommand{\high}[1]{\textbf{#1}} % highlighted text
\newcommand{\high}[1]{\textit{#1}} % highlighted text
%\newcommand{\term}[1]{\textit{#1}\/} % technical term
\newcommand{\qterm}[1]{`{#1}'} % technical term, quotes
%\newcommand{\trns}[1]{\strut `#1'} % translation in glossed example
\newcommand{\trnss}[1]{\strut \phantom{\sqz{}} `#1'} % translation in ungrammatical glossed example
\newcommand{\ttrns}[1]{(`#1')} % an in-text translation of a word
%\newcommand{\feat}[1]{\mbox{\textsc{\MakeLowercase{#1}}}}     % feature name
%\newcommand{\val}[1]{\mbox{\textsc{\MakeLowercase{#1}}}}    % f-structure value
\newcommand{\featt}[1]{\mbox{\textsc{\MakeLowercase{#1}}}}     % feature name
\newcommand{\vall}[1]{\mbox{\textsc{\textup{\MakeLowercase{#1}}}}}    % f-structure value
\newcommand{\mg}[1]{\mbox{\textsc{\MakeLowercase{#1}}}}    % morphological gloss
%\newcommand{\word}[1]{\textit{#1}}       % mention of word
\providecommand{\kstar}[1]{{#1}\ensuremath{^*}}
\providecommand{\kplus}[1]{{#1}\ensuremath{^+}}
\newcommand{\template}[1]{@\textsc{\MakeLowercase{#1}}}
\newcommand{\templaten}[1]{\textsc{\MakeLowercase{#1}}}
\newcommand{\templatenn}[1]{\MakeUppercase{#1}}
\newcommand{\tempeq}{\ensuremath{=}}
\newcommand{\predval}[1]{\ensuremath{\langle}\textsc{#1}\ensuremath{\rangle}}
\newcommand{\predvall}[1]{{\rm `#1'}}
\newcommand{\lfgfst}[1]{\ensuremath{#1\,}}
\newcommand{\scare}[1]{`#1'} % scare quotes
\newcommand{\bracket}[1]{\ensuremath{\left\langle\mathit{#1}\right\rangle}}
\newcommand{\sectionw}[1][]{Section#1} % section word: for cap/non-cap
\newcommand{\tablew}[1][]{Table#1} % table word: for cap/non-cap
\newcommand{\lfgglue}{LFG+Glue}
\newcommand{\hpsgglue}{HPSG+Glue}
\newcommand{\gs}{GS}
%\newcommand{\func}[1]{\ensuremath{\mathbf{#1}}}
\newcommand{\func}[1]{\textbf{#1}}
\renewcommand{\glue}{Glue}
%\newcommand{\exr}[1]{(\ref{ex:#1}}
\newcommand{\exra}[1]{(\ref{ex:#1})}


%%%%%%%%%%%%%%%%%%%%%%%%%%%%%%
% Notation
%\newcommand{\xbar}[1]{$_{\mbox{\textsc{#1}$^{\raisebox{1ex}{}}$}}$}
\newcommand{\xprime}[2][]{\textup{\mbox{{#2}\ensuremath{^\prime_{\hspace*{-.0em}\mbox{\footnotesize\ensuremath{\mathit{#1}}}}}}}}
\providecommand{\xzero}[2][]{#2\ensuremath{^0_{\mbox{\footnotesize\ensuremath{\mathit{#1}}}}}}



\let\leftangle\langle
\let\rightangle\rangle

%\newcommand{\pslabel}[1]{}



  %add all your local new commands to this file


% Don't do this at home. I do not like the smaller font for captions.
% I just removed loading the caption packege in langscibook.cls
%% \captionsetup{%
%% font={%
%% stretch=1%.8%
%% ,normalsize%,small%
%% },%
%% width=.8\textwidth
%% }

\makeatletter
\def\blx@maxline{77}
\makeatother


\newcommand{\page}{}

\newcommand{\todostefan}[1]{\todo[color=orange!80]{\footnotesize #1}\xspace}
\newcommand{\todosatz}[1]{\todo[color=red!40]{\footnotesize #1}\xspace}

\newcommand{\inlinetodostefan}[1]{\todo[color=green!40,inline]{\footnotesize #1}\xspace}

\newcommand{\addpages}{\todostefan{add pages}}
\newcommand{\addglosses}{\todostefan{add glosses}}


\newcommand{\spacebr}{\hspaceThis{[}}

\newcommand{\danish}{\jambox{(\ili{Danish})}}
\newcommand{\english}{\jambox{(\ili{English})}}
\newcommand{\german}{\jambox{(\ili{German})}}
\newcommand{\yiddish}{\jambox{(\ili{Yiddish})}}
\newcommand{\welsh}{\jambox{(\ili{Welsh})}}

% Cite and cross-reference other chapters
\newcommand{\crossrefchaptert}[2][]{\citet*[#1]{chapters/#2}, Chapter~\ref{chap-#2} of this volume} 
\newcommand{\crossrefchapterp}[2][]{(\citealp*[#1]{chapters/#2}, Chapter~\ref{chap-#2} of this volume)}
\newcommand{\crossrefchapteralt}[2][]{\citealt*[#1]{chapters/#2}, Chapter~\ref{chap-#2} of this volume}
\newcommand{\crossrefchapteralp}[2][]{\citealp*[#1]{chapters/#2}, Chapter~\ref{chap-#2} of this volume}
% example of optional argument:
% \crossrefchapterp[for something, see:]{name}
% gives: (for something, see: Author 2018, Chapter~X of this volume)

\let\crossrefchapterw\crossrefchaptert



% Davis Koenig

\let\ig=\textsc
\let\tc=\textcolor

% evolution, Flickinger, Pollard, Wasow

\let\citeNP\citet

% Adam P

%\newcommand{\toappear}{Forthcoming}
\newcommand{\pg}[1]{p.\,#1}
\renewcommand{\implies}{\rightarrow}

\newcommand*{\rref}[1]{(\ref{#1})}
\newcommand*{\aref}[1]{(\ref{#1}a)}
\newcommand*{\bref}[1]{(\ref{#1}b)}
\newcommand*{\cref}[1]{(\ref{#1}c)}

\newcommand{\msadam}{.}
\newcommand{\morsyn}[1]{\textsc{#1}}

\newcommand{\nom}{\morsyn{nom}}
\newcommand{\acc}{\morsyn{acc}}
\newcommand{\dat}{\morsyn{dat}}
\newcommand{\gen}{\morsyn{gen}}
\newcommand{\ins}{\morsyn{ins}}
%\newcommand{\aploc}{\morsyn{loc}}
\newcommand{\voc}{\morsyn{voc}}
\newcommand{\ill}{\morsyn{ill}}
\renewcommand{\inf}{\morsyn{inf}}
\newcommand{\passprc}{\morsyn{passp}}

%\newcommand{\Nom}{\msadam\nom}
%\newcommand{\Acc}{\msadam\acc}
%\newcommand{\Dat}{\msadam\dat}
%\newcommand{\Gen}{\msadam\gen}
\newcommand{\Ins}{\msadam\ins}
\newcommand{\Loc}{\msadam\loc}
\newcommand{\Voc}{\msadam\voc}
\newcommand{\Ill}{\msadam\ill}
\newcommand{\PassP}{\msadam\passprc}

\newcommand{\Aux}{\textsc{aux}}

\newcommand{\princ}[1]{\textnormal{\textsc{#1}}} % for constraint names
\newcommand{\notion}[1]{\emph{#1}}
\renewcommand{\path}[1]{\textnormal{\textsc{#1}}}
\newcommand{\ftype}[1]{\textit{#1}}
\newcommand{\fftype}[1]{{\scriptsize\textit{#1}}}
\newcommand{\la}{$\langle$}
\newcommand{\ra}{$\rangle$}
%\newcommand{\argst}{\path{arg-st}}
\newcommand{\phtm}[1]{\setbox0=\hbox{#1}\hspace{\wd0}}
\newcommand{\prep}[1]{\setbox0=\hbox{#1}\hspace{-1\wd0}#1}

%%%%%%%%%%%%%%%%%%%%%%%%%%%%%%%%%%%%%%%%%%%%%%%%%%%%%%%%%%%%%%%%%%%%%%%%%%%

% FROM FS.STY:

%%%
%%% Feature structures
%%%

% \fs         To print a feature structure by itself, type for example
%             \fs{case:nom \\ person:P}
%             or (better, for true italics),
%             \fs{\it case:nom \\ \it person:P}
%
% \lfs        To print the same feature structure with the category
%             label N at the top, type:
%             \lfs{N}{\it case:nom \\ \it person:P}

%    Modified 1990 Dec 5 so that features are left aligned.
\newcommand{\fs}[1]%
{\mbox{\small%
$
\!
\left[
  \!\!
  \begin{tabular}{l}
    #1
  \end{tabular}
  \!\!
\right]
\!
$}}

%     Modified 1990 Dec 5 so that features are left aligned.
%\newcommand{\lfs}[2]
%   {
%     \mbox{$
%           \!\!
%           \begin{tabular}{c}
%           \it #1
%           \\
%           \mbox{\small%
%                 $
%                 \left[
%                 \!\!
%                 \it
%                 \begin{tabular}{l}
%                 #2
%                 \end{tabular}
%                 \!\!
%                 \right]
%                 $}
%           \end{tabular}
%           \!\!
%           $}
%   }

\newcommand{\ft}[2]{\path{#1}\hspace{1ex}\ftype{#2}}
\newcommand{\fsl}[2]{\fs{{\fftype{#1}} \\ #2}}

\newcommand{\fslt}[2]
 {\fst{
       {\fftype{#1}} \\
       #2 
     }
 }

\newcommand{\fsltt}[2]
 {\fstt{
       {\fftype{#1}} \\
       #2 
     }
 }

\newcommand{\fslttt}[2]
 {\fsttt{
       {\fftype{#1}} \\
       #2 
     }
 }


% jak \ft, \fs i \fsl tylko nieco ciasniejsze

\newcommand{\ftt}[2]
% {{\sc #1}\/{\rm #2}}
 {\textsc{#1}\/{\rm #2}}

\newcommand{\fst}[1]
  {
    \mbox{\small%
          $
          \left[
          \!\!\!
%          \sc
          \begin{tabular}{l} #1
          \end{tabular}
          \!\!\!\!\!\!\!
          \right]
          $
          }
   }

%\newcommand{\fslt}[2]
% {\fst{#2\\
%       {\scriptsize\it #1}
%      }
% }


% superciasne

\newcommand{\fstt}[1]
  {
    \mbox{\small%
          $
          \left[
          \!\!\!
%          \sc
          \begin{tabular}{l} #1
          \end{tabular}
          \!\!\!\!\!\!\!\!\!\!\!
          \right]
          $
          }
   }

%\newcommand{\fsltt}[2]
% {\fstt{#2\\
%       {\scriptsize\it #1}
%      }
% }

\newcommand{\fsttt}[1]
  {
    \mbox{\small%
          $
          \left[
          \!\!\!
%          \sc
          \begin{tabular}{l} #1
          \end{tabular}
          \!\!\!\!\!\!\!\!\!\!\!\!\!\!\!\!
          \right]
          $
          }
   }



% %add all your local new commands to this file

% \newcommand{\smiley}{:)}

% you are not supposed to mess with hardcore stuff, St.Mü. 22.08.2018
%% \renewbibmacro*{index:name}[5]{%
%%   \usebibmacro{index:entry}{#1}
%%     {\iffieldundef{usera}{}{\thefield{usera}\actualoperator}\mkbibindexname{#2}{#3}{#4}{#5}}}

% % \newcommand{\noop}[1]{}



% Rui

\newcommand{\spc}[0]{\hspace{-1pt}\underline{\hspace{6pt}}\,}
\newcommand{\spcs}[0]{\hspace{-1pt}\underline{\hspace{6pt}}\,\,}
\newcommand{\bad}[1]{\leavevmode\llap{#1}}
\newcommand{\COMMENT}[1]{}


% Rui coordination
\newcommand{\subl}[1]{$_{\scriptstyle \textsc{#1}}$}



% Andy Lücking gesture.tex
\newcommand{\Pointing}{\ding{43}}
% Giotto: "Meeting of Joachim and Anne at the Golden Gate" - 1305-10 
\definecolor{GoldenGate1}{rgb}{.13,.09,.13} % Dress of woman in black
\definecolor{GoldenGate2}{rgb}{.94,.94,.91} % Bridge
\definecolor{GoldenGate3}{rgb}{.06,.09,.22} % Blue sky
\definecolor{GoldenGate4}{rgb}{.94,.91,.87} % Dress of woman with shawl
\definecolor{GoldenGate5}{rgb}{.52,.26,.26} % Joachim's robe
\definecolor{GoldenGate6}{rgb}{.65,.35,.16} % Anne's robe
\definecolor{GoldenGate7}{rgb}{.91,.84,.42} % Joachim's halo

\makeatletter
\newcommand{\@Depth}{1} % x-dimension, to front
\newcommand{\@Height}{1} % z-dimension, up
\newcommand{\@Width}{1} % y-dimension, rightwards
%\GGS{<x-start>}{<y-start>}{<z-top>}{<z-bottom>}{<Farbe>}{<x-width>}{<y-depth>}{<opacity>}
\newcommand{\GGS}[9][]{%
\coordinate (O) at (#2-1,#3-1,#5);
\coordinate (A) at (#2-1,#3-1+#7,#5);
\coordinate (B) at (#2-1,#3-1+#7,#4);
\coordinate (C) at (#2-1,#3-1,#4);
\coordinate (D) at (#2-1+#8,#3-1,#5);
\coordinate (E) at (#2-1+#8,#3-1+#7,#5);
\coordinate (F) at (#2-1+#8,#3-1+#7,#4);
\coordinate (G) at (#2-1+#8,#3-1,#4);
\draw[draw=black, fill=#6, fill opacity=#9] (D) -- (E) -- (F) -- (G) -- cycle;% Front
\draw[draw=black, fill=#6, fill opacity=#9] (C) -- (B) -- (F) -- (G) -- cycle;% Top
\draw[draw=black, fill=#6, fill opacity=#9] (A) -- (B) -- (F) -- (E) -- cycle;% Right
}
\makeatother


% pragmatics
\newcommand{\speaking}[1]{\eqparbox{name}{\textsc{\lowercase{#1}\space}}}
\newcommand{\alname}[1]{\eqparbox{name}{\textsc{\lowercase{#1}}}}
\newcommand{\HPSGTTR}{HPSG$_{\text{TTR}}$\xspace}

\newcommand{\ttrtype}[1]{\textit{#1}}
\newcommand{\avmel}{\q<\quad\q>} %% shortcut for empty lists in AVM
\newcommand{\ttrmerge}{\ensuremath{\wedge_{\textit{merge}}}}
\newcommand{\Cat}[2][0.1pt]{%
  \begin{scope}[y=#1,x=#1,yscale=-1, inner sep=0pt, outer sep=0pt]
   \path[fill=#2,line join=miter,line cap=butt,even odd rule,line width=0.8pt]
  (151.3490,307.2045) -- (264.3490,307.2045) .. controls (264.3490,291.1410) and (263.2021,287.9545) .. (236.5990,287.9545) .. controls (240.8490,275.2045) and (258.1242,244.3581) .. (267.7240,244.3581) .. controls (276.2171,244.3581) and (286.3490,244.8259) .. (286.3490,264.2045) .. controls (286.3490,286.2045) and (323.3717,321.6755) .. (332.3490,307.2045) .. controls (345.7277,285.6390) and (309.3490,292.2151) .. (309.3490,240.2046) .. controls (309.3490,169.0514) and (350.8742,179.1807) .. (350.8742,139.2046) .. controls (350.8742,119.2045) and (345.3490,116.5037) .. (345.3490,102.2045) .. controls (345.3490,83.3070) and (361.9972,84.4036) .. (358.7581,68.7349) .. controls (356.5206,57.9117) and (354.7696,49.2320) .. (353.4652,36.1439) .. controls (352.5396,26.8573) and (352.2445,16.9594) .. (342.5985,17.3574) .. controls (331.2650,17.8250) and (326.9655,37.7742) .. (309.3490,39.2045) .. controls (291.7685,40.6320) and (276.7783,24.2380) .. (269.9740,26.5795) .. controls (263.2271,28.9013) and (265.3490,47.2045) .. (269.3490,60.2045) .. controls (275.6359,80.6368) and (289.3490,107.2045) .. (264.3490,111.2045) .. controls (239.3490,115.2045) and (196.3490,119.2045) .. (165.3490,160.2046) .. controls (134.3490,201.2046) and (135.4934,249.3212) .. (123.3490,264.2045) .. controls (82.5907,314.1553) and (40.8239,293.6463) .. (40.8239,335.2045) .. controls (40.8239,353.8102) and (72.3490,367.2045) .. (77.3490,361.2045) .. controls (82.3490,355.2045) and (34.8638,337.3259) .. (87.9955,316.2045) .. controls (133.3871,298.1601) and   (137.4391,294.4766) .. (151.3490,307.2045) -- cycle;
\end{scope}%
}


% KdK
\newcommand{\smiley}{:)}

\renewbibmacro*{index:name}[5]{%
  \usebibmacro{index:entry}{#1}
    {\iffieldundef{usera}{}{\thefield{usera}\actualoperator}\mkbibindexname{#2}{#3}{#4}{#5}}}

% \newcommand{\noop}[1]{}

% chngcntr.sty otherwise gives error that these are already defined
%\let\counterwithin\relax
%\let\counterwithout\relax

% the space of a left bracket for glossings
\newcommand{\LB}{\hspaceThis{[}}

\newcommand{\LF}{\mbox{$[\![$}}

\newcommand{\RF}{\mbox{$]\!]_F$}}

\newcommand{\RT}{\mbox{$]\!]_T$}}





% Manfred's

\newcommand{\kommentar}[1]{}

\newcommand{\bsp}[1]{\emph{#1}}
\newcommand{\bspT}[2]{\bsp{#1} `#2'}
\newcommand{\bspTL}[3]{\bsp{#1} (lit.: #2) `#3'}

\newcommand{\noidi}{§}

\newcommand{\refer}[1]{(\ref{#1})}

%\newcommand{\avmtype}[1]{\multicolumn{2}{l}{\type{#1}}}
\newcommand{\attr}[1]{\textsc{#1}}

\newcommand{\srdefault}{\mbox{\begin{tabular}{c}{\large <}\\[-1.5ex]$\sqcap$\end{tabular}}}

%% \newcommand{\myappcolumn}[2]{
%% \begin{minipage}[t]{#1}#2\end{minipage}
%% }

%% \newcommand{\appc}[1]{\myappcolumn{3.7cm}{#1}}


% Jong-Bok


% clean that up and do not use \def (killing other stuff defined before)
%\if 0
\newcommand\DEL{\textsc{del}}
\newcommand\del{\textsc{del}}

\newcommand\conn{\textsc{conn}}
\newcommand\CONN{\textsc{conn}}
\newcommand\CONJ{\textsc{conj}}
\newcommand\LITE{\textsc{lex}}
\newcommand\lite{\textsc{lex}}
\newcommand\HON{\textsc{hon}}

%\newcommand\CAUS{\textsc{caus}}
%\newcommand\PASS{\textsc{pass}}
\newcommand\NPST{\textsc{npst}}
%\newcommand\COND{\textsc{cond}}



\newcommand\hdlite{\textsc{head-lex construction}}
\newcommand\hdlight{\textsc{head-light} Schema}
\newcommand\NFORM{\textsc{nform}}

\newcommand\RELS{\textsc{rels}}
%\newcommand\TENSE{\textsc{tense}}


%\newcommand\ARG{\textsc{arg}}
\newcommand\ARGs{\textsc{arg0}}
\newcommand\ARGa{\textsc{arg}}
\newcommand\ARGb{\textsc{arg2}}
\newcommand\TPC{\textsc{top}}
%\newcommand\PROG{\textsc{prog}}

\newcommand\LIGHT{\textsc{light}\xspace}
\newcommand\pst{\textsc{pst}}
%\newcommand\PAST{\textsc{pst}}
%\newcommand\DAT{\textsc{dat}}
%\newcommand\CONJ{\textsc{conj}}
\newcommand\nominal{\textsc{nominal}}
\newcommand\NOMINAL{\textsc{nominal}}
\newcommand\VAL{\textsc{val}}
%\newcommand\val{\textsc{val}}
\newcommand\MODE{\textsc{mode}}
\newcommand\RESTR{\textsc{restr}}
\newcommand\SIT{\textsc{sit}}
\newcommand\ARG{\textsc{arg}}
\newcommand\RELN{\textsc{rel}}
%\newcommand\REL{\textsc{rel}}
%\newcommand\RELS{\textsc{rels}}
%\newcommand\arg-st{\textsc{arg-st}}
\newcommand\xdel{\textsc{xdel}}
\newcommand\zdel{\textsc{zdel}}
\newcommand\sug{\textsc{sug}}
%\newcommand\IMP{\textsc{imp}}
%\newcommand\conn{\textsc{conn}}
%\newcommand\CONJ{\textsc{conj}}
%\newcommand\HON{\textsc{hon}}
\newcommand\BN{\textsc{bn}}
\newcommand\bn{\textsc{bn}}
\newcommand\pres{\textsc{pres}}
\newcommand\PRES{\textsc{pres}}
\newcommand\prs{\textsc{pres}}
%\newcommand\PRS{\textsc{pres}}
\newcommand\agt{\textsc{agt}}
%\newcommand\DEL{\textsc{del}}
%\newcommand\PRED{\textsc{pred}}
\newcommand\AGENT{\textsc{agent}}
\newcommand\THEME{\textsc{theme}}
%\newcommand\AUX{\textsc{aux}}
%\newcommand\THEME{\textsc{theme}}
%\newcommand\PL{\textsc{pl}}
\newcommand\SRC{\textsc{src}}
\newcommand\src{\textsc{src}}
\newcommand{\FORMjb}{\textsc{form}}
\newcommand{\formjb}{\FORM}
\newcommand\GCASE{\textsc{gcase}}
\newcommand\gcase{\textsc{gcase}}
\newcommand\SCASE{\textsc{scase}}
\newcommand\PHON{\textsc{phon}}
%\newcommand\SS{\textsc{ss}}
\newcommand\SYN{\textsc{syn}}
%\newcommand\LOC{\textsc{loc}}
\newcommand\MOD{\textsc{mod}}
\newcommand\INV{\textsc{inv}}
%\newcommand\L{\textsc{l}}
%\newcommand\CASE{\textsc{case}}
\newcommand\SPR{\textsc{spr}}
\newcommand\COMPS{\textsc{comps}}
%\newcommand\comps{\textsc{comps}}
\newcommand\SEM{\textsc{sem}}
\newcommand\CONT{\textsc{cont}}
\newcommand\SUBCAT{\textsc{subcat}}
\newcommand\CAT{\textsc{cat}}
%\newcommand\C{\textsc{c}}
%\newcommand\SUBJ{\textsc{subj}}
\newcommand\subjjb{\textsc{subj}}
%\newcommand\SLASH{\textsc{slash}}
\newcommand\LOCAL{\textsc{local}}
%\newcommand\ARG-ST{\textsc{arg-st}}
%\newcommand\AGR{\textsc{agr}}
\newcommand\PER{\textsc{per}}
%\newcommand\NUM{\textsc{num}}
%\newcommand\IND{\textsc{ind}}
\newcommand\VFORM{\textsc{vform}}
\newcommand\PFORM{\textsc{pform}}
\newcommand\decl{\textsc{decl}}
%\newcommand\loc{\textsc{loc   }}
% \newcommand\   {\textsc{  }}

%\newcommand\NEG{\textsc{neg}}
\newcommand\FRAMES{\textsc{frames}}
%\newcommand\REFL{\textsc{refl}}

\newcommand\MKG{\textsc{mkg}}

%\newcommand\BN{\textsc{bn}}
\newcommand\HD{\textsc{hd}}
\newcommand\NP{\textsc{np}}
\newcommand\PF{\textsc{pf}}
%\newcommand\PL{\textsc{pl}}
\newcommand\PP{\textsc{pp}}
%\newcommand\SS{\textsc{ss}}
\newcommand\VF{\textsc{vf}}
\newcommand\VP{\textsc{vp}}
%\newcommand\bn{\textsc{bn}}
\newcommand\cl{\textsc{cl}}
%\newcommand\pl{\textsc{pl}}
\newcommand\Wh{\ital{Wh}}
%\newcommand\ng{\textsc{neg}}
\newcommand\wh{\ital{wh}}
%\newcommand\ACC{\textsc{acc}}
%\newcommand\AGR{\textsc{agr}}
\newcommand\AGT{\textsc{agt}}
\newcommand\ARC{\textsc{arc}}
%\newcommand\ARG{\textsc{arg}}
\newcommand\ARP{\textsc{arc}}
%\newcommand\AUX{\textsc{aux}}
%\newcommand\CAT{\textsc{cat}}
%\newcommand\COP{\textsc{cop}}
%\newcommand\DAT{\textsc{dat}}
\newcommand\NEWCOMMAND{\textsc{def}}
%\newcommand\DEL{\textsc{del}}
\newcommand\DOM{\textsc{dom}}
\newcommand\DTR{\textsc{dtr}}
%\newcommand\FUT{\textsc{fut}}
\newcommand\GAP{\textsc{gap}}
%\newcommand\GEN{\textsc{gen}}
%\newcommand\HON{\textsc{hon}}
%\newcommand\IMP{\textsc{imp}}
%\newcommand\IND{\textsc{ind}}
%\newcommand\INV{\textsc{inv}}
\newcommand\LEX{\textsc{lex}}
\newcommand\Lex{\textsc{lex}}
%\newcommand\LOC{\textsc{loc}}
%\newcommand\MOD{\textsc{mod}}
\newcommand\MRK{{\nr MRK}}
%\newcommand\NEG{\textsc{neg}}
\newcommand\NEW{\textsc{new}}
%\newcommand\NOM{\textsc{nom}}
%\newcommand\NUM{\textsc{num}}
%\newcommand\PER{\textsc{per}}
%\newcommand\PST{\textsc{pst}}
\newcommand\QUE{\textsc{que}}
%\newcommand\REL{\textsc{rel}}
\newcommand\SEL{\textsc{sel}}
%\newcommand\SEM{\textsc{sem}}
%\newcommand\SIT{\textsc{arg0}}
%\newcommand\SPR{\textsc{spr}}
%\newcommand\SRC{\textsc{src}}
\newcommand\SUG{\textsc{sug}}
%\newcommand\SYN{\textsc{syn}}
%\newcommand\TPC{\textsc{top}}
%\newcommand\VAL{\textsc{val}}
%\newcommand\acc{\textsc{acc}}
%\newcommand\agt{\textsc{agt}}
\newcommand\cop{\textsc{cop}}
%\newcommand\dat{\textsc{dat}}
\newcommand\foc{\textsc{focus}}
%\newcommand\FOC{\textsc{focus}}
\newcommand\fut{\textsc{fut}}
\newcommand\hon{\textsc{hon}}
\newcommand\imp{\textsc{imp}}
\newcommand\kes{\textsc{kes}}
%\newcommand\lex{\textsc{lex}}
%\newcommand\loc{\textsc{loc}}
\newcommand\mrk{{\nr MRK}}
%\newcommand\nom{\textsc{nom}}
%\newcommand\num{\textsc{num}}
\newcommand\plu{\textsc{plu}}
\newcommand\pne{\textsc{pne}}
%\newcommand\pst{\textsc{pst}}
\newcommand\pur{\textsc{pur}}
%\newcommand\que{\textsc{que}}
%\newcommand\src{\textsc{src}}
%\newcommand\sug{\textsc{sug}}
\newcommand\tpc{\textsc{top}}
%\newcommand\utt{\textsc{utt}}
%\newcommand\val{\textsc{val}}
%% \newcommand\LITE{\textsc{lex}}
%% \newcommand\PAST{\textsc{pst}}
%% \newcommand\POSP{\textsc{pos}}
%% \newcommand\PRS{\textsc{pres}}
%% \newcommand\mod{\textsc{mod}}%
%% \newcommand\newuse{{`kes'}}
%% \newcommand\posp{\textsc{pos}}
%% \newcommand\prs{\textsc{pres}}
%% \newcommand\psp{{\it en\/}}
%% \newcommand\skes{\textsc{kes}}
%% \newcommand\CASE{\textsc{case}}
%% \newcommand\CASE{\textsc{case}}
%% \newcommand\COMP{\textsc{comp}}
%% \newcommand\CONJ{\textsc{conj}}
%% \newcommand\CONN{\textsc{conn}}
%% \newcommand\CONT{\textsc{cont}}
%% \newcommand\DECL{\textsc{decl}}
%% \newcommand\FOCUS{\textsc{focus}}
%% %\newcommand\FORM{\textsc{form}} duplicate
%% \newcommand\FREL{\textsc{frel}}
%% \newcommand\GOAL{\textsc{goal}}
\newcommand\HEAD{\textsc{head}}
%% \newcommand\INDEX{\textsc{ind}}
%% \newcommand\INST{\textsc{inst}}
%% \newcommand\MODE{\textsc{mode}}
%% \newcommand\MOOD{\textsc{mood}}
%% \newcommand\NMLZ{\textsc{nmlz}}
%% \newcommand\PHON{\textsc{phon}}
%% \newcommand\PRED{\textsc{pred}}
%% %\newcommand\PRES{\textsc{pres}}
%% \newcommand\PROM{\textsc{prom}}
%% \newcommand\RELN{\textsc{pred}}
%% \newcommand\RELS{\textsc{rels}}
%% \newcommand\STEM{\textsc{stem}}
%% \newcommand\SUBJ{\textsc{subj}}
%% \newcommand\XARG{\textsc{xarg}}
%% \newcommand\bse{{\it bse\/}}
%% \newcommand\case{\textsc{case}}
%% \newcommand\caus{\textsc{caus}}
%% \newcommand\comp{\textsc{comp}}
%% \newcommand\conj{\textsc{conj}}
%% \newcommand\conn{\textsc{conn}}
%% \newcommand\decl{\textsc{decl}}
%% \newcommand\fin{{\it fin\/}}
%% %\newcommand\form{\textsc{form}}
%% \newcommand\gend{\textsc{gend}}
%% \newcommand\inf{{\it inf\/}}
%% \newcommand\mood{\textsc{mood}}
%% \newcommand\nmlz{\textsc{nmlz}}
%% \newcommand\pass{\textsc{pass}}
%% \newcommand\past{\textsc{past}}
%% \newcommand\perf{\textsc{perf}}
%% \newcommand\pln{{\it pln\/}}
%% \newcommand\pred{\textsc{pred}}


%% %\newcommand\pres{\textsc{pres}}
%% \newcommand\proc{\textsc{proc}}
%% \newcommand\nonfin{{\it nonfin\/}}
%% \newcommand\AGENT{\textsc{agent}}
%% \newcommand\CFORM{\textsc{cform}}
%% %\newcommand\COMPS{\textsc{comps}}
%% \newcommand\COORD{\textsc{coord}}
%% \newcommand\COUNT{\textsc{count}}
%% \newcommand\EXTRA{\textsc{extra}}
%% \newcommand\GCASE{\textsc{gcase}}
%% \newcommand\GIVEN{\textsc{given}}
%% \newcommand\LOCAL{\textsc{local}}
%% \newcommand\NFORM{\textsc{nform}}
%% \newcommand\PFORM{\textsc{pform}}
%% \newcommand\SCASE{\textsc{scase}}
%% \newcommand\SLASH{\textsc{slash}}
%% \newcommand\SLASH{\textsc{slash}}
%% \newcommand\THEME{\textsc{theme}}
%% \newcommand\TOPIC{\textsc{topic}}
%% \newcommand\VFORM{\textsc{vform}}
%% \newcommand\cause{\textsc{cause}}
%% %\newcommand\comps{\textsc{comps}}
%% \newcommand\gcase{\textsc{gcase}}
%% \newcommand\itkes{{\it kes\/}}
%% \newcommand\pass{{\it pass\/}}
%% \newcommand\vform{\textsc{vform}}
%% \newcommand\CCONT{\textsc{c-cont}}
%% \newcommand\GN{\textsc{given-new}}
%% \newcommand\INFO{\textsc{info-st}}
%% \newcommand\ARG-ST{\textsc{arg-st}}
%% \newcommand\SUBCAT{\textsc{subcat}}
%% \newcommand\SYNSEM{\textsc{synsem}}
%% \newcommand\VERBAL{\textsc{verbal}}
%% \newcommand\arg-st{\textsc{arg-st}}
%% \newcommand\plain{{\it plain}\/}
%% \newcommand\propos{\textsc{propos}}
%% \newcommand\ADVERBIAL{\textsc{advl}}
%% \newcommand\HIGHLIGHT{\textsc{prom}}
%% \newcommand\NOMINAL{\textsc{nominal}}

\newenvironment{myavm}{\begingroup\avmvskip{.1ex}
  \selectfont\begin{avm}}%
{\end{avm}\endgroup\medskip}
\newcommand\pfix{\vspace{-5pt}}


\newcommand{\jbsub}[1]{\lower4pt\hbox{\small #1}}
\newcommand{\jbssub}[1]{\lower4pt\hbox{\small #1}}
\newcommand\jbtr{\underbar{\ \ \ }\ }


%\fi

% cl

\newcommand{\delphin}{\textsc{delph-in}}


% YK -- CG chapter

\newcommand{\grey}[1]{\colorbox{mycolor}{#1}}
\definecolor{mycolor}{gray}{0.8}

\newcommand{\GQU}[2]{\raisebox{1.6ex}{\ensuremath{\rotatebox{180}{\textbf{#1}}_{\scalebox{.7}{\textbf{#2}}}}}}

\newcommand{\SetInfLen}{\setpremisesend{0pt}\setpremisesspace{10pt}\setnamespace{0pt}}

\newcommand{\pt}[1]{\ensuremath{\mathsf{#1}}}
\newcommand{\ptv}[1]{\ensuremath{\textsf{\textsl{#1}}}}

\newcommand{\sv}[1]{\ensuremath{\bm{\mathcal{#1}}}}
\newcommand{\sX}{\sv{X}}
\newcommand{\sF}{\sv{F}}
\newcommand{\sG}{\sv{G}}

\newcommand{\syncat}[1]{\textrm{#1}}
\newcommand{\syncatVar}[1]{\ensuremath{\mathit{#1}}}

\newcommand{\RuleName}[1]{\textrm{#1}}

\newcommand{\SemTyp}{\textsf{Sem}}

\newcommand{\E}{\ensuremath{\bm{\epsilon}}\xspace}

\newcommand{\greeka}{\upalpha}
\newcommand{\greekb}{\upbeta}
\newcommand{\greekd}{\updelta}
\newcommand{\greekp}{\upvarphi}
\newcommand{\greekr}{\uprho}
\newcommand{\greeks}{\upsigma}
\newcommand{\greekt}{\uptau}
\newcommand{\greeko}{\upomega}
\newcommand{\greekz}{\upzeta}

\newcommand{\Lemma}{\ensuremath{\hskip.5em\vdots\hskip.5em}\noLine}
\newcommand{\LemmaAlt}{\ensuremath{\hskip.5em\vdots\hskip.5em}}

\newcommand{\I}{\iota}

\newcommand{\sem}{\ensuremath}

\newcommand{\NoSem}{%
\renewcommand{\LexEnt}[3]{##1; \syncat{##3}}
\renewcommand{\LexEntTwoLine}[3]{\renewcommand{\arraystretch}{.8}%
\begin{array}[b]{l} ##1;  \\ \syncat{##3} \end{array}}
\renewcommand{\LexEntThreeLine}[3]{\renewcommand{\arraystretch}{.8}%
\begin{array}[b]{l} ##1; \\ \syncat{##3} \end{array}}}

\newcommand{\hypml}[2]{\left[\!\!#1\!\!\right]^{#2}}

%%%%for bussproof
\def\defaultHypSeparation{\hskip0.1in}
\def\ScoreOverhang{0pt}

\newcommand{\MultiLine}[1]{\renewcommand{\arraystretch}{.8}%
\ensuremath{\begin{array}[b]{l} #1 \end{array}}}

\newcommand{\MultiLineMod}[1]{%
\ensuremath{\begin{array}[t]{l} #1 \end{array}}}

\newcommand{\hypothesis}[2]{[ #1 ]^{#2}}

\newcommand{\LexEnt}[3]{#1; \ensuremath{#2}; \syncat{#3}}

\newcommand{\LexEntTwoLine}[3]{\renewcommand{\arraystretch}{.8}%
\begin{array}[b]{l} #1; \\ \ensuremath{#2};  \syncat{#3} \end{array}}

\newcommand{\LexEntThreeLine}[3]{\renewcommand{\arraystretch}{.8}%
\begin{array}[b]{l} #1; \\ \ensuremath{#2}; \\ \syncat{#3} \end{array}}

\newcommand{\LexEntFiveLine}[5]{\renewcommand{\arraystretch}{.8}%
\begin{array}{l} #1 \\ #2; \\ \ensuremath{#3} \\ \ensuremath{#4}; \\ \syncat{#5} \end{array}}

\newcommand{\LexEntFourLine}[4]{\renewcommand{\arraystretch}{.8}%
\begin{array}{l} \pt{#1} \\ \pt{#2}; \\ \syncat{#4} \end{array}}

\newcommand{\ManySomething}{\renewcommand{\arraystretch}{.8}%
\raisebox{-3mm}{\begin{array}[b]{c} \vdots \,\,\,\,\,\, \vdots \\
\vdots \end{array}}}

\newcommand{\lemma}[1]{\renewcommand{\arraystretch}{.8}%
\begin{array}[b]{c} \vdots \\ #1 \end{array}}

\newcommand{\lemmarev}[1]{\renewcommand{\arraystretch}{.8}%
\begin{array}[b]{c} #1 \\ \vdots \end{array}}

\newcommand{\p}{\ensuremath{\upvarphi}}

% clashes with soul package
\newcommand{\yusukest}{\textbf{\textsf{st}}}

\newcommand{\shortarrow}{\xspace\hskip-1.2ex\scalebox{.5}[1]{\ensuremath{\bm{\rightarrow}}}\hskip-.5ex\xspace}

\newcommand{\SemInt}[1]{\mbox{$[\![ \textrm{#1} ]\!]$}}

\newcommand{\HypSpace}{\hskip-.8ex}
\newcommand{\RaiseHeight}{\raisebox{2.2ex}}
\newcommand{\RaiseHeightLess}{\raisebox{1ex}}

\newcommand{\ThreeColHyp}[1]{\RaiseHeight{\Bigg[}\HypSpace#1\HypSpace\RaiseHeight{\Bigg]}}
\newcommand{\TwoColHyp}[1]{\RaiseHeightLess{\Big[}\HypSpace#1\HypSpace\RaiseHeightLess{\Big]}}

\newcommand{\LemmaShort}{\ensuremath{ \ \vdots} \ \noLine}
\newcommand{\LemmaShortAlt}{\ensuremath{ \ \vdots} \ }

\newcommand{\fail}{**}
\newcommand{\vs}{\raisebox{.05em}{\ensuremath{\upharpoonright}}}
\newcommand{\DerivSize}{\small}

\def\maru#1{{\ooalign{\hfil
  \ifnum#1>999 \resizebox{.25\width}{\height}{#1}\else%
  \ifnum#1>99 \resizebox{.33\width}{\height}{#1}\else%
  \ifnum#1>9 \resizebox{.5\width}{\height}{#1}\else #1%
  \fi\fi\fi%
\/\hfil\crcr%
\raise.167ex\hbox{\mathhexbox20D}}}}

\newenvironment{samepage2}%
 {\begin{flushleft}\begin{minipage}{\linewidth}}
 {\end{minipage}\end{flushleft}}

\newcommand{\cmt}[1]{\textsl{\textbf{[#1]}}}
\newcommand{\trns}[1]{\textbf{#1}\xspace}
\newcommand{\ptfont}{}
\newcommand{\gp}{\underline{\phantom{oo}}}
\newcommand{\mgcmt}{\marginnote}

\newcommand{\term}[1]{\emph{#1}}

\newcommand{\citeposs}[1]{\citeauthor{#1}'s \citeyearpar{#1}}

% for standalone compilations Felix: This is in the class already
%\let\thetitle\@title
%\let\theauthor\@author 
\makeatletter
\newcommand{\togglepaper}[1][0]{ 
\bibliography{../Bibliographies/stmue,../localbibliography,
../Bibliographies/properties,
../Bibliographies/np,
../Bibliographies/negation,
../Bibliographies/ellipsis,
../Bibliographies/binding,
../Bibliographies/complex-predicates,
../Bibliographies/control-raising,
../Bibliographies/coordination,
../Bibliographies/morphology,
../Bibliographies/lfg,
collection.bib}
  %% hyphenation points for line breaks
%% Normally, automatic hyphenation in LaTeX is very good
%% If a word is mis-hyphenated, add it to this file
%%
%% add information to TeX file before \begin{document} with:
%% %% hyphenation points for line breaks
%% Normally, automatic hyphenation in LaTeX is very good
%% If a word is mis-hyphenated, add it to this file
%%
%% add information to TeX file before \begin{document} with:
%% \include{localhyphenation}
\hyphenation{
A-la-hver-dzhie-va
anaph-o-ra
ana-phor
ana-phors
an-te-ced-ent
an-te-ced-ents
affri-ca-te
affri-ca-tes
ap-proach-es
Atha-bas-kan
Athe-nä-um
Bona-mi
Chi-che-ŵa
com-ple-ments
con-straints
Cope-sta-ke
Da-ge-stan
Dor-drecht
er-klä-ren-de
Ginz-burg
Gro-ning-en
Jap-a-nese
Jon-a-than
Ka-tho-lie-ke
Ko-bon
krie-gen
Le-Sourd
moth-er
Mül-ler
Nie-mey-er
Par-a-digm
Prze-piór-kow-ski
phe-nom-e-non
re-nowned
Rie-he-mann
un-bound-ed
with-in
}

% listing within here does not have any effect for lfg.tex % 2020-05-14

% why has "erklärende" be listed here? I specified langid in bibtex item. Something is still not working with hyphenation.


% to do: check
%  Alahverdzhieva


% biblatex:

% This is a LaTeX frontend to TeX’s \hyphenation command which defines hy- phenation exceptions. The ⟨language⟩ must be a language name known to the babel/polyglossia packages. The ⟨text ⟩ is a whitespace-separated list of words. Hyphenation points are marked with a dash:

% \DefineHyphenationExceptions{american}{%
% hy-phen-ation ex-cep-tion }

\hyphenation{
A-la-hver-dzhie-va
anaph-o-ra
ana-phor
ana-phors
an-te-ced-ent
an-te-ced-ents
affri-ca-te
affri-ca-tes
ap-proach-es
Atha-bas-kan
Athe-nä-um
Bona-mi
Chi-che-ŵa
com-ple-ments
con-straints
Cope-sta-ke
Da-ge-stan
Dor-drecht
er-klä-ren-de
Ginz-burg
Gro-ning-en
Jap-a-nese
Jon-a-than
Ka-tho-lie-ke
Ko-bon
krie-gen
Le-Sourd
moth-er
Mül-ler
Nie-mey-er
Par-a-digm
Prze-piór-kow-ski
phe-nom-e-non
re-nowned
Rie-he-mann
un-bound-ed
with-in
}

% listing within here does not have any effect for lfg.tex % 2020-05-14

% why has "erklärende" be listed here? I specified langid in bibtex item. Something is still not working with hyphenation.


% to do: check
%  Alahverdzhieva


% biblatex:

% This is a LaTeX frontend to TeX’s \hyphenation command which defines hy- phenation exceptions. The ⟨language⟩ must be a language name known to the babel/polyglossia packages. The ⟨text ⟩ is a whitespace-separated list of words. Hyphenation points are marked with a dash:

% \DefineHyphenationExceptions{american}{%
% hy-phen-ation ex-cep-tion }

  \memoizeset{
    memo filename prefix={hpsg-handbook.memo.dir/},
    % readonly
  }
  \papernote{\scriptsize\normalfont
    \@author.
    \@title. 
    To appear in: 
    Stefan Müller, Anne Abeillé, Robert D. Borsley \& Jean-Pierre Koenig (eds.)
    HPSG Handbook
    Berlin: Language Science Press. [preliminary page numbering]
  }
  \pagenumbering{roman}
  \setcounter{chapter}{#1}
  \addtocounter{chapter}{-1}
}
\makeatother

\makeatletter
\newcommand{\togglepaperminimal}[1][0]{ 
  \bibliography{../Bibliographies/stmue,
                ../localbibliography,
  ../Bibliographies/coordination,
collection.bib}
  %% hyphenation points for line breaks
%% Normally, automatic hyphenation in LaTeX is very good
%% If a word is mis-hyphenated, add it to this file
%%
%% add information to TeX file before \begin{document} with:
%% %% hyphenation points for line breaks
%% Normally, automatic hyphenation in LaTeX is very good
%% If a word is mis-hyphenated, add it to this file
%%
%% add information to TeX file before \begin{document} with:
%% \include{localhyphenation}
\hyphenation{
A-la-hver-dzhie-va
anaph-o-ra
ana-phor
ana-phors
an-te-ced-ent
an-te-ced-ents
affri-ca-te
affri-ca-tes
ap-proach-es
Atha-bas-kan
Athe-nä-um
Bona-mi
Chi-che-ŵa
com-ple-ments
con-straints
Cope-sta-ke
Da-ge-stan
Dor-drecht
er-klä-ren-de
Ginz-burg
Gro-ning-en
Jap-a-nese
Jon-a-than
Ka-tho-lie-ke
Ko-bon
krie-gen
Le-Sourd
moth-er
Mül-ler
Nie-mey-er
Par-a-digm
Prze-piór-kow-ski
phe-nom-e-non
re-nowned
Rie-he-mann
un-bound-ed
with-in
}

% listing within here does not have any effect for lfg.tex % 2020-05-14

% why has "erklärende" be listed here? I specified langid in bibtex item. Something is still not working with hyphenation.


% to do: check
%  Alahverdzhieva


% biblatex:

% This is a LaTeX frontend to TeX’s \hyphenation command which defines hy- phenation exceptions. The ⟨language⟩ must be a language name known to the babel/polyglossia packages. The ⟨text ⟩ is a whitespace-separated list of words. Hyphenation points are marked with a dash:

% \DefineHyphenationExceptions{american}{%
% hy-phen-ation ex-cep-tion }

\hyphenation{
A-la-hver-dzhie-va
anaph-o-ra
ana-phor
ana-phors
an-te-ced-ent
an-te-ced-ents
affri-ca-te
affri-ca-tes
ap-proach-es
Atha-bas-kan
Athe-nä-um
Bona-mi
Chi-che-ŵa
com-ple-ments
con-straints
Cope-sta-ke
Da-ge-stan
Dor-drecht
er-klä-ren-de
Ginz-burg
Gro-ning-en
Jap-a-nese
Jon-a-than
Ka-tho-lie-ke
Ko-bon
krie-gen
Le-Sourd
moth-er
Mül-ler
Nie-mey-er
Par-a-digm
Prze-piór-kow-ski
phe-nom-e-non
re-nowned
Rie-he-mann
un-bound-ed
with-in
}

% listing within here does not have any effect for lfg.tex % 2020-05-14

% why has "erklärende" be listed here? I specified langid in bibtex item. Something is still not working with hyphenation.


% to do: check
%  Alahverdzhieva


% biblatex:

% This is a LaTeX frontend to TeX’s \hyphenation command which defines hy- phenation exceptions. The ⟨language⟩ must be a language name known to the babel/polyglossia packages. The ⟨text ⟩ is a whitespace-separated list of words. Hyphenation points are marked with a dash:

% \DefineHyphenationExceptions{american}{%
% hy-phen-ation ex-cep-tion }

  \memoizeset{
    memo filename prefix={hpsg-handbook.memo.dir/},
    % readonly
  }
  \papernote{\scriptsize\normalfont
    \@author.
    \@title. 
    To appear in: 
    Stefan Müller, Anne Abeillé, Robert D. Borsley \& Jean-Pierre Koenig (eds.)
    HPSG Handbook
    Berlin: Language Science Press. [preliminary page numbering]
  }
  \pagenumbering{roman}
  \setcounter{chapter}{#1}
  \addtocounter{chapter}{-1}
}
\makeatother




% In case that year is not given, but pubstate. This mainly occurs for titles that are forthcoming, in press, etc.
\renewbibmacro*{addendum+pubstate}{% Thanks to https://tex.stackexchange.com/a/154367 for the idea
  \printfield{addendum}%
  \iffieldequalstr{labeldatesource}{pubstate}{}
  {\newunit\newblock\printfield{pubstate}}
}

\DeclareLabeldate{%
    \field{date}
    \field{year}
    \field{eventdate}
    \field{origdate}
    \field{urldate}
    \field{pubstate}
    \literal{nodate}
}

%\defbibheading{diachrony-sources}{\section*{Sources}} 

% if no langid is set, it is English:
% https://tex.stackexchange.com/a/279302
\DeclareSourcemap{
  \maps[datatype=bibtex]{
    \map{
      \step[fieldset=langid, fieldvalue={english}]
    }
  }
}


% for bibliographies
% biber/biblatex could use sortname field rather than messing around this way.
\newcommand{\SortNoop}[1]{}


% Doug Ball

\newcommand{\elist}{\q<\ \ \q>}

\newcommand{\esetDB}{\q\{\ \ \q\}}


\makeatletter

\newcommand{\nolistbreak}{%

  \let\oldpar\par\def\par{\oldpar\nobreak}% Any \par issues a \nobreak

  \@nobreaktrue% Don't break with first \item

}

\makeatother


% intermediate before Frank's trees are fixed
% This will be removed!!!!!
%\newcommand{\tree}[1]{} % ignore them blody trees
%\usepackage{tree-dvips}


\newcommand{\nodeconnect}[2]{}
\newcommand{\nodetriangle}[2]{}



% Doug relative clauses
%% I've compiled out almost all my private LaTeX command, but there are some
%% I found hard to get rid of. They are defined here.
%% There are few others which defined in places in the document where they have only
%% local effect (e.g. within figures); their names all end in DA, e.g. \MotherDA
%% There are a lot of \labels -- they are all of the form \label{sec:rc-...} or
%% \label{x:rc-...} or similar, so there should be no clashes.

% Subscripts -- scriptsize italic shape lowered by .25ex 
\newcommand{\subscr}[1]{\raisebox{-.5ex}{\protect{\scriptsize{\itshape #1\/}}}}
% A boxed subscript, for avm tags in normal text
\newcommand{\subtag}[1]{\subscr{\idx{#1}}}

%% Sets and tuples: I use \setof{} to get brackets that are upright, not slanted
%\newcommand{\setof}[1]{\ensuremath{\lbrace\,\mathit{#1}\,\rbrace}}
% 11.10.2019 EP: Doug requested replacement of existing \setof definition with the following:
%\newcommand{\setof}[1]{\begin{avm}\{\textcolor{red}{#1}\}\end{avm}}
% 31.1.2019 EP: Doug requested re-replacement of the above \textcolour version with the following:
\newcommand{\setof}[1]{\begin{avm}\{#1\}\end{avm}}

\newcommand{\tuple}[1]{\ensuremath{\left\langle\,\mbox{\textit{#1}}\,\right\rangle}}

% Single pile of stuff, optional arugment is psn (e.g. t or b)
% e.g. to put a over b over c in a centered column, top aligned, do:
%   \cPile[t]{a\\b\\c} 
\newcommand{\cPile}[2][]{%
  \begingroup%
  \renewcommand{\arraystretch}{.5}\begin{tabular}[#1]{c}#2\end{tabular}%
  \endgroup%
}

%% for linguistic examples in running text (`linguistic citation'):
\newcommand{\lic}[1]{\textit{#1}}

%% A gap marked by an underline, raised slightly
%% Default argument indicates how long the line should be:
\newcommand{\uGap}[1][3ex]{\raisebox{.25em}{\underline{\hspace{#1}}}\xspace}

%% \TnodeDA{XP}{avmcontents} -- in a Tree, put a node label next to an AVM
\newcommand{\TnodeDA}[2]{#1~\begin{avm}{#2}\end{avm}}

%% This allows tipa stuff to be put in \emph -- we need to change to cmr first.
%% It is used in the discussion of Arabic.
\newcommand{\emphtipa}[1]{{\fontfamily{cmr}\emph{\tipaencoding #1}}} 



 
 
\definecolor{lsDOIGray}{cmyk}{0,0,0,0.45}


% morphology.tex:
% Berthold

\newcommand{\dnode}[1]{\rnode{#1}{\fbox{#1}}}
\newcommand{\tnode}[1]{\rnode{#1}{\textit{#1}}}

\newcommand{\tl}[2]{#2}

\newcommand{\rrr}[3]{%
  \psframebox[linestyle=none]{%
    \avmoptions{center}
    \begin{avm}
      \[mud & \{ #1 \}\\
      ms & \{ #2 \}\\
      mph & \<  #3 \> \]
    \end{avm}
  }
}
\newcommand{\rr}[2]{%
  \psframebox[linestyle=none]{%
    \avmoptions{center}
    \begin{avm}
      \[mud & \{ #1 \}\\
      mph & \<  #2 \> \]
    \end{avm}
  }
}
 

% Frank Richter
\newtheorem{mydef}{Definition}

\long\def\set[#1\set=#2\set]%
{%
\left\{%
\tabcolsep 1pt%
\begin{tabular}{l}%
#1%
\end{tabular}%
\left|%
\tabcolsep 1pt%
\begin{tabular}{l}%
#2%
\end{tabular}%
\right.%
\right\}%
}

\newcommand{\einruck}{\\ \hspace*{1em}}


%\newcommand{\NatNum}{\mathrm{I\hspace{-.17em}N}}
\newcommand{\NatNum}{\mathbb{N}}
\newcommand{\Aug}[1]{\widehat{#1}}
%\newcommand{\its}{\mathrm{:}}
% Felix 14.02.2020
\DeclareMathOperator{\its}{:}

\newcommand{\sequence}[1]{\langle#1\rangle}

\newcommand{\INTERPRETATION}[2]{\sequence{#1\mathsf{U}#2,#1\mathsf{S}#2,#1\mathsf{A}#2,#1\mathsf{R}#2}}
\newcommand{\Interpretation}{\INTERPRETATION{}{}}

\newcommand{\Inte}{\mathsf{I}}
\newcommand{\Unive}{\mathsf{U}}
\newcommand{\Speci}{\mathsf{S}}
\newcommand{\Atti}{\mathsf{A}}
\newcommand{\Reli}{\mathsf{R}}
\newcommand{\ReliT}{\mathsf{RT}}

\newcommand{\VarInt}{\mathsf{G}}
\newcommand{\CInt}{\mathsf{C}}
\newcommand{\Tinte}{\mathsf{T}}
\newcommand{\Dinte}{\mathsf{D}}

% this was missing from ash's stuff.

%% \def \optrulenode#1{
%%   \setbox1\hbox{$\left(\hbox{\begin{tabular}{@{\strut}c@{\strut}}#1\end{tabular}}\right)$}
%%   \raisebox{1.9ex}{\raisebox{-\ht1}{\copy1}}}



\newcommand{\pslabel}[1]{}

\newcommand{\addpagesunless}{\todostefan{add pages unless you cite the
 work as such}}

% dg.tex
% framed boxes as used in dg.tex
% original idea from stackexchange, but modified by Saso
% http://tex.stackexchange.com/questions/230300/doing-something-like-psframebox-in-tikz#230306
\tikzset{
  frbox/.style={
    rounded corners,
    draw,
    thick,
    inner sep=5pt,
    anchor=base,
  },
}

% get rid of these morewrite messages:
% https://tex.stackexchange.com/questions/419489/suppressing-messages-to-standard-output-from-package-morewrites/419494#419494
\ExplSyntaxOn
\cs_set_protected:Npn \__morewrites_shipout_ii:
  {
    \__morewrites_before_shipout:
    \__morewrites_tex_shipout:w \tex_box:D \g__morewrites_shipout_box
    \edef\tmp{\interactionmode\the\interactionmode\space}\batchmode\__morewrites_after_shipout:\tmp
  }
\ExplSyntaxOff


% This is for places where authors used bold. I replace them by \emph
% but have the information where the bold was. St. Mü. 09.05.2020
\newcommand{\textbfemph}[1]{\emph{#1}}



% Felix 09.06.2020: copy code from the third line into localcommands.tex: https://github.com/langsci/langscibook#defined-environments-commands-etc
\patchcmd{\mkbibindexname}{\ifdefvoid{#3}{}{\MakeCapital{#3} }}{\ifdefvoid{#3}{}{#3 }}{}{\AtEndDocument{\typeout{mkbibindexname could not be patched.}}}

  %% -*- coding:utf-8 -*-

%%%%%%%%%%%%%%%%%%%%%%%%%%%%%%%%%%%%%%%%%%%%%%%%%%%%%%%%%%%%
%
% gb4e

% fixes problem with to much vertical space between \zl and \eal due to the \nopagebreak
% command.
\makeatletter
\def\@exe[#1]{\ifnum \@xnumdepth >0%
                 \if@xrec\@exrecwarn\fi%
                 \if@noftnote\@exrecwarn\fi%
                 \@xnumdepth0\@listdepth0\@xrectrue%
                 \save@counters%
              \fi%
                 \advance\@xnumdepth \@ne \@@xsi%
                 \if@noftnote%
                        \begin{list}{(\thexnumi)}%
                        {\usecounter{xnumi}\@subex{#1}{\@gblabelsep}{0em}%
                        \setcounter{xnumi}{\value{equation}}}
% this is commented out here since it causes additional space between \zl and \eal 06.06.2020
%                        \nopagebreak}%
                 \else%
                        \begin{list}{(\roman{xnumi})}%
                        {\usecounter{xnumi}\@subex{(iiv)}{\@gblabelsep}{\footexindent}%
                        \setcounter{xnumi}{\value{fnx}}}%
                 \fi}
\makeatother

% the texlive 2020 langsci-gb4e adds a newline after \eas, the texlive 2017 version was OK.
\makeatletter
\def\eas{\ifnum\@xnumdepth=0\begin{exe}[(34)]\else\begin{xlist}[iv.]\fi\ex\begin{tabular}[t]{@{}b{.99\linewidth}@{}}}
\makeatother


%%%%%%%%%%%%%%%%%%%%%%%%%%%%%%%%%%%%%%%%%%%%%%%%%%%%%%%%%%
%
% biblatex

% biblatex sets the option autolang=hyphens
%
% This disables language shorthands. To avoid this, the hyphens code can be redefined
%
% https://tex.stackexchange.com/a/548047/18561

\makeatletter
\def\hyphenrules#1{%
  \edef\bbl@tempf{#1}%
  \bbl@fixname\bbl@tempf
  \bbl@iflanguage\bbl@tempf{%
    \expandafter\bbl@patterns\expandafter{\bbl@tempf}%
    \expandafter\ifx\csname\bbl@tempf hyphenmins\endcsname\relax
      \set@hyphenmins\tw@\thr@@\relax
    \else
      \expandafter\expandafter\expandafter\set@hyphenmins
      \csname\bbl@tempf hyphenmins\endcsname\relax
    \fi}}
\makeatother


% the package defined \attop in a way that produced a box that has textwidth
%
\def\attop#1{\leavevmode\begin{minipage}[t]{.995\linewidth}\strut\vskip-\baselineskip\begin{minipage}[t]{.995\linewidth}#1\end{minipage}\end{minipage}}


%%%%%%%%%%%%%%%%%%%%%%%%%%%%%%%%%%%%%%%%%%%%%%%%%%%%%%%%%%%%%%%%%%%%


% Don't do this at home. I do not like the smaller font for captions.
% This does not work. Throw out package caption in langscibook
% \captionsetup{%
% font={%
% stretch=1%.8%
% ,normalsize%,small%
% },%
% width=\textwidth%.8\textwidth
% }
% \setcaphanging


  \togglepaper[6]
}{}


\author{%
	Stephen Wechsler\affiliation{The University of Texas at Austin}%
}
\title{Agreement}
\abstract{
Agreement is modeled in HPSG by assigning agreement features such as person, number, and gender (``phi features'') to specified positions  in the feature structures representing the agreement trigger and target. The locality conditions on agreement follow from the normal operation of the grammar in which those phi features are embedded.  In anaphoric agreement, phi features appear on referential indices; in verb agreement, phi features appear on the verb's \argst list items; and in modifier agreement, phi features appear on the \textsc{mod} value of the modifier.  Selective underspecification of agreement features  accounts for the alternation between formal and semantic agreement.  Within the HPSG framework, long-distance agreement has been analyzed as anaphoric agreement in a special clausal construction, while superficial agreement has been modeled using linearization theory.  
}



\begin{document}
\maketitle
\label{chap-agreement}
% \chapterDOI{} %will be filled in at production

%\epigram{\textit{There is nothing more likely to start disagreement among people or countries than an agreement.} \\ -- E. B. White }


\section{Introduction} 


Agreement is the systematic covariation between a semantic or formal property of one element (called the agreement \textit{trigger}) and a formal property of another (called the agreement \textit{target}).  In the sentences \textit{I am here} and \textit{They are here}, the subjects (\textit{I} and \textit{they}, respectively) are the triggers; the target verb forms (\textit{am} and \textit{are}, respectively) covary with them.  Research on  agreement systems within HPSG has been devoted to describing and explaining a number of observed aspects of such systems.  Regarding the grammatical relationship between the trigger and the target, we may first of all ask how local that relationship is, and in what grammatical terms it is defined.  Having determined the prevailing locality conditions on agreement in a given language, we attempt to explain observed exceptions, that is, cases of apparent ``long-distance agreement'', as well as cases of superficial agreement defined on string adjacency.  Agreement features across languages include person, number, and gender (known as \textit{phi} features), as well as deictic features and case, but various different subsets of those features are involved in particular agreement relations.  How can we explain the distribution of features?  How are locality and feature distribution related to the diachronic origin of agreement systems?  Also, as indicated in the definition of agreement provided in the first sentence of this paper, the features of the target are sometimes determined by the trigger's form and sometimes by its meaning.  What regulates this choice?   In some cases a single trigger in a sentence determines different features on two different targets.  Why does such ``mixed agreement'' exist, and what does its existence tell us about the grammatical representation of agreement?  
%How does the grammar resolve the target form when the trigger is a coordinate NP whose conjuncts have conflicting agreement values?  
This chapter reviews HPSG approaches to these questions of locality, grammatical representation,
feature distribution, diachrony, semantic versus formal agreement, and mixed agreement. Agreement
with coordinate phrases is discussed by \crossrefchaptert{coordination}.

HPSG offers an integrated account of these phenomena.  In most cases the analysis of agreement phenomena does not involve any  special formal devices dedicated for agreement, comparable to the \textit{probe} and \textit{goal}, or the \textsc{agree} relation, found in \isi{Minimalist} accounts \citep{Chomsky2000b-u}.  Instead, the observed agreement phenomena arise as a side effect of other grammatical mechanisms responsible for valence saturation, the semantics of modification, and coreference.  

\section{Agreement as unification } 
\label{unif-sec}


Constraint-based formalisms such as HPSG are uniquely well-suited for modeling agreement.  
Within such formalisms, agreement occurs when 
multiple feature sets
 arising from distinct elements of a sentence specify information about a single abstract object, so that the information must be mutually consistent \citep{Kay:1984}.  
The two forms are said to agree when the values imposed by the two constraints are compatible, while ungrammaticality results when they are incompatible.  For example the \ili{English} verb \textit{is} in (\ref{is}) specifies that its initial \argst list item, which is identified with the \subj list item, has third person, singular features.  In the mechanism of valence saturation, the NP list item in the value of \subj unifies with the feature description representing the \textsc{synsem} value of the subject NP.  The features specified by the verb for its subject and by the subject NP must be compatible; otherwise the representation for the resulting sentence is ill-formed, predicting ungrammaticality as in (\ref{sober}a).  

\ea		
\label{is} 
Simplified lexical sign for the verb \textit{is}:\\*
\avm{
[\phon < is > \\
subj & < \1 > \\ 
comps & < \2 > \\
arg-st & < \1 NP	[pers & 3rd \\
					num & sg], \2 XP > ] 
}
\z

\ea
\label{entryfori}
Simplified lexical signs for  \textit{I} and \textit{she}: \\*
\avm{
[\phon < I > \\
head &	[\type*{noun}
		pers & 1st \\
		num & sg ] ] 
}
\avm{
[\phon  < she > \\
head &	[\type*{noun}
		pers & 3rd \\
		num & sg \\
		gen & fem ] ] 
}
\z

\eal
\label{sober}
\ex[*]{I is sober.}
\ex[ ]{She is sober.}
\zl
 
\noindent
The features supplied by the trigger and target must be consistent, but 
there is no general minimum requirement on how many features they specify.  Both of them can be, and typically are, underspecified for some agreement features.  For example, \isi{gender} is not specified by the verb in (\ref{is}) or the first pronoun in (\ref{entryfori}).   

Since unification
is commutative, the representation of an agreement construction is the same regardless of whether a feature originates from the trigger or the target.  This immediately accounts for common agreement behavior observed when triggers are underspecified  \citep{Barlow:1988}.  
For example, \ili{Serbo-Croatian} is a grammatical gender language, where common nouns are assigned to the  masculine, feminine, or neuter gender.   The noun \textit{knjiga} `book’ in (\ref{old-book}) is feminine, so the modifying determiner and adjective appear in feminine form \citep[4, example\,(1)]{Wechsler+Zlatic:2003}.   

\begin{exe}
\ex \label{old-book}
\gll Ov-a	star-a	knjig-a stalno  pad-a. \\
this-\textsc{nom.f.sg}	old-\textsc{nom.f.sg}	book-\textsc{nom.sg} always fall-\textsc{3sg} \\
\glt `This old book keeps falling.'  
\end{exe} 


\noindent
However, some nouns are unspecified for gender, such as \textit{sudija}  `judge’.  Interestingly, the gender of an agreeing adjective actually adds semantic information, indicating the sex of the judge \citep[42, example\,(23)]{Wechsler+Zlatic:2003}.

\begin{exe}
\ex \label{sudija}
\begin{xlist}
\ex
\gll 	Taj	stari	sudija	je	dobro	sudio.	 \\
that.\textsc{m}	old.\textsc{m}	judge	\textsc{aux}	well	judged.\textsc{m} \\
\glt `That old (male) judge judged well.'
\ex 
 \gll
	Ta	 stara	sudija	je	dobro	sudila. \\
that.\textsc{f}	old.\textsc{f}	judge	\textsc{aux}	well	judged.\textsc{f}\\
	\glt `That old (female) judge  judged well.’
\end{xlist}
\end{exe}

\noindent 
Here the gender feature comes from the targets instead of the trigger.   This illustrates an advantage of constraint-based theories like HPSG over transformational accounts in which a feature is copied from the trigger, where it originates, to the target, where it is then realized.  The usual source of the feature (the noun) lacks it in (\ref{sudija}), a problem for the feature-copying view.   

The same problem occurs even more dramatically in \emph{pro}-drop\is{pro-drop@\emph{pro}-drop} languages.  Many languages
allow subject pronouns to drop, and distinguish person, number, and/""or gender on the verb.  If those
features originate from the null subject, then there would have to be distinct null pronouns, one
for each verbal and predicate adjective inflection \citep[64]{Pollard+Sag:1994}.  This would be more
complex and stipulative, and moreover the paradigm of putative null pronouns would have to exactly
match the set of distinctions drawn in the verb and adjective systems, rather than reflecting the
pronoun paradigm.    HPSG avoids this suspicious assumption.  Null anaphora is modeled by allowing
the \textit{pro}-dropped argument to appear on the \argst list but not a valence list like \subj or \comps (see \crossrefchapteralt{arg-st}).  For example, in the context given in (\ref{prodrop}) a \ili{Serbo-Croatian} speaker could omit the subject pronoun.  

\begin{exe}
\ex 
\label{prodrop}
Context: Speaker comes home to find her bookcase mysteriously empty. \\
\gll  Gde su (one) nestale? \\
where did (they.\textsc{f.pl}) disappear.\textsc{f.pl}     \\
\glt `Where did they (i.e. the books) go?' 
\end{exe}


\noindent
The sign for the inflected participle specifies feminine plural features on the initial item in its \argst list.  The \subj list item is optional: 

\begin{exe} 
\ex		\label{nestale} 
Simplified lexical sign for the participle form \textit{nestale}:\\*
\avm{
[\phon < nestale > \\
subj & <( \1 )> \\ 
comps & < > \\
arg-st & <\1 NP	[num & pl \\
				gen & fem] > ] 
}
\end{exe}

\noindent
The feminine plural features are specified regardless of whether  the subject pronoun appears.  When the pronoun is dropped we have the usual underspecification, only in this case the trigger does not exist, so it is effectively fully underspecified, realizing no features at all.  


\section{Locality in agreement} 

\subsection{Argument and modifier agreement}
\label{arg-mod-agr}

In HPSG, the grammatical agreement of a predicator with its subject or object, or an adjective, determiner, or other modifier with its head noun, piggy-backs on the mechanism of valence saturation and modification.   Agreement is encoded in the grammar by adding features of person, number, gender, case, and deixis to the existing feature descriptions involved in syntactic and semantic composition.  This simple assumption is sufficient to explain the broad patterning of distribution of agreement, in contrast to the transformational approach where complex locality conditions must be stipulated.   

In HPSG, predicate-argument agreement arises directly from the  valence saturation, as illustrated already in (\ref{is}) above.  Thus the locality conditions on the trigger-target relation follow from the conditions on the subject-head or complement"=head relation.   Similarly, attributive adjectives agree with nouns directly through the composition of the modifier with the head that it selects via the \textsc{mod} feature.  For example, the \ili{Serbo-Croatian} feminine adjective form \textit{stara} `old.\textsc{f}' in (\ref{sudija}b) specifies feminine singular features for the common noun phrase (N$'$) that it modifies.  

\begin{exe} 
\ex	\label{stara}  Simplified lexical sign for  \textit{stara}:\\*
\avm{
[\phon < stara > \\
mod &	[head &	[\type*{noun}
				num & sg \\
				gend & fem] \\
		comps & < > ] ]
}
\end{exe}

\noindent
In head-adjunct phrases, the \textsc{mod} value of the adjunct daughter is token-identical with the \textit{synsem} value of the head daughter.  So \textit{stara}'s feminine singular features  cannot conflict with the features of the noun it modifies.  

The predicted locality conditions are also affected by the percolation of features from words to phrasal nodes, and this depends on the location of the features within the feature description.   Agreement features of the \textit{trigger} appear either within the \textsc{head} value or the semantic \textsc{content} value (these give rise to \textsc{concord} and \textsc{index} agreement, respectively; see Section~\ref{ind-con}).  In either case these features percolate from the trigger's head word to its maximal phrasal projection, due to the Head Feature Principle in the former case and the Semantics Principle in the latter.  For example the noun phrase \textit{the books} inherits its [\textsc{num} \textit{pl}] feature from the head word \textit{books}.  This determines plural agreement on a verb:  \textit{These books are/*is interesting.}  Apparent exceptions, where a target seems to fail to agree with the head of the trigger, are discussed below.   

However, agreement features of the \textit{target} appear in neither the \textsc{head} nor the \textsc{content} value of the target form, but rather appear embedded in an  \textsc{arg-st} list item or \textsc{mod} features.  So agreement features of the target do not project to the  target's phrasal projection such as VP, S, or AP.  This is a welcome consequence.  If the subject agreement features of the verb projected to the VP, for example, we would expect to find VP-modifying adverbs that consistently agree with them, but we do not.\footnote{VP-modifying secondary predicates sometimes agree with their own subjects.  What we do not find are adjuncts that consistently agree with the subject agreement features of the VP even when the adjunct is not predicated of that subject.}  


\section{Varieties of agreement target} 

\subsection{Anaphoric agreement}
\label{ana-agr}
In anaphoric agreement, an anaphoric pronoun agrees in person, number, and gender with its antecedent.  Since \citet{Pollard+Sag:1992,Pollard+Sag:1994}, anaphoric agreement has been analyzed in HPSG by assuming that person, number, and gender are formal features of the referential index associated with an NP.  Anaphoric binding in HPSG is modeled as coindexation, i.e.\ sharing of the \textsc{index} value, between the binder and bindee.  Thus any specifications for agreement features of the \textsc{index} contributed by the binder and bindee must be mutually consistent.  In (\ref{admire}), Principle A of the \isi{Binding Theory} requires the reflexive pronoun to be coindexed with an o-commanding item, here the subject pronoun: 

\begin{exe}
\ex  
 \begin{xlist}  \label{admire}
\ex She admires herself.  
\ex \textit{admire}:\\*
\avm{
[arg-st & < NP:	[\type*{ppro}
				index & \1	[pers & 3rd \\
							num & sg \\
							gen & fem] ], NP:[\type*{ana}
											index & \1	[pers & 3rd \\
														num & sg \\
														gen & fem] ]
> ]
}
\end{xlist}
\end{exe}

\noindent
The agreement features are formal features and not semantic ones, but the semantic correlates of person (speaker, addressee, other), number (cardinality), and gender (male, female, inanimate, etc.) are invoked under certain conditions (described in Section~\ref{pancake-sec}).   
Thus \textsc{index} agreement is distinct from \textit{pragmatic agreement} whereby semantic features of two
coreferential expressions must be semantically consistent in order for them to refer to a single
entity.  \textsc{index} agreement is enforced only within the  syntactic domain defined by binding theory,
while pragmatic agreement applies everywhere.   For example, feminine pronouns are sometimes used
for ships, in addition to neuter pronouns.  Whichever gender is chosen, it must be consistent in
binding contexts (example based on \citeauthor{Pollard+Sag:1994}'s \citeyear[79]{Pollard+Sag:1994} example (46a)):


\begin{exe}
\ex   \label{lurch}
\begin{xlist}
\ex[ ]{The ship lurched, and then it righted itself.  She is a fine ship.}
\ex[ ]{The ship lurched, and then she righted herself.  It is a fine ship.}
\ex[*]{The ship lurched, and then she righted itself.}
\ex[*]{The ship lurched, and then it righted herself.}
\end{xlist}
\end{exe}

\noindent
The bound reflexive must agree formally with its antecedent, while other
coreferential pronouns need not agree, as they are not coarguments of the antecedent and not subject to the structural binding theory.  

In grammatical gender languages, where common nouns are conventionally assigned to a gender, an anaphoric pronoun appearing outside the binding domain of its antecedent can generally agree with that antecedent either formally or, if it is semantically appropriate (such as an animate, sexed entity), it can alternatively agree pragmatically.  In most situations pronouns allow either pragmatic or \textsc{index} agreement with their
antecedents.  For example, pronouns coreferential with the \ili{Serbo-Croatian} grammatically neuter diminutive noun \textit{devoj\v{c}e} `girl' can appear in either neuter or feminine gender \citep[from][198]{Wechsler+Zlatic:2003}:

\begin{exe}
\ex \label{girl}
\gll 	Ovo	                      malo                           devoj\v{c}e$_i$         je          u\v{s}lo.  \\
         this.\textsc{n.sg} 	 little.\textsc{n.sg}   girl.\textsc{n.sg} 	\textsc{aux.3sg} entered.\textsc{n.sg}  \\
 %        \glt this is the translation  
%\end{exe}
\begin{xlist}
\ex  
\gll 	Ono$_i$	je	htelo	da	telefonira. \\
		it.\textsc{n.sg}  	\textsc{aux.sg}  	wanted.\textsc{n.sg}  	that	telephone   \\
\ex  
\gll 	Ona$_i$	je	htela	da	telefonira. \\
		she.\textsc{f.sg}  	\textsc{aux.sg}  	wanted.\textsc{f.sg}  	that	telephone \\
\glt `This little $\mbox{girl}_i$ came in.  $\mbox{She}_i$ wanted to use the telephone.'
\end{xlist}
\end{exe}

\noindent
The neuter pronoun in (\ref{girl}a) reflects \textsc{index} agreement
with the antecedent while the feminine pronoun (\ref{girl}b) reflects its reference to a female (pragmatic agreement).  But when a reflexive pronoun is locally bound  by a nominative subject, agreement in  formal \textsc{index} features is preferred:

\begin{exe}
\ex \label{girl2}
\gll 	Devoj\v{c}e            je               volelo              samo                  /  ?* samu              sebe. \\
        girl.\textsc{nom.n.sg} \textsc{aux3.sg} liked.\textsc{n.sg} own.\textsc{acc.n.sg} {} {} \textsc{acc.f.sg} self.\textsc{acc} \\
\glt `The girl liked herself.'
\end{exe}

\noindent
Again, this illustrates \textsc{index} agreement in the domain defined by the structural binding theory.





\subsection{Grammatical agreement: \textsc{index} and \textsc{concord}}
\label{ind-con}\label{agreement-sec-index-concord}
As noted above, in  HPSG agreement effectively piggy-backs on other independently justified grammatical processes.  Anaphoric agreement is a side-effect of binding (Section~\ref{ana-agr}) while  grammatical agreement is a side-effect of valence saturation and modification (Section~\ref{arg-mod-agr}).  The formal HPSG analysis of a particular agreement process mainly consists of positing agreement features somewhere in the feature structure; the observed properties follow from the location of those agreement features.  With regard to the location of the features, grammatical agreement bifurcates into two types, \textsc{index} and \textsc{concord}.\footnote{The \textsc{index}/""\textsc{concord} theory is sketched in  \citet[Chapter~2]{Pollard+Sag:1994} and \citet{Kathol99b}, and developed in detail in  \citet{Wechsler+Zlatic:2000,Wechsler+Zlatic:2003}, all in the HPSG framework.  It has since been adopted into LFG (\citealt{king+dalrymple:2004}, inter alia) and GB/Minimalism \citep{Danon:2009}.}  (The attribute name  \textsc{concord} was introduced by \citealt[799]{Wechsler+Zlatic:2000}, \citealt[14]{Wechsler+Zlatic:2003}; precursors to the idea were treated as \textsc{head} features in \citealt{Pollard+Sag:1994}, and called \textsc{agr} by \citealt{Kathol99b}.)   The best way to understand this bifurcation of agreement, and indeed the operation of grammatical agreement systems generally, is by considering their diachronic origin.   Although our primary goal is the description of synchronic grammar, a look at diachrony can help explain the forms that the grammar takes, and can also provide clues as to the best formalization of it.  

Within the diachronic literature on agreement there are thought to be two different lexical sources for agreement inflections: (i) incorporated pronouns and (ii) incorporated noun classifiers \citep{greenberg:1978}.  
%Going back further, the noun classifiers derive from semantically superordinate common nouns meaning `animal’, `man’, `woman’, and so on.  
These two sources, ultimately traced to pronouns and common nouns, give rise to \textsc{index} and \textsc{concord} target inflections, respectively, as explained next.    


\subsubsection{\textsc{index} agreement}
Taking pronouns first, many grammatical agreement systems evolve historically from the incorporation of pronominal arguments into the predicates selecting those arguments, such as verbs and nouns (\citealt{bopp:1842,givon:1976,wald:1979}, inter alia).  When a phrase serving as antecedent of the incorporated pronoun is reanalyzed as the true subject or object of the predicate,  the pronominal affix effectively becomes an agreement marker.  With this reanalysis the only change in the affix is that it loses its ability to refer: it no longer functions as a pronoun.   The affix retains its agreement features, and what was formerly anaphoric agreement with the topic becomes grammatical agreement with the subject or object.  This explains why the features of grammatical agreement match those of pronominal anaphora: typically person, number, and gender, with occasional deictic features  \citep[752]{bresnan+mchombo:1987}.   

As explained above, structural anaphoric binding involves identifying (structure sharing) the referential indices of the pronoun and its binder.   Therefore grammatical agreement derived from it is also \textsc{index} agreement.   For example, the signs for \ili{English} \textit{is} and \textit{I} in (\ref{is}) and (\ref{entryfori}) above should be rewritten as follows:

\begin{exe} 

\ex 	\label{is2}
	Simplified sign for \textit{is}, illustrating \textsc{index} agreement:\\*
\avm{
[\phon < is > \\
subj & < NP [content|index &	[pers & 3rd \\
								num & sg ] ] > \\ 
comps & < XP > ]
}
\end{exe}

\begin{exe} 
\ex	\label{entryfori2}
Sign for  \textit{I}, illustrating  \textsc{index} features:\\*
\avm{
[\phon < I > \\
content|index & \1	[pers & 1st \\
					num & sg ] \\
context & speaker!(\1)! ]
}
\end{exe}

\noindent
This finite verb form specifies third person singular features of its subject's referential index.  

One salient distinguishing characteristic of \textsc{index} agreement is that it includes the \textsc{person} feature.  The only known diachronic source of the \textsc{person} feature is from pronouns.  Therefore, the other type of agreement, \textsc{concord}, lacks the \textsc{person} feature (as we will see below).  

By modeling verb agreement in a way that reflects its historical origin, we are able to explain an array of facts concerning particular agreement systems.   Some of these facts and explanations are presented in Section~\ref{mismatch} below.  
 



\subsubsection{\textsc{concord}}
\label{concord-sec}

The agreement inflections on modifiers of nouns, such as adjectives and determiners, are thought to derive historically not from pronouns, but from noun classifiers 
(\citealt{greenberg:1978,reid:1997,Seifart:2009,Grinevald+Seifart:2004}, \citealt[268--269]{corbett:2006}).
The classifier morphemes in turn derive historically from lexical common nouns denoting  superordinate categories like animal, woman, man, etc.  For example \citet{reid:1997} posits the following historical development of \ili{Ngan’gityemerri} (southern Daly; southwest of Darwin, Australia), a language where the historical stages continue to cooccur in the current synchronic grammar.   Originally the language had general-specific pairings of nouns as a common syntactic construction, such as  \textit{gagu wamanggal}  `animal wallaby’ in (\ref{wallaby1}) (from \citealt[216]{reid:1997}, examples (162)--(165)).  The specific noun can be omitted when reference to it is established in discourse, leaving the general noun and modifier, to form NPs like \textit{gagu kerre}, literally `animal big’ but functioning roughly like nominal ellipsis `big one’.  Then, where the specific noun is also included, both noun and modifier attract the generic term (\ref{wallaby2}).  The gender markers then reduce phonologically and incorporate, producing modifier gender agreement (\ref{wallaby3}). 

\begin{exe}
\ex
\begin{xlist}
\ex \label{wallaby1} Stage I: \\
\gll 	Gagu	   wamanggal	  kerre    ngeben-da. \\
	     animal	wallaby	      big	      1\textsc{sg.sb.aux}-shoot \\
\glt `I shot a big wallaby.’	
\ex \label{wallaby2} Stage II: \\
\gll 	Gagu	   wamanggal	   gagu	kerre	    ngeben-da. \\
	animal	wallaby	  animal     	big	1\textsc{sg.sb.aux}-shoot\\
\glt `I shot a big wallaby.’
\ex \label{wallaby3} Stage III: \\
\gll 	wa=ngurmumba	wa=ngayi	darany-fipal-nyine. \\
	male=youth		male=mine	\textsc{3sg.aux}-return-\textsc{foc} \\
\glt 	`My initiand son has just returned.’ 
\end{xlist}
\end{exe}

\noindent
If the same affix is retained on the modifiers and the noun they modify, then the result is symmetrical agreement (also known as alliterative agreement), like the feminine \textit{-a} endings in \ili{Spanish} \textit{zona rosa} \citep[87--88]{corbett:2006}.  But often an asymmetry between the affixes on the  noun and the modifiers develops: the noun affix becomes obligatory and is subject to morphophonological processes that do not affect the modifier affix \citep[216]{reid:1997}.    This process may further progress to ``prefix absorption'' into the common noun, as evidenced by ``gender prefixed nominal roots being interpreted as stems for further gender marking'' \citep[217]{reid:1997}.


Agreement marked with inflections from such nominal sources is called \emph{concord}, which is described using the HPSG \textsc{concord} feature.  What is the proper HPSG formalization of this type of agreement, given its provenance?   The last stages of the diachronic development, described in the previous paragraph, imply that the \textit{form} of the trigger (the noun) is influenced by the agreement features.  That is, noun declension classes tend to correlate with gender assignment (and more generally, phonological and morphological characteristics of nouns correlate with gender assignment); and number is marked on nouns as well.  (This close relation between declension class and \textsc{concord} is demonstrated in detail in \citealt[Chapter~2]{Wechsler+Zlatic:2003}.) Thus the agreement features must appear both on the head noun (to inform its form and/""or its gender selection and number value) and on the phrasal projection of that noun (to trigger agreement via the \textsc{mod} feature of the agreement targets).  Ergo \textsc{concord} is a \textsc{head} feature of the trigger.  

Along with the number and gender features, the \textsc{concord} value is assumed to include the case feature when case is a feature of NPs  realized on both the head noun and its modifying adjectives or determiner.   \textsc{concord} lacks the person feature, since  common nouns, from which the agreement inflections on the targets derive, lack the person feature (common nouns do not distinguish person values, since they are all in the third person).    Meanwhile, \textsc{index} agreement preserves the pronominal features of person, number, and gender, reflecting its origins.  In the usual case the number and gender values found in \textsc{concord} match those found in \textsc{index}.  The \ili{Serbo-Croatian} noun form \textit{knjiga} triggers feminine singular nominative \textsc{concord} on its adjectival possessive specifier and modifier, and third person singular \textsc{index} agreement on the finite auxiliary.  (The status of the participle is discussed below.)  

\begin{exe}
\ex  \label{fell}
\gll 	Moja 	stara	 knjiga	je pala.  \\
my.\textsc{f.nom.sg}  old.\textsc{f.nom} 	book.\textsc{nom.sg} 	\textsc{aux.3.sg}  fall.\textsc{pprt.f.sg} \\
\glt`My old book fell.' \citep[18]{Wechsler+Zlatic:2003}
\end{exe}

\noindent
The nominative singular noun form \emph{knjiga} specifies its agreement features in both \textsc{concord} (a \textsc{head} feature) and \textsc{index}, with the respective values for number and gender shared:

\eas
\label{knjiga-avm} Lexical sign for \emph{knjiga} `book' \citep[from][18]{Wechsler+Zlatic:2003}: \\*
\avm{
 [ phon   & \phonliste{ knjiga }  \\
   synsem & [ category & [ head & [\type*{noun}
	                           concord & \3 [ case & nom\\
                                                  num  & \1 sing\\
                                                  gen  & \2 fem ]]\\ 
                           spr  & < (AP [\type{poss}, concord \3 ] ) > ]\\ 
              content & [ index & \tag{i} [pers & 3rd\\ 
                                           num  & \1\\ 
                                           gen  & \2 ]\\ 
                          restr & \{ !$book(i)$! \} ]]] 
}
%% \[ phonology & \< $knjiga$ \>  \\
%%    synsem & \[
%% 	category & \[ head \, \[\asort{noun}
%% 	                      concord & \ibox{3}\[ case & \it{nom}\\
%%                                            num & \ibox{1}\it{sing}\\
%%                                            gen & \ibox{2}\it{fem}\]\]\\ 
%%                       spr \, \<\ \, (AP\[$poss$; concord \ibox{3}\]\) \, \> \]\\ 
%% content & \[ 
%% index & \@i\[pers & \it{3rd}\\ num & \ibox{1}\\ gen & \ibox{2}\]\\ 
%% restr & \{ $book(i)$ \} 
%% \]\] 
%% \end{avm}
\zs

\noindent
The specifier (\textsc{spr}) is shown as AP because the possessive phrase is categorically an adjective phrase in \ili{Serbo-Croatian}.  The features in the overlap between \textsc{concord} and \textsc{index} are normally shared as in this example.  But with some special nouns, features can be asymmetrically specified in only one of the two values (with no reentrancy linking them, of course).  This leads to mismatches between \textsc{concord} and \textsc{index} targets, discussed in Section~\ref{mismatch} below.

The phi features also appear within the \textsc{head} value, as shown in (\ref{knjiga-avm}), so that adjunct APs can agree with those features.  For example, concord by the attributive adjective \textit{stara} `old' is guaranteed because its \textsc{mod} feature is specified for feminine singular features, as shown in (\ref{stara}) in Section~\ref{arg-mod-agr} above.   


\subsection{Conclusion}
To summarize this section, we have seen the two main historical paths to agreement, and shown how HPSG formalizes these two types of agreement so as to capture the syntactic and semantic properties that follow directly from their origins.  Agreement that descends from anaphoric agreement of pronouns with their antecedents, through the incorporation of personal pronouns into verbs and other predicators, inherits the \textsc{index} matching process found in the anaphoric agreement from which it descends.  Agreement that descends from the incorporation of noun classifiers involves features located in the \textsc{head} value that connect a trigger noun form to its phrasal projection.  The feature sets differ for the same reason; \textsc{person} is a feature only of the first type, and \textsc{case} only of the second.  \textsc{concord} correlates strongly with declension class, while \textsc{index} agreement need not correlate as strongly (for evidence see \citealt[Chapter~2]{Wechsler+Zlatic:2003}).  The differences in feature sets and morphology further correlate with systematic syntactic differences, described in the following section.  


\section{Syntactic, semantic, and default agreement}
\label{pancake-sec}
 
This chapter has so far focused mainly on formal agreement, as opposed to semantic agreement.   But this is one of three different ways in which the form of an agreement target may be determined by a grammar:	

\begin{exe}
\ex   Formal, semantic, and default determinants of target form. 
\begin{xlist}
\ex	Formal agreement: The target form depends on the trigger's formal phi features.
\ex	Semantic `agreement’: The target form depends on the trigger's meaning.
\ex	Failure of agreement: The target fails to agree and hence takes its default form.
\end{xlist}
\end{exe}

\noindent
 In formal agreement, the trigger is grammatically specified for certain features as a consequence of the words making up the trigger phrase: for example a nominal may be marked for a gender as a consequence of the lexical gender of the head noun.  In semantic agreement, the target is sensitive to the meaning of the trigger instead of its formal features.  \ili{English} number agreement can be formal as in (\ref{clothes}), from \citet[92]{Wechsler:2013}, or semantic as in (\ref{impeached}), from \citet[92]{Mccloskey:1991}:

\begin{exe} 
\ex \label{clothes}
 \begin{xlist}
\ex   His clothes are/*is dirty.
\ex   His clothing is/*are dirty.
\end{xlist}
\end{exe}

\begin{exe} 
\ex \label{impeached}
 \begin{xlist}
\ex   That the position will be funded and that Mary will be hired now seems/??seem likely.
\ex  	That the president will be reelected and that he will be impeached are/??is equally likely at this point.
\end{xlist}
\end{exe}

\noindent
Regarding (\ref{impeached}), \citet[564--565]{Mccloskey:1991} observes that singular is used for ``a single complex state of affairs or situation-type'', while plural is possible for ``a plurality of distinct states of affairs or situation-types''.  The latter sort of interpretation is facilitated by the use of the adverb \textit{equally}.   Formal and semantic gender agreement are illustrated by the \ili{French} examples in (\ref{barbe}):

\begin{exe} 
\ex \label{barbe}
 \begin{xlist}
\ex   
\gll   La sentinelle	{\`{a} la barbe}	a \'{e}t\'{e}	\{ prise / *pris \} 	{en otage}.  \\
		the.\textsc{f} sentry	bearded	\textsc{aux} been	
		{} taken.\textsc{f.sg} {} taken.\textsc{m}	 {} hostage \\
\glt		`The bearded sentry was taken hostage.’
\ex   
\gll   Dupont	est	\{ comp\'{e}tent /		comp\'{e}tente \}. \\
		Dupont	is	{} competent.\textsc{m.sg} {}	competent.\textsc{f.sg} {} \\
\glt		`Dupont \{ a man / a woman \} is competent.’
 \end{xlist}
\end{exe} 

\noindent
The grammatically feminine noun \textit{sentinelle} `sentry'  triggers feminine agreement regardless of the sex of the sentry; but in (\ref{barbe}b) feminine agreement indicates that Dupont is female while masculine agreement indicates that Dupont is male.  

How does the grammar negotiate between formal and semantic agreement?  In HPSG, syntactic and semantic representations are composed in tandem, making the framework well suited to address this question.   It was addressed in early HPSG work, including \cite[Chapter~1]{Pollard+Sag:1994}.   The specific approach due to \cite{Wechsler:2011} exploits the underspecification of agreement features (see Section~\ref{unif-sec}).  I posit the Agreement Marking Principle (AMP), which states that target agreement features are semantically interpreted whenever the trigger is underspecified for the formal grammatical features to which the target would normally be sensitive.   The subject phrases in (\ref{clothes}) are specified for number due to the formal features of the head nouns, but those in (\ref{impeached}) are not, as a (coordinate) clause has no grammatical source for those features.  Consequently, by the AMP, the verb's number feature is semantically interpreted in (\ref{impeached}).  Similarly, \textit{sentinelle} in (\ref{barbe}a) gives its formal feminine gender feature to the subject, while \textit{Dupont} lacks a gender specification, triggering the semantic interpretation of the target adjectives in (\ref{barbe}b): feminine is interpreted as `female'.    

Agreement targets generally have a default form for use when there is no trigger or the normal agreement relation is blocked for some reason.   
Blocking of agreement comes about in various situations; here we  consider a case where the trigger is interpreted metonymically, \is{metonymy} apparently resulting in a reassignment of the referential index.  \ili{Swedish} predicate adjectives normally agree with their subjects in number (either singular or plural) and grammatical gender, either neuter (\textsc{nt}) or `common' gender (\textsc{com}), the gender held in common between masculine and feminine: 

\begin{exe} 
\ex \label{huset}
 \begin{xlist}
\ex 
\gll 	Hus-et 	är	gott. \\
		house-\textsc{def.n.sg}	is	good.\textsc{n.sg} \\
\glt	`The house is good.’
\ex   
\gll  Pannkaka-n 	är	god. \\
	pancake-\textsc{def.com.sg} 	be.\textsc{pres} 	good.\textsc{com.sg} \\
\glt	`The pancake is good.’
\ex   
\gll   \{   Hus-en / Pannkak-orna 	\}	är	god-a. \\
	   {}   house-\textsc{pl.def} {} pancake-\textsc{pl.def} {} be.\textsc{pres} good-\textsc{pl}  \\
\glt	`The houses / The pancakes are good.’
\end{xlist}
\end{exe}

\noindent
As shown in (\ref{huset}), a predicate adjective is inflected for number, and, in the singular, for gender, and agrees with its subject. But in sentences like (\ref{pannkakor}), the adjective appears in the neuter singular form, regardless of the number and gender features of the subject.  Note that \textit{pannkakor} is the plural form of a common gender noun \citep{Faarlund:1977, Enger:2004,Josefsson:2009}:

\begin{exe} 
\ex \label{pannkakor}
\gll   Pannkak-or 	är	gott. \\
	pancake-\textsc{pl}	be.\textsc{pres}	good.\textsc{n.sg} \\
\glt	`Situations involving pancakes are good.’ (e.g. `Eating pancakes is good.’)
\end{exe}

\noindent
In general, \ili{Swedish} predicate adjectives appear in neuter singular when there is no triggering NP, such as with clausal subjects (see (\ref{extrap}a) below).  \citet{Wechsler+Zlatic:2003} posit the index type \textit{unm} (`unmarked') for  referential indices that lack phi features, such as those introduced by verbs.   So \textit{gott} has a \subj list item whose index  is disjunctively specified for either neuter singular or type \textit{unm}.  

The lack of agreement in (\ref{pannkakor}) then arises because the subject phrase refers, not to the pancakes, but to a situation involving them; hence its referential index is distinct from the one lexically introduced by the noun \textit{pannkakor}.  A rule shifts the index and encodes the metonymic relation between the entity and the situation involving it.  This  is implemented with a non-branching phrasal construction in \citet[82, example\,(20)]{Wechsler:2013}:

\inlinetodostefan{Redo as Sag 1997 without \mother. In any case the \dtrs have to be a list.}
\begin{exe} 
\ex	
\label{metonymy}
\textit{metonymy-ctx:}\\
\avm{
[mtr &	[syn & NP \\
		sem &	[index & s$_{unm}$ \\
				restr & \{\type{involve}($s, i$)\} $\cup$ \1 ] ] \\ 
dtrs &	[syn & NP \\
		sem &	[index & i \\
				restr & \1 ] ]
] 
}
\end{exe}

\noindent
The noun \emph{pannkakor} in (\ref{pannkakor}) has an index marked with the features [\textsc{person}  \textit{3rd}], [\textsc{gender} \textit{com}], and [\textsc{number} \textit{pl}], which, by the Semantics Principle, are therefore shared with the index of the daughter NP node in a structure licensed by rule (\ref{metonymy}).  But the construction  specifies the mother NP node's index is unmarked for those features, thus explaining the neuter singular adjective.    

On the alternative ellipsis analysis, sentence (\ref{pannkakor}) has an elliptical clausal or infinitival subject, with a structure like (\ref{extrap}a) except that \emph{att \"{a}ta} is silent \citep{Faarlund:1977, Enger:2004,Josefsson:2009}:  

\begin{exe} 
\ex\label{extrap}
\begin{xlist}
\ex[ ]{ 
\gll   Att äta pannkakor är gott. \\
to eat pancakes be.\textsc{pres} good.\textsc{n.sg}  \\
\glt `Eating pancakes is good.’}
\ex[ ]{ 
\gll Det är gott att äta pannkakor. \\
it be.\textsc{pres} good.\textsc{n.sg}  to eat pancakes \\
\glt `It is good to eat pancakes.'}
\ex[*]{ 
\gll  Det är gott pannkakor.\\
 it be.\textsc{pres} good.\textsc{n.sg}  pancakes \\
\glt  Intended: `It is good  to eat pancakes.'}
\end{xlist}
\end{exe}

\noindent
But the metonymic subject behaves in all respects like an NP, and unlike a clause or infinitival phrase.  For example, unlike an infinitival it resists extraposition, as shown in (\ref{extrap}b, c).  The metonymy analysis captures the fact that the subject has a clause-like meaning but not clause-like syntax.  
  


\section{Mixed agreement}
\label{mismatch}
 

The two-feature (\textsc{index/concord}) theory of agreement was originally motivated by
\textit{mixed agreement}, where a single phrase triggers different features on distinct targets
\parencites[Chapter~2]{Pollard+Sag:1994}{Kathol99b}.  For example, the \ili{French} second person plural pronoun \textit{vous} refers to multiple addressees, and also has an honorific or polite use for a single (or multiple) addressee.  When used to refer politely to one addressee, \textit{vous} triggers singular on a predicate adjective but plural on the verb, as in (\ref{loyal}a):

\begin{exe} 
\ex\label{loyal}
\begin{xlist}
\ex
\gll   Vous		\^{e}tes		loyal. \\
		you.\textsc{pl}		be.\textsc{2pl}	loyal.\textsc{m.sg} \\
\glt		`You (singular, formal, male) are loyal.’ 
\ex 
\gll	Vous		\^{e}tes		loyaux. \\
		you.\textsc{pl}		be.\textsc{2pl}	loyal.\textsc{pl} \\
\glt		`You (plural) are loyal.’  
\end{xlist}
\end{exe}

\noindent
\citet{Wechsler:2011} analyzes this by adopting the following suppositions: (i) \textit{vous} has a  second person plural marked referential index; (ii) \textit{vous} lacks phi features for \textsc{concord}; (iii) finite verbs agree with their subjects in \textsc{index}; and (iv) predicate adjectives agree with their subjects in \textsc{concord}.  Suppositions (i) and (iii) need not be stipulated, as they follow from the theory:  the pronoun must have \textsc{index} phi features since it shows anaphoric agreement (when it serves as binder or bindee); and the verb must agree in \textsc{index} since it includes the \textsc{person} feature.  By the Agreement Marking Principle (see Section~\ref{pancake-sec}), the (\textsc{concord}) number and gender features of the predicate adjective are interpreted semantically, which is what is shown by example (\ref{loyal}). 

``Polite plural pronouns'' of this kind are found in many languages of the world \citep{Head:1978}.   The cross-linguistic agreement patterns observed in typological studies \citep{Comrie:1975,Wechsler:2011} confirm the predictions of the theory.  Taken together, suppositions (i) and (iii) from the previous paragraph entail that any person agreement targets agreeing with polite pronouns should show formal, rather than semantic, agreement.  Targets lacking person, meanwhile, can vary across languages.  This pattern is confirmed for all languages with polite plurals that have been surveyed, including \ili{Romance} languages; Modern \ili{Greek}; \ili{Germanic} (\ili{Icelandic}); West, South and East \ili{Slavic}; \ili{Hindi}; Gbaya (Niger-Congo);  Kobon and Usan (Papuan); and Sakha (Turkic) (see \citealt{Comrie:1975} and \citealt{Wechsler:2011}).   
 
The \textsc{index/concord} distinction plays a crucial role in this account of mixed agreement.  An earlier hypothesis, proposed by \citet{Kathol99b}, is that \ili{French} predicate adjectives are grammatically specified for semantic  agreement with their subjects, while finite verbs show formal agreement.  But a plurale tantum noun such as \textit{ciseaux} `scissors’ triggers syntactic agreement on the predicate adjective:
	
\begin{exe} 
\ex\label{ciseaux}
\gll   Ces	ciseaux	sont	g\'{e}niaux!	(*g\'{e}nial!) \\
		these.\textsc{pl}	scissors(\textsc{m.pl})	are.\textsc{pl} 	brilliant.\textsc{m.pl}	\hphantom{(*}brilliant.\textsc{m.sg} \\
\glt		`These scissors are cool!’ 
\end{exe}

\noindent
As far as the syntax is concerned, \textit{ciseaux} `scissors’ is an ordinary common noun with masculine plural \textsc{concord} features, so it triggers those features on the adjective.  More generally, agreement target types cannot be split into ``formal'' and ``semantic'' agreement targets; both formal and semantic agreement are found across all target types.  Which of the two is observed for a given agreement feature depends, according to the  \textsc{index/concord}  theory,  on whether the trigger is specified for the grammatical feature, together with the  \textsc{index}  versus  \textsc{concord}  status of the target.  
 

\section{Agreement defined on other structures}
So far our look at grammatical agreement has focused primarily on agreement defined on local grammatical relations like subject, object, and modifier.  In this section we look at HPSG analyses of two other types of agreement, namely long-distance and superficial agreement.  

\subsection{Long-distance agreement}
\label{LDA}

The simple picture of locality in the previous sections is challenged by the phenomenon of long-distance agreement, where the trigger appears within a clause subordinate to the one headed by the target verb.  Long-distance agreement has been observed in a number of languages, including \ili{Tsez} (Nakh-Dagestanian; \citealt{polinsky+potsdam:2001}), \ili{Hindi}-Urdu \citep{bhatt:2005}, and \ili{Passamaquoddy} (Athabaskan; \citealt{bruening:2001,LeSourd:2018}).  

\ili{Passamaquoddy} long-distance agreement is illustrated by this sentence \citep[example\,(5)]{LeSourd:2018}, with the relevant elements indicated in bold:
\inlinetodostefan{I would like to replace bold by italics.}

\begin{exe}
\ex\label{baskets}
\gll N-kosic\'{i}y-a-\textbf{k}  [eli- P\'{i}yel -litah\'{a}si-t           {[eli-kis-ankum-\'{i}-hti-t}   \textbf{n\`{i}kt}  \textbf{ehp\'{i}c-ik}	{poson\'{u}ti-yil]]} \\
     1-know-\textsc{dir-prox.pl} \spacebr{}thus- Peter -think-\textsc{3an} { \spacebr{}thus-\textsc{past}-sell-\textsc{3/1-prox.pl-3an}}         those.\textsc{prox}  woman-\textsc{prox.pl}     basket-\textsc{in.pl}\\
\glt `I know that Peter thinks that those women sold me the baskets.’
\end{exe}

\noindent
The \textit{-k} suffix on the matrix verb \textit{kosic\'{i}y} `know' marks plural, deictically proximate agreement with the phrase \textit{n\`{i}kt ehp\'{i}cik} `those women'  in the doubly embedded subordinate clause.  \citet{LeSourd:2018} analyzes \ili{Passamaquoddy} long distance agreement in the HPSG framework.   He notes that  \ili{Passamaquoddy} long distance agreement is parallelled by long-distance raising, in which an NP in the matrix clause is coreferential with an implicit argument of a subordinate clause \citep[example\,(4)]{LeSourd:2018}:  

\begin{exe}
\ex\label{baskets2}
\gll N-kosic\'{i}y-a-\textbf{k} \textbf{n\`{i}kt}  \textbf{ehp\'{i}c-ik}$_i$ [eli- P\'{i}yel -litah\'{a}si-t  [eli-kis-ankum-\'{i}-hti-t  \textbf{e$_i$} 	{poson\'{u}ti-yil]]} \\
     1-know-\textsc{dir-prox.pl} those.\textsc{prox}  woman-\textsc{prox.pl} \spacebr{}thus- Peter -think-\textsc{3an} \spacebr{}thus-\textsc{past}-sell-\textsc{3/1-prox.pl-3an}         {}    basket-\textsc{in.pl}\\
\glt `I know about those women$_i$ that Peter thinks that they$_i$ sold me the baskets.’
\end{exe}

\noindent
\ili{Passamaquoddy} speakers report that sentences (\ref{baskets}) and (\ref{baskets2}) suggest the subject of `know' (the speaker) is familiar with the women.  This provides evidence that the phrase `those women' in (\ref{baskets2}) is an argument of the matrix verb `know', as implied by the translation.  
%(In other words, this is a prolepsis construction.)
Similarly, the matrix clause (\ref{baskets}) contains a null argument (cross-referenced by the proximate plural \textit{-k}  suffix), which is  cataphoric to `those women'.  Hence a more literal translation of (\ref{baskets}) is `I know about them$_i$ that Peter thinks that those women$_i$ sold me the baskets.'\footnote{LeSourd notes that \ili{Passamaquoddy} lacks Principle~C effects, so cataphora of this kind is permitted.}  What the long-distance agreement and raising constructions share is simply that the matrix object is coreferential with some argument contained in the subordinate clause.  The following lexical entry for the verb root \textit{kosic\'{i}y} `know' captures that:

 \begin{exe}
\ex\label{know}
\textit{kosic\'{i}y} `know': \\
\avm{
[\phon < kosic\'{i}y > \\
arg-st & < NP$_i$ , NP$_j$, S:[restr < \ldots [prd|arg & j ] \ldots> ]> ]
}  
\end{exe}

\noindent
LeSourd adopts the version of HPSG described in the \citet{Sag+etal:2003} textbook, which uses a simplified Minimal Recursion Semantics. The semantic restrictions feature (\textsc{restr}) takes as its value a list of elementary predications (\textsc{prd}).  The list for each node is a concatenation of the restrictions of the daughter nodes.  Thus every semantic argument contained within the S complement, whether overt or null, will correspond to some argument (\textsc{arg}) of an elementary predication (\textsc{prd}) in S's \textsc{restr} list.  The lexical entry in (\ref{know}) stipulates that the matrix object NP corefers with some such argument.    In conclusion, \ili{Passamaquoddy} long-distance agreement is really the anaphoric agreement of a pronoun with an antecedent in a higher clause.  


\subsection{Superficial agreement}

 In some languages, string adjacency of the trigger and target, rather than a grammatical relation such as subject or modifier, is a grammatical condition on agreement. This may arise because person agreement derives historically from pronoun incorporation, and a basic syntactic precondition for incorporation is string adjacency between the pronoun and the head into which it incorporates \citep{givon:1976,ariel:1999,wechsler+epps+coppock:2010,fuss:2005}.  If the trigger occupies the syntactic position that the pronoun occupied prior to incorporation (for example because the trigger is itself a pronoun) then the result is that trigger and target are adjacent.  For example, West \ili{Flemish} complementizers agree with an immediately following subject, even though the complementizer and subject are not related by any grammatical relation \citep{Haegeman:1992}.  To take another example, \citet{Borsley:2009} analyzes \ili{Welsh} superficial agreement in the HPSG framework, citing examples like the following:

\begin{exe}
\ex \label{welsh}
\begin{xlist}
\ex
\gll 	Gwelon nhw ddraig. \\
see.\textsc{past.3pl} they dragon \\
\glt `They saw a dragon.’
\ex 
\gll 	arno fo \\
on.\textsc{3sg.m} he \\
\glt `on him’
\ex 
\gll 	Gweles i a Megan geffyl. \\
see.\textsc{past.1sg} I and Megan horse \\
\glt `Megan and I saw a horse.’
\end{xlist}
\end{exe}

\noindent
The trigger is the subject in (\ref{welsh}a), object in (\ref{welsh}b), and the first conjunct of a coordinate subject in (\ref{welsh}c).  But in every case, ``An agreeing element agrees with an immediately following noun phrase if and only if the latter is a pronoun'' \citep[example\,(48)]{Borsley:2009}. \citet[example\,(99)]{Borsley:2009} expresses this as an HPSG implicational constraint using the \textsc{dom}ain feature from linearization theory \citep{reape:1994,Mueller95c,Mueller99a,Kathol:2000}:

\begin{exe}
\ex \label{welshrule}
{}[\textsc{dom}  \liste{[\textsc{agr} \ibox{1}], NP: $ppro$\ind{2}, \ldots} ]  %$\rightarrow$
\impl  \ \ibox{1} = \ibox{2}
\end{exe}

\noindent
The \textsc{dom}ain list encodes linear precedence between constituents that are not necessarily sisters.  In (\ref{welshrule}) the \textsc{agr} value is the set of phi features of the target; the colon following NP represents the semantic \textsc{content} attribute; and the subscripted tag \ibox{2} is the \textsc{index} value.  The rule states that when a constituent bearing the \textsc{agr} attribute is immediately followed by a personal pronoun (content of type \textit{ppro}), then the \textsc{agr} value is identified with the pronoun's index (shown here as  \ibox{2}), that is, it agrees with a right-adjacent pronoun.  



\section{Conclusion} 

Agreement is analyzed in HPSG by assigning phi features to specific locations in the feature descriptions representing the grammar.  Anaphoric agreement results from phi features appearing on the referential indices of the binder and bindee, together with the assumption that binding consists of the identification of those indices.  Verbal agreement with subjects and objects results when phi features appear on the verb's \argst list items that are identified with the \textsc{synsem} values of the subject and object phrases.  Modifier agreement with heads occurs when phi features appear within the \textsc{mod} value of the modifier.  According to the \textsc{index}/\textsc{concord} theory, when agreement is historically descended from anaphoric agreement of incorporated pronouns, then those features within the \argst list or \textsc{mod} items are located  on the referential index; while otherwise they are collected in the \textsc{concord} feature and placed within the value of the \textsc{head} features.   The locality conditions on agreement follow from the normal operation of the grammar in which those phi features are embedded.  Some cases of agreement seem to exist outside those conditions.  Long-distance agreement has been analyzed as a kind of anaphoric agreement within a prolepsis construction, and superficial agreement has been defined on string adjacency and precedence, within linearization theory.  

\section*{Abbreviations}

\inlinetodostefan{add \textsc{pprt}}

%\section*{Acknowledgements}

%\printbibliography[heading=subbibliography,notkeyword=this]

%\end{document}


{\sloppy
\printbibliography[heading=subbibliography,notkeyword=this] 
}
\end{document}
