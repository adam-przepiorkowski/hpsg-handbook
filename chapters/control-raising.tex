%% -*- coding:utf-8 -*-

\documentclass[output=paper
	        ,collection
	        ,collectionchapter
 	        ,biblatex
                ,babelshorthands
                ,newtxmath
                ,draftmode
                ,colorlinks, citecolor=brown
]{langscibook}

\IfFileExists{../localcommands.tex}{%hack to check whether this is being compiled as part of a collection or standalone
  % add all extra packages you need to load to this file 

% the ISBN assigned to the digital edition
\usepackage[ISBN=9783961102556]{ean13isbn} 

\usepackage{graphicx}
\usepackage{tabularx}
\usepackage{amsmath} 

%\usepackage{tipa}      % Davis Koenig
\usepackage{xunicode} % Provide tipa macros (BC)

\usepackage{multicol}

% Berthold morphology
\usepackage{relsize}
%\usepackage{./styles/rtrees-bc} % forbidden forest 08.12.2019


\usepackage{langsci-optional} 
% used to be in this package
\providecommand{\citegen}{}
\renewcommand{\citegen}[2][]{\citeauthor{#2}'s (\citeyear*[#1]{#2})}
\providecommand{\lsptoprule}{}
\renewcommand{\lsptoprule}{\midrule\toprule}
\providecommand{\lspbottomrule}{}
\renewcommand{\lspbottomrule}{\bottomrule\midrule}
\providecommand{\largerpage}{}
\renewcommand{\largerpage}[1][1]{\enlargethispage{#1\baselineskip}}


\usepackage{langsci-lgr}

\newcommand{\MAS}{\textsc{m}\xspace} % \M is taken by somebody

%\usepackage{./styles/forest/forest}
\usepackage{langsci-forest-setup}

\usepackage{./styles/memoize/memoize} 
\memoizeset{
  memo filename prefix={chapters/hpsg-handbook.memo.dir/},
  register=\todo{O{}+m},
  prevent=\todo,
}

\usepackage{tikz-cd}

\usepackage{./styles/tikz-grid}
\usetikzlibrary{shadows}


% removed with texlive 2020 06.05.2020
% %\usepackage{pgfplots} % for data/theory figure in minimalism.tex
% % fix some issue with Mod https://tex.stackexchange.com/a/330076
% \makeatletter
% \let\pgfmathModX=\pgfmathMod@
% \usepackage{pgfplots}%
% \let\pgfmathMod@=\pgfmathModX
% \makeatother

\usepackage{subcaption}

% Stefan Müller's styles
\usepackage{./styles/merkmalstruktur,german,./styles/makros.2020,./styles/my-xspace,./styles/article-ex,
./styles/eng-date}

\selectlanguage{USenglish}

\usepackage{./styles/abbrev}


% Has to be loaded late since otherwise footnotes will not work

%%%%%%%%%%%%%%%%%%%%%%%%%%%%%%%%%%%%%%%%%%%%%%%%%%%%
%%%                                              %%%
%%%           Examples                           %%%
%%%                                              %%%
%%%%%%%%%%%%%%%%%%%%%%%%%%%%%%%%%%%%%%%%%%%%%%%%%%%%
% remove the percentage signs in the following lines
% if your book makes use of linguistic examples
\usepackage{langsci-gb4e} 


%% St. Mü.: 03.04.2020
%% these two versions of the command can be used for series of sets of examples:
%% \eal
%% \ex
%% \ex
%% \zlcont
%% \ealcont
%% \ex
%% \ex
%% \zl

\let\zlcont\z
\def\ealcont{\exnrfont\ex\begin{xlist}[iv.]\raggedright}

% original version of \z
%\def\z{\ifnum\@xnumdepth=1\end{exe}\else\end{xlist}\fi}
% \zcont just removes \end{exe}
%\def\zcont{\ifnum\@xnumdepth=1\else\end{xlist}\fi}
\def\zcont{}

% Crossing out text
% uncomment when needed
%\usepackage{ulem}

\usepackage{./styles/additional-langsci-index-shortcuts}

% this is the completely redone avm package
\usepackage{./styles/langsci-avm}
\avmsetup{columnsep=.3ex,style=narrow}

%\let\asort\type*


\usepackage{./styles/avm+}


\renewcommand{\tpv}[1]{{\avmjvalfont\itshape #1}}

% no small caps please
\renewcommand{\phonshape}[0]{\normalfont\itshape}

\regAvmFonts

\usepackage{theorem}

\newtheorem{mydefinition}{Def.}
\newtheorem{principle}{Principle}

{\theoremstyle{break}
%\newtheorem{schema}{Schema}
\newtheorem{mydefinition-break}[mydefinition]{Def.}
\newtheorem{principle-break}[principle]{Principle}
}


%% \newcommand{schema}[2]{
%% \begin{minipage}{\textwidth}
%% {\textbf{Schema~\theschema}}]\hspace{.5em}\textbf{(#1)}\\
%% #2
%% \end{minipage}}


% This avoids linebreaks in the Schema
\newcounter{schemacounter}
\makeatletter
\newenvironment{schema}[1][]
  {%
   \refstepcounter{schemacounter}%
   \par\bigskip\noindent
   \minipage{\linewidth}%
   \textbf{Schema~\theschemacounter\hspace{.5em} \ifx&#1&\else(#1)\fi}\par
  }{\endminipage\par\bigskip\@endparenv}%
\makeatother

%\usepackage{subfig}





% Davis Koenig Lexikon

\usepackage{tikz-qtree,tikz-qtree-compat} % Davis Koenig remove

\usepackage{shadow}



\usepackage[english]{isodate} % Andy Lücking
\usepackage[autostyle]{csquotes} % Andy
%\usepackage[autolanguage]{numprint}

%\defaultfontfeatures{
%    Path = /usr/local/texlive/2017/texmf-dist/fonts/opentype/public/fontawesome/ }

%% https://tex.stackexchange.com/a/316948/18561
%\defaultfontfeatures{Extension = .otf}% adds .otf to end of path when font loaded without ext parameter e.g. \newfontfamily{\FA}{FontAwesome} > \newfontfamily{\FA}{FontAwesome.otf}
%\usepackage{fontawesome} % Andy Lücking
\usepackage{pifont} % Andy Lücking -> hand

\usetikzlibrary{decorations.pathreplacing} % Andy Lücking
\usetikzlibrary{matrix} % Andy 
\usetikzlibrary{positioning} % Andy
\usepackage{tikz-3dplot} % Andy

% pragmatics
\usepackage{eqparbox} % Andy
\usepackage{enumitem} % Andy
\usepackage{longtable} % Andy
\usepackage{tabu} % Andy              needs to be loaded before hyperref as of texlive 2020

% tabu-fix
% to make "spread 0pt" work
% -----------------------------
\RequirePackage{etoolbox}
\makeatletter
\patchcmd
	\tabu@startpboxmeasure
	{\bgroup\begin{varwidth}}%
	{\bgroup
	 \iftabu@spread\color@begingroup\fi\begin{varwidth}}%
	{}{}
\def\@tabarray{\m@th\def\tabu@currentgrouptype
    {\currentgrouptype}\@ifnextchar[\@array{\@array[c]}}
%
%%% \pdfelapsedtime bug 2019-12-15
\patchcmd
	\tabu@message@etime
	{\the\pdfelapsedtime}%
	{\pdfelapsedtime}%
	{}{}
%
%
\makeatother
% -----------------------------


% Manfred's packages

%\usepackage{shadow}

\usepackage{tabularx}
\newcolumntype{L}[1]{>{\raggedright\arraybackslash}p{#1}} % linksbündig mit Breitenangabe


% Jong-Bok

%\usepackage{xytree}

\newcommand{\xytree}[2][dummy]{Let's do the tree!}

% seems evil, get rid of it
% defines \ex is incompatible with gb4e
%\usepackage{lingmacros}

% taken from lingmacros:
\makeatletter
% \evnup is used to line up the enumsentence number and an entry along
% the top.  It can take an argument to improve lining up.
\def\evnup{\@ifnextchar[{\@evnup}{\@evnup[0pt]}}

\def\@evnup[#1]#2{\setbox1=\hbox{#2}%
\dimen1=\ht1 \advance\dimen1 by -.5\baselineskip%
\advance\dimen1 by -#1%
\leavevmode\lower\dimen1\box1}
\makeatother


% YK -- CG chapter

%\usepackage{xspace}
\usepackage{bm}
\usepackage{ebproof}


% Antonio Branco, remove this
\usepackage{epsfig}

% now unicode
%\usepackage{alphabeta}





\usepackage{pst-node}


% fmr: additional packages
%\usepackage{amsthm}


% Ash and Steve: LFG
\usepackage{./styles/lfg/dalrymple}

\RequirePackage{graphics}
%\RequirePackage{./styles/lfg/trees}
%% \RequirePackage{avm}
%% \avmoptions{active}
%% \avmfont{\sc}
%% \avmvalfont{\sc}
\RequirePackage{./styles/lfg/lfgmacrosash}

\usepackage{./styles/lfg/glue}

%%%%%%%%%%%%%%%%%%%%%%%%%%%%%%
%% Markup
%%%%%%%%%%%%%%%%%%%%%%%%%%%%%%
\usepackage[normalem]{ulem} % For thinks like strikethrough, using \sout

% \newcommand{\high}[1]{\textbf{#1}} % highlighted text
\newcommand{\high}[1]{\textit{#1}} % highlighted text
%\newcommand{\term}[1]{\textit{#1}\/} % technical term
\newcommand{\qterm}[1]{`{#1}'} % technical term, quotes
%\newcommand{\trns}[1]{\strut `#1'} % translation in glossed example
\newcommand{\trnss}[1]{\strut \phantom{\sqz{}} `#1'} % translation in ungrammatical glossed example
\newcommand{\ttrns}[1]{(`#1')} % an in-text translation of a word
%\newcommand{\feat}[1]{\mbox{\textsc{\MakeLowercase{#1}}}}     % feature name
%\newcommand{\val}[1]{\mbox{\textsc{\MakeLowercase{#1}}}}    % f-structure value
\newcommand{\featt}[1]{\mbox{\textsc{\MakeLowercase{#1}}}}     % feature name
\newcommand{\vall}[1]{\mbox{\textsc{\textup{\MakeLowercase{#1}}}}}    % f-structure value
\newcommand{\mg}[1]{\mbox{\textsc{\MakeLowercase{#1}}}}    % morphological gloss
%\newcommand{\word}[1]{\textit{#1}}       % mention of word
\providecommand{\kstar}[1]{{#1}\ensuremath{^*}}
\providecommand{\kplus}[1]{{#1}\ensuremath{^+}}
\newcommand{\template}[1]{@\textsc{\MakeLowercase{#1}}}
\newcommand{\templaten}[1]{\textsc{\MakeLowercase{#1}}}
\newcommand{\templatenn}[1]{\MakeUppercase{#1}}
\newcommand{\tempeq}{\ensuremath{=}}
\newcommand{\predval}[1]{\ensuremath{\langle}\textsc{#1}\ensuremath{\rangle}}
\newcommand{\predvall}[1]{{\rm `#1'}}
\newcommand{\lfgfst}[1]{\ensuremath{#1\,}}
\newcommand{\scare}[1]{`#1'} % scare quotes
\newcommand{\bracket}[1]{\ensuremath{\left\langle\mathit{#1}\right\rangle}}
\newcommand{\sectionw}[1][]{Section#1} % section word: for cap/non-cap
\newcommand{\tablew}[1][]{Table#1} % table word: for cap/non-cap
\newcommand{\lfgglue}{LFG+Glue}
\newcommand{\hpsgglue}{HPSG+Glue}
\newcommand{\gs}{GS}
%\newcommand{\func}[1]{\ensuremath{\mathbf{#1}}}
\newcommand{\func}[1]{\textbf{#1}}
\renewcommand{\glue}{Glue}
%\newcommand{\exr}[1]{(\ref{ex:#1}}
\newcommand{\exra}[1]{(\ref{ex:#1})}


%%%%%%%%%%%%%%%%%%%%%%%%%%%%%%
% Notation
%\newcommand{\xbar}[1]{$_{\mbox{\textsc{#1}$^{\raisebox{1ex}{}}$}}$}
\newcommand{\xprime}[2][]{\textup{\mbox{{#2}\ensuremath{^\prime_{\hspace*{-.0em}\mbox{\footnotesize\ensuremath{\mathit{#1}}}}}}}}
\providecommand{\xzero}[2][]{#2\ensuremath{^0_{\mbox{\footnotesize\ensuremath{\mathit{#1}}}}}}



\let\leftangle\langle
\let\rightangle\rangle

%\newcommand{\pslabel}[1]{}



  %add all your local new commands to this file


% Don't do this at home. I do not like the smaller font for captions.
% I just removed loading the caption packege in langscibook.cls
%% \captionsetup{%
%% font={%
%% stretch=1%.8%
%% ,normalsize%,small%
%% },%
%% width=.8\textwidth
%% }

\makeatletter
\def\blx@maxline{77}
\makeatother


\newcommand{\page}{}

\newcommand{\todostefan}[1]{\todo[color=orange!80]{\footnotesize #1}\xspace}
\newcommand{\todosatz}[1]{\todo[color=red!40]{\footnotesize #1}\xspace}

\newcommand{\inlinetodostefan}[1]{\todo[color=green!40,inline]{\footnotesize #1}\xspace}

\newcommand{\addpages}{\todostefan{add pages}}
\newcommand{\addglosses}{\todostefan{add glosses}}


\newcommand{\spacebr}{\hspaceThis{[}}

\newcommand{\danish}{\jambox{(\ili{Danish})}}
\newcommand{\english}{\jambox{(\ili{English})}}
\newcommand{\german}{\jambox{(\ili{German})}}
\newcommand{\yiddish}{\jambox{(\ili{Yiddish})}}
\newcommand{\welsh}{\jambox{(\ili{Welsh})}}

% Cite and cross-reference other chapters
\newcommand{\crossrefchaptert}[2][]{\citet*[#1]{chapters/#2}, Chapter~\ref{chap-#2} of this volume} 
\newcommand{\crossrefchapterp}[2][]{(\citealp*[#1]{chapters/#2}, Chapter~\ref{chap-#2} of this volume)}
\newcommand{\crossrefchapteralt}[2][]{\citealt*[#1]{chapters/#2}, Chapter~\ref{chap-#2} of this volume}
\newcommand{\crossrefchapteralp}[2][]{\citealp*[#1]{chapters/#2}, Chapter~\ref{chap-#2} of this volume}
% example of optional argument:
% \crossrefchapterp[for something, see:]{name}
% gives: (for something, see: Author 2018, Chapter~X of this volume)

\let\crossrefchapterw\crossrefchaptert



% Davis Koenig

\let\ig=\textsc
\let\tc=\textcolor

% evolution, Flickinger, Pollard, Wasow

\let\citeNP\citet

% Adam P

%\newcommand{\toappear}{Forthcoming}
\newcommand{\pg}[1]{p.\,#1}
\renewcommand{\implies}{\rightarrow}

\newcommand*{\rref}[1]{(\ref{#1})}
\newcommand*{\aref}[1]{(\ref{#1}a)}
\newcommand*{\bref}[1]{(\ref{#1}b)}
\newcommand*{\cref}[1]{(\ref{#1}c)}

\newcommand{\msadam}{.}
\newcommand{\morsyn}[1]{\textsc{#1}}

\newcommand{\nom}{\morsyn{nom}}
\newcommand{\acc}{\morsyn{acc}}
\newcommand{\dat}{\morsyn{dat}}
\newcommand{\gen}{\morsyn{gen}}
\newcommand{\ins}{\morsyn{ins}}
%\newcommand{\aploc}{\morsyn{loc}}
\newcommand{\voc}{\morsyn{voc}}
\newcommand{\ill}{\morsyn{ill}}
\renewcommand{\inf}{\morsyn{inf}}
\newcommand{\passprc}{\morsyn{passp}}

%\newcommand{\Nom}{\msadam\nom}
%\newcommand{\Acc}{\msadam\acc}
%\newcommand{\Dat}{\msadam\dat}
%\newcommand{\Gen}{\msadam\gen}
\newcommand{\Ins}{\msadam\ins}
\newcommand{\Loc}{\msadam\loc}
\newcommand{\Voc}{\msadam\voc}
\newcommand{\Ill}{\msadam\ill}
\newcommand{\PassP}{\msadam\passprc}

\newcommand{\Aux}{\textsc{aux}}

\newcommand{\princ}[1]{\textnormal{\textsc{#1}}} % for constraint names
\newcommand{\notion}[1]{\emph{#1}}
\renewcommand{\path}[1]{\textnormal{\textsc{#1}}}
\newcommand{\ftype}[1]{\textit{#1}}
\newcommand{\fftype}[1]{{\scriptsize\textit{#1}}}
\newcommand{\la}{$\langle$}
\newcommand{\ra}{$\rangle$}
%\newcommand{\argst}{\path{arg-st}}
\newcommand{\phtm}[1]{\setbox0=\hbox{#1}\hspace{\wd0}}
\newcommand{\prep}[1]{\setbox0=\hbox{#1}\hspace{-1\wd0}#1}

%%%%%%%%%%%%%%%%%%%%%%%%%%%%%%%%%%%%%%%%%%%%%%%%%%%%%%%%%%%%%%%%%%%%%%%%%%%

% FROM FS.STY:

%%%
%%% Feature structures
%%%

% \fs         To print a feature structure by itself, type for example
%             \fs{case:nom \\ person:P}
%             or (better, for true italics),
%             \fs{\it case:nom \\ \it person:P}
%
% \lfs        To print the same feature structure with the category
%             label N at the top, type:
%             \lfs{N}{\it case:nom \\ \it person:P}

%    Modified 1990 Dec 5 so that features are left aligned.
\newcommand{\fs}[1]%
{\mbox{\small%
$
\!
\left[
  \!\!
  \begin{tabular}{l}
    #1
  \end{tabular}
  \!\!
\right]
\!
$}}

%     Modified 1990 Dec 5 so that features are left aligned.
%\newcommand{\lfs}[2]
%   {
%     \mbox{$
%           \!\!
%           \begin{tabular}{c}
%           \it #1
%           \\
%           \mbox{\small%
%                 $
%                 \left[
%                 \!\!
%                 \it
%                 \begin{tabular}{l}
%                 #2
%                 \end{tabular}
%                 \!\!
%                 \right]
%                 $}
%           \end{tabular}
%           \!\!
%           $}
%   }

\newcommand{\ft}[2]{\path{#1}\hspace{1ex}\ftype{#2}}
\newcommand{\fsl}[2]{\fs{{\fftype{#1}} \\ #2}}

\newcommand{\fslt}[2]
 {\fst{
       {\fftype{#1}} \\
       #2 
     }
 }

\newcommand{\fsltt}[2]
 {\fstt{
       {\fftype{#1}} \\
       #2 
     }
 }

\newcommand{\fslttt}[2]
 {\fsttt{
       {\fftype{#1}} \\
       #2 
     }
 }


% jak \ft, \fs i \fsl tylko nieco ciasniejsze

\newcommand{\ftt}[2]
% {{\sc #1}\/{\rm #2}}
 {\textsc{#1}\/{\rm #2}}

\newcommand{\fst}[1]
  {
    \mbox{\small%
          $
          \left[
          \!\!\!
%          \sc
          \begin{tabular}{l} #1
          \end{tabular}
          \!\!\!\!\!\!\!
          \right]
          $
          }
   }

%\newcommand{\fslt}[2]
% {\fst{#2\\
%       {\scriptsize\it #1}
%      }
% }


% superciasne

\newcommand{\fstt}[1]
  {
    \mbox{\small%
          $
          \left[
          \!\!\!
%          \sc
          \begin{tabular}{l} #1
          \end{tabular}
          \!\!\!\!\!\!\!\!\!\!\!
          \right]
          $
          }
   }

%\newcommand{\fsltt}[2]
% {\fstt{#2\\
%       {\scriptsize\it #1}
%      }
% }

\newcommand{\fsttt}[1]
  {
    \mbox{\small%
          $
          \left[
          \!\!\!
%          \sc
          \begin{tabular}{l} #1
          \end{tabular}
          \!\!\!\!\!\!\!\!\!\!\!\!\!\!\!\!
          \right]
          $
          }
   }



% %add all your local new commands to this file

% \newcommand{\smiley}{:)}

% you are not supposed to mess with hardcore stuff, St.Mü. 22.08.2018
%% \renewbibmacro*{index:name}[5]{%
%%   \usebibmacro{index:entry}{#1}
%%     {\iffieldundef{usera}{}{\thefield{usera}\actualoperator}\mkbibindexname{#2}{#3}{#4}{#5}}}

% % \newcommand{\noop}[1]{}



% Rui

\newcommand{\spc}[0]{\hspace{-1pt}\underline{\hspace{6pt}}\,}
\newcommand{\spcs}[0]{\hspace{-1pt}\underline{\hspace{6pt}}\,\,}
\newcommand{\bad}[1]{\leavevmode\llap{#1}}
\newcommand{\COMMENT}[1]{}


% Rui coordination
\newcommand{\subl}[1]{$_{\scriptstyle \textsc{#1}}$}



% Andy Lücking gesture.tex
\newcommand{\Pointing}{\ding{43}}
% Giotto: "Meeting of Joachim and Anne at the Golden Gate" - 1305-10 
\definecolor{GoldenGate1}{rgb}{.13,.09,.13} % Dress of woman in black
\definecolor{GoldenGate2}{rgb}{.94,.94,.91} % Bridge
\definecolor{GoldenGate3}{rgb}{.06,.09,.22} % Blue sky
\definecolor{GoldenGate4}{rgb}{.94,.91,.87} % Dress of woman with shawl
\definecolor{GoldenGate5}{rgb}{.52,.26,.26} % Joachim's robe
\definecolor{GoldenGate6}{rgb}{.65,.35,.16} % Anne's robe
\definecolor{GoldenGate7}{rgb}{.91,.84,.42} % Joachim's halo

\makeatletter
\newcommand{\@Depth}{1} % x-dimension, to front
\newcommand{\@Height}{1} % z-dimension, up
\newcommand{\@Width}{1} % y-dimension, rightwards
%\GGS{<x-start>}{<y-start>}{<z-top>}{<z-bottom>}{<Farbe>}{<x-width>}{<y-depth>}{<opacity>}
\newcommand{\GGS}[9][]{%
\coordinate (O) at (#2-1,#3-1,#5);
\coordinate (A) at (#2-1,#3-1+#7,#5);
\coordinate (B) at (#2-1,#3-1+#7,#4);
\coordinate (C) at (#2-1,#3-1,#4);
\coordinate (D) at (#2-1+#8,#3-1,#5);
\coordinate (E) at (#2-1+#8,#3-1+#7,#5);
\coordinate (F) at (#2-1+#8,#3-1+#7,#4);
\coordinate (G) at (#2-1+#8,#3-1,#4);
\draw[draw=black, fill=#6, fill opacity=#9] (D) -- (E) -- (F) -- (G) -- cycle;% Front
\draw[draw=black, fill=#6, fill opacity=#9] (C) -- (B) -- (F) -- (G) -- cycle;% Top
\draw[draw=black, fill=#6, fill opacity=#9] (A) -- (B) -- (F) -- (E) -- cycle;% Right
}
\makeatother


% pragmatics
\newcommand{\speaking}[1]{\eqparbox{name}{\textsc{\lowercase{#1}\space}}}
\newcommand{\alname}[1]{\eqparbox{name}{\textsc{\lowercase{#1}}}}
\newcommand{\HPSGTTR}{HPSG$_{\text{TTR}}$\xspace}

\newcommand{\ttrtype}[1]{\textit{#1}}
\newcommand{\avmel}{\q<\quad\q>} %% shortcut for empty lists in AVM
\newcommand{\ttrmerge}{\ensuremath{\wedge_{\textit{merge}}}}
\newcommand{\Cat}[2][0.1pt]{%
  \begin{scope}[y=#1,x=#1,yscale=-1, inner sep=0pt, outer sep=0pt]
   \path[fill=#2,line join=miter,line cap=butt,even odd rule,line width=0.8pt]
  (151.3490,307.2045) -- (264.3490,307.2045) .. controls (264.3490,291.1410) and (263.2021,287.9545) .. (236.5990,287.9545) .. controls (240.8490,275.2045) and (258.1242,244.3581) .. (267.7240,244.3581) .. controls (276.2171,244.3581) and (286.3490,244.8259) .. (286.3490,264.2045) .. controls (286.3490,286.2045) and (323.3717,321.6755) .. (332.3490,307.2045) .. controls (345.7277,285.6390) and (309.3490,292.2151) .. (309.3490,240.2046) .. controls (309.3490,169.0514) and (350.8742,179.1807) .. (350.8742,139.2046) .. controls (350.8742,119.2045) and (345.3490,116.5037) .. (345.3490,102.2045) .. controls (345.3490,83.3070) and (361.9972,84.4036) .. (358.7581,68.7349) .. controls (356.5206,57.9117) and (354.7696,49.2320) .. (353.4652,36.1439) .. controls (352.5396,26.8573) and (352.2445,16.9594) .. (342.5985,17.3574) .. controls (331.2650,17.8250) and (326.9655,37.7742) .. (309.3490,39.2045) .. controls (291.7685,40.6320) and (276.7783,24.2380) .. (269.9740,26.5795) .. controls (263.2271,28.9013) and (265.3490,47.2045) .. (269.3490,60.2045) .. controls (275.6359,80.6368) and (289.3490,107.2045) .. (264.3490,111.2045) .. controls (239.3490,115.2045) and (196.3490,119.2045) .. (165.3490,160.2046) .. controls (134.3490,201.2046) and (135.4934,249.3212) .. (123.3490,264.2045) .. controls (82.5907,314.1553) and (40.8239,293.6463) .. (40.8239,335.2045) .. controls (40.8239,353.8102) and (72.3490,367.2045) .. (77.3490,361.2045) .. controls (82.3490,355.2045) and (34.8638,337.3259) .. (87.9955,316.2045) .. controls (133.3871,298.1601) and   (137.4391,294.4766) .. (151.3490,307.2045) -- cycle;
\end{scope}%
}


% KdK
\newcommand{\smiley}{:)}

\renewbibmacro*{index:name}[5]{%
  \usebibmacro{index:entry}{#1}
    {\iffieldundef{usera}{}{\thefield{usera}\actualoperator}\mkbibindexname{#2}{#3}{#4}{#5}}}

% \newcommand{\noop}[1]{}

% chngcntr.sty otherwise gives error that these are already defined
%\let\counterwithin\relax
%\let\counterwithout\relax

% the space of a left bracket for glossings
\newcommand{\LB}{\hspaceThis{[}}

\newcommand{\LF}{\mbox{$[\![$}}

\newcommand{\RF}{\mbox{$]\!]_F$}}

\newcommand{\RT}{\mbox{$]\!]_T$}}





% Manfred's

\newcommand{\kommentar}[1]{}

\newcommand{\bsp}[1]{\emph{#1}}
\newcommand{\bspT}[2]{\bsp{#1} `#2'}
\newcommand{\bspTL}[3]{\bsp{#1} (lit.: #2) `#3'}

\newcommand{\noidi}{§}

\newcommand{\refer}[1]{(\ref{#1})}

%\newcommand{\avmtype}[1]{\multicolumn{2}{l}{\type{#1}}}
\newcommand{\attr}[1]{\textsc{#1}}

\newcommand{\srdefault}{\mbox{\begin{tabular}{c}{\large <}\\[-1.5ex]$\sqcap$\end{tabular}}}

%% \newcommand{\myappcolumn}[2]{
%% \begin{minipage}[t]{#1}#2\end{minipage}
%% }

%% \newcommand{\appc}[1]{\myappcolumn{3.7cm}{#1}}


% Jong-Bok


% clean that up and do not use \def (killing other stuff defined before)
%\if 0
\newcommand\DEL{\textsc{del}}
\newcommand\del{\textsc{del}}

\newcommand\conn{\textsc{conn}}
\newcommand\CONN{\textsc{conn}}
\newcommand\CONJ{\textsc{conj}}
\newcommand\LITE{\textsc{lex}}
\newcommand\lite{\textsc{lex}}
\newcommand\HON{\textsc{hon}}

%\newcommand\CAUS{\textsc{caus}}
%\newcommand\PASS{\textsc{pass}}
\newcommand\NPST{\textsc{npst}}
%\newcommand\COND{\textsc{cond}}



\newcommand\hdlite{\textsc{head-lex construction}}
\newcommand\hdlight{\textsc{head-light} Schema}
\newcommand\NFORM{\textsc{nform}}

\newcommand\RELS{\textsc{rels}}
%\newcommand\TENSE{\textsc{tense}}


%\newcommand\ARG{\textsc{arg}}
\newcommand\ARGs{\textsc{arg0}}
\newcommand\ARGa{\textsc{arg}}
\newcommand\ARGb{\textsc{arg2}}
\newcommand\TPC{\textsc{top}}
%\newcommand\PROG{\textsc{prog}}

\newcommand\LIGHT{\textsc{light}\xspace}
\newcommand\pst{\textsc{pst}}
%\newcommand\PAST{\textsc{pst}}
%\newcommand\DAT{\textsc{dat}}
%\newcommand\CONJ{\textsc{conj}}
\newcommand\nominal{\textsc{nominal}}
\newcommand\NOMINAL{\textsc{nominal}}
\newcommand\VAL{\textsc{val}}
%\newcommand\val{\textsc{val}}
\newcommand\MODE{\textsc{mode}}
\newcommand\RESTR{\textsc{restr}}
\newcommand\SIT{\textsc{sit}}
\newcommand\ARG{\textsc{arg}}
\newcommand\RELN{\textsc{rel}}
%\newcommand\REL{\textsc{rel}}
%\newcommand\RELS{\textsc{rels}}
%\newcommand\arg-st{\textsc{arg-st}}
\newcommand\xdel{\textsc{xdel}}
\newcommand\zdel{\textsc{zdel}}
\newcommand\sug{\textsc{sug}}
%\newcommand\IMP{\textsc{imp}}
%\newcommand\conn{\textsc{conn}}
%\newcommand\CONJ{\textsc{conj}}
%\newcommand\HON{\textsc{hon}}
\newcommand\BN{\textsc{bn}}
\newcommand\bn{\textsc{bn}}
\newcommand\pres{\textsc{pres}}
\newcommand\PRES{\textsc{pres}}
\newcommand\prs{\textsc{pres}}
%\newcommand\PRS{\textsc{pres}}
\newcommand\agt{\textsc{agt}}
%\newcommand\DEL{\textsc{del}}
%\newcommand\PRED{\textsc{pred}}
\newcommand\AGENT{\textsc{agent}}
\newcommand\THEME{\textsc{theme}}
%\newcommand\AUX{\textsc{aux}}
%\newcommand\THEME{\textsc{theme}}
%\newcommand\PL{\textsc{pl}}
\newcommand\SRC{\textsc{src}}
\newcommand\src{\textsc{src}}
\newcommand{\FORMjb}{\textsc{form}}
\newcommand{\formjb}{\FORM}
\newcommand\GCASE{\textsc{gcase}}
\newcommand\gcase{\textsc{gcase}}
\newcommand\SCASE{\textsc{scase}}
\newcommand\PHON{\textsc{phon}}
%\newcommand\SS{\textsc{ss}}
\newcommand\SYN{\textsc{syn}}
%\newcommand\LOC{\textsc{loc}}
\newcommand\MOD{\textsc{mod}}
\newcommand\INV{\textsc{inv}}
%\newcommand\L{\textsc{l}}
%\newcommand\CASE{\textsc{case}}
\newcommand\SPR{\textsc{spr}}
\newcommand\COMPS{\textsc{comps}}
%\newcommand\comps{\textsc{comps}}
\newcommand\SEM{\textsc{sem}}
\newcommand\CONT{\textsc{cont}}
\newcommand\SUBCAT{\textsc{subcat}}
\newcommand\CAT{\textsc{cat}}
%\newcommand\C{\textsc{c}}
%\newcommand\SUBJ{\textsc{subj}}
\newcommand\subjjb{\textsc{subj}}
%\newcommand\SLASH{\textsc{slash}}
\newcommand\LOCAL{\textsc{local}}
%\newcommand\ARG-ST{\textsc{arg-st}}
%\newcommand\AGR{\textsc{agr}}
\newcommand\PER{\textsc{per}}
%\newcommand\NUM{\textsc{num}}
%\newcommand\IND{\textsc{ind}}
\newcommand\VFORM{\textsc{vform}}
\newcommand\PFORM{\textsc{pform}}
\newcommand\decl{\textsc{decl}}
%\newcommand\loc{\textsc{loc   }}
% \newcommand\   {\textsc{  }}

%\newcommand\NEG{\textsc{neg}}
\newcommand\FRAMES{\textsc{frames}}
%\newcommand\REFL{\textsc{refl}}

\newcommand\MKG{\textsc{mkg}}

%\newcommand\BN{\textsc{bn}}
\newcommand\HD{\textsc{hd}}
\newcommand\NP{\textsc{np}}
\newcommand\PF{\textsc{pf}}
%\newcommand\PL{\textsc{pl}}
\newcommand\PP{\textsc{pp}}
%\newcommand\SS{\textsc{ss}}
\newcommand\VF{\textsc{vf}}
\newcommand\VP{\textsc{vp}}
%\newcommand\bn{\textsc{bn}}
\newcommand\cl{\textsc{cl}}
%\newcommand\pl{\textsc{pl}}
\newcommand\Wh{\ital{Wh}}
%\newcommand\ng{\textsc{neg}}
\newcommand\wh{\ital{wh}}
%\newcommand\ACC{\textsc{acc}}
%\newcommand\AGR{\textsc{agr}}
\newcommand\AGT{\textsc{agt}}
\newcommand\ARC{\textsc{arc}}
%\newcommand\ARG{\textsc{arg}}
\newcommand\ARP{\textsc{arc}}
%\newcommand\AUX{\textsc{aux}}
%\newcommand\CAT{\textsc{cat}}
%\newcommand\COP{\textsc{cop}}
%\newcommand\DAT{\textsc{dat}}
\newcommand\NEWCOMMAND{\textsc{def}}
%\newcommand\DEL{\textsc{del}}
\newcommand\DOM{\textsc{dom}}
\newcommand\DTR{\textsc{dtr}}
%\newcommand\FUT{\textsc{fut}}
\newcommand\GAP{\textsc{gap}}
%\newcommand\GEN{\textsc{gen}}
%\newcommand\HON{\textsc{hon}}
%\newcommand\IMP{\textsc{imp}}
%\newcommand\IND{\textsc{ind}}
%\newcommand\INV{\textsc{inv}}
\newcommand\LEX{\textsc{lex}}
\newcommand\Lex{\textsc{lex}}
%\newcommand\LOC{\textsc{loc}}
%\newcommand\MOD{\textsc{mod}}
\newcommand\MRK{{\nr MRK}}
%\newcommand\NEG{\textsc{neg}}
\newcommand\NEW{\textsc{new}}
%\newcommand\NOM{\textsc{nom}}
%\newcommand\NUM{\textsc{num}}
%\newcommand\PER{\textsc{per}}
%\newcommand\PST{\textsc{pst}}
\newcommand\QUE{\textsc{que}}
%\newcommand\REL{\textsc{rel}}
\newcommand\SEL{\textsc{sel}}
%\newcommand\SEM{\textsc{sem}}
%\newcommand\SIT{\textsc{arg0}}
%\newcommand\SPR{\textsc{spr}}
%\newcommand\SRC{\textsc{src}}
\newcommand\SUG{\textsc{sug}}
%\newcommand\SYN{\textsc{syn}}
%\newcommand\TPC{\textsc{top}}
%\newcommand\VAL{\textsc{val}}
%\newcommand\acc{\textsc{acc}}
%\newcommand\agt{\textsc{agt}}
\newcommand\cop{\textsc{cop}}
%\newcommand\dat{\textsc{dat}}
\newcommand\foc{\textsc{focus}}
%\newcommand\FOC{\textsc{focus}}
\newcommand\fut{\textsc{fut}}
\newcommand\hon{\textsc{hon}}
\newcommand\imp{\textsc{imp}}
\newcommand\kes{\textsc{kes}}
%\newcommand\lex{\textsc{lex}}
%\newcommand\loc{\textsc{loc}}
\newcommand\mrk{{\nr MRK}}
%\newcommand\nom{\textsc{nom}}
%\newcommand\num{\textsc{num}}
\newcommand\plu{\textsc{plu}}
\newcommand\pne{\textsc{pne}}
%\newcommand\pst{\textsc{pst}}
\newcommand\pur{\textsc{pur}}
%\newcommand\que{\textsc{que}}
%\newcommand\src{\textsc{src}}
%\newcommand\sug{\textsc{sug}}
\newcommand\tpc{\textsc{top}}
%\newcommand\utt{\textsc{utt}}
%\newcommand\val{\textsc{val}}
%% \newcommand\LITE{\textsc{lex}}
%% \newcommand\PAST{\textsc{pst}}
%% \newcommand\POSP{\textsc{pos}}
%% \newcommand\PRS{\textsc{pres}}
%% \newcommand\mod{\textsc{mod}}%
%% \newcommand\newuse{{`kes'}}
%% \newcommand\posp{\textsc{pos}}
%% \newcommand\prs{\textsc{pres}}
%% \newcommand\psp{{\it en\/}}
%% \newcommand\skes{\textsc{kes}}
%% \newcommand\CASE{\textsc{case}}
%% \newcommand\CASE{\textsc{case}}
%% \newcommand\COMP{\textsc{comp}}
%% \newcommand\CONJ{\textsc{conj}}
%% \newcommand\CONN{\textsc{conn}}
%% \newcommand\CONT{\textsc{cont}}
%% \newcommand\DECL{\textsc{decl}}
%% \newcommand\FOCUS{\textsc{focus}}
%% %\newcommand\FORM{\textsc{form}} duplicate
%% \newcommand\FREL{\textsc{frel}}
%% \newcommand\GOAL{\textsc{goal}}
\newcommand\HEAD{\textsc{head}}
%% \newcommand\INDEX{\textsc{ind}}
%% \newcommand\INST{\textsc{inst}}
%% \newcommand\MODE{\textsc{mode}}
%% \newcommand\MOOD{\textsc{mood}}
%% \newcommand\NMLZ{\textsc{nmlz}}
%% \newcommand\PHON{\textsc{phon}}
%% \newcommand\PRED{\textsc{pred}}
%% %\newcommand\PRES{\textsc{pres}}
%% \newcommand\PROM{\textsc{prom}}
%% \newcommand\RELN{\textsc{pred}}
%% \newcommand\RELS{\textsc{rels}}
%% \newcommand\STEM{\textsc{stem}}
%% \newcommand\SUBJ{\textsc{subj}}
%% \newcommand\XARG{\textsc{xarg}}
%% \newcommand\bse{{\it bse\/}}
%% \newcommand\case{\textsc{case}}
%% \newcommand\caus{\textsc{caus}}
%% \newcommand\comp{\textsc{comp}}
%% \newcommand\conj{\textsc{conj}}
%% \newcommand\conn{\textsc{conn}}
%% \newcommand\decl{\textsc{decl}}
%% \newcommand\fin{{\it fin\/}}
%% %\newcommand\form{\textsc{form}}
%% \newcommand\gend{\textsc{gend}}
%% \newcommand\inf{{\it inf\/}}
%% \newcommand\mood{\textsc{mood}}
%% \newcommand\nmlz{\textsc{nmlz}}
%% \newcommand\pass{\textsc{pass}}
%% \newcommand\past{\textsc{past}}
%% \newcommand\perf{\textsc{perf}}
%% \newcommand\pln{{\it pln\/}}
%% \newcommand\pred{\textsc{pred}}


%% %\newcommand\pres{\textsc{pres}}
%% \newcommand\proc{\textsc{proc}}
%% \newcommand\nonfin{{\it nonfin\/}}
%% \newcommand\AGENT{\textsc{agent}}
%% \newcommand\CFORM{\textsc{cform}}
%% %\newcommand\COMPS{\textsc{comps}}
%% \newcommand\COORD{\textsc{coord}}
%% \newcommand\COUNT{\textsc{count}}
%% \newcommand\EXTRA{\textsc{extra}}
%% \newcommand\GCASE{\textsc{gcase}}
%% \newcommand\GIVEN{\textsc{given}}
%% \newcommand\LOCAL{\textsc{local}}
%% \newcommand\NFORM{\textsc{nform}}
%% \newcommand\PFORM{\textsc{pform}}
%% \newcommand\SCASE{\textsc{scase}}
%% \newcommand\SLASH{\textsc{slash}}
%% \newcommand\SLASH{\textsc{slash}}
%% \newcommand\THEME{\textsc{theme}}
%% \newcommand\TOPIC{\textsc{topic}}
%% \newcommand\VFORM{\textsc{vform}}
%% \newcommand\cause{\textsc{cause}}
%% %\newcommand\comps{\textsc{comps}}
%% \newcommand\gcase{\textsc{gcase}}
%% \newcommand\itkes{{\it kes\/}}
%% \newcommand\pass{{\it pass\/}}
%% \newcommand\vform{\textsc{vform}}
%% \newcommand\CCONT{\textsc{c-cont}}
%% \newcommand\GN{\textsc{given-new}}
%% \newcommand\INFO{\textsc{info-st}}
%% \newcommand\ARG-ST{\textsc{arg-st}}
%% \newcommand\SUBCAT{\textsc{subcat}}
%% \newcommand\SYNSEM{\textsc{synsem}}
%% \newcommand\VERBAL{\textsc{verbal}}
%% \newcommand\arg-st{\textsc{arg-st}}
%% \newcommand\plain{{\it plain}\/}
%% \newcommand\propos{\textsc{propos}}
%% \newcommand\ADVERBIAL{\textsc{advl}}
%% \newcommand\HIGHLIGHT{\textsc{prom}}
%% \newcommand\NOMINAL{\textsc{nominal}}

\newenvironment{myavm}{\begingroup\avmvskip{.1ex}
  \selectfont\begin{avm}}%
{\end{avm}\endgroup\medskip}
\newcommand\pfix{\vspace{-5pt}}


\newcommand{\jbsub}[1]{\lower4pt\hbox{\small #1}}
\newcommand{\jbssub}[1]{\lower4pt\hbox{\small #1}}
\newcommand\jbtr{\underbar{\ \ \ }\ }


%\fi

% cl

\newcommand{\delphin}{\textsc{delph-in}}


% YK -- CG chapter

\newcommand{\grey}[1]{\colorbox{mycolor}{#1}}
\definecolor{mycolor}{gray}{0.8}

\newcommand{\GQU}[2]{\raisebox{1.6ex}{\ensuremath{\rotatebox{180}{\textbf{#1}}_{\scalebox{.7}{\textbf{#2}}}}}}

\newcommand{\SetInfLen}{\setpremisesend{0pt}\setpremisesspace{10pt}\setnamespace{0pt}}

\newcommand{\pt}[1]{\ensuremath{\mathsf{#1}}}
\newcommand{\ptv}[1]{\ensuremath{\textsf{\textsl{#1}}}}

\newcommand{\sv}[1]{\ensuremath{\bm{\mathcal{#1}}}}
\newcommand{\sX}{\sv{X}}
\newcommand{\sF}{\sv{F}}
\newcommand{\sG}{\sv{G}}

\newcommand{\syncat}[1]{\textrm{#1}}
\newcommand{\syncatVar}[1]{\ensuremath{\mathit{#1}}}

\newcommand{\RuleName}[1]{\textrm{#1}}

\newcommand{\SemTyp}{\textsf{Sem}}

\newcommand{\E}{\ensuremath{\bm{\epsilon}}\xspace}

\newcommand{\greeka}{\upalpha}
\newcommand{\greekb}{\upbeta}
\newcommand{\greekd}{\updelta}
\newcommand{\greekp}{\upvarphi}
\newcommand{\greekr}{\uprho}
\newcommand{\greeks}{\upsigma}
\newcommand{\greekt}{\uptau}
\newcommand{\greeko}{\upomega}
\newcommand{\greekz}{\upzeta}

\newcommand{\Lemma}{\ensuremath{\hskip.5em\vdots\hskip.5em}\noLine}
\newcommand{\LemmaAlt}{\ensuremath{\hskip.5em\vdots\hskip.5em}}

\newcommand{\I}{\iota}

\newcommand{\sem}{\ensuremath}

\newcommand{\NoSem}{%
\renewcommand{\LexEnt}[3]{##1; \syncat{##3}}
\renewcommand{\LexEntTwoLine}[3]{\renewcommand{\arraystretch}{.8}%
\begin{array}[b]{l} ##1;  \\ \syncat{##3} \end{array}}
\renewcommand{\LexEntThreeLine}[3]{\renewcommand{\arraystretch}{.8}%
\begin{array}[b]{l} ##1; \\ \syncat{##3} \end{array}}}

\newcommand{\hypml}[2]{\left[\!\!#1\!\!\right]^{#2}}

%%%%for bussproof
\def\defaultHypSeparation{\hskip0.1in}
\def\ScoreOverhang{0pt}

\newcommand{\MultiLine}[1]{\renewcommand{\arraystretch}{.8}%
\ensuremath{\begin{array}[b]{l} #1 \end{array}}}

\newcommand{\MultiLineMod}[1]{%
\ensuremath{\begin{array}[t]{l} #1 \end{array}}}

\newcommand{\hypothesis}[2]{[ #1 ]^{#2}}

\newcommand{\LexEnt}[3]{#1; \ensuremath{#2}; \syncat{#3}}

\newcommand{\LexEntTwoLine}[3]{\renewcommand{\arraystretch}{.8}%
\begin{array}[b]{l} #1; \\ \ensuremath{#2};  \syncat{#3} \end{array}}

\newcommand{\LexEntThreeLine}[3]{\renewcommand{\arraystretch}{.8}%
\begin{array}[b]{l} #1; \\ \ensuremath{#2}; \\ \syncat{#3} \end{array}}

\newcommand{\LexEntFiveLine}[5]{\renewcommand{\arraystretch}{.8}%
\begin{array}{l} #1 \\ #2; \\ \ensuremath{#3} \\ \ensuremath{#4}; \\ \syncat{#5} \end{array}}

\newcommand{\LexEntFourLine}[4]{\renewcommand{\arraystretch}{.8}%
\begin{array}{l} \pt{#1} \\ \pt{#2}; \\ \syncat{#4} \end{array}}

\newcommand{\ManySomething}{\renewcommand{\arraystretch}{.8}%
\raisebox{-3mm}{\begin{array}[b]{c} \vdots \,\,\,\,\,\, \vdots \\
\vdots \end{array}}}

\newcommand{\lemma}[1]{\renewcommand{\arraystretch}{.8}%
\begin{array}[b]{c} \vdots \\ #1 \end{array}}

\newcommand{\lemmarev}[1]{\renewcommand{\arraystretch}{.8}%
\begin{array}[b]{c} #1 \\ \vdots \end{array}}

\newcommand{\p}{\ensuremath{\upvarphi}}

% clashes with soul package
\newcommand{\yusukest}{\textbf{\textsf{st}}}

\newcommand{\shortarrow}{\xspace\hskip-1.2ex\scalebox{.5}[1]{\ensuremath{\bm{\rightarrow}}}\hskip-.5ex\xspace}

\newcommand{\SemInt}[1]{\mbox{$[\![ \textrm{#1} ]\!]$}}

\newcommand{\HypSpace}{\hskip-.8ex}
\newcommand{\RaiseHeight}{\raisebox{2.2ex}}
\newcommand{\RaiseHeightLess}{\raisebox{1ex}}

\newcommand{\ThreeColHyp}[1]{\RaiseHeight{\Bigg[}\HypSpace#1\HypSpace\RaiseHeight{\Bigg]}}
\newcommand{\TwoColHyp}[1]{\RaiseHeightLess{\Big[}\HypSpace#1\HypSpace\RaiseHeightLess{\Big]}}

\newcommand{\LemmaShort}{\ensuremath{ \ \vdots} \ \noLine}
\newcommand{\LemmaShortAlt}{\ensuremath{ \ \vdots} \ }

\newcommand{\fail}{**}
\newcommand{\vs}{\raisebox{.05em}{\ensuremath{\upharpoonright}}}
\newcommand{\DerivSize}{\small}

\def\maru#1{{\ooalign{\hfil
  \ifnum#1>999 \resizebox{.25\width}{\height}{#1}\else%
  \ifnum#1>99 \resizebox{.33\width}{\height}{#1}\else%
  \ifnum#1>9 \resizebox{.5\width}{\height}{#1}\else #1%
  \fi\fi\fi%
\/\hfil\crcr%
\raise.167ex\hbox{\mathhexbox20D}}}}

\newenvironment{samepage2}%
 {\begin{flushleft}\begin{minipage}{\linewidth}}
 {\end{minipage}\end{flushleft}}

\newcommand{\cmt}[1]{\textsl{\textbf{[#1]}}}
\newcommand{\trns}[1]{\textbf{#1}\xspace}
\newcommand{\ptfont}{}
\newcommand{\gp}{\underline{\phantom{oo}}}
\newcommand{\mgcmt}{\marginnote}

\newcommand{\term}[1]{\emph{#1}}

\newcommand{\citeposs}[1]{\citeauthor{#1}'s \citeyearpar{#1}}

% for standalone compilations Felix: This is in the class already
%\let\thetitle\@title
%\let\theauthor\@author 
\makeatletter
\newcommand{\togglepaper}[1][0]{ 
\bibliography{../Bibliographies/stmue,../localbibliography,
../Bibliographies/properties,
../Bibliographies/np,
../Bibliographies/negation,
../Bibliographies/ellipsis,
../Bibliographies/binding,
../Bibliographies/complex-predicates,
../Bibliographies/control-raising,
../Bibliographies/coordination,
../Bibliographies/morphology,
../Bibliographies/lfg,
collection.bib}
  %% hyphenation points for line breaks
%% Normally, automatic hyphenation in LaTeX is very good
%% If a word is mis-hyphenated, add it to this file
%%
%% add information to TeX file before \begin{document} with:
%% %% hyphenation points for line breaks
%% Normally, automatic hyphenation in LaTeX is very good
%% If a word is mis-hyphenated, add it to this file
%%
%% add information to TeX file before \begin{document} with:
%% \include{localhyphenation}
\hyphenation{
A-la-hver-dzhie-va
anaph-o-ra
ana-phor
ana-phors
an-te-ced-ent
an-te-ced-ents
affri-ca-te
affri-ca-tes
ap-proach-es
Atha-bas-kan
Athe-nä-um
Bona-mi
Chi-che-ŵa
com-ple-ments
con-straints
Cope-sta-ke
Da-ge-stan
Dor-drecht
er-klä-ren-de
Ginz-burg
Gro-ning-en
Jap-a-nese
Jon-a-than
Ka-tho-lie-ke
Ko-bon
krie-gen
Le-Sourd
moth-er
Mül-ler
Nie-mey-er
Par-a-digm
Prze-piór-kow-ski
phe-nom-e-non
re-nowned
Rie-he-mann
un-bound-ed
with-in
}

% listing within here does not have any effect for lfg.tex % 2020-05-14

% why has "erklärende" be listed here? I specified langid in bibtex item. Something is still not working with hyphenation.


% to do: check
%  Alahverdzhieva


% biblatex:

% This is a LaTeX frontend to TeX’s \hyphenation command which defines hy- phenation exceptions. The ⟨language⟩ must be a language name known to the babel/polyglossia packages. The ⟨text ⟩ is a whitespace-separated list of words. Hyphenation points are marked with a dash:

% \DefineHyphenationExceptions{american}{%
% hy-phen-ation ex-cep-tion }

\hyphenation{
A-la-hver-dzhie-va
anaph-o-ra
ana-phor
ana-phors
an-te-ced-ent
an-te-ced-ents
affri-ca-te
affri-ca-tes
ap-proach-es
Atha-bas-kan
Athe-nä-um
Bona-mi
Chi-che-ŵa
com-ple-ments
con-straints
Cope-sta-ke
Da-ge-stan
Dor-drecht
er-klä-ren-de
Ginz-burg
Gro-ning-en
Jap-a-nese
Jon-a-than
Ka-tho-lie-ke
Ko-bon
krie-gen
Le-Sourd
moth-er
Mül-ler
Nie-mey-er
Par-a-digm
Prze-piór-kow-ski
phe-nom-e-non
re-nowned
Rie-he-mann
un-bound-ed
with-in
}

% listing within here does not have any effect for lfg.tex % 2020-05-14

% why has "erklärende" be listed here? I specified langid in bibtex item. Something is still not working with hyphenation.


% to do: check
%  Alahverdzhieva


% biblatex:

% This is a LaTeX frontend to TeX’s \hyphenation command which defines hy- phenation exceptions. The ⟨language⟩ must be a language name known to the babel/polyglossia packages. The ⟨text ⟩ is a whitespace-separated list of words. Hyphenation points are marked with a dash:

% \DefineHyphenationExceptions{american}{%
% hy-phen-ation ex-cep-tion }

  \memoizeset{
    memo filename prefix={hpsg-handbook.memo.dir/},
    % readonly
  }
  \papernote{\scriptsize\normalfont
    \@author.
    \@title. 
    To appear in: 
    Stefan Müller, Anne Abeillé, Robert D. Borsley \& Jean-Pierre Koenig (eds.)
    HPSG Handbook
    Berlin: Language Science Press. [preliminary page numbering]
  }
  \pagenumbering{roman}
  \setcounter{chapter}{#1}
  \addtocounter{chapter}{-1}
}
\makeatother

\makeatletter
\newcommand{\togglepaperminimal}[1][0]{ 
  \bibliography{../Bibliographies/stmue,
                ../localbibliography,
  ../Bibliographies/coordination,
collection.bib}
  %% hyphenation points for line breaks
%% Normally, automatic hyphenation in LaTeX is very good
%% If a word is mis-hyphenated, add it to this file
%%
%% add information to TeX file before \begin{document} with:
%% %% hyphenation points for line breaks
%% Normally, automatic hyphenation in LaTeX is very good
%% If a word is mis-hyphenated, add it to this file
%%
%% add information to TeX file before \begin{document} with:
%% \include{localhyphenation}
\hyphenation{
A-la-hver-dzhie-va
anaph-o-ra
ana-phor
ana-phors
an-te-ced-ent
an-te-ced-ents
affri-ca-te
affri-ca-tes
ap-proach-es
Atha-bas-kan
Athe-nä-um
Bona-mi
Chi-che-ŵa
com-ple-ments
con-straints
Cope-sta-ke
Da-ge-stan
Dor-drecht
er-klä-ren-de
Ginz-burg
Gro-ning-en
Jap-a-nese
Jon-a-than
Ka-tho-lie-ke
Ko-bon
krie-gen
Le-Sourd
moth-er
Mül-ler
Nie-mey-er
Par-a-digm
Prze-piór-kow-ski
phe-nom-e-non
re-nowned
Rie-he-mann
un-bound-ed
with-in
}

% listing within here does not have any effect for lfg.tex % 2020-05-14

% why has "erklärende" be listed here? I specified langid in bibtex item. Something is still not working with hyphenation.


% to do: check
%  Alahverdzhieva


% biblatex:

% This is a LaTeX frontend to TeX’s \hyphenation command which defines hy- phenation exceptions. The ⟨language⟩ must be a language name known to the babel/polyglossia packages. The ⟨text ⟩ is a whitespace-separated list of words. Hyphenation points are marked with a dash:

% \DefineHyphenationExceptions{american}{%
% hy-phen-ation ex-cep-tion }

\hyphenation{
A-la-hver-dzhie-va
anaph-o-ra
ana-phor
ana-phors
an-te-ced-ent
an-te-ced-ents
affri-ca-te
affri-ca-tes
ap-proach-es
Atha-bas-kan
Athe-nä-um
Bona-mi
Chi-che-ŵa
com-ple-ments
con-straints
Cope-sta-ke
Da-ge-stan
Dor-drecht
er-klä-ren-de
Ginz-burg
Gro-ning-en
Jap-a-nese
Jon-a-than
Ka-tho-lie-ke
Ko-bon
krie-gen
Le-Sourd
moth-er
Mül-ler
Nie-mey-er
Par-a-digm
Prze-piór-kow-ski
phe-nom-e-non
re-nowned
Rie-he-mann
un-bound-ed
with-in
}

% listing within here does not have any effect for lfg.tex % 2020-05-14

% why has "erklärende" be listed here? I specified langid in bibtex item. Something is still not working with hyphenation.


% to do: check
%  Alahverdzhieva


% biblatex:

% This is a LaTeX frontend to TeX’s \hyphenation command which defines hy- phenation exceptions. The ⟨language⟩ must be a language name known to the babel/polyglossia packages. The ⟨text ⟩ is a whitespace-separated list of words. Hyphenation points are marked with a dash:

% \DefineHyphenationExceptions{american}{%
% hy-phen-ation ex-cep-tion }

  \memoizeset{
    memo filename prefix={hpsg-handbook.memo.dir/},
    % readonly
  }
  \papernote{\scriptsize\normalfont
    \@author.
    \@title. 
    To appear in: 
    Stefan Müller, Anne Abeillé, Robert D. Borsley \& Jean-Pierre Koenig (eds.)
    HPSG Handbook
    Berlin: Language Science Press. [preliminary page numbering]
  }
  \pagenumbering{roman}
  \setcounter{chapter}{#1}
  \addtocounter{chapter}{-1}
}
\makeatother




% In case that year is not given, but pubstate. This mainly occurs for titles that are forthcoming, in press, etc.
\renewbibmacro*{addendum+pubstate}{% Thanks to https://tex.stackexchange.com/a/154367 for the idea
  \printfield{addendum}%
  \iffieldequalstr{labeldatesource}{pubstate}{}
  {\newunit\newblock\printfield{pubstate}}
}

\DeclareLabeldate{%
    \field{date}
    \field{year}
    \field{eventdate}
    \field{origdate}
    \field{urldate}
    \field{pubstate}
    \literal{nodate}
}

%\defbibheading{diachrony-sources}{\section*{Sources}} 

% if no langid is set, it is English:
% https://tex.stackexchange.com/a/279302
\DeclareSourcemap{
  \maps[datatype=bibtex]{
    \map{
      \step[fieldset=langid, fieldvalue={english}]
    }
  }
}


% for bibliographies
% biber/biblatex could use sortname field rather than messing around this way.
\newcommand{\SortNoop}[1]{}


% Doug Ball

\newcommand{\elist}{\q<\ \ \q>}

\newcommand{\esetDB}{\q\{\ \ \q\}}


\makeatletter

\newcommand{\nolistbreak}{%

  \let\oldpar\par\def\par{\oldpar\nobreak}% Any \par issues a \nobreak

  \@nobreaktrue% Don't break with first \item

}

\makeatother


% intermediate before Frank's trees are fixed
% This will be removed!!!!!
%\newcommand{\tree}[1]{} % ignore them blody trees
%\usepackage{tree-dvips}


\newcommand{\nodeconnect}[2]{}
\newcommand{\nodetriangle}[2]{}



% Doug relative clauses
%% I've compiled out almost all my private LaTeX command, but there are some
%% I found hard to get rid of. They are defined here.
%% There are few others which defined in places in the document where they have only
%% local effect (e.g. within figures); their names all end in DA, e.g. \MotherDA
%% There are a lot of \labels -- they are all of the form \label{sec:rc-...} or
%% \label{x:rc-...} or similar, so there should be no clashes.

% Subscripts -- scriptsize italic shape lowered by .25ex 
\newcommand{\subscr}[1]{\raisebox{-.5ex}{\protect{\scriptsize{\itshape #1\/}}}}
% A boxed subscript, for avm tags in normal text
\newcommand{\subtag}[1]{\subscr{\idx{#1}}}

%% Sets and tuples: I use \setof{} to get brackets that are upright, not slanted
%\newcommand{\setof}[1]{\ensuremath{\lbrace\,\mathit{#1}\,\rbrace}}
% 11.10.2019 EP: Doug requested replacement of existing \setof definition with the following:
%\newcommand{\setof}[1]{\begin{avm}\{\textcolor{red}{#1}\}\end{avm}}
% 31.1.2019 EP: Doug requested re-replacement of the above \textcolour version with the following:
\newcommand{\setof}[1]{\begin{avm}\{#1\}\end{avm}}

\newcommand{\tuple}[1]{\ensuremath{\left\langle\,\mbox{\textit{#1}}\,\right\rangle}}

% Single pile of stuff, optional arugment is psn (e.g. t or b)
% e.g. to put a over b over c in a centered column, top aligned, do:
%   \cPile[t]{a\\b\\c} 
\newcommand{\cPile}[2][]{%
  \begingroup%
  \renewcommand{\arraystretch}{.5}\begin{tabular}[#1]{c}#2\end{tabular}%
  \endgroup%
}

%% for linguistic examples in running text (`linguistic citation'):
\newcommand{\lic}[1]{\textit{#1}}

%% A gap marked by an underline, raised slightly
%% Default argument indicates how long the line should be:
\newcommand{\uGap}[1][3ex]{\raisebox{.25em}{\underline{\hspace{#1}}}\xspace}

%% \TnodeDA{XP}{avmcontents} -- in a Tree, put a node label next to an AVM
\newcommand{\TnodeDA}[2]{#1~\begin{avm}{#2}\end{avm}}

%% This allows tipa stuff to be put in \emph -- we need to change to cmr first.
%% It is used in the discussion of Arabic.
\newcommand{\emphtipa}[1]{{\fontfamily{cmr}\emph{\tipaencoding #1}}} 



 
 
\definecolor{lsDOIGray}{cmyk}{0,0,0,0.45}


% morphology.tex:
% Berthold

\newcommand{\dnode}[1]{\rnode{#1}{\fbox{#1}}}
\newcommand{\tnode}[1]{\rnode{#1}{\textit{#1}}}

\newcommand{\tl}[2]{#2}

\newcommand{\rrr}[3]{%
  \psframebox[linestyle=none]{%
    \avmoptions{center}
    \begin{avm}
      \[mud & \{ #1 \}\\
      ms & \{ #2 \}\\
      mph & \<  #3 \> \]
    \end{avm}
  }
}
\newcommand{\rr}[2]{%
  \psframebox[linestyle=none]{%
    \avmoptions{center}
    \begin{avm}
      \[mud & \{ #1 \}\\
      mph & \<  #2 \> \]
    \end{avm}
  }
}
 

% Frank Richter
\newtheorem{mydef}{Definition}

\long\def\set[#1\set=#2\set]%
{%
\left\{%
\tabcolsep 1pt%
\begin{tabular}{l}%
#1%
\end{tabular}%
\left|%
\tabcolsep 1pt%
\begin{tabular}{l}%
#2%
\end{tabular}%
\right.%
\right\}%
}

\newcommand{\einruck}{\\ \hspace*{1em}}


%\newcommand{\NatNum}{\mathrm{I\hspace{-.17em}N}}
\newcommand{\NatNum}{\mathbb{N}}
\newcommand{\Aug}[1]{\widehat{#1}}
%\newcommand{\its}{\mathrm{:}}
% Felix 14.02.2020
\DeclareMathOperator{\its}{:}

\newcommand{\sequence}[1]{\langle#1\rangle}

\newcommand{\INTERPRETATION}[2]{\sequence{#1\mathsf{U}#2,#1\mathsf{S}#2,#1\mathsf{A}#2,#1\mathsf{R}#2}}
\newcommand{\Interpretation}{\INTERPRETATION{}{}}

\newcommand{\Inte}{\mathsf{I}}
\newcommand{\Unive}{\mathsf{U}}
\newcommand{\Speci}{\mathsf{S}}
\newcommand{\Atti}{\mathsf{A}}
\newcommand{\Reli}{\mathsf{R}}
\newcommand{\ReliT}{\mathsf{RT}}

\newcommand{\VarInt}{\mathsf{G}}
\newcommand{\CInt}{\mathsf{C}}
\newcommand{\Tinte}{\mathsf{T}}
\newcommand{\Dinte}{\mathsf{D}}

% this was missing from ash's stuff.

%% \def \optrulenode#1{
%%   \setbox1\hbox{$\left(\hbox{\begin{tabular}{@{\strut}c@{\strut}}#1\end{tabular}}\right)$}
%%   \raisebox{1.9ex}{\raisebox{-\ht1}{\copy1}}}



\newcommand{\pslabel}[1]{}

\newcommand{\addpagesunless}{\todostefan{add pages unless you cite the
 work as such}}

% dg.tex
% framed boxes as used in dg.tex
% original idea from stackexchange, but modified by Saso
% http://tex.stackexchange.com/questions/230300/doing-something-like-psframebox-in-tikz#230306
\tikzset{
  frbox/.style={
    rounded corners,
    draw,
    thick,
    inner sep=5pt,
    anchor=base,
  },
}

% get rid of these morewrite messages:
% https://tex.stackexchange.com/questions/419489/suppressing-messages-to-standard-output-from-package-morewrites/419494#419494
\ExplSyntaxOn
\cs_set_protected:Npn \__morewrites_shipout_ii:
  {
    \__morewrites_before_shipout:
    \__morewrites_tex_shipout:w \tex_box:D \g__morewrites_shipout_box
    \edef\tmp{\interactionmode\the\interactionmode\space}\batchmode\__morewrites_after_shipout:\tmp
  }
\ExplSyntaxOff


% This is for places where authors used bold. I replace them by \emph
% but have the information where the bold was. St. Mü. 09.05.2020
\newcommand{\textbfemph}[1]{\emph{#1}}



% Felix 09.06.2020: copy code from the third line into localcommands.tex: https://github.com/langsci/langscibook#defined-environments-commands-etc
\patchcmd{\mkbibindexname}{\ifdefvoid{#3}{}{\MakeCapital{#3} }}{\ifdefvoid{#3}{}{#3 }}{}{\AtEndDocument{\typeout{mkbibindexname could not be patched.}}}

  \togglepaper[13]
}{}
%\forestset{external/readonly}


\author{%
	Anne Abeillé\affiliation{Université de Paris}%
}
\title{Control and Raising}

% \chapterDOI{} %will be filled in at production

%\epigram{Change epigram in chapters/03.tex or remove it there }
\abstract{The distinction between raising and control predicates has been a hallmark of syntactic theory since \citet{Rosenbaum67a-u}, \citet{Postal1974}. Unlike transformational analyses, HPSG treats the difference as mainly a semantic one: raising verbs (\word{seem}, \word{begin}, \word{expect}) do not semantically select their subject (or object) nor assign them a semantic role, while control verbs (\word{want}, \word{promise}, \word{persuade}) semantically select all their syntactic arguments. On the syntactic side, raising verbs share their subject (or object) with the unexpressed subject of their non"=finite complement while control verbs only coindex them \citet{PollardandSag1994}. We will provide creole data (from Mauritian) which support a phrasal analysis of their complement, and argue against a clausal (or small clause) analysis (\citet{HenriandLaurens2011}. The distinction is also relevant for non"=verbal predicates such as adjectives (\word{likely} vs \word{eager}). The raising analysis naturally extends to copular constructions (\word{become}, \word{consider}) and most auxiliary verbs \citet{PollardandSag1994 \citet{Sagetal2020}.}}

\begin{document}
\maketitle
\label{chap-control-raising}
\avmoptions{center}


%\section{Introduction}
%\label{control-sec-intro}
\todostefan{everywhere:  Please change the ref of Bresnan 1982 to Control and complementation. Linguistic Inquiry Vol. 13, No. 3, pp. 343-434}

\section{The distinction between raising and control predicates}

\subsection{The main distinction between raising and control verbs}

In a broad sense, 'control' refers to a relation of referential dependence between an unexpressed subject (the controlled element) and an expressed or unexpressed constituent (the controller); the referential properties of the controlled element, including possibly the property of having no reference at all, are determined by those of the controller (Bresnan 1982:372 ). Verbs taking non"=finite complements usually determine the interpretation of the unexpressed subject of the non"=finite verb. With \word{want}, the subject is understood as the subject of the infinitive, while with \word{persuade} it is the object, as shown by the reflexives (\ref{equi1}), (\ref{equi2}). They are called 'control' verbs, and John is called the 'controller' in (\ref{equi1}) while it is Mary in (\ref{equi2}).

	\begin{exe}
	\ex \begin{xlist}
	\ex John wants to buy himself a coat. \label{equi1}
   \ex 	John persuaded Mary to buy herself / * himself a coat.\label{equi2}
 \end{xlist}
 \end{exe}


Another type of verb also takes a non-finite complement and identifies its subject (or its object) with the unexpressed subject of the non-finite verb. Since \citet{Postal1974}, they are called 'raising' verbs. What semantic role the missing subject has if any is determined by the lower verb, of if that is a raising verb, an even lower verb. In (\ref{seem}) the subject of the infinitive (\emph{like}) is understood to the be the subject of \word{seem}, while in (\ref{exp}) the subject of the non-finite verb (\emph{buy}) is understood to be the object of \word{expect}. Verbs like \word{seem} are called 'subject-to-subject-raising' (or 'subject-raising') verbs, while
verbs like \word{expect} are called 'subject-to-object-raising' (or 'object-raising') verbs.

\begin{exe}
	\ex \begin{xlist}
	\ex John seemed to like himself.\label{seem}
\ex  John expected Mary to buy herself / * himself a coat. \label{exp}
\end{xlist}
 \end{exe}
 
 Raising and control constructions differ from other constructions in which the missing subject remains vague (\ref{arbitrary}) and which are a case of 'arbitrary' or 'anaphoric' control \citep{Chomsky1981:75-76, Bresnan1982:379} \footnote{Bresnan 1982 proposes a non-transformational analysis and renames 'raising' 'functional' control and 'control' (obligatory) 'anaphoric' control.}.
 
\begin{exe}
 \ex Buying a coat can be expensive.\label{arbitrary}
  \end{exe}
  
 A number of syntactic and semantic properties show how control verbs like \emph{want, hope, force, persuade, promise}, differ from raising verbs like \emph{see, seem, start, believe, expect} \citep{Rosenbaum67a-u,Postal1974,Bresnan1982}.\footnote{The same distinction is available for verbs taking a gerund-participle complement:
Kim remembered seeing Lee. (control) vs  Kim started singing. (raising).}
 The key point is that there is a semantic role associated with the subject position with verbs like \emph{want} but not with verbs like \emph{seem} and with the post-verbal position with verbs like \emph{persuade} but not with verbs like \emph{expect}.  The consequence is that more or less any NP is possible as subject of \emph{seem} and as the post-verbal NP after \emph{expect}. This includes expletive \emph{it} and \emph{there} and idiom-chunks. \\
     Let us first consider non"=referential subjects: meteorological \word{it} is selected
 by predicates such as \word{rain}. It can be the subject of \word{start}, \word{seem}, but not of
 \word{hope}, \word{want}. It can be the object of \word{expect}, \word{believe} but not of \word{force}, \word{persuade}.
	
\eal
\ex[]{
It rained this morning.
}
\ex[]{
It seems/started to rain this morning.
} \label{rain1}
\ex[]{
We expect it to rain tomorrow. 
} \label{rain2}
\zl
\eal
\ex[\#]{
It wants/hopes to rain tomorrow.
} \label{rain3}
\ex[\#]{
The sorcerer forced it to rain.
} \label{rain4}
\zl
 	
 The same contrast holds with an idiomatic subject such as \word{the cat} in the expression \word{the cat is out of the bag} (the secret is out). It can be the subject of \word{seem} or the object of \word{expect}, with its idiomatic meaning. If it is the subject of \word{want} or the object of \word{persuade}, the idiomatic meaning is lost and only the literal meaning remains.
 
\eal
\judgewidth{\#}
\ex[]{
The cat is out of the bag.
}
\ex[]{
The cat seems to be out of the bag. 
} \label{cat1}
\ex[]{
We expected the cat to be out of the bag. 
} \label{cat2}
\ex[\#]{
The cat wants to be out of the bag.\hfill(non"=idiomatic)
} \label{cat3}
\ex[\#]{
We persuaded the cat to be out of the bag.\hfill(non"=idiomatic)
} \label{cat4}
\zl

Let us now look at non-nominal subjects : \emph{be obvious} allows for a sentential subject and \emph{be a good place to hide} allows for a prepositional subject. They are possible with raising verbs, as in the following:
 
\eal 
\ex[]{
[That Kim is a spy] seemed to be obvious.
}
\ex[]{
 [Under the bed] is a good place to hide.}
\ex[]{
Kim expects [under the bed] to be a good place to hide.
} \label{under}
\zl

But they would not be possible with control verbs.
\eal
\ex[\#]{
[That Kim is a spy] wanted to be obvious.
}
\ex[\#]{
Kim persuaded [under the bed] to be a good place to hide.
}
\zl

In languages such as German which allow them, subjectless constructions can be embedded under
raising verbs but not under control verbs \citep[\page 48]{Mueller2002b}; subjectless passive \emph{gearbeitet} ('worked') can thus appear under \emph{scheinen} ('seem') but not under \emph{versuchen} ('try'):

\eal
\label{german1}
\ex[]{ 
\gll weil gearbeitet wurde. [German]\\
     because worked was\\
\glt 'because work was being done'
}
\ex[]{ 
\gll Dort schien noch gearbeitet zu werden.\\
     there seemed yet worked to be\\
\glt `Work seemed to still be being done there.’
}
\ex[*]{
\gll Der student versucht, gearbeitet zu werden\\
     the student tries worked to be\\
\glt Intended: `The student tries to get the work done.'
}
\zl
 
 All this shows that the subject (or the object) of a raising verb is only selected by the non"=finite verb. 

A related difference is that when control and raising sentences have a corresponding sentence with a finite clause complement, they have rather different related sentences.
%It seemed [that Kim impressed Sandy]
%Kim hoped [that he would go home]
%Kim expected [that Sandy would go home]
%Kim persuaded Sandy [that he/she should home]
With control verbs, the non"=finite complement may often be replaced by a sentential complement (with is own subject), while it is not possible with raising verbs:

\eal
\ex[]{
Kim hoped [to impress Sandy] / [that he impressed Sandy].\label{hope1}
}
\ex[]{
Kim seemed [to impress Sandy] / *[that he impressed Sandy].
}
\zl

\eal
\ex[]{
Kim promised Sandy [to come] / [that he will come]. \label{promise1}
}
\ex[]{
Kim expected Sandy [to come]/ *[that she will come].
}
\zl

With some raising verbs, on the other hand, a sentential complement is possible with an expletive subject (\ref{seem4}), or with no postverbal object (\ref{expect3}).

\eal
\ex[]{
It seemed [that Kim impressed Sandy]. \label{seem4}
}
\ex[]{
Kim expected [that Sandy will come].\label{expect3}
}
\zl

This shows that the control verbs can have a subject (or an object) different from the subject of the embedded verb, but not the raising verbs.\footnote{Another contrast proposed by \citet{Jacobson1990} is that control verbs may allow for a null complement (\emph{She tried.}), or a non"=verbal complement (\emph{They wanted a raise.}), while raising verbs may not (\emph{*She seemed}). However, some raising verbs may have a null complement (\emph{She just started.}) as well as some auxiliaries (\emph{She doesn't.}) which can be analysed as raising verbs (see Section~\ref{sec-auxiliaries-as-raising-verbs} below).}

\subsection{More on control verbs}

For control verbs, the choice of the controller is determined by the semantic class of the verb \citet{PollardandSag1992}(chapter 3) (see also \citealt{JackendoffandCulicover2003}).  Verbs of influence (\word{permit, forbid}) are object-control
while verbs of commitment (\word{promise, try}) as in (\ref{commit}) and orientation (\word{want,
  hate}) as in (\ref{orient}) are subject-control, as shown by the reflexive in the following examples:

\begin{exe}
	\ex \begin{xlist}
	\ex John promised Mary to buy himself / * herself a coat. \label{commit}
   \ex 	John permitted Mary to buy herself / * himself a coat.\label{orient}
 \end{xlist}
 \end{exe}
 
  The classification of control verbs is cross-linguistically widespread \citep{VanValinandLapolla1997}, but Romance verbs of mental representation and speech report are an exception in being subject-control without having a commitment or an orientation component.


\begin{exe}
\ex \begin{xlist}
\ex \gll Marie dit {ne pas} \^etre convaincue. [French] \\
Mary says \ig{neg} be convinced \\
\glt `Mary says she is not convinced.'	
\ex \gll Paul pensait  avoir compris. \\
Paul thought have understood \\
\glt `Paul thought he understood.'
 \end{xlist}
\end{exe}

It is worth noting that for object-control verbs, the controller may also be the complement of a preposition \citep[\page 139]{PollardandSag1994}:

\begin{exe}
\ex Kim appealed [to Sandy] to cooperate. \label{to}
\end{exe}


%\eal
%\ex[]{
%Leslie wants this / a raise.
%}
%\ex[]{
%Leslie tried.
%}
%\ex[*]{
%Leslie seemed.
%}
%\ex[*]{
%Leslie seemed this.
%}
%\zl
 
 \citet{Bresnan1982:401}, who attributes the generalization to Visser, also suggests that object-control verbs may passivize (and become subject-control) while subject-control verbs do not (with a verbal complement).

\eal
\ex[]{
Mary was persuaded to leave (by John).
}\label{persuade-pass}
\ex[*]{
Mary was promised to leave (by John).
}\label{promise-pass}
\ex[]{Pat was promised to be allowed to leave. (Pollard and Sag 1994:285)
}\label{promise-pass2}
\zl
However, there are counterexamples (\ref{promise-pass2}) and the generalization does not seem to hold crosslinguistically (see \citew[\page 129]{Mueller2002b} for counterexamples in German).
 
\subsection{More on raising verbs}

From a cross-linguistic point of view, raising verbs usually belong to other semantic classes than control verbs. The distinction between subject-raising and object-raising also has some semantic basis: verbs marking tense, aspect, modality (\word{start, cease, keep}) are subject-raising, while
causative and perception verbs (\word{let, see}) are usually object-raising:

	\begin{exe}
\ex  \begin{xlist}
\ex John started to like himself.
\ex It started to rain.
\ex John let it appear that he was tired.
\ex John let Mary buy herself / * himself a coat.
	 \end{xlist}
	 \end{exe}
	

Transformational analyses posit distinct syntactic structures for raising and control sentences: subject-raising verbs select a sentential complement (and no subject), while subject-control verbs select a subject and a sentential complement \citep{Postal1974, Chomsky81a}. With subject-raising verbs, the embedded clause's subject is supposed to move to the position of matrix verb subject, hence the  term 'raising'. They also posit two distinct structures for object-control and object-raising verbs: while object-control verbs select two complements, object-raising verbs only select a sentential complement and
an exceptional case marking (ECM) rule assigns case to the embedded clause's subject of \word{expect} verbs).
% a co-indexing rule between the empty subject (PRO) of the infinitive and the subject (\word{promise}) or the object (\word{persuade}) of the matrix verb for control verbs.
In this approach, both subject- and object-raising verbs have a sentential complement:
	
\begin{exe}
\ex  \begin{xlist}
\ex 	subject-raising:\\
{}[\sub{NP} $e$ ] seems [\sub{S} John to leave ] 
$\leadsto$  
{}[\sub{NP} John$_{i}$ ] seems [\sub{S} $e_{i}$ to leave ]	
%\ex subject-control:  
%{}[\sub{NP} John$_{i}$ ] wants [\sub{S} PRO$_{i}$ to leave ]	
% \end{xlist}
% \end{exe}

%\begin{exe}
%\ex  \begin{xlist}
\ex 	object-raising (ECM): We expected [\sub{S} John to leave ] 	
%\ex object-control: We persuaded  
%{}[\sub{NP} John$_{i}$ ]  [\sub{S} PRO$_{i}$ to leave ]	
 \end{xlist}
 \end{exe}

 As observed by  \citet{SagandPollard1994}, the putative correspondence between source and target for raising structures is not systematic: \word{seem} may take a sentential complement (with an expletive subject) (\ref{seem4}) but the other subject-raising verbs (aspectual and modal verbs) do not. 


\eal
\ex[]{
Paul started to understand.
}
\ex[*]{
It started [that Paul understands].
}
\zl
 
Similarly, while some object-raising verbs (\word{expect, see}) may take a sentential complement (\ref{expect3}), others do not (\word{let, make, prevent}).
 
\eal
\ex[]{
We expect Paul to understand.	\label{ex-we-expect-paul-to-unterstand}
}
\ex[]{
We expect [that Paul understands]. \label{ex-we-expect-that-paul-understands}
}
\zl
\eal
\ex[]{
We let Paul sleep.
}
\ex[*]{
We let [that Paul sleeps].
}
\zl

Furthermore, in transformational analyses, it is often assumed that the subject of the non"=finite verb  is supposed to raise to receive case from the matrix verb.
 But the subject of \word{seem} or \word{start} need not bear case  since it can be a non-nominal subject (\ref{under}).
%\eal
%\ex[]{[Drinking one liter of water each day] seems to benefit your health.}
%\zl
Data from languages with ``quirky'' case such as Icelandic, also show that subjects of subject-raising verbs in fact keep the quirky case assigned by the embedded verb  \citep{Zaenenetal1985}, contrary to the subject of subject-control verbs which are assigned case by the matrix verb and are thus in the nominative. A verb like \word{need} takes an accusative subject, and a raising verb (\word{seem}) takes an accusative subject as well when combined with \word{need} (\ref{need}). With a control verb (\word{hope}), on the other hand, the subject must be nominative (\ref{hope-i}).

\begin{exe}
\ex 
\begin{xlist}
\ex \gll Mig vantar peninga. [Icelandic, PollardandSag1994:138-139]\\
I.\textsc{acc} need money.\textsc{acc} \\
\ex \gll Mig virdast vanta peninga. \label{need} \\
I.\textsc{acc} seem need money.\textsc{acc} \\
\ex \gll Eg vonast till ad vanta ekki peninga. \label{hope-i} \\
I.\textsc{nom} hope for to need not money.\textsc{acc} \\
\glt `I hope I won't need money.'
	\end{xlist}
		
\end{exe}

Turning now to object-raising verbs, when a finite sentential complement is possible, the structure is not the same as  with a non"=finite complement. Heavy NP shift is possible with a non"=finite complement, and not with a sentential complement \citet{Bresnan1982:423} \citet{PollardandSag1994}: this shows that \word{expect} has two complements in (\ref{ex-we-expect-paul-to-unterstand}) and only one in (\ref{ex-we-expect-that-paul-understands}).

\eal
\ex[]{
We expected [all students] [to understand].
}
\ex[]{
We expected [to understand] [all those who attended the class]. \label{HNPS}
}
\ex[]{ 
We expected [that [all those who attended the class] understand].
}
\ex[*]{
We expected [that understand [all those who attended the class]].
}
\zl

Fronting also shows that the NP VP sequence does not behave as a single constituent, unlike the finite complement:

\eal
\ex[]{
That Paul understood, I did not expect.
}
\ex[*]{
Paul to understand, I did not expect.
}
\zl


This shows that object-raising verbs are better analysed as ditransitive verbs and that the subject of the non"=finite verb has all properties of an object of the matrix verb. It is an accusative in English (\word{him, her}) \ref{pro} and it can passivize, like the object of an object-control verb (\ref{passive}).

\begin{exe}
\ex
\begin{xlist} \label{pro}
\ex We expect him to understand.
\ex  We persuaded him to work on this.
\end{xlist}
\ex \begin{xlist} \label{passive}
\ex  He was expected to understand.
\ex  He was persuaded to work on this.
\end{xlist}
	
\end{exe}


To conclude, the movement (raising) analysis of subject-raising verbs, as well as the ECM analysis of object-raising verbs are motivated by semantic considerations: an NP which receives a semantic role from a verb should be a syntactic argument of this verb. But they lead to syntactic structures which are not motivated (assuming a systematic availability of a sentential complementation) and/or make wrong empirical predictions (that the postverbal sequence of ECM verb behaves as one constituent instead of two).
 
\subsection{Raising and control non-verbal predicates}\label{nonverbal}

Non-verbal predicates taking a non"=finite complement may also fall under the raising/control distinction.  Adjectives such as \word{likely} have raising properties: they do not select the category of their subject, nor assign it a semantic role, contrary to adjectives like \word{eager}. Meteorological \word{it} is thus compatible with \word{likely}, but not with \word{eager}. In the following examples, the subject of the adjective is the same as the subject of the copula (see Section~\ref{sec-copular-constructions} below).

\eal
\ex[]{
It is likely to rain.
}
\ex[]{
John is likely / eager to work here.
}
\ex[*]{
It is eager to rain.
}
\zl

The same contrast may be found with  nouns taking a non finite complement. Nouns such as \word{tendency} have raising properties: they do not select the category of their subject, nor assign it a semantic role, contrary to nouns like \word{desire}. Meteorological \word{it} is thus compatible with the former, but not with the latter. In the following examples, the subject of the predicative noun is the same as the subject of the light verb \emph{have}.


\eal
\ex[]{
John has a tendency to lie.
}
\ex[]{
John has a desire to win.
}
\ex[]{
It has a tendency /*desire to rain at this time of year.
}
\zl

\section{An HPSG analysis}


In a nutshell, the HPSG analysis rests on a few leading ideas: non"=finite complements are unsaturated VPs (a verb phrase with a non"=empty \subjl); a syntactic argument need not be assigned a semantic role; control and raising verbs have the same syntactic arguments; raising verbs do not assign a semantic role to the syntactic argument that functions as the subject of their non finite complement. We continue to use the term ``raising'', but it is just a cover term, since no raising is taking place in HPSG analyses.

As a result,  raising means full identity of syntactic and semantic information (\type{synsem}) \crossrefchapterp{properties} with the unexpressed subject, while control involves identity of semantic indices (discourse referents) between the controller and the unexpressed subject. Co-indexing is compatible with the controller and the controlled subject not bearing the same case (\ref{hope}) or having different parts of speech (\ref{to}), as it is the case for pronouns and antecedents (see \crossrefchapteralp{binding}). This would not be possible with raising verbs, where there is full sharing of syntactic and semantic features between the subject (or the object) of the matrix verb and the (expected) subject of the non"=finite verb. In German, the nominal complement of a raising verb like \emph{sehen} (see) must agree in case with the subject of the infinitive, as shown by the adverbial phrase (one after the other) which agrees in case with the unexpressed subject of the infinitive, but it can have a different case with a control verb like \emph{erlauben} (allow) ((\ref{german2}) from \citew[\page 47--48]{Mueller2002b}):


\eal
\label{german2}
\ex 
\gll Der Wächter  sah den Einbrecher     und seinen Helfer            einen       / *  einer nach dem anderen weglaufen [German]\\
     the watchman saw the burglar.\ACC{} and his    accomplice.\ACC{} one.\ACC{} {} {} one.\NOM{} after the other run.away\\
\glt `The watchman saw the burglar and his accomplice run away, one after the other.'
\ex
\gll Der Wächter erlaubte den Einbrechern einer nach dem anderen weglaufen.\\
     the watchman  allowed the burglars.\DAT{} one.\NOM{} after the other run.away\\
\glt `The watchman allowed the burglars to run away, one after the other.'
\zl

\subsection{The HPSG analysis of 'raising' verbs}
Subject-raising-verbs (and object-raising verbs) can be defined as subtypes inheriting from verb"=lexeme and subject-raising"=lexeme (or object raising"=lexeme) types.


\begin{figure}[htbp!]
	\begin{forest}
       [{\type{lexeme}} 
      					[{\fbox{\attrib{part-of-speech}}}
      						[{\type{verb-lx}}, name=A1 
      							[, no edge ]
      							[, no edge ] ] 
      						 [{\type{adj-lx}}]
      						 [{\type{noun-lx}}] 
      						 [{\ldots}]   		
      					] 
      					[{\fbox{\attrib{arg-selection}}} 
      					    [{\type{intr-lx}}
      					 		[{\type{subj-rsg-lx}}
      					 			[{\type{sr-v-lx}}, name=B1 ]
      					 		]
      					 		[{\ldots}]
      					 	]
      					 	 [{\type{tr-lx}}
      					 		[{\type{obj-rsg-lx}}
      					 			[{\type{or-v-lx}}, name=B2 ]
      					 		]
      					 		[{\ldots}]
      					 	]
      					 	[{\ldots}]
      					]  
      	]
      	\draw (A1.south)-- (B1.north);
      	\draw (B2) to [bend left= 6] (A1);
\end{forest}
\caption{\label{verb-hier2}A type hierarchy for subject- and object-raising verbs}
\end{figure}


As in \crossrefchapterw[Section~]{properties}\todostefan{Crossref Section~4.1 once this chapter is transfered to latex}, upper case letters are used for the two dimensions of classification, and \type{verb-lx}, \type{intr-lx}, \type{tr-lx}, \type{subj-rsg-lx}, \type{obj-rsg-lx}, \type{or-v-lx} and \type{sr-v-lx} abbreviate \type{verb"=lexeme}, \type{intransitive"=lexeme}, \type{transitive"=lexeme}, \type{subject-raising"=lexeme}, \type{object"=raising"=lexeme}, \type{object"=raising"=verb"=lexeme} and \type{subject-raising"=verb"=lexeme}, respectively. 
Constraints on types \type{subj-rsg-lx} and  \type{obj-rsg-lx} are as follows:\footnote{\append  is used for list concatenation. The category of the complement is not specified as a VP since it may be a V in some Romance languages with a flat structure (AbeilléandGodard2003) and in some verb final langages where the matrix verb and the non finite verb form a verbal complex  (German, Dutch, Japanese, Persian, Korean, see \crossrefchapterw{order} on constituent order). Furthermore, the same lexical types will also be used for copular verbs that take non"=verbal predicative complements, see Section~\ref{sec-copular-constructions}.}

\eal
\label{rsg}
\ex	\type{subj-rsg-lx}	\impl  \argst  \begin{avm} \@1 \append \<\[subj & \@1\]\> \end{avm} \label{rais-1}
\ex \type{obj-rsg-lx} \impl \argst  \begin{avm} \<NP\> \append \@1 \append \<\[subj & \@1\]\> \end{avm} \label{rais2}
\zl


The subject of the non finite complement shares its \type{synsem} with the subject of subject-raising verb in \ref{rais-1}, and with the object of object raising verb in \ref{rais2}. This means that they share their syntactic and semantic features: they have the same semantic index (if any), but also the same part of speech, the same case etc. Thus a subject appropriate for the non finite verb is appropriate as a subject (or an object) of the raising verb: this allows for expletive (\ref{rain1},\ref{rain2}) or idiomatic (\ref{cat1},\ref{cat2}) subjects, as well as non nominal subjects (\ref{under}). If the embedded verb is subjectless, as in \ref{german1}, this information is shared too ([1] can be an empty list).

A subject-raising verb (\word{seem}) and an object"=raising verb (\word{expect}) inherit from \type{subj-rsg-v} and \type{obj-rsg-v} respectively; their lexical descriptions look as follows, assuming a MRS semantics (Copestake et al 2005, and Koenig and Richter 2020, chapter on semantics):

\eas
\word{seem}:\\
\begin{avm}
	\[subj-rsg-v-word\\
	subj & \<\@1 \> \\
	comps & \<\@2\VP\[head & \[vform & inf\] \\
		       subj & \<\@1\> \\
		       cont & \[ind & \@3\] \]\>\\
	arg-st & <\@1,\@2>\\
		cont & \[ind & s \\
 		rels & \{\[\asort{seem-rel}
			soa & \@3\]\}\]
	\]
\end{avm}
\zs

\eas
\word{expect}:\\
\begin{avm}
	\[obj-rsg-v-word\\
	subj & \<\@1\NP$_{\@i}$ \> \\
	comps & \<\@2, \@3\VP\[head & \[vform & inf\] \\
		subj & \<\@2\> \\
		cont & \[ind & \@4\] \]\>\\
	arg-st & <\@1,\@2, \@3>\\
	cont & \[ind & s \\
			rels & \{\[\asort{expect-rel}
			exp & \@i\\
			soa & \@4\]\}\]
	\]
\end{avm}
\zs
\inlinetodostefan{AA arg-st added: the types are more general and the words are more specified}
They take a VP and not a clausal complement, which means that the complements expected by the infinitive are realized locally but not its subject. The  corresponding simplified trees are as shown in Figures~\ref{cons-1} and~\ref{cons2}. Notice that the syntactic structures are the same.
\begin{figure}
% \begin{tikzpicture}[baseline, sibling distance=2pt, level distance=60pt, scale=.9]
% 	\Tree
% 	[.{\begin{avm}
% 		\[phon & \phonliste{ Paul seems to sleep }\\
% 			subj & \eliste \\
% 			comps & \eliste\]
% 		\end{avm}}
% 		{\begin{avm}\[phon & \phonliste{ Paul } \\
% 			synsem & \@1 \]
% 		\end{avm}}
% 		[.{\begin{avm}
% 			\[phon & \phonliste{ seems to sleep }\\
% 			subj & \<\@1\>\]
% 			\end{avm}}
% 		 {\begin{avm}
% 			\[phon & \phonliste{ seems } \\
% 			subj & \<\@1\>\\
% 			comps & \<\@2 \[subj & \< \@1 \>\]\>\\
% 			\]
% 			\end{avm}} 
% 		{\begin{avm}
% 			\[phon \phonliste{ to sleep }\\
% 				synsem \@2 \]	
% 			\end{avm}}  
% 		]
% 	]
% \end{tikzpicture}
\begin{forest}
[{\begin{avm}
    \[\type{S}\\
    phon & \phonliste{ Paul seems to sleep }\\
      subj & \eliste \\
      comps & \eliste\]
  \end{avm}}
  [{\begin{avm}
  \[\NP\\
  phon & \phonliste{ Paul } \\
         	 synsem & \@1 \]
    \end{avm}}]
  [{\begin{avm}
      \[\VP\\
      phon & \phonliste{ seems to sleep }\\
        subj & \<\@1\>\\
        comps & \eliste\]
    \end{avm}}
    [{\begin{avm}
        \[\type{V}\\
        phon & \phonliste{ seems } \\
          subj & \<\@1\>\\
          comps & \<\@2 \[subj & \< \@1 \>\]\>\\
        \]
      \end{avm}}]
    [{\begin{avm}
        \[\VP\\
        phon \phonliste{ to sleep }\\
          synsem \@2 \]	
      \end{avm}}] ] ]
\end{forest}
\caption{\label{cons-1}A sentence with a subject-raising verb}
\end{figure}

\begin{figure}
\begin{forest}
  [{\begin{avm}
      \[\type{S}\\
      phon & \phonliste{ Mary expected Paul to work }\\
        subj & \eliste\\
        comps & \eliste\]		
    \end{avm}}
	[{\begin{avm} 
	\[\NP\\
	phon & \phonliste{ Mary } \\
			synsem & \@3 \]
		\end{avm}}]
	[{\begin{avm}
            \[\VP\\
            phon & \phonliste{ expected Paul to work }\\
              subj & \<\@3 \>\\
              comps & \eliste\]		
          \end{avm}}
	[{\begin{avm}
            \[\type{V}\\
            phon & \phonliste{ expected } \\
              subj & \<\@3 \>\\
              comps & \<\@1, \@2 \[
                subj & \<\@1 \>\]\>\]		
          \end{avm}}]
	[{\begin{avm} 
	\[\NP\\
	phon & \phonliste{ Paul } \\
			synsem & \@1 \]
		\end{avm}}]
	[{\begin{avm}
            \[\VP\\
            phon & \phonliste{ to work }\\
              synsem & \@2 \]	
          \end{avm}}]
	] ]
\end{forest}	
\caption{\label{cons2}A sentence with an object"=raising verb}
\end{figure}

Raising verbs have in common a mismatch between syntactic and
semantic arguments: the raising verb has a subject (or an object) which is not one of its semantic
arguments (its INDEX does not appear in the CONT feature of the raising verb). To constrain this type of
mismatch, \citet[140]{PollardandSag1994} propose the Raising Principle.

\begin{exe}
\ex Raising Principle: Let X be a non"=expletive element subcategorized by Y, X is not assigned any semantic role by Y iff Y also subcategorizes a complement which has X as its first argument.
\end{exe}

This principle was meant to prevent raising verbs from omitting their VP complement, unlike control verbs \citep{Jacobson1990}. Without a non"=finite complement, the subject of \word{seem} is not assigned any semantic role, which violates the Raising principle. However, some unexpressed (null) complements are possible with some subject-raising verbs 
as well as VP ellipsis with English auxiliaries, which are analysed as subject-raising verbs (see section 4 below and Nykiel and Kim this volume chapter on ellipsis). So the Raising principle should be reformulated in terms of Argument-structure and not Valence features.

\eal
\ex[]{John tried /* seems.
}
\ex[]{
John just started.
}
\ex[]{
John did.
}
\zl

For subject-raising verbs which allow for a sentential complement as well (with an expletive subject) (\ref{seem4}), another lexical description is needed, and the same holds for object"=raising verbs which allow a sentential complement (with no object) (\ref{expect3}). These can be seen as valence alternations, which are available for some items (or some classes of items) but not all (see Wechsler, Koenig and Davis, 2020, chapter on arg structure).

\eal
\ex \emph{seem}:   \argst \sliste{ NP[\type{it}], S }
\ex \emph{expect}: \argst \sliste{ NP, S }
\zl

\subsection{The HPSG analysis of control verbs}

\citet{SagandPollard1991} propose a semantics-based control theory. The semantic class of the verb determines whether it is subject-control or object-control: they distinguish verbs of orientation (\emph{want, hope}), verbs of commitment (\emph{promise, try}) and verbs of influence (persuade, forbid) based on the type of relation and semantic roles of their arguments. 

\begin{figure}
	\begin{forest}
       [{\type{control-relation}} 
      					[{\type{orientation-rel}}
      						[{\type{want-rel}} ] 
      						 [{\type{hope-rel}}]
      						 [{\ldots}]   		
      					] 
      					[{\type{commitment-rel}} 
      					 		[{\type{promise-rel}}
      					 			[{\type{try-rel}}]
      					 		[{\ldots}]
      					 	]
      					 	 [{\type{influence-rel}}
      					 		[{\type{persuade-rel}}
      					 			[{\type{forbid-rel}}]
      					 		]
      					 		[{\ldots}]
      					 	]
      					 	[{\ldots}]
      					]  
      	]
      %	\draw (A1.south)-- (B1.north);
      	%\draw (B2) to [bend left= 6] (A1);
\end{forest}
\caption{\label{verb-hier3}A type hierarchy for control predicates}
\end{figure}

For example, \emph{want}, \emph{promise} and \emph{persuade} have a semantic content such as the following, with SOA meaning state-of-affairs and denoting the content of the non"=finite complement:

\eas
%\word{want}:\\
\begin{avm}
	\[\type{want-rel} \\
	experiencer & \@1 \\
	SOA & \[relation &  \\
			arg & \@1\]\]
\end{avm}
\zs

\eas
%\word{promise}:\\
\begin{avm}
	\[\type{promise-rel} \\
	commitor & \@1 \\
		commitee & \@2 \\
	SOA & \[relation &  \\
			arg & \@1\]\]
\end{avm}
\zs
\eas 
%\word{persuade}:\\*
\begin{avm}
\[\type{persuade-rel} \\
	influencer & \@1 \\
		influenced & \@2 \\
	SOA & \[relation &  \\
			arg & \@2\]\]
\end{avm}	
\zs

According to this theory, the controller is the experiencer with verbs of orientation, the commitor with verbs of commitment, and the influencer with verbs of influence. 
From the syntactic point of view, two types of control predicates, \type{subject-cont-lx} and  \type{object-cont-lx}, can be defined as follows:

\eal
\label{cont}
\ex	\type{subj-cont-lx}	\impl \argst  \begin{avm}  \<NP$_{\@i}$, ...\[subj & \<\[ind & \@i\]\>\]\> \end{avm}
\ex \type{obj-cont-lx} \impl  \argst  \begin{avm} \<\@0, \@1 \[ind & \@i\], \[subj & \<\[ind &\@i\]\>\]\> \end{avm}
\zl

The controller is the first argument with subject-control verbs, while it is the second argument with object-control verbs. Contrary to the types defined for raising predicates in (\ref{rsg}), the controller here is simply coindexed with the subject of the non finite complement. This means it must have a semantic role (since it has a semantic index), thus expletives and idiom parts are not allowed ((\ref{rain3}), (\ref{rain4}), (\ref{cat3}), (\ref{cat4})). This also implies that its syntactic features may differ from those of the subject of the non finite complement: it may have a different part of speech (a NP subject can be coindexed with a PP controller) as well as a different case ((\ref{to}), (\ref{hope})).\\
Verbs of orientation and commitment inherit from the type \type{subj-cont-lx}, while verbs of influence inherit from the type \type{subj-cont-lx}.
A subject-control verb (\word{want}) and an object-control verb (\word{persuade}) inherit from \type{subj-cont-v} and \type{obj-cont-v} respectively; their lexical descriptions are as follows:
\footnote{To account for Visser's generalization (object-control verbs passivize  while subject-control verbs do not), \citet{SagandPollard1991} analyse the subject of the infinitive as a reflexive, which must be bound by the controller. According to Binding Theory (see \crossrefchapteralp{binding}), the controller must be less oblique than the reflexive, hence less oblique than the VP complement which contains the reflexive: the controller can be the subject and the VP a complement as in (\ref{per}); it can be the first complement when the VP is the second complement as in (\ref{persuade-pass}), but it cannot be a  \emph{by}-phrase, which is more oblique than the VP complement, as in (\ref{promise-pass}) (the by-phrase should not be bound according to principle C, and the subject of infinitive should be bound according to principle A).}.

\begin{exe}
\ex \word{want}:\\
\begin{avm}
	\[\type{subj-cont-v-word}\\
	subj & \<\@1\NP$_{\@i}$ \> \\
	comps & \<\@2\VP\[head & \[vform & inf\] \\
		subj & \<\[ind & \@i\]\> \\
		cont & \[ind & \@3\] \]\>\\
	arg-st & <\@1,\@2> \\
	cont & \[ind & s \\
			rels & \{\[\asort{want-rel}
			exp & \@i \\
			soa & \@3\]\}\]
	\]
\end{avm}	
\ex \word{persuade}:\\*
\begin{avm}
	\[\type{obj-cont-v-word}\\
	subj & \<\@1\NP$_{\@i}$ \> \\
	comps & \<\@2\NP$_{\@j}$, \@3\VP\[head & \[vform & inf\] \\
		subj & \<\[ind & \@j\]\> \\
		cont & \[ind & \@4\] \]\>\\
	arg-st & <\@1,\@2, \@3> \\
	cont & \[ind & s \\
			rels & \{\[\asort{persuade-rel}
			agent & \@i \\
			patient & \@j \\
			soa & \@4\]\}\]
	\]
\end{avm}	
\end{exe}

The corresponding structures for subject-control and object-control sentences are illustrated in Figures~\ref{sleep3} and~\ref{cons3}:

\begin{figure}
% fails 03.05.2020 23:23
% \tikzexternaldisable
%  \begin{tikzpicture}[baseline, sibling distance=2pt, level distance=60pt, scale=.9]
% 	\Tree
% 	[.{\begin{avm}
% 		\[phon & \phonliste{ Paul wants to sleep }\\
% 			subj & \eliste \\
% 			comps & \eliste\]
% 		\end{avm}}
% 		{\begin{avm}\[phon \phonliste{ Paul } \\
% 			synsem \@1 \]
% 		\end{avm}}
% 		[.{\begin{avm}
% 			\[phon & \phonliste{ wants to sleep }\\
% 			subj & \<\@1\>\]
% 			\end{avm}}
% 		 {\begin{avm}
% 			\[phon & \phonliste{ wants } \\
% 			subj & \<\@1 \[ cont|ind  \type{i} \] \>\\
% 			comps & \<\@2 \[subj & \< \normalfont NP_{i} \> \]\>\\
% 			\]
% 			\end{avm}} 
% 		{\begin{avm}
% 			\[phon & \phonliste{ to sleep }\\
% 				synsem & \@2  \]	
% 			\end{avm}}  
% 		]
% 	]
% \end{tikzpicture}
% This one fails forest even without externalization
\begin{forest}
[{\begin{avm}
    \[\type{S}\\
    phon & \phonliste{ Paul wants to sleep }\\
      subj & \eliste \\
      comps & \eliste\]
  \end{avm}}
  [{\begin{avm}
    \[\NP\\
    phon \phonliste{ Paul } \\
      synsem \@1 \]
    \end{avm}}]
  [{\begin{avm}
      \[\VP\\
      phon & \phonliste{ wants to sleep }\\
        subj & \<\@1\>
        comps & \eliste\]
    \end{avm}}
    [{\begin{avm}
        \[\type{V}\\
        phon & \phonliste{ wants } \\
          subj & \<\@1 \[ind  \type{i} \] \>\\
          comps & \<\@2 \[subj & \< \normalfont NP$_{i}$ \> \]\>\\
        \]
      \end{avm}}] 
    [{\begin{avm}
        \[\VP\\
        phon & \phonliste{ to sleep }\\
          synsem & \@2  \]	
      \end{avm}}] ] ]
\end{forest}
% \begin{forest}
% [\avmtmp{
%    [phon & \phonliste{ Paul wants to sleep }\\
%     subj & \eliste \\
%     comps & \eliste ]
%   }
%   [\avmtmp{
%     [phon \phonliste{ Paul } \\
%      synsem \1 ]
%     }]
%   [\avmtmp{
%       [phon & \phonliste{ wants to sleep }\\
%        subj & < \1 >]
%     }
%     [\avmtmp{
%       [phon  & \phonliste{ wants } \\
%        subj  & < \1 [ cont|ind  \type{i} ] >\\
%        comps & < \2 [ subj & < \normalfont NP$_{i}$ > ] >\\
%         \]
%       }] 
%     [\avmtmp{
%         [phon   & \phonliste{ to sleep }\\
%          synsem & \2  ]	
%      }] ] ]
% \end{forest}
\caption{\label{sleep3}A sentence with a subject-control verb}
\end{figure}



%The corresponding trees are given in Figure~\ref{cons2} and~\ref{cons3}. Notice that the syntactic structures are the same.

%\begin{figure}
% \begin{tikzpicture}[baseline, sibling distance=2pt, level distance=60pt, scale=.9]
% 	\Tree
% 	[.{\begin{avm}
% \[phon & \phonliste{ Mary expected Paul to work }\\
% subj & \eliste\\
% comps & \eliste\]		
% \end{avm}}
% 	{\begin{avm} \[phon & \phonliste{ Mary } \\
% 			synsem & \@3 \]
% 		\end{avm}}
% 	[.{\begin{avm}
% \[phon & \phonliste{ expected Paul to work }\\
% subj & \<\@3 NP\>\\
% comps & \eliste\]		
% \end{avm}}
% 	{\begin{avm}
% \[phon & \phonliste{ expected } \\
% subj & \<\@3 NP\>\\
% comps & \<\@1, \@2 \[
% 		 subj & \@1 \]\>\]		
% \end{avm}}
% 	{\begin{avm} \[phon & \phonliste{ Paul } \\
% 			synsem & \@1 \]
% 		\end{avm}}
% 	{\begin{avm}
% 			\[phon & \phonliste{ to work }\\
% 				synsem & \@2 \]	
% 			\end{avm}}
% 	] ]
% \end{tikzpicture}	
\begin{figure}
% fails 04.05.2020 06:30
% \tikzexternaldisable
% \begin{tikzpicture}[baseline, sibling distance=2pt, level distance=60pt, scale=.9]
% 	\Tree
% 	[.{\begin{avm}
% \[phon & \phonliste{ Mary persuaded Paul to work }\\
% subj & \eliste\\
% comps & \eliste\]		
% \end{avm}}
% 	{\begin{avm}\[phon & \phonliste{ Mary } \\
% 	synsem & \@3
% 			\]
% 		\end{avm}}
% 	[.{\begin{avm}
% \[phon & \phonliste{ persuaded Paul to work }\\
% subj & \<\@3 NP\>\\
% comps & \eliste\]		
% \end{avm}}
% 	{\begin{avm}
% \[phon & \phonliste{ persuaded } \\
% subj & \<\@3 NP\>\\
% comps & \<\@1 \[cont|ind \type{i} \], \@2 \[
% 		 subj & \< NP$_{i}$ \> \]\>\]		
% \end{avm}}
% 	{\begin{avm}\[phon  \phonliste{ Paul } \\
% 		synsem \@1 \]
% 		\end{avm}}
% 	{\begin{avm}
% 			\[phon & \phonliste{ to work }\\
% 				synsem & \@2  \]	
% 			\end{avm}}
% 	] ]
% \end{tikzpicture}	
\begin{forest}
[{\begin{avm}
    \[\type{S}\\
    phon & \phonliste{ Mary persuaded Paul to work }\\
      subj & \eliste\\
      comps & \eliste\]		
  \end{avm}}
  [{\begin{avm}
  \[\NP\\
  phon & \phonliste{ Mary } \\
	synsem & \@3
      \]
    \end{avm}}]
  [{\begin{avm}
      \[\VP\\
      phon & \phonliste{ persuaded Paul to work }\\
        subj & \<\@3 NP\>\\
        comps & \eliste\]		
    \end{avm}}
    [{\begin{avm}
        \[\type{V}\\
        phon & \phonliste{ persuaded } \\
          subj & \<\@3 NP\>\\
          comps & \<\@1 \[ind \type{i} \], \@2 \[
            subj & \< NP$_{i}$ \> \]\>\]		
      \end{avm}}]
    [{\begin{avm}
    \[\NP\\
    phon  \phonliste{ Paul } \\
          synsem \@1 \]
      \end{avm}}]
    [{\begin{avm}
        \[\VP\\
        phon & \phonliste{ to work }\\
          synsem & \@2  \]	
      \end{avm}}] ] ]
\end{forest}	
\caption{\label{cons3}A sentence with an object-control verb}
\end{figure}

In some Slavic languages (Russian, Czech, Polish), some subject-control verbs may allow case sharing as well, as shown by predicate case agreement with quantified (non nominative) subjects. As observed by \cite{Przepiorkowski2004}, \cite{PrzepiorkowskiandRosen2005}, coindexing does not prevent full sharing: so the analysis may allow for both cases, and a specific constraint may be added to enforce only case sharing and prevent default (instrumental) case assignment to the embedded predicate.

\begin{exe}
\ex \begin{xlist}
\ex 
\gll Janek chce byé miły. [Polish]\\
     Janek.\textsc{nom} wants be.\textsc{inf} nice.\textsc{nom} \\
\glt `Janek wants to be nice.’
\ex 
\gll Pięć dziewcząt chce być miłe / miłych. \\
     five.\textsc{acc} girls.\textsc{gen} wants be.\textsc{inf} nice.\textsc{acc} {} nice.\textsc{gen}\\
\glt `Five girls want to be nice.’ \citep[ex (6)--(7)]{Przepiorkowski2004}
	\end{xlist}
		
\end{exe}


For control verbs which allow for a sentential complement as well  ((\ref{hope1}), (\ref{promise1})), another lexical description is needed. These can be seen as valence alternations, which are available for some items (or some classes of items) but not all (see Wechsler, Koenig and Davis, 2020, chapter on arg structure).

\eal
\ex \emph{want}: \argst <NP, S>
\ex \emph{persuade}: \argst <NP, NP, S>
\zl




\subsection{Raising and control in Mauritian} \label{sec-maurit}


Mauritian is a French-based creole, which has raising and control verbs, belonging roughly to the same semantic classes as in English or French. Verbs marking aspect or
modality (\word{kontign} 'continue', \word{aret} 'stop') are subject-raising verbs and causative and perception verbs (\emph{get} 'watch') are
object"=raising. Raising verbs differ from TMA (tense modality aspect) markers by different properties: they are preceded by
the negation, which follows TMA; they can be coordinated unlike TMA \citep[\page 209]{HenriandLaurens2011}:

\eal
\ex[]{ 
\gll To pou kontign ou aret bwar? [Mauritian] \\
     2\SG{} \IRR{} continue.\textsc{sf} or stop.\textsc{sf} drink.\textsc{lf}\\
\glt `You will continue or stop drinking?'
}
\ex[*]{
\gll To'nn ou pou aret bwar? \\
     2\SG{}'\PRF{} or \IRR{} stop.\textsc{sf} drink.\textsc{lf}\\
\glt  `You have or will stop drinking?'
}
\zl
 
If their verbal complement has no external argument, as is the case with impersonal expressions such as \word{ena lapli} `to rain', then the raising verb itself has no external argument, contrary to a control verb like \word{sey} `try':

\eal
\ex[]{
\gll Kontign     ena lapli. \\
     continue.\textsc{sf} have.\textsc{sf} rain \\
\glt `It continued to rain.'
}
\ex[*]{
\gll Sey ena lapli. \\
     try have.\textsc{sf} rain \\
\glt Literally: `It tries to rain.'
}
\zl

Unlike in French, its superstrate, in Mauritian,  verbs neither inflect for tense, mood and aspect nor for person, number, and
gender. But they have a short form and a long form (henceforth \textsc{sf} and \textsc{lf}), with
30\,\% verbs showing a syncretic form (\emph{bwar}). The following list of examples provides pairs of short and
long forms respectively:

\eal
\ex manz/manze `eat', koz/koze `talk', sant/sante `sing'
\ex pans/panse `think', kontign/kontigne `continue', konn/kone `know'
\zl

As described in \citet{Henri2010}, the verb form is determined by the construction: the short form is required before a phrasal complement and the long form appears otherwise.\footnote{\textit{yer} `yesterday' is an adjunct. See \citew{Hassamal2017} for an analysis of Mauritian adverbs which treats as complements those triggering the verb short form.}


\begin{exe}
\ex \begin{xlist}
\ex 
\gll Zan sant sega / manz pom / trov so mama / pans Paris. \\
     Zan sing.\textsc{sf} sega {} eat.\textsc{sf} apple {} find.\textsc{sf} \POSS{} mother {} think.\textsc{sf} Paris \\
\glt `Zan sings a sega / eats an apple / finds his mother / thinks about Paris.'	
\ex 
\gll Zan sante / manze.\\
     Zan sing.\textsc{lf} {} eat.\textsc{lf}\\
\glt `Zan sings / eats.'
\ex 
\gll Zan ti zante yer. \\
Zan  \PRF{} sing.\textsc{lf} yesterday\\
\glt `Zan sang yesterday.'
\end{xlist}
\end{exe}


\citet{Henri2010} proposes to define two possible values (\textsc{sf} and \textsc{lf}) for the head
feature \vform, with the following lexical constraint (\type{nelist} stands for non-empty list):

\begin{exe}       
\ex \begin{avm} \[vform & \type{sf} \]~ \impl~  \[comps & nelist\] 
\end{avm}
\end{exe}
Interestingly, clausal complements do not trigger the verb short form (\citet{Henri2010} analyses them as extraposed). The complementizer (\emph{ki}) is optional.

\eal
\ex 
\gll Zan panse             (ki)               Mari pou    vini.\\
     Zan think.\textsc{lf} \hspaceThis{(}that Mari \FUT{} come.\textsc{lf}\\
\glt `Zan thinks that Mari will come.'
\ex 
\gll Mari trouve           (ki)                so      mama   tro      manze.\\
     Mari find.\textsc{lf} \hspaceThis{(}that  \POSS{} mother too.much eat.\textsc{lf}\\
\glt `Mari finds that her mother eats too much.'
\zl

On the other hand, subject-raising and subject-control verbs occur in a short form before a verbal complement.

\begin{exe}
\ex \begin{xlist}
\ex \gll Zan kontign sante.\\
Zan continue.\textsc{sf} sing.\textsc{lf}\\\jambox*{(subject-raising verb, p.\,198)}
\glt `Zan continues to sing.'
\ex \gll Zan sey sante.\\
Zan try.\textsc{sf} sing.\textsc{lf}\\\jambox{(subject-control verb)}
\glt `Zan tries to sing.'
\end{xlist}
\end{exe}

The same is true with object-control and object"=raising verbs:
\eal
\settowidth\jamwidth{(object"=raising verb, p.\,200)}
\ex \gll Zan inn fors Mari vini.\\
Zan \PRF{} force.\textsc{sf} Mari come.\textsc{lf}\\\jambox{(object-control verb)}
\glt `Zan has forced Mari to come.'
\ex \gll Zan pe get Mari dormi.\\
Zan \PROG{} watch.\textsc{sf} Mari sleep.\textsc{lf}\\\jambox{(object"=raising verb, p.\,200)}
\glt `Zan is watching Mari sleep.'
\end{xlist}
\end{exe}


Raising  and control verbs thus differ from verbs taking sentential complements. Their \textsc{sf} form is
predicted if they take unsaturated VP complements. Assuming the same lexical type hierarchy as
defined above, verbs like \word{kontign} `continue' and \word{sey} `try' inherit from
\type{subj-rsg-v} and \type{subj-cont-v} respectively 
%and have the following lexical entries 
\footnote{Henri \& Laurens use Sign-based Construction Grammar (SBCG) (see \crossrefchapteralp{properties}\todostefan{section 7.2} and \crossrefchapteralp{cxg}), but their analyses can be adapted to the feature geometry of Constructional HPSG
\citep{Sag97a} assumed in this volume. The analysis of control verbs sketched here will be revised in \ref{section-xarg} below.} \citet[\page 197]{HenriandLaurens2011} conclude that ``while Mauritian data can be brought in accordance with the open complement analysis, both morphological data on the control or raising verb and the existence of genuine verbless clauses put up a big challenge for both the clause and small clause analysis.''

%\begin{exe}
%\ex \word{kontign} `continue':\\
%\begin{avm}
%	\[head & sf\\
%subj & \<\@1 \> \\
%	comps & \<VP\[%head & verb \\
%		%marking & unmark\\
%		subj & \<\@1\> \\
%		cont & \[ind & \@2\] \]\>\\
%	cont & \[ind & s \\
%			rels & \{\[\asort{continue-rel}
%			arg & \@2\]\}\]
%	\]
%\end{avm}
%\ex \word{sey} `try':\\*
%\begin{avm}
%	\[head & sf\\
%subj & \<NP$_{\@i}$ \> \\
%	comps & \<VP\[%head & verb \\
%		%marking & unmark\\
%		subj & \<\[ind & \@i\]\> \\
%		cont & \[ind & \@2\] \]\>\\
%	cont & \[ind & s \\
%			rels & \{\[\asort{try-rel} \\
%			exp & \@i \\
%			arg & \@2\]\}\]
%	\]
%\end{avm}	
%\end{exe}



\subsection{Raising and control in prodrop and ergative languages}
The theory of raising and control presented above naturally extends to prodrop and ergative languages. 
Since Bouma et al (2001), it is assumed that syntactic arguments are listed in ARG-ST and that only canonical ones are present in the valence lists (SUBJ, SPR and COMPS). See for example Borsley and Crysmann this volume for an analysis of UDC with non canonical \synsem . For prodrop languages, it has been proposed, e.g. in \citep{ManningandSag1998:65} that null subject sentences have an element representing the understood subject in the ARG-ST list of the main verb but nothing in the SUBJ list. 

\eal
\ex 
\gll Vengo. \label{Italian} \\
come.PRS.1SG\\
\glt 'I come'
\ex 
\gll Posso venire. \label{Italian-raising} \\
can.1SG come.INF\\
'\glt I can come'
\ex 
\gll Voglio venire. \label{Italian-control} \\
want.1SG come.INF\\
\glt 'I want to come'
\zl

Assuming the lexical types for subj-rsg-lexmes and subj-cont-lexemes in (\ref{rsg}), (\ref{cont}), the verbal descriptions for (\ref{Italian-raising}) and (\ref{Italian-control})  are as follows:
%Vengo : SUBJ <>, Comps <>, ARG-ST<[pro]>
%Posso : SUBJ  <>, Comps <2>, Arg-st <1[pro], 2VP[SUBJ <1>]>
%Voglio SUBJ  <>, Comps <2>, Arg-st <NPi[pro], 2VP[SUBJ <NPi>]>
\eal
\ex	\type{posso} \impl \begin{avm} \[subj & elist\\
comps & \< \@2\> \\ arg-st & \<\@1[pro] ,\@2\[subj & \@1\]\>\] \end{avm} \label{rais1}
\ex \type{voglio}  \impl \begin{avm} \[subj & elist\\
comps & \< \@2\> \\
arg-st & \<NP$_{\@i}$[pro], \@2\[subj & \<\[ind & \@i\]\>\]\>\] \end{avm}
\zl


Balinese offers an intriguing case of syntactic ergativity. It displays rigid SVO order, regardless of the verb's voice form \citep{WechslerandArka1998}. In the agentive voice (AV), the subject is the \argst initial member, while In the objective voice (OV), the verb is transitive, and the subject is the initial NP, although it is not the first argument.  (see \crossrefchapteralp[Section~\ref{arg-st-sec-ergativity}]{arg-st}):

\begin{exe}
\ex \begin{xlist}
\ex  \gll Ida ng-adol bawi.\\
3sg \textsc{av}-sell pig\\
\glt `He/She sold a pig.'
\ex \gll Bawi adol ida. [Balinese]\\
pig \textsc{ov}.sell 3sg \\
\glt `He/She sold a pig.' 
\end{xlist}
\end{exe}

Different properties argue in favor of a subject status of the first NP in the objective voice. Binding properties show that the agent is always the first element on the \argst list, see
\citew{WechslerandArka1998}, \citew{ManningandSag1998} and \crossrefchaptert{binding}. The objective voice is also different from the passive: the passive may have a passive prefix, an agent by-phrase, and does not constrain the thematic role of its subject. The two verbal types can be defined as follows (see \crossrefchapteralp[Section~\ref{arg-st-sec-ergativity}]{arg-st}):

\begin{exe}       
\ex \type{av-verb} \begin{avm}  \impl~  \[subj & \@1\\
comps & \@2\\
arg-st & \@1 \append \@2 \] 
\end{avm}
\ex \type{ov-verb} \begin{avm}  \impl~  \[subj & \@1\\
comps & \@2\\
arg-st & \@2 \append \@1 \]
\end{avm}
\end{exe}

\todostefan{See Chapter on linking for the two verbal types? it is just refered to before the example}

 In this analysis, the preverbal argument, whether the theme of an OV verb or the agent of an AV verb, is the subject, and in many languages, only a subject
can be raised or controlled \citep{Zaenenetal1985}. Thus the first argument of the verb is controlled when the embedded verb is in the agentive voice, and the second argument is controlled when the verb is in the objective voice. 


\begin{exe}
\ex \begin{xlist}
\ex 
\gll Tiang edot [ \trace{} teka].\\
     1 want     {} {} come\\\hfill\citep[ex 25]{WechslerandArka1998}
\glt `I want to come.'
\ex 
\gll Tiang edot [ \trace{}  meriksa dokter].\\
     1     want {} {}     \textsc{av}.examine doctor\\
\glt `I want to examine a doctor.'
\ex 
\gll Tiang edot [ \trace{} periksa dokter].\\
     1     want {} {}    \textsc{ov}.examine doctor\\
\glt `I want to be examined by a doctor.'
\end{xlist}
\end{exe}

Turning to \word{majanji} `promise', in this type of commitment relation, the promiser must have semantic control over the action promised \citep{Farkas1988,Kroeger1993,SagandPollard1991}. The promiser should therefore be the actor of the downstairs verb. This semantic constraint interacts with the syntactic constraint that the controllee must be the subject, to predict that the controlled VP must be in AV voice, which places the Agent in subject role. The same facts obtain for other control verbs such as \word{paksa} `force'.

\eal
\ex[]{
\gll Tiang majanji maang Nyoman pipis.\\
     1 promise \textsc{av}.give Nyoman money\\\hfill\citep[ex 27]{WechslerandArka1998}
\glt `I promised to give Nyoman money.' 
}
\ex[*]{ 
\gll Tiang majanji Nyoman baang pipis. \\
     1 promise Nyoman \textsc{ov}.give money \\
}
\ex[*]{ 
\gll Tiang majanji pipis baang Nyoman. \\
     1 promise money \textsc{ov}.give Nyoman\\ 
}
\zl

Balinese also has subject-raising verbs like \word{ngenah} `seem':

\begin{exe}
\ex \begin{xlist}
\ex 
\gll Ngenah ia mobog.\\
     seem 3 lie\\\hfill\citep[ex 7]{WechslerandArka1998}
\glt `It seems that (s)he is lying.'
\ex 
\gll  Ia ngenah mobog.\\
      3 seem lie\\
\glt `(S)he seems to be lying.'
\end{xlist}
\end{exe}

As predicted, the agent can be ``raised'' when the embedded verb is in the agentive voice, since it is the subject:
%The same applies to a transitive verb in the agentive voice: the agent can %appear as the subject of \emph{ngenah} `seem' but not the patient.

\eal
\judgewidth{?*}
%\ex[]{ 
%\gll Ngenah sajan [ci ngengkebang kapelihan-ne].\\
  %   seem much \spacebr{}2 \textsc{av}.hide mistake-3\POSS\\\hfill\citep[ex 9]{WechslerandArka1998}
%\glt `It is very apparent that you are hiding his/her wrongdoing.'
%}
\ex[]{
\gll Ci ngenah sajan ngengkebang kapelihan-ne.\\
     2 seem much \textsc{av}.hide mistake-3\POSS\\
\glt `You seem to be hiding his/her wrongdoing.' \citep[ex 9]{WechslerandArka1998}
}
\ex[?*]{ 
\gll Kapelihan-ne ngenah sajan ci ngengkebang.\\
     mistake-3\POSS{} seem much 2 \textsc{av}.hide\\
}
\zl

On the other hand, only the patient can be ``raised'' (because that is the subject) when the embedded verb is in the objective voice:

\eal
\judgewidth{?*}
%\ex[]{ 
%\gll Ngenah sajan [kapelihan-ne engkebang ci].\\
%     seem much \spacebr{}mistake-3\POSS{} \textsc{ov}.hide 2\\\hfill
%\glt `It is very apparent that you are hiding his/her wrongdoing.'
%}
\ex[]{
\gll Kapelihan-ne ngenah sajan engkebang ci.\\
     mistake-3\POSS{} seem much \textsc{ov}.hide 2 \\
\glt `His/her wrongdoings seem to be hidden by you.' \citep[ex 8]{WechslerandArka1998}
}
\ex[?*]{
\gll Ci ngenah sajan kapelihan-ne engkebang.\\
     2 seem much mistake-3\POSS{} \textsc{ov}.hide \\
}
\zl


Turning now to object-raising verbs, like \emph{tawang} `know',  they can occur in the agentive voice with an embedded AV verb \ref{av}, and with an embedded OV verb \ref{ov}, unlike control verbs like 'promise'. 
%Balinese also displays object"=raising. While the subject of \emph{mulih} %`go home' has been ``raised'' to the
They can also occur in the objective voice, when the subject of the embedded verb is raised.
In \ref{rais-av}, the embedded verb (arrest) is in the agentive voice and its subject (the police) is raised to the subject of \emph{tawang} `know' in the objective voice; in \ref{rais-ov}, the embedded verb (arrest) is in the objective voice and its subject (Wayan) is raised to the subject of \emph{nawang} 'know' in the objective voice (Wechsler and Arka ex 23).

\begin{exe}
\ex \begin{xlist}
\ex 
%\gll
% Nyoman Santosa tawang           tiang  mulih.\\
  %   Nyoman Santosa \textsc{ov}.know 1      go.home\\\hfill\citep[ex 22]%{WechslerandArka1998}
%\glt `I knew that Nyoman Santosa went home.'
%\ex 
%\gll Tiang nawang           Nyoman Santosa mulih.\\
%     1     \textsc{av}.know Nyoman Santosa go.home\\
%\glt `I knew that Nyoman Santosa went home.'
\gll
Ia nawang polisi lakar nangkep Wayan. \\
3 AV.know police FUT AV.arrest Wayan \\
\glt 'He knew that the police would arrest Wayan.' \label{av}
\ex
\label{rais-av} 
\gll Polisi tawang=a  lakar nangkep Wayan. \\
     police OV.know=3 FUT   AV.arrest Wayan\\

\ex
\gll Ia nawang Wayan lakar tangkep polisi. \\
     3 AV.know Wayan FUT OV.arrest police\\
\glt 'He knew that the police would arrest Wayan.' \label{ov}
\ex
\gll Wayan tawang=a lakar tangkep polisi. \\
     Wayan OV.know=3 FUT OV.arrest police\\ \label{rais-ov}
\end{xlist}
\end{exe}


In Balinese, the subject is always the controlled (or 'raised') element but it is not necessarily the first argument of the embedded verb. The semantic difference between control verbs and raising verbs has a consequence for their complementation: raising verbs (which do not constrain the semantic role of the raised argument) can take verbal complements either in the agentive or objective voice, while object control verbs (which select an agentive argument) can only take a verbal complement in the agentive voice. This difference is a result of the analysis of raising and control presented above, and nothing else has to be added.
%As a result, it is always the subject of the embedded predicate that is coindexed or shared with an argument of the matrix verb, but the subject is not always the first syntactic argument.

%\eal
%\label{rsg-two}
%\ex	\type{subj-rsg-lx}	\impl \begin{avm} \[arg-st & \@1 \append \<\[subj & \@1\]\>\] \end{avm}
%\ex \type{obj-rsg-lx} \impl \begin{avm} \[arg-st & \<NP\> \append \@1 \append \<\[subj & \@1\]\>\] \end{avm}
%\zl


\subsection{\xarg and a revised HPSG analysis}\label{section-xarg}

\todostefan{please add Gerdts, D. and Hukari, T. 2000. A-Subjects and Control in Halkomelem. In D. Flickinger and A. Kathol (eds.),
The Proceedings of the 7th International Conference on Head-Driven Phrase Structure Grammar 
, pp 100–123, Stanford:CSLI Publications}
Sometimes, obligatory control is also attested for verbal complements with an expressed subject. 
As noted by \citet{Zec87a-u,Farkas1988} and \citet[\page 115--116]{GH2000a-u}, in some languages,  such as Romanian, Japanese  \citep{Kuno76a-u,Iida96a-u} or Persian \citep{Karimi2008},  the expressed subject of a verbal complement
may display obligatory control. This may be a challenge for the theory of control presented here, since a clausal complement is a
saturated complement, with an empty \subjl, and the matrix verb cannot access the \subjv of the
embedded verb. Sag and Pollard (1991: 89) proposed a semantic feature external-argument (Ext-Arg), which makes the index of the subject argument available at the clausal level.  \citet{Sag2007a} proposed to introduce a Head
feature \xarg that takes as its value the first syntactic argument of the head verb, and is
accessible at the clause level. 

This is adopted by \citet{HenriandLaurens2011} for Mauritian.  After some subject-control verbs
like \word{pans} `think', the VP complement may have an optional pronominal subject which must be coindexed with the matrix subject. 
%It is not a clausal complement since the matrix verb is in the short form (\textsc{sf}) and not in the long form (see above).

\begin{exe}
\ex \gll Zan$_{i}$ pans pou (li$_{i}$) vini.\\
Zan think.\textsc{sf} COMP 3\SG{} come.\textsc{lf}  \\\hfill(p.\,202)
 \glt `Zan thinks about coming.'
\end{exe}

Using \xarg, they propose for \word{pans} `think' 
the following description. The complement of \word{pans} must have an \xarg coindexed with the subject of \word{pans}, but its \subjl is not constrained: it can be a saturated verbal complement (whose \subjv is the empty list) or a VP complement (whose \subjv is not the empty list).

\begin{exe}
\ex \word{pans} `think':\\
\begin{avm}
	\[subj & \<NP$_{\@i}$ \> \\
	comps & \<\[head & \[verb\\
	xarg & \[ind & \@i\]\]\\
		marking & pou  
		%cont & \[ind & \@2\]
		 \]\>
	%cont & \[ind & s \\
		%	rels & \{\[\asort{think-rel}
		%	arg \@i \\
		%	arg & \@2\]\}\]
	\]
\end{avm}
\end{exe}

 See also \citet{Sag2007a}:408-409, \citet{KaySag2009} for the obligatory control of possessive determiners in English expressions such as \emph{keep one's cool}, \emph{lose one's temper}, with an \xarg feature on nouns and NPs:
\begin{exe}
\ex \begin{xlist}
\ex John lost his / * her temper.
\ex Mary lost * his / her temper.
\end{xlist}
\end{exe}

Raising may also involve verbs taking a finite complement with a pronominal subject. It is the case in English with \emph{look like} which has been called ``copy raising'' (\citealp{Rogers74a-u,Hornstein99a-u} a.o.): it takes a finite complement with an overt subject, but this subject must be coindexed with the matrix subject; it is a raising predicate, as shown by the possibility of the expletive \emph{there}:

\eal
\ex Peter looks like he's tired. / \# Mary is coming.
\ex There looks like there's going to be a storm. \citep[ex 17]{Sag2007a}\\
\zl

The verb \emph{look like} can thus have the subject of its sentential complement  be shared with its own subject:\\
\emph{look like}: \argst \sliste{ \ibox{1}, S[\xarg \ibox{1}] }

This is also the case in English tag questions, since the subject of the tag question must be expressed and shared with that of the matrix clause: 

\eal
\ex Paul left, didn't he?
\ex It rained yesterday, didn't it ?
\zl


The types for subject-raising and subject-control verbs in  \ref{rais-1} \ref{cont-1} will thus be revised as follows:\\
\eal
\ex \type{subj-rsg-verb} \impl \argst \begin{avm}  \@1 \append \<\[x-arg & \@1\]\> \end{avm} 
\ex \type{subj-cont-verb} \impl \argst \begin{avm}  \<NP$_{\@i}$, ...\[x-arg & \<\[ind & \@i\]\>\]\> \end{avm}
\zl

\section{Copular constructions}
\label{sec-copular-constructions}

Copular verbs can also be considered as ``raising'' verbs \citep{Chomsky81a:106}. 
While attributive adjectives are adjoined to N or NP, predicative adjectives are complements of copular verbs and share their subject with these verbs. Like raising verbs (Section~\ref{control-sec-intro}), copular verbs come in two varieties: subject copular verbs (\word{be}, \word{get}, \word{seem}, \ldots), and object copular verbs (\word{consider}, \word{prove}, \word{expect}, \ldots).

Let us review a few properties of copular constructions.
The adjective selects for the verb's subject or object: \word{likely} may selects a nominal or a sentential argument, while \word{expensive} only takes a nominal argument. As a result, \word{seem} combined with \word{expensive} only takes a nominal subject, and \word{consider} combined with the same adjective only takes a nominal object.


\begin{exe}
\ex \label{storm}
\begin{xlist}
\ex{} [A storm] / [That it rains] seems likely.
\ex{} [This trip] / * [That he comes ] seems expensive.
\end{xlist}
\ex \begin{xlist}
\ex 	I consider [a storm] likely / likely [that it rains].
\ex 	I consider [this trip] expensive/ * expensive [that he comes].
\end{xlist}	
\end{exe}


A copular verb thus takes any subject (or object) allowed by the predicate: \emph{be} can take a PP subject in English (\ref{under2}), \emph{werden} takes no subject when combined with a subjectless predicate like \emph{schlecht} (bad) in German (\ref{german3}):

\eal
\ex{}[Under the bed] is a good place to hide \label{under2}
\ex 
\gll Ihm wurde schlecht [German] \citep[\page 72]{Mueller2002b}\label{german3}\\
     him.\DAT{} got bad\\
\glt `He got sick.'
\zl

 In English, \word{be} also has the properties of an auxiliary, see Section~\ref{control-sec-copula-verbs}.

\subsection{The problems with a small clause analysis}

To account for these properties, Transformational Grammar since \citet{Stowell1983} and
\citet{Chomsky1986} has proposed a clausal or \emph{small clause} analysis: the predicative
adjective heads a (small) clause; the subject of the adjective raises to the subject position of the
embedding clause (\ref{rais1}) or stays in its subject position and receives accusative case from
the matrix verb via so-called Exceptional Case Marking\is{Exceptional Case Marking} (ECM) (\ref{ecm}).


\begin{exe}
\ex  {}[\sub{NP} e] be [\sub{S} John sick] $\leadsto$  [\sub{NP} John ] is  [\sub{S} $e_{i}$ sick] \label{rais1}
\ex   We consider [\sub{S} John sick] \label{ecm}
\end{exe}

It is true that the adjective may combine with its subject to form a verbless sentence. It is the
case in AAVE \citep{Bender2001a}, in French \citet{Laurens2008} and creole languages
\citet{HenriandAbeille2007}, in Slavic languages \citep{Zec87a-u}, in Semitic languages (see
\citealp{Alqurashi:Borsley:14}), among others. 

\begin{exe}
\ex \gll Magnifique ce chapeau !\\
beautiful this hat\\\hfill{(French)}
\glt `what a beautiful hat'
\end{exe}

But this does not entail that \emph{be} takes a sentential complement. 


%a problem for the raising principle? In French, and other Romance languages (Abeillé and Godard 2000), the predicate can be pronominalized as a complement:\\

%\begin{exe}
%\ex \gll Paul est malade / médecin / en forme.
%Paul is sick / a doctor / in a good shape\\
%\ex \gll Paul l'est.
%Paul it is\\
%\glt Paul is so
%\end{exe}

\citet[Chapter~3]{PollardandSag1994} present several arguments against a (small) clause analysis. The putative sentential source is sometimes attested (\ref{cons1}) but more often ungrammatical:

	
\eal
\ex[]{
John gets / becomes sick.
}
\ex[*]{
It gets / becomes that John is sick.
}
\ex[]{
\label{cons1}
John considers Lou a friend / that Lou is a friend.
}
\ex[]{
Paul regards Mary as crazy.
}
\ex[*]{
Paul regards that Mary is crazy.
}
\zl

	
When a clausal complement is possible, its properties differ from those of the putative small clause. Pseudo-clefting shows that \textit{Lou a friend} is not a constituent in (\ref{consider}).

\eal
\ex[]{
We consider Lou a friend.\label{consider}
}
\ex[*]{
What we consider is Lou a friend.
}
\ex[]{
We consider [that Lou is a friend].
}
\ex[]{
What we consider is [that Lou is a friend].
}
\zl

Following \citet{Bresnan1982}, \citet[113]{PollardandSag1994} also show that Heavy-NP shift applies to the putative subject of the small clause, exactly as it applies to the first complement of a ditransitive verb:

\begin{exe}
\ex \begin{xlist}
\ex   We woud consider [any candidat] [acceptable]
\ex We would consider [acceptable]  [any candidate who supports the proposed amendment].
\ex   I showed [all the cookies] [to Dana]
\ex I showed [to Dana]  [all the cookies that could be made from betel nuts and molasses].  
\end{xlist}

\end{exe}

 Indeed, the ``subject'' of the adjective with object"=raising verbs has all the properties of an
 object: it bears accusative case and it can be the subject of a passive:

\begin{exe}
\ex \begin{xlist}
\ex We consider him / * he guilty.
\ex 	We consider that he / * him is guilty.
\ex 	He was proved guilty (by the jury).	
\end{xlist}
\end{exe}
	

Furthermore, the matrix verb may select the head of the putative small clause, which is not the case
with verbs taking a clausal complement, and which violates the locality of subcategorization. The
verb \word{expect} takes a predicative adjective but not a preposition or a nominal predicate (\ref{ex-expect}),
\word{get} selects a predicative adjective or a preposition (\ref{ex-get}), but not a predicative nominal, while
\word{prove} selects a predicative noun or adjective but not a preposition (\ref{ex-prove}).


\eal
\label{ex-expect}
\ex I expect that man (to be) dead  by tomorrow. \citep[\page 102]{PollardandSag1994}
\ex I expect that island *(to be) off the route. (p.\,103)
\ex I expect that island *(to be) a good vacation spot.(p.\,103)
\zl
\ea
\label{ex-get}
John got political / * a success. (p.\,105)	
\z
\eal
\label{ex-prove}
\ex Tracy proved the theorem (to be) false. (p.\,100)
\ex I proved the weapon *(to be) in his possession.	(p.\,101)
\zl
	


\subsection{An HPSG analysis of copular verbs}
\label{control-sec-copula-verbs}
	
Copular verbs such as \word{be} or \word{consider} are analysed as subtypes of subject-raising verbs and object"=raising verbs respectively (\ref{rsg}). They share their subject (or object) with the unexpressed subject of their predicative complement. Instead of taking a VP complement, they take a predicative complement (\prd $+$), which they may select the category of.
 The two
lexical types for verbs that take a predicative complement are as follows:

\begin{exe} 
\ex	\type{subj-pred-v}	\impl \argst \begin{avm}  \@1 \append \<\[subj & \@1\\
prd & $+$\]\> \end{avm}
\ex \type{obj-pred-v} \impl \argst \begin{avm}  \<\NP\> \append \@1 \append \<\[subj & \@1\\
prd & $+$\]\> \end{avm}
\end{exe}

A copular verb like \word{be} or \word{seem} does not assign any semantic role to its subject, while
verbs like \word{consider} or \word{expect} do not assign any semantic role to their object. For more details, see \citew{PollardandSag1994} and  \citew[Section~2.2.7]{Mueller2002b}, \citew{MuellerPredication, VanEynde2015}. 
The lexical descriptions for predicative \word{seem} and predicative 
\word{consider} inherit from the \type{subject-pred-v} type and \type{object-pred-v} type
respectively, and are as follows:
\inlinetodostefan{Stefan: fix these lexical items. Raise everything? Use append
  for consider. Should this be \argst? it is easier to show a word that is later used in the figure
  ; AA: I use append in the types, but for English verbs, we can assume that there are no
  subjectless predicates.\\
Stefan: But this assumption is not in the theory. Hence you have to use append to reflect what you
stated in the types, don't you? AA arg-st added: the types are more general and the words are more specified}


\eas
\word{seem}:\\
\begin{avm}
\[\tp{subj-pred-v-word}\\
subj & \<\@1\>\\
comps & \<\@2\[head & \[prd & $+$\]\\
		 subj & \<\@1\> \\
		 cont & \[ind & \@3\]\]\>\\
arg-st & \<\@1, \@2\> \\
cont & \[ind & s \\
		rels & \{\[\asort{seem-rel} 
				soa & \@3\]\}\]\]		
\end{avm}
\zs

\eas
\word{consider}:\\
\begin{avm}
\[\tp{obj-pred-v-word}\\
subj & \<\@1\NP$_{i}$\>\\
comps & \<\@2, \@3\[head & \[prd & $+$\]\\
		 subj & \<\@2\> \\
		 cont & \[ind & \@4\]\]\>\\
arg-st & \<\@1, \@2, \@3\> \\
cont & \[ind & s \\
		rels & \{\[\asort{consider-rel} 
				exp & i \\
				soa & \@4\]\}\]\]		
\end{avm}	
\zs

	
The subject of \word{seem} is unspecified: it can be any category selected by the predicative complement; the same holds for the first complement of \word{consider}: it can be any category selected by the predicative complement (see examples (\ref{storm}) above).
\word{Consider} selects a subject and two complements, but only takes two semantic arguments: one corresponding to its subject, and one corresponding to its predicative complement. It does not assign a semantic role to its non"=predicative complement.\\
Let us take the example
	\textit{Paul seems happy}. As a predicative adjective, \word{happy} has a \headf [\prd $+$] and its \subjf is not the empty list: it subcategorizes for a nominal subject and assigns a semantic role to it, as shown in (\ref{happy2}).
	
\eas
\label{happy2}
\word{happy}:\\
\begin{avm}
\[phon & \phonliste{ happy }\\
head & \[\asort{adj}
	 prd & $+$\]\\
subj & \<\NP$_{i}$\> \\
comps & \eliste \\
cont & \[ind & s \\
rels & \{\[\asort{happy-rel}
exp & i\]\}\]
\]	
\end{avm}
\zs

In the trees in the Figures~\ref{fig-happy} and~\ref{fig-cons}, the \subjf of \word{happy} is
shared with the \subjf of \word{seem} and the first element of the \comps list of
\word{consider}.\footnote{In what follows, we ignore adjectives taking complements. As noted in section 1 ref\, adjectives may take a non finite VP complement and fall under a control or raising type: as a subject-raising adjective, \word{likely} shares the \textsc{synsem} value of its subject with the expected subject of its \VP complement; as a subject-control adjective, \word{eager} coindexes both subjects.
Such adjectives thus inherit from subj-rsg"=lexeme and subj-control"=lexeme type, respectively, as well as from adjective"=lexeme type. In some languages, copular constructions are complex predicates, which means that the copular verb inherits the complements of the adjective as well, see Abeillé and Godard 2000}
\inlinetodostefan{please add: Abeillé and Godard 2000. Varieties of esse in Romance languages, Proceedings of the 7th HPSG conference, 2-22\\
fix figures. Is \ibox{1} a list or an element of a list? If the complete
\subjv is supposed to be raised, my fix is technically not correct. AA seems ok to me, the numbers are synsem descriptions, these figures are similar to the previous ones}


\begin{figure}
% \begin{tikzpicture}[baseline, sibling distance=2pt, level distance=60pt, scale=.9]
% 	\Tree
% 	[.{\begin{avm}
% 		\[phon & \phonliste{ Paul seems happy }\\
% 			subj & \eliste \\
% 			comps & \eliste\]
% 		\end{avm}}
% 		{\begin{avm} \[phon & \phonliste{ Paul } \\
% 			synsem & \@1 \]
% 		\end{avm}}
% 		[.{\begin{avm}
% 			\[phon & \phonliste{ seems happy }\\
% 			subj & \<\@1\>\]
% 			\end{avm}}
% 		 {\begin{avm}
% 			\[phon & \phonliste{ seems } \\
% 			subj & \<\@1\>\\
% 			comps & \<\@2 \[subj & \<\@1\>\]\>\\
% 			\]
% 			\end{avm}} 
% 		{\begin{avm}
% 			\[phon & \phonliste{ happy }\\
% 				synsem & \@2  \]	
% 			\end{avm}}  
% 		]
% 	]
% \end{tikzpicture}
\begin{forest}
[{\begin{avm}
    \[\type{S}\\
    phon & \phonlist{ Paul seems happy }\\
      subj & elist \\
      comps & elist\]
  \end{avm}}
   [{\begin{avm} 
     \[\type{NP}\\
     phon & \phonlist{ Paul } \\
       synsem & \@1 \]
     \end{avm}}]
   [{\begin{avm}
       \[\type{VP}\\
       phon & \phonlist{ seems happy }\\
         subj & \<\@1\>\\
         comps & \elist\]
     \end{avm}}
     [{\begin{avm}
         \[\type{V}\\
         phon & \phonlist{ seems } \\
           subj & \<\@1\>\\
           comps & \<\@2 \[subj & \<\@1\>\]\>\\
         \]
       \end{avm}}] 
     [{\begin{avm}
         \[\type{AP}\\
         phon & \phonlist{ happy }\\
           synsem & \@2  \]	
       \end{avm}} ]]]
\end{forest}
\caption{\label{fig-happy}A sentence with an intransitive copular verb}
\end{figure}



\begin{figure}
% \begin{tikzpicture}[baseline, sibling distance=2pt, level distance=60pt, scale=.9]
% 	\Tree
% 	[.{\begin{avm}
% \[phon & \phonliste{ Mary considers Paul happy }\\
% subj & \eliste\\
% comps & \eliste\]		
% \end{avm}}
% 	{\begin{avm}\[phon & \phonliste{ Mary }\\
% 			synsem & \@3 \]
% 		\end{avm}}
% 	[.{\begin{avm}
% \[phon & \phonliste{ considers Paul happy }\\
% subj & \<\@3 NP\>\\
% comps & \eliste\]		
% \end{avm}}
% 	{\begin{avm}
% \[phon & \phonliste{ considers } \\
% subj & \<\@3 NP\>\\
% comps & \<\@1, \@2 \[
% 		 subj & \<\@1\> \]\>\]		
% \end{avm}}
% 	{\begin{avm} \[phon & \phonliste{ Paul } \\
% 			synsem & \@1 \]
% 		\end{avm}}
% 	{\begin{avm}
% 			\[phon & \phonliste{ happy }\\
% 				synsem & \@2 \]	
% 			\end{avm}}
% 	] ]
% \end{tikzpicture}
\begin{forest}
[{\begin{avm}
    \[\type{S}\\
    phon & \phonlist{ Mary considers Paul happy }\\
      subj & \elist\\
      comps & \elist\]		
  \end{avm}}
   [{\begin{avm}
   \[\type{NP}\\
   phon & \phonliste{ Mary }\\
			synsem & \@3 \]
		\end{avm}}]
   [{\begin{avm}
       \[\type{VP}//
       phon & \phonliste{ considers Paul happy }\\
         subj & \<\@3 \>\\
         comps & \elist\]		
     \end{avm}}
	[\avmtmp{
            [\type{V}\\
            phon  & \phonlist{ considers } \\
             subj  & < \@3 >\\
             comps & < \@1, \@2 [ subj & < \@1 >  ] > ]		
          }]
	[{\begin{avm}
	 \[\type{NP}\\
	 phon & \phonlist{ Paul } \\
			synsem & \@1 \]
		\end{avm}}]
	[{\begin{avm}
			\[\type{AP}\\
			phon & \phonlist{ happy }\\
				synsem & \@2 
				\]	
			\end{avm}}]
	] ]
\end{forest}	
\caption{\label{fig-cons}A sentence with a transitive copular verb}
\end{figure}

\citet{PollardandSag1994} mention a few verbs taking a predicative complement which can be considered as control verbs. A verb like \word{feel} selects a nominal subject and assigns a semantic role to it. 

\begin{exe}
\ex John feels tired / in a good mood.
\end{exe}

\noindent
It inherits from the subject-control-verb type (\ref{cont}); its lexical description is as follows:

\begin{exe}
\ex 	\word{feel}: 
\begin{avm}
	\[subj & \<\@1NP$_{\@i}$ \> \\
	comps & \<\@2\[head & \[prd & $+$\] \\
		subj & \<\[ind & \@i\]\> \\
		cont & \[ind & \@3\] \]\>\\
	arg-st & \<\@1,\@2\> \\
	cont & \[ind & s \\
			rels & \{\[\asort{feel-rel}
			exp & \@i \\
			soa & \@3\]\}\]
	\]
\end{avm}
\end{exe}


\subsection{Copular verbs in Mauritian}

As shown by \citet{HenriandLaurens2011}, and as noted earlier, Mauritian data argue in favor of a non"=clausal
analysis. A copular verb takes a short form before an attributive complement, and a long form before
a clausal one. Despite the lack of inflection on the embedded verb, and the possibility of subject
prodrop,  clausal complements differ from non"=clausal complements by the following properties: they
do not trigger the matrix verb short form (SF), they may be introduced by the complementizer \word{ki},
their subject is a weak pronoun (\word{mo} `I', \word{to} `you'). On the other hand, a VP or AP complement
cannot be introduced by \word{ki}, and an NP complement must be realized as a strong pronoun (\word{mwa} `me',
\word{twa} `you'). See section \ref{sec-maurit} above for the alternation between verb short form (SF) and long form (LF).

\begin{exe}
\ex \begin{xlist}
\ex 
\gll Mari ti res  malad.\\
     Mari \textsc{pst} remain.\textsc{sf} sick\\\hfill\citep[\page 198]{HenriandLaurens2011}
\glt `Mari remained sick.'

\ex 
\gll Mari trouv  so mama malad\\
     Mari find.\textsc{sf} \POSS{} mother sick\\
\glt `Mari finds her mother sick.'

\ex 
\gll Mari trouve (ki) mo malad\\
     Mari find.\textsc{lf} \hspaceThis{(}that 1\SG.\textsc{wk} sick\\
\glt `Mari finds that I am sick.'

\ex 
\gll Mari trouv ki mwa malad\\
     Mari find.\textsc{sf} \hspaceThis{(*}that 1\SG.\textsc{str} sick\\
\glt `Mari finds me sick.'
\end{xlist}
\end{exe}

\citet[\page 218]{HenriandLaurens2011} conclude that ``Complements of raising and control verbs systematically pattern with non-clausal phrases such as NPs or PPs. This kind of evidence is seldom available in world's languages because heads are not usually sensitive to the properties of their complements. The analysis as clause or small clauses is also problematic because of the existence of genuine verbless clauses in Mauritian which pattern with verbal clauses and not with complements of raising and control verbs.''




\section{Auxiliaries as raising verbs}
\label{sec-auxiliaries-as-raising-verbs}

Following \citep{Ross69a-u,Gazdaretal1982, Sagetal2020}, 
 \word{be}, \word{do}, \word{have}, and modals (e.g., \word{can}, \word{should}) in HPSG are not considered a special part of speech (\type{Aux} or \type{Infl}) but verbs with the Head property in (\ref{ex-head-value-of-aux-elements}):

\begin{exe}
\ex \label{ex-head-value-of-aux-elements}
  \type{auxiliary-verb} \impl \begin{avm}
 \[head & \[aux & $+$\]	\]
 \end{avm}
 \end{exe}
 
 English auxiliaries take \VP (or XP) complements and do not select their subject, just like subject-raising verbs. They are thus compatible with non"=referential subjects, such as meteorological \word{it} and existential \textit{there}. They select the verb form of their non"=finite complements: \textit{have} selects a past participle, \textit{be} a gerund-participle, \textit{can} and \textit{will} a bare form.

	
\begin{exe}
\ex \begin{xlist}
\ex Paul has left.
\ex Paul is leaving.
\ex Paul can leave.
\ex It will rain.
\ex There can be a riot.
\end{xlist}	
\end{exe}

In this approach, English auxiliaries are subtypes of subject-raising-verbs, and thus take a \VP (or
XP) complement and share their subject with the unexpressed subject of the non"=finite verb.\footnote{ \emph{Be} is an auxiliary and a subj-raising verb with a \prd$+$ complement, see
  section \ref{control-sec-copula-verbs} above, or a gerund-participle \VP complement,
  different from the identity \emph{be} which is not a raising verb (Van
  Eynde 1995, Müller 2009 on Predication). A verb like \emph{dare}, shown to be an auxiliary by its postnominal
  negation, is not a raising verb but a subject-control verb:
\eal
\ex[]{
He is lazy and sleeping.
}
\ex[]{
I dare not be late.
}
\ex[\#]{
It will not dare rain.
}
\zllast
}
The lexical descriptions for the auxiliaries \word{will} and \word{have} look as follows: 

\ea
\word{will}:\\*
\begin{avm}
	\[head & \[aux &  $+$\]\\
	subj & \<\@1 \> \\
	comps & \<\@2\VP \[head & \[vform & bse\]  \\
						subj & \<\@1\> \\
						cont & \[ind & \@3\] \]\>\\
	arg-st & <\@1,\@2>\\
	cont & \[ind & s \\
			rels & \{\[\asort{future-rel}
			soa & \@3\]\}\]
	\]
\end{avm}
\z
\eas
\word{have}:\\*
\begin{avm}
		\[head & \[aux & $+$\]\\
		subj & \<\@1 \> \\
	comps & \<\@2\VP \[head & \[vform & past-part\] \\
		subj & \<\@1\> \\
		cont & \[ind & \@3\] \]\>\\
	arg-st & <\@1,\@2>\\
	cont & \[ind & s \\
			rels & \{\[\asort{perfect-rel}
			soa & \@3\]\}\]
	\]
\end{avm}	
\zs

To account for their NICE (\isi{negation}, \isi{inversion}, \isi{contraction}, \isi{ellipsis}) properties, Kim and Sag (2002) use a binary head feature \aux, so that only [\aux $+$] verbs may allow for subject inversion (\ref{inv}), sentential negation (\ref{neg}), contraction or VP ellipsis (\ref{ell}). See  \crossrefchapterw[Section~\ref{sec-head-movement-vs-flat}]{order} on subject inversion, \crossrefchapterw[Section~\ref{sec-sentential-negation}]{negation} on negation and \crossrefchapterw{ellipsis} on post-auxiliary ellipsis (section 5).\footnote{Copular \word{be} has the NICE properties (\textit{Is John happy?}), it is an auxiliary verb with [\prd $+$] complement. Since \emph{to} allows for VP ellipsis, it is also analysed as an auxiliary verb: \emph{John promised to work and he started to}. See \citew*{GPS82a-u}.}

\eal
\ex[]{
Is Paul working? \label{inv}
}
\ex[*]{
Keeps Paul working?
}
\ex[]{
Paul is (probably) not working.\label{neg}
}
\ex[*]{
Paul keeps (probably) not working.
}
\ex[]{
John promised to come and he will. \label{ell}
}
\ex[*]{
John promised to come and he seems.
}
\zl

\noindent
Subject raising verbs such as \word{seem}, \word{keep} or \word{start} are [\aux $-$].

\citet{Sagetal2020} revised this analysis and proposed a new analysis couched in \sbcg (\citealp{Sag2012a}; see also \crossrefchapteralp[Section~\ref{sec-sbcg}]{cxg}). The descriptions used below were translated into the feature geometry of Constructional HPSG \citep{Sag97a}, which is used in this volume. In their approach, the head feature \aux is both lexical and constructional: the constructions restricted to auxiliaries require their head to be [\aux $+$], while the constructions available for all verbs are [\aux $-$]. In this approach, non"=auxiliary verbs are lexically specified as [\aux $-$]:

\begin{exe}
\ex \type{non-auxiliary-verb} \impl \begin{avm}\[head & \[aux & $-$\\
 inv & $-$ \] \]\end{avm}
\end{exe}

 Auxiliary verbs, on the other hand are unspecified for the feature \aux, and are contextually specified; except for unstressed \word{do}  which is [\aux $+$] and must occur in constructions restricted to auxiliaries.

\eal
\ex[]{
Paul is working. [\aux $-$]
}
\ex[]{
Is Paul working? [\aux $+$]
} \label{inv1}
\ex[*]{
John does work. [\aux $-$]
}
\ex[]{
Does John work? [\aux $+$]
}\label{inv2}
\zl

Subject inversion is handled by a subtype of head-subject-complement phrase, which is independently needed for verb initial languages like \ili{Welsh} \parencites[]{Borsley1999}[]{SWB2003a}.\footnote{As noted in Abeillé and Borsley chapter 1, in some HPSG work, e.g. Sag, Wasow, and Bender (2003: 409-414), examples like \ref{inv1} \ref{inv2} are analysed as involving an auxiliary verb with two complements and no subject. This approach has no need for an additional phrase type, but it requires an alternative valence description for auxiliary verbs.} It is a specific (non"=binary) construction, of which other constructions such as \type{polar-interrogative-clause} are subtypes, and whose head must be [\textsc{inv} $+$].  
\todostefan{please add Borsley, R. D. (1999), Mutation and constituent structure in Welsh, Lingua109, 263-300}
\begin{exe}
\ex \type{initial-aux-cx} \impl \begin{avm}
		\[subj & \eliste \\
                  comps & \eliste \\
                  head-dtr & \@0 \[aux & $+$ \\
                   subj & \@1\\
                    comps & \@2 \]\\
                  dtrs & \< \@0 \>~$\oplus$ \@1 $\oplus$ \@2
                  \] \end{avm}
  \end{exe}          
       
Most auxiliaries are lexically unspecified for the feature INV and allow for both constructions (non"=inverted and inverted), while the 1st person \word{aren't} is obligatory inverted (lexically marked as [\textsc{inv} $+$]) and the modal \word{better} obligatory non"=inverted (lexically marked as [\textsc{inv} $-$]):

\eal
\ex[]{
Aren't I dreaming?
}
\ex[*]{
I aren't dreaming.
}
\ex[]{
We better be carefull.
}
\ex[*]{
Better we be carefull?
}
\zl

While the distinction is not always easy to make between VP ellipsis and Null complement anaphora (\textit{Paul tried}), \citeauthor{Sagetal2020} observe that certain elliptical constructions are restricted to auxiliaries, for example pseudogapping (see also \crossrefchaptert{ellipsis} Kim and Nykiel this volume, \citep{Miller2014a-u}).

\eal
\ex[]{
John can eat more pizza than Mary can tacos.
}
\ex[]{Larry might read the short story, but he won’t the play.
}
\ex[*]{
Ann seems to buy more bagels than Sue seems cupcakes.
}
\zl

This could be captured by having the relevant auxiliairies optionally inherit the complements of their verbal complement. 
\footnote{See \citew{KimandSag2002} for a comparison of French and English auxilaries, \citew{AG2002b-u} for a thorough analysis of French auxiliaries as ``generalized'' raising verbs, inheriting not only the subject but also any complement from the past participle; such generalized raising was first suggested by \citet{HN89a,HN94a} for German and has been adopted since in various analyses of verbal complexes in German \citep{Kiss95a,Meurers2000b,Kathol2001a,Mueller99a,Mueller2002b}, Dutch \citep{BvN98a} and Persian \citep[Section~4]{MuellerPersian}. See also \crossrefchaptert{complex-predicates}.}
A revised version of 'will' with complement inheritance could be the following:
\\
will: ARG-ST <[1], VP[SUBJ <[1]>, COMPS [2]]> + [2]
\\
As observed by \citet{ArnoldandBorsley2008}, auxiliaries can be stranded in certain non-restrictive relative clauses such as (\ref{aux1}), no such possibility is open to non-auxiliary verbs (\ref{nonaux}) \crossrefchapterp{relative-clauses}:

\eal
\ex[]{
Kim was singing, which Lee wasn't. \label{aux1}
}
\ex[*]{
Kim tried to impress Lee, which Sandy didn't try. \label{nonaux}
}
\zl

The HPSG analysis sketched here captures a very wide range of facts, and expresses both generalizations (English auxiliaries are subtypes of subject-raising verbs) and lexical idiosyncrasies (copula \emph{be} takes non verbal complements, 1st person \emph{aren't} triggers obligatory inversion etc).


	
\section{Conclusion}
Complements of 'raising' and control verbs have been either analyzed as clauses \citep{Chomsky81a} or small clauses \citep{Stowell81a-u,Stowell1983} in Mainstream Generative Grammar.
As in LFG \citep{Bresnan1982}, 'raising' and control predicates are analysed as taking non-clausal open complements in HPSG \citep{PollardandSag1994}, with sharing or coindexing the (unexpressed) subject of the embedded predicate with their own subject (or object). This leads to a more accurate analysis of 'object"=raising' verbs as ditransitive, without the need for an exceptional case marking device. This analysis naturally extends to pro-drop and ergative languages; it also makes correct empirical predictions for languages marking clausal complementation differently from VP complementation. A rich hierarchy of lexical types enables verbs and adjectives taking non"=finite or predicative complements to inherit from a raising-type or a control-type. The Raising Principle prevents any other kind of non"=canonical linking between semantic argument and syntactic argument. A semantic-based control theory predicts which predicates are subject-control and which object-control. The ``subject-raising'' analysis has been successfully extended to copular and auxiliary verbs, without the need for an Infl category.




\section*{Abbreviations}

\begin{tabularx}{.45\textwidth}{lX}
\textsc{av} & Agentive Voice\\
\textsc{lf} & long form\\ 
\textsc{ov} & Objective Voice\\
\textsc{sf} & short form\\
\textsc{str} & strong\\
\textsc{wk} & weak\\

\end{tabularx}

\section*{Acknowledgements}

I am grateful to the reviewers and the coeditors for the volume for their helpful comments.
{\sloppy
\printbibliography[heading=subbibliography,notkeyword=this] 
}

\end{document}


%      <!-- Local IspellDict: en_US-w_accents -->
