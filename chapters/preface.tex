\addchap{Preface}
\begin{refsection}

Head-driven Phrase Structure Grammar (HPSG) is a declarative (or, as is often said,
constraint-based) monostratal approach to grammar which dates back to early 1985, when Carl Pollard
presented his Lectures on HPSG. It was developed initially in joint work by Pollard and Ivan Sag,
but many other people have made important contributions to its development over the decades. It
provides a framework for the formulation and implementation of natural language grammars which are
(i) linguistically motivated, (ii) formally explicit, and (iii) computationally tractable. From the
very beginning it has involved both theoretical and computational work seeking both to address the
theoretical concerns of linguists and the practical issues involved in building a useful natural
language processing system.

HPSG is an eclectic framework which has drawn ideas from the earlier Generalized Phrase Structure
Grammar (GPSG, \citealp{GKPS85a}), Categorial Grammar \citep{Ajdukiewicz35a-u}, and Lexical"=Functional
Grammar (LFG, \citealp{Bresnan82a-ed}), among others. It has naturally evolved over the decades. Thus, the construction"=based version of
HPSG, which emerged in the mid-1990s \citep{Sag97a,GSag2000a-u}, differs from earlier work
\citep{ps,ps2} in employing complex hierarchies of phrase types or
constructions. Similarly, the more recent Sign-Based Construction Grammar approach differs from
earlier versions of HPSG in making a distinction between signs and constructions and using to make a
number of simplifications \citep{Sag2012a}.

Over the years, there have been groups of HPSG researchers in many locations engaged in both
descriptive and theoretical work and often in building HPSG-based computational systems. There have
also been various research and teaching networks, and an annual conference since 1993. The result of
this work is a rich and varied body of research focusing on a variety of languages and offering a
variety of insights. The present volume seeks to provide a picture of where HPSG is today. It begins
with a number of introductory chapters dealing with various general issues. These are followed by
chapters outlining HPSG ideas about some of the most important syntactic phenomena. Next are a
series of chapters on other levels of description, and then chapters on other areas of
linguistics. A final group of chapters considers the relation between HPSG and other theoretical
frameworks.

It should be noted that for various reasons not all areas of HPSG research are covered in the
handbook (e.g., phonology). So, the fact that a particular topic is not addressed in the handbook
should not be interpreted as an absence of research on the topic. Readers interested in such topics
can refer to the HPSG online bibliography maintained at the Humboldt Universität zu Berlin.\footnote{%
\url{https://hpsg.hu-berlin.de/HPSG-Bib/}, 2020-03-18.
}

All chapters were reviewed by one author and at least one of the editors. All chapters were reviewed
by Stefan Müller. Jean-Pierre Koenig and Stefan Müller did a final round of reading all papers and
checked for consistency and cross-linking between the chapters.


\section*{Open access}


Many authors of this handbook have previously been involved in several other handbook projects (some that cover various aspects of HPSG), and by now there are at least five handbook articles on HPSG available. But the editors felt that writing one authoritative resource describing the framework and being available free of charge to everybody was an important service to the linguistic community. We hence decided to publish the book open access with Language Science Press.

% militant version starts here: =:-)
%% The authors of this handbook were involved in many, many other handbook projects before. By now
%% there are at least five handbook articles on HPSG available.
%% % Detmar Bob Levine
%% % Stefan (HSK)
%% % Stefan (Artenvielfalt)
%% % Stefan & Felix
%% % Stefan & Antonio
%% % Adam Przepiórkowski and Anna Kupść  in journal
%% The editors felt that writing these handbook articles for commercial publishers who will hide them
%% behind paywalls is a waste of time. Established researchers do not need further handbook articles
%% that people cannot read. What is needed instead is one authoritative resource describing the framework
%% and being available free of charge to everybody. We hence decided to publish the book open access
%% with Language Science Press.

\section*{Acknowledgements}

We thank all the authors for their great contributions to the book, and for
reviewing chapters and chapter outlines of the other authors. We thank
Frank Richter, Bob Levine, and Roland Schäfer for dicussion of points related to the handbook, and
Elizabeth Pankratz for extremely careful proofreading and help with typesetting issues. We also
thank Elisabeth Eberle and Luisa Kalvelage for doing bibliographies and typesetting trees of several
chapters and for converting a complicated chapter from Word into \LaTeX.

We thank Sebastian Nordhoff and Felix Kopecky for constant support regarding \LaTeX{} issues, both for
the book project overall and for individual authors. Felix implemented a new \LaTeX{} class for
typesetting AVMs, \texttt{langsci-avm}, which was used for typesetting this book. It is compatible with more
modern font management systems and with the \texttt{forest} package, which is used for most of the trees in this book.

We thank Sašo Živanović for writing and maintaining the \texttt{forest} package and for help
specifying particular styles with very advanced features. His package turned typesetting trees from a
nightmare into pure fun! To make the handling of this large book possible, Stefan Müller asked Sašo
for help with externalization of \texttt{forest} trees, which led to the development of
the \texttt{memoize} package. The HPSG handbook and other book projects by Stefan were an
ideal testing ground for externalization of \texttt{tikz} pictures. Stefan wants to thank
Sašo for the intense collaboration that led to a package of great value for everybody
living in the woods.

The code of the book is available on GitHub, and we hope that it may serve as a role model for future
publications of HPSG papers.


\printbibliography[heading=subbibliography]
\end{refsection}

%      <!-- Local IspellDict: en_US-w_accents -->
