%% -*- coding:utf-8 -*-

\documentclass[output=paper
	        ,collection
	        ,collectionchapter
 	        ,biblatex
                ,babelshorthands
                ,newtxmath
                ,draftmode
                ,colorlinks, citecolor=brown
]{./langsci/langscibook}

\IfFileExists{../localcommands.tex}{%hack to check whether this is being compiled as part of a collection or standalone
  % add all extra packages you need to load to this file 

% the ISBN assigned to the digital edition
\usepackage[ISBN=9783961102556]{ean13isbn} 

\usepackage{graphicx}
\usepackage{tabularx}
\usepackage{amsmath} 

%\usepackage{tipa}      % Davis Koenig
\usepackage{xunicode} % Provide tipa macros (BC)

\usepackage{multicol}

% Berthold morphology
\usepackage{relsize}
%\usepackage{./styles/rtrees-bc} % forbidden forest 08.12.2019


\usepackage{langsci-optional} 
% used to be in this package
\providecommand{\citegen}{}
\renewcommand{\citegen}[2][]{\citeauthor{#2}'s (\citeyear*[#1]{#2})}
\providecommand{\lsptoprule}{}
\renewcommand{\lsptoprule}{\midrule\toprule}
\providecommand{\lspbottomrule}{}
\renewcommand{\lspbottomrule}{\bottomrule\midrule}
\providecommand{\largerpage}{}
\renewcommand{\largerpage}[1][1]{\enlargethispage{#1\baselineskip}}


\usepackage{langsci-lgr}

\newcommand{\MAS}{\textsc{m}\xspace} % \M is taken by somebody

%\usepackage{./styles/forest/forest}
\usepackage{langsci-forest-setup}

\usepackage{./styles/memoize/memoize} 
\memoizeset{
  memo filename prefix={chapters/hpsg-handbook.memo.dir/},
  register=\todo{O{}+m},
  prevent=\todo,
}

\usepackage{tikz-cd}

\usepackage{./styles/tikz-grid}
\usetikzlibrary{shadows}


% removed with texlive 2020 06.05.2020
% %\usepackage{pgfplots} % for data/theory figure in minimalism.tex
% % fix some issue with Mod https://tex.stackexchange.com/a/330076
% \makeatletter
% \let\pgfmathModX=\pgfmathMod@
% \usepackage{pgfplots}%
% \let\pgfmathMod@=\pgfmathModX
% \makeatother

\usepackage{subcaption}

% Stefan Müller's styles
\usepackage{./styles/merkmalstruktur,german,./styles/makros.2020,./styles/my-xspace,./styles/article-ex,
./styles/eng-date}

\selectlanguage{USenglish}

\usepackage{./styles/abbrev}


% Has to be loaded late since otherwise footnotes will not work

%%%%%%%%%%%%%%%%%%%%%%%%%%%%%%%%%%%%%%%%%%%%%%%%%%%%
%%%                                              %%%
%%%           Examples                           %%%
%%%                                              %%%
%%%%%%%%%%%%%%%%%%%%%%%%%%%%%%%%%%%%%%%%%%%%%%%%%%%%
% remove the percentage signs in the following lines
% if your book makes use of linguistic examples
\usepackage{langsci-gb4e} 


%% St. Mü.: 03.04.2020
%% these two versions of the command can be used for series of sets of examples:
%% \eal
%% \ex
%% \ex
%% \zlcont
%% \ealcont
%% \ex
%% \ex
%% \zl

\let\zlcont\z
\def\ealcont{\exnrfont\ex\begin{xlist}[iv.]\raggedright}

% original version of \z
%\def\z{\ifnum\@xnumdepth=1\end{exe}\else\end{xlist}\fi}
% \zcont just removes \end{exe}
%\def\zcont{\ifnum\@xnumdepth=1\else\end{xlist}\fi}
\def\zcont{}

% Crossing out text
% uncomment when needed
%\usepackage{ulem}

\usepackage{./styles/additional-langsci-index-shortcuts}

% this is the completely redone avm package
\usepackage{./styles/langsci-avm}
\avmsetup{columnsep=.3ex,style=narrow}

%\let\asort\type*


\usepackage{./styles/avm+}


\renewcommand{\tpv}[1]{{\avmjvalfont\itshape #1}}

% no small caps please
\renewcommand{\phonshape}[0]{\normalfont\itshape}

\regAvmFonts

\usepackage{theorem}

\newtheorem{mydefinition}{Def.}
\newtheorem{principle}{Principle}

{\theoremstyle{break}
%\newtheorem{schema}{Schema}
\newtheorem{mydefinition-break}[mydefinition]{Def.}
\newtheorem{principle-break}[principle]{Principle}
}


%% \newcommand{schema}[2]{
%% \begin{minipage}{\textwidth}
%% {\textbf{Schema~\theschema}}]\hspace{.5em}\textbf{(#1)}\\
%% #2
%% \end{minipage}}


% This avoids linebreaks in the Schema
\newcounter{schemacounter}
\makeatletter
\newenvironment{schema}[1][]
  {%
   \refstepcounter{schemacounter}%
   \par\bigskip\noindent
   \minipage{\linewidth}%
   \textbf{Schema~\theschemacounter\hspace{.5em} \ifx&#1&\else(#1)\fi}\par
  }{\endminipage\par\bigskip\@endparenv}%
\makeatother

%\usepackage{subfig}





% Davis Koenig Lexikon

\usepackage{tikz-qtree,tikz-qtree-compat} % Davis Koenig remove

\usepackage{shadow}



\usepackage[english]{isodate} % Andy Lücking
\usepackage[autostyle]{csquotes} % Andy
%\usepackage[autolanguage]{numprint}

%\defaultfontfeatures{
%    Path = /usr/local/texlive/2017/texmf-dist/fonts/opentype/public/fontawesome/ }

%% https://tex.stackexchange.com/a/316948/18561
%\defaultfontfeatures{Extension = .otf}% adds .otf to end of path when font loaded without ext parameter e.g. \newfontfamily{\FA}{FontAwesome} > \newfontfamily{\FA}{FontAwesome.otf}
%\usepackage{fontawesome} % Andy Lücking
\usepackage{pifont} % Andy Lücking -> hand

\usetikzlibrary{decorations.pathreplacing} % Andy Lücking
\usetikzlibrary{matrix} % Andy 
\usetikzlibrary{positioning} % Andy
\usepackage{tikz-3dplot} % Andy

% pragmatics
\usepackage{eqparbox} % Andy
\usepackage{enumitem} % Andy
\usepackage{longtable} % Andy
\usepackage{tabu} % Andy              needs to be loaded before hyperref as of texlive 2020

% tabu-fix
% to make "spread 0pt" work
% -----------------------------
\RequirePackage{etoolbox}
\makeatletter
\patchcmd
	\tabu@startpboxmeasure
	{\bgroup\begin{varwidth}}%
	{\bgroup
	 \iftabu@spread\color@begingroup\fi\begin{varwidth}}%
	{}{}
\def\@tabarray{\m@th\def\tabu@currentgrouptype
    {\currentgrouptype}\@ifnextchar[\@array{\@array[c]}}
%
%%% \pdfelapsedtime bug 2019-12-15
\patchcmd
	\tabu@message@etime
	{\the\pdfelapsedtime}%
	{\pdfelapsedtime}%
	{}{}
%
%
\makeatother
% -----------------------------


% Manfred's packages

%\usepackage{shadow}

\usepackage{tabularx}
\newcolumntype{L}[1]{>{\raggedright\arraybackslash}p{#1}} % linksbündig mit Breitenangabe


% Jong-Bok

%\usepackage{xytree}

\newcommand{\xytree}[2][dummy]{Let's do the tree!}

% seems evil, get rid of it
% defines \ex is incompatible with gb4e
%\usepackage{lingmacros}

% taken from lingmacros:
\makeatletter
% \evnup is used to line up the enumsentence number and an entry along
% the top.  It can take an argument to improve lining up.
\def\evnup{\@ifnextchar[{\@evnup}{\@evnup[0pt]}}

\def\@evnup[#1]#2{\setbox1=\hbox{#2}%
\dimen1=\ht1 \advance\dimen1 by -.5\baselineskip%
\advance\dimen1 by -#1%
\leavevmode\lower\dimen1\box1}
\makeatother


% YK -- CG chapter

%\usepackage{xspace}
\usepackage{bm}
\usepackage{ebproof}


% Antonio Branco, remove this
\usepackage{epsfig}

% now unicode
%\usepackage{alphabeta}





\usepackage{pst-node}


% fmr: additional packages
%\usepackage{amsthm}


% Ash and Steve: LFG
\usepackage{./styles/lfg/dalrymple}

\RequirePackage{graphics}
%\RequirePackage{./styles/lfg/trees}
%% \RequirePackage{avm}
%% \avmoptions{active}
%% \avmfont{\sc}
%% \avmvalfont{\sc}
\RequirePackage{./styles/lfg/lfgmacrosash}

\usepackage{./styles/lfg/glue}

%%%%%%%%%%%%%%%%%%%%%%%%%%%%%%
%% Markup
%%%%%%%%%%%%%%%%%%%%%%%%%%%%%%
\usepackage[normalem]{ulem} % For thinks like strikethrough, using \sout

% \newcommand{\high}[1]{\textbf{#1}} % highlighted text
\newcommand{\high}[1]{\textit{#1}} % highlighted text
%\newcommand{\term}[1]{\textit{#1}\/} % technical term
\newcommand{\qterm}[1]{`{#1}'} % technical term, quotes
%\newcommand{\trns}[1]{\strut `#1'} % translation in glossed example
\newcommand{\trnss}[1]{\strut \phantom{\sqz{}} `#1'} % translation in ungrammatical glossed example
\newcommand{\ttrns}[1]{(`#1')} % an in-text translation of a word
%\newcommand{\feat}[1]{\mbox{\textsc{\MakeLowercase{#1}}}}     % feature name
%\newcommand{\val}[1]{\mbox{\textsc{\MakeLowercase{#1}}}}    % f-structure value
\newcommand{\featt}[1]{\mbox{\textsc{\MakeLowercase{#1}}}}     % feature name
\newcommand{\vall}[1]{\mbox{\textsc{\textup{\MakeLowercase{#1}}}}}    % f-structure value
\newcommand{\mg}[1]{\mbox{\textsc{\MakeLowercase{#1}}}}    % morphological gloss
%\newcommand{\word}[1]{\textit{#1}}       % mention of word
\providecommand{\kstar}[1]{{#1}\ensuremath{^*}}
\providecommand{\kplus}[1]{{#1}\ensuremath{^+}}
\newcommand{\template}[1]{@\textsc{\MakeLowercase{#1}}}
\newcommand{\templaten}[1]{\textsc{\MakeLowercase{#1}}}
\newcommand{\templatenn}[1]{\MakeUppercase{#1}}
\newcommand{\tempeq}{\ensuremath{=}}
\newcommand{\predval}[1]{\ensuremath{\langle}\textsc{#1}\ensuremath{\rangle}}
\newcommand{\predvall}[1]{{\rm `#1'}}
\newcommand{\lfgfst}[1]{\ensuremath{#1\,}}
\newcommand{\scare}[1]{`#1'} % scare quotes
\newcommand{\bracket}[1]{\ensuremath{\left\langle\mathit{#1}\right\rangle}}
\newcommand{\sectionw}[1][]{Section#1} % section word: for cap/non-cap
\newcommand{\tablew}[1][]{Table#1} % table word: for cap/non-cap
\newcommand{\lfgglue}{LFG+Glue}
\newcommand{\hpsgglue}{HPSG+Glue}
\newcommand{\gs}{GS}
%\newcommand{\func}[1]{\ensuremath{\mathbf{#1}}}
\newcommand{\func}[1]{\textbf{#1}}
\renewcommand{\glue}{Glue}
%\newcommand{\exr}[1]{(\ref{ex:#1}}
\newcommand{\exra}[1]{(\ref{ex:#1})}


%%%%%%%%%%%%%%%%%%%%%%%%%%%%%%
% Notation
%\newcommand{\xbar}[1]{$_{\mbox{\textsc{#1}$^{\raisebox{1ex}{}}$}}$}
\newcommand{\xprime}[2][]{\textup{\mbox{{#2}\ensuremath{^\prime_{\hspace*{-.0em}\mbox{\footnotesize\ensuremath{\mathit{#1}}}}}}}}
\providecommand{\xzero}[2][]{#2\ensuremath{^0_{\mbox{\footnotesize\ensuremath{\mathit{#1}}}}}}



\let\leftangle\langle
\let\rightangle\rangle

%\newcommand{\pslabel}[1]{}



  %add all your local new commands to this file


% Don't do this at home. I do not like the smaller font for captions.
% I just removed loading the caption packege in langscibook.cls
%% \captionsetup{%
%% font={%
%% stretch=1%.8%
%% ,normalsize%,small%
%% },%
%% width=.8\textwidth
%% }

\makeatletter
\def\blx@maxline{77}
\makeatother


\newcommand{\page}{}

\newcommand{\todostefan}[1]{\todo[color=orange!80]{\footnotesize #1}\xspace}
\newcommand{\todosatz}[1]{\todo[color=red!40]{\footnotesize #1}\xspace}

\newcommand{\inlinetodostefan}[1]{\todo[color=green!40,inline]{\footnotesize #1}\xspace}

\newcommand{\addpages}{\todostefan{add pages}}
\newcommand{\addglosses}{\todostefan{add glosses}}


\newcommand{\spacebr}{\hspaceThis{[}}

\newcommand{\danish}{\jambox{(\ili{Danish})}}
\newcommand{\english}{\jambox{(\ili{English})}}
\newcommand{\german}{\jambox{(\ili{German})}}
\newcommand{\yiddish}{\jambox{(\ili{Yiddish})}}
\newcommand{\welsh}{\jambox{(\ili{Welsh})}}

% Cite and cross-reference other chapters
\newcommand{\crossrefchaptert}[2][]{\citet*[#1]{chapters/#2}, Chapter~\ref{chap-#2} of this volume} 
\newcommand{\crossrefchapterp}[2][]{(\citealp*[#1]{chapters/#2}, Chapter~\ref{chap-#2} of this volume)}
\newcommand{\crossrefchapteralt}[2][]{\citealt*[#1]{chapters/#2}, Chapter~\ref{chap-#2} of this volume}
\newcommand{\crossrefchapteralp}[2][]{\citealp*[#1]{chapters/#2}, Chapter~\ref{chap-#2} of this volume}
% example of optional argument:
% \crossrefchapterp[for something, see:]{name}
% gives: (for something, see: Author 2018, Chapter~X of this volume)

\let\crossrefchapterw\crossrefchaptert



% Davis Koenig

\let\ig=\textsc
\let\tc=\textcolor

% evolution, Flickinger, Pollard, Wasow

\let\citeNP\citet

% Adam P

%\newcommand{\toappear}{Forthcoming}
\newcommand{\pg}[1]{p.\,#1}
\renewcommand{\implies}{\rightarrow}

\newcommand*{\rref}[1]{(\ref{#1})}
\newcommand*{\aref}[1]{(\ref{#1}a)}
\newcommand*{\bref}[1]{(\ref{#1}b)}
\newcommand*{\cref}[1]{(\ref{#1}c)}

\newcommand{\msadam}{.}
\newcommand{\morsyn}[1]{\textsc{#1}}

\newcommand{\nom}{\morsyn{nom}}
\newcommand{\acc}{\morsyn{acc}}
\newcommand{\dat}{\morsyn{dat}}
\newcommand{\gen}{\morsyn{gen}}
\newcommand{\ins}{\morsyn{ins}}
%\newcommand{\aploc}{\morsyn{loc}}
\newcommand{\voc}{\morsyn{voc}}
\newcommand{\ill}{\morsyn{ill}}
\renewcommand{\inf}{\morsyn{inf}}
\newcommand{\passprc}{\morsyn{passp}}

%\newcommand{\Nom}{\msadam\nom}
%\newcommand{\Acc}{\msadam\acc}
%\newcommand{\Dat}{\msadam\dat}
%\newcommand{\Gen}{\msadam\gen}
\newcommand{\Ins}{\msadam\ins}
\newcommand{\Loc}{\msadam\loc}
\newcommand{\Voc}{\msadam\voc}
\newcommand{\Ill}{\msadam\ill}
\newcommand{\PassP}{\msadam\passprc}

\newcommand{\Aux}{\textsc{aux}}

\newcommand{\princ}[1]{\textnormal{\textsc{#1}}} % for constraint names
\newcommand{\notion}[1]{\emph{#1}}
\renewcommand{\path}[1]{\textnormal{\textsc{#1}}}
\newcommand{\ftype}[1]{\textit{#1}}
\newcommand{\fftype}[1]{{\scriptsize\textit{#1}}}
\newcommand{\la}{$\langle$}
\newcommand{\ra}{$\rangle$}
%\newcommand{\argst}{\path{arg-st}}
\newcommand{\phtm}[1]{\setbox0=\hbox{#1}\hspace{\wd0}}
\newcommand{\prep}[1]{\setbox0=\hbox{#1}\hspace{-1\wd0}#1}

%%%%%%%%%%%%%%%%%%%%%%%%%%%%%%%%%%%%%%%%%%%%%%%%%%%%%%%%%%%%%%%%%%%%%%%%%%%

% FROM FS.STY:

%%%
%%% Feature structures
%%%

% \fs         To print a feature structure by itself, type for example
%             \fs{case:nom \\ person:P}
%             or (better, for true italics),
%             \fs{\it case:nom \\ \it person:P}
%
% \lfs        To print the same feature structure with the category
%             label N at the top, type:
%             \lfs{N}{\it case:nom \\ \it person:P}

%    Modified 1990 Dec 5 so that features are left aligned.
\newcommand{\fs}[1]%
{\mbox{\small%
$
\!
\left[
  \!\!
  \begin{tabular}{l}
    #1
  \end{tabular}
  \!\!
\right]
\!
$}}

%     Modified 1990 Dec 5 so that features are left aligned.
%\newcommand{\lfs}[2]
%   {
%     \mbox{$
%           \!\!
%           \begin{tabular}{c}
%           \it #1
%           \\
%           \mbox{\small%
%                 $
%                 \left[
%                 \!\!
%                 \it
%                 \begin{tabular}{l}
%                 #2
%                 \end{tabular}
%                 \!\!
%                 \right]
%                 $}
%           \end{tabular}
%           \!\!
%           $}
%   }

\newcommand{\ft}[2]{\path{#1}\hspace{1ex}\ftype{#2}}
\newcommand{\fsl}[2]{\fs{{\fftype{#1}} \\ #2}}

\newcommand{\fslt}[2]
 {\fst{
       {\fftype{#1}} \\
       #2 
     }
 }

\newcommand{\fsltt}[2]
 {\fstt{
       {\fftype{#1}} \\
       #2 
     }
 }

\newcommand{\fslttt}[2]
 {\fsttt{
       {\fftype{#1}} \\
       #2 
     }
 }


% jak \ft, \fs i \fsl tylko nieco ciasniejsze

\newcommand{\ftt}[2]
% {{\sc #1}\/{\rm #2}}
 {\textsc{#1}\/{\rm #2}}

\newcommand{\fst}[1]
  {
    \mbox{\small%
          $
          \left[
          \!\!\!
%          \sc
          \begin{tabular}{l} #1
          \end{tabular}
          \!\!\!\!\!\!\!
          \right]
          $
          }
   }

%\newcommand{\fslt}[2]
% {\fst{#2\\
%       {\scriptsize\it #1}
%      }
% }


% superciasne

\newcommand{\fstt}[1]
  {
    \mbox{\small%
          $
          \left[
          \!\!\!
%          \sc
          \begin{tabular}{l} #1
          \end{tabular}
          \!\!\!\!\!\!\!\!\!\!\!
          \right]
          $
          }
   }

%\newcommand{\fsltt}[2]
% {\fstt{#2\\
%       {\scriptsize\it #1}
%      }
% }

\newcommand{\fsttt}[1]
  {
    \mbox{\small%
          $
          \left[
          \!\!\!
%          \sc
          \begin{tabular}{l} #1
          \end{tabular}
          \!\!\!\!\!\!\!\!\!\!\!\!\!\!\!\!
          \right]
          $
          }
   }



% %add all your local new commands to this file

% \newcommand{\smiley}{:)}

% you are not supposed to mess with hardcore stuff, St.Mü. 22.08.2018
%% \renewbibmacro*{index:name}[5]{%
%%   \usebibmacro{index:entry}{#1}
%%     {\iffieldundef{usera}{}{\thefield{usera}\actualoperator}\mkbibindexname{#2}{#3}{#4}{#5}}}

% % \newcommand{\noop}[1]{}



% Rui

\newcommand{\spc}[0]{\hspace{-1pt}\underline{\hspace{6pt}}\,}
\newcommand{\spcs}[0]{\hspace{-1pt}\underline{\hspace{6pt}}\,\,}
\newcommand{\bad}[1]{\leavevmode\llap{#1}}
\newcommand{\COMMENT}[1]{}


% Rui coordination
\newcommand{\subl}[1]{$_{\scriptstyle \textsc{#1}}$}



% Andy Lücking gesture.tex
\newcommand{\Pointing}{\ding{43}}
% Giotto: "Meeting of Joachim and Anne at the Golden Gate" - 1305-10 
\definecolor{GoldenGate1}{rgb}{.13,.09,.13} % Dress of woman in black
\definecolor{GoldenGate2}{rgb}{.94,.94,.91} % Bridge
\definecolor{GoldenGate3}{rgb}{.06,.09,.22} % Blue sky
\definecolor{GoldenGate4}{rgb}{.94,.91,.87} % Dress of woman with shawl
\definecolor{GoldenGate5}{rgb}{.52,.26,.26} % Joachim's robe
\definecolor{GoldenGate6}{rgb}{.65,.35,.16} % Anne's robe
\definecolor{GoldenGate7}{rgb}{.91,.84,.42} % Joachim's halo

\makeatletter
\newcommand{\@Depth}{1} % x-dimension, to front
\newcommand{\@Height}{1} % z-dimension, up
\newcommand{\@Width}{1} % y-dimension, rightwards
%\GGS{<x-start>}{<y-start>}{<z-top>}{<z-bottom>}{<Farbe>}{<x-width>}{<y-depth>}{<opacity>}
\newcommand{\GGS}[9][]{%
\coordinate (O) at (#2-1,#3-1,#5);
\coordinate (A) at (#2-1,#3-1+#7,#5);
\coordinate (B) at (#2-1,#3-1+#7,#4);
\coordinate (C) at (#2-1,#3-1,#4);
\coordinate (D) at (#2-1+#8,#3-1,#5);
\coordinate (E) at (#2-1+#8,#3-1+#7,#5);
\coordinate (F) at (#2-1+#8,#3-1+#7,#4);
\coordinate (G) at (#2-1+#8,#3-1,#4);
\draw[draw=black, fill=#6, fill opacity=#9] (D) -- (E) -- (F) -- (G) -- cycle;% Front
\draw[draw=black, fill=#6, fill opacity=#9] (C) -- (B) -- (F) -- (G) -- cycle;% Top
\draw[draw=black, fill=#6, fill opacity=#9] (A) -- (B) -- (F) -- (E) -- cycle;% Right
}
\makeatother


% pragmatics
\newcommand{\speaking}[1]{\eqparbox{name}{\textsc{\lowercase{#1}\space}}}
\newcommand{\alname}[1]{\eqparbox{name}{\textsc{\lowercase{#1}}}}
\newcommand{\HPSGTTR}{HPSG$_{\text{TTR}}$\xspace}

\newcommand{\ttrtype}[1]{\textit{#1}}
\newcommand{\avmel}{\q<\quad\q>} %% shortcut for empty lists in AVM
\newcommand{\ttrmerge}{\ensuremath{\wedge_{\textit{merge}}}}
\newcommand{\Cat}[2][0.1pt]{%
  \begin{scope}[y=#1,x=#1,yscale=-1, inner sep=0pt, outer sep=0pt]
   \path[fill=#2,line join=miter,line cap=butt,even odd rule,line width=0.8pt]
  (151.3490,307.2045) -- (264.3490,307.2045) .. controls (264.3490,291.1410) and (263.2021,287.9545) .. (236.5990,287.9545) .. controls (240.8490,275.2045) and (258.1242,244.3581) .. (267.7240,244.3581) .. controls (276.2171,244.3581) and (286.3490,244.8259) .. (286.3490,264.2045) .. controls (286.3490,286.2045) and (323.3717,321.6755) .. (332.3490,307.2045) .. controls (345.7277,285.6390) and (309.3490,292.2151) .. (309.3490,240.2046) .. controls (309.3490,169.0514) and (350.8742,179.1807) .. (350.8742,139.2046) .. controls (350.8742,119.2045) and (345.3490,116.5037) .. (345.3490,102.2045) .. controls (345.3490,83.3070) and (361.9972,84.4036) .. (358.7581,68.7349) .. controls (356.5206,57.9117) and (354.7696,49.2320) .. (353.4652,36.1439) .. controls (352.5396,26.8573) and (352.2445,16.9594) .. (342.5985,17.3574) .. controls (331.2650,17.8250) and (326.9655,37.7742) .. (309.3490,39.2045) .. controls (291.7685,40.6320) and (276.7783,24.2380) .. (269.9740,26.5795) .. controls (263.2271,28.9013) and (265.3490,47.2045) .. (269.3490,60.2045) .. controls (275.6359,80.6368) and (289.3490,107.2045) .. (264.3490,111.2045) .. controls (239.3490,115.2045) and (196.3490,119.2045) .. (165.3490,160.2046) .. controls (134.3490,201.2046) and (135.4934,249.3212) .. (123.3490,264.2045) .. controls (82.5907,314.1553) and (40.8239,293.6463) .. (40.8239,335.2045) .. controls (40.8239,353.8102) and (72.3490,367.2045) .. (77.3490,361.2045) .. controls (82.3490,355.2045) and (34.8638,337.3259) .. (87.9955,316.2045) .. controls (133.3871,298.1601) and   (137.4391,294.4766) .. (151.3490,307.2045) -- cycle;
\end{scope}%
}


% KdK
\newcommand{\smiley}{:)}

\renewbibmacro*{index:name}[5]{%
  \usebibmacro{index:entry}{#1}
    {\iffieldundef{usera}{}{\thefield{usera}\actualoperator}\mkbibindexname{#2}{#3}{#4}{#5}}}

% \newcommand{\noop}[1]{}

% chngcntr.sty otherwise gives error that these are already defined
%\let\counterwithin\relax
%\let\counterwithout\relax

% the space of a left bracket for glossings
\newcommand{\LB}{\hspaceThis{[}}

\newcommand{\LF}{\mbox{$[\![$}}

\newcommand{\RF}{\mbox{$]\!]_F$}}

\newcommand{\RT}{\mbox{$]\!]_T$}}





% Manfred's

\newcommand{\kommentar}[1]{}

\newcommand{\bsp}[1]{\emph{#1}}
\newcommand{\bspT}[2]{\bsp{#1} `#2'}
\newcommand{\bspTL}[3]{\bsp{#1} (lit.: #2) `#3'}

\newcommand{\noidi}{§}

\newcommand{\refer}[1]{(\ref{#1})}

%\newcommand{\avmtype}[1]{\multicolumn{2}{l}{\type{#1}}}
\newcommand{\attr}[1]{\textsc{#1}}

\newcommand{\srdefault}{\mbox{\begin{tabular}{c}{\large <}\\[-1.5ex]$\sqcap$\end{tabular}}}

%% \newcommand{\myappcolumn}[2]{
%% \begin{minipage}[t]{#1}#2\end{minipage}
%% }

%% \newcommand{\appc}[1]{\myappcolumn{3.7cm}{#1}}


% Jong-Bok


% clean that up and do not use \def (killing other stuff defined before)
%\if 0
\newcommand\DEL{\textsc{del}}
\newcommand\del{\textsc{del}}

\newcommand\conn{\textsc{conn}}
\newcommand\CONN{\textsc{conn}}
\newcommand\CONJ{\textsc{conj}}
\newcommand\LITE{\textsc{lex}}
\newcommand\lite{\textsc{lex}}
\newcommand\HON{\textsc{hon}}

%\newcommand\CAUS{\textsc{caus}}
%\newcommand\PASS{\textsc{pass}}
\newcommand\NPST{\textsc{npst}}
%\newcommand\COND{\textsc{cond}}



\newcommand\hdlite{\textsc{head-lex construction}}
\newcommand\hdlight{\textsc{head-light} Schema}
\newcommand\NFORM{\textsc{nform}}

\newcommand\RELS{\textsc{rels}}
%\newcommand\TENSE{\textsc{tense}}


%\newcommand\ARG{\textsc{arg}}
\newcommand\ARGs{\textsc{arg0}}
\newcommand\ARGa{\textsc{arg}}
\newcommand\ARGb{\textsc{arg2}}
\newcommand\TPC{\textsc{top}}
%\newcommand\PROG{\textsc{prog}}

\newcommand\LIGHT{\textsc{light}\xspace}
\newcommand\pst{\textsc{pst}}
%\newcommand\PAST{\textsc{pst}}
%\newcommand\DAT{\textsc{dat}}
%\newcommand\CONJ{\textsc{conj}}
\newcommand\nominal{\textsc{nominal}}
\newcommand\NOMINAL{\textsc{nominal}}
\newcommand\VAL{\textsc{val}}
%\newcommand\val{\textsc{val}}
\newcommand\MODE{\textsc{mode}}
\newcommand\RESTR{\textsc{restr}}
\newcommand\SIT{\textsc{sit}}
\newcommand\ARG{\textsc{arg}}
\newcommand\RELN{\textsc{rel}}
%\newcommand\REL{\textsc{rel}}
%\newcommand\RELS{\textsc{rels}}
%\newcommand\arg-st{\textsc{arg-st}}
\newcommand\xdel{\textsc{xdel}}
\newcommand\zdel{\textsc{zdel}}
\newcommand\sug{\textsc{sug}}
%\newcommand\IMP{\textsc{imp}}
%\newcommand\conn{\textsc{conn}}
%\newcommand\CONJ{\textsc{conj}}
%\newcommand\HON{\textsc{hon}}
\newcommand\BN{\textsc{bn}}
\newcommand\bn{\textsc{bn}}
\newcommand\pres{\textsc{pres}}
\newcommand\PRES{\textsc{pres}}
\newcommand\prs{\textsc{pres}}
%\newcommand\PRS{\textsc{pres}}
\newcommand\agt{\textsc{agt}}
%\newcommand\DEL{\textsc{del}}
%\newcommand\PRED{\textsc{pred}}
\newcommand\AGENT{\textsc{agent}}
\newcommand\THEME{\textsc{theme}}
%\newcommand\AUX{\textsc{aux}}
%\newcommand\THEME{\textsc{theme}}
%\newcommand\PL{\textsc{pl}}
\newcommand\SRC{\textsc{src}}
\newcommand\src{\textsc{src}}
\newcommand{\FORMjb}{\textsc{form}}
\newcommand{\formjb}{\FORM}
\newcommand\GCASE{\textsc{gcase}}
\newcommand\gcase{\textsc{gcase}}
\newcommand\SCASE{\textsc{scase}}
\newcommand\PHON{\textsc{phon}}
%\newcommand\SS{\textsc{ss}}
\newcommand\SYN{\textsc{syn}}
%\newcommand\LOC{\textsc{loc}}
\newcommand\MOD{\textsc{mod}}
\newcommand\INV{\textsc{inv}}
%\newcommand\L{\textsc{l}}
%\newcommand\CASE{\textsc{case}}
\newcommand\SPR{\textsc{spr}}
\newcommand\COMPS{\textsc{comps}}
%\newcommand\comps{\textsc{comps}}
\newcommand\SEM{\textsc{sem}}
\newcommand\CONT{\textsc{cont}}
\newcommand\SUBCAT{\textsc{subcat}}
\newcommand\CAT{\textsc{cat}}
%\newcommand\C{\textsc{c}}
%\newcommand\SUBJ{\textsc{subj}}
\newcommand\subjjb{\textsc{subj}}
%\newcommand\SLASH{\textsc{slash}}
\newcommand\LOCAL{\textsc{local}}
%\newcommand\ARG-ST{\textsc{arg-st}}
%\newcommand\AGR{\textsc{agr}}
\newcommand\PER{\textsc{per}}
%\newcommand\NUM{\textsc{num}}
%\newcommand\IND{\textsc{ind}}
\newcommand\VFORM{\textsc{vform}}
\newcommand\PFORM{\textsc{pform}}
\newcommand\decl{\textsc{decl}}
%\newcommand\loc{\textsc{loc   }}
% \newcommand\   {\textsc{  }}

%\newcommand\NEG{\textsc{neg}}
\newcommand\FRAMES{\textsc{frames}}
%\newcommand\REFL{\textsc{refl}}

\newcommand\MKG{\textsc{mkg}}

%\newcommand\BN{\textsc{bn}}
\newcommand\HD{\textsc{hd}}
\newcommand\NP{\textsc{np}}
\newcommand\PF{\textsc{pf}}
%\newcommand\PL{\textsc{pl}}
\newcommand\PP{\textsc{pp}}
%\newcommand\SS{\textsc{ss}}
\newcommand\VF{\textsc{vf}}
\newcommand\VP{\textsc{vp}}
%\newcommand\bn{\textsc{bn}}
\newcommand\cl{\textsc{cl}}
%\newcommand\pl{\textsc{pl}}
\newcommand\Wh{\ital{Wh}}
%\newcommand\ng{\textsc{neg}}
\newcommand\wh{\ital{wh}}
%\newcommand\ACC{\textsc{acc}}
%\newcommand\AGR{\textsc{agr}}
\newcommand\AGT{\textsc{agt}}
\newcommand\ARC{\textsc{arc}}
%\newcommand\ARG{\textsc{arg}}
\newcommand\ARP{\textsc{arc}}
%\newcommand\AUX{\textsc{aux}}
%\newcommand\CAT{\textsc{cat}}
%\newcommand\COP{\textsc{cop}}
%\newcommand\DAT{\textsc{dat}}
\newcommand\NEWCOMMAND{\textsc{def}}
%\newcommand\DEL{\textsc{del}}
\newcommand\DOM{\textsc{dom}}
\newcommand\DTR{\textsc{dtr}}
%\newcommand\FUT{\textsc{fut}}
\newcommand\GAP{\textsc{gap}}
%\newcommand\GEN{\textsc{gen}}
%\newcommand\HON{\textsc{hon}}
%\newcommand\IMP{\textsc{imp}}
%\newcommand\IND{\textsc{ind}}
%\newcommand\INV{\textsc{inv}}
\newcommand\LEX{\textsc{lex}}
\newcommand\Lex{\textsc{lex}}
%\newcommand\LOC{\textsc{loc}}
%\newcommand\MOD{\textsc{mod}}
\newcommand\MRK{{\nr MRK}}
%\newcommand\NEG{\textsc{neg}}
\newcommand\NEW{\textsc{new}}
%\newcommand\NOM{\textsc{nom}}
%\newcommand\NUM{\textsc{num}}
%\newcommand\PER{\textsc{per}}
%\newcommand\PST{\textsc{pst}}
\newcommand\QUE{\textsc{que}}
%\newcommand\REL{\textsc{rel}}
\newcommand\SEL{\textsc{sel}}
%\newcommand\SEM{\textsc{sem}}
%\newcommand\SIT{\textsc{arg0}}
%\newcommand\SPR{\textsc{spr}}
%\newcommand\SRC{\textsc{src}}
\newcommand\SUG{\textsc{sug}}
%\newcommand\SYN{\textsc{syn}}
%\newcommand\TPC{\textsc{top}}
%\newcommand\VAL{\textsc{val}}
%\newcommand\acc{\textsc{acc}}
%\newcommand\agt{\textsc{agt}}
\newcommand\cop{\textsc{cop}}
%\newcommand\dat{\textsc{dat}}
\newcommand\foc{\textsc{focus}}
%\newcommand\FOC{\textsc{focus}}
\newcommand\fut{\textsc{fut}}
\newcommand\hon{\textsc{hon}}
\newcommand\imp{\textsc{imp}}
\newcommand\kes{\textsc{kes}}
%\newcommand\lex{\textsc{lex}}
%\newcommand\loc{\textsc{loc}}
\newcommand\mrk{{\nr MRK}}
%\newcommand\nom{\textsc{nom}}
%\newcommand\num{\textsc{num}}
\newcommand\plu{\textsc{plu}}
\newcommand\pne{\textsc{pne}}
%\newcommand\pst{\textsc{pst}}
\newcommand\pur{\textsc{pur}}
%\newcommand\que{\textsc{que}}
%\newcommand\src{\textsc{src}}
%\newcommand\sug{\textsc{sug}}
\newcommand\tpc{\textsc{top}}
%\newcommand\utt{\textsc{utt}}
%\newcommand\val{\textsc{val}}
%% \newcommand\LITE{\textsc{lex}}
%% \newcommand\PAST{\textsc{pst}}
%% \newcommand\POSP{\textsc{pos}}
%% \newcommand\PRS{\textsc{pres}}
%% \newcommand\mod{\textsc{mod}}%
%% \newcommand\newuse{{`kes'}}
%% \newcommand\posp{\textsc{pos}}
%% \newcommand\prs{\textsc{pres}}
%% \newcommand\psp{{\it en\/}}
%% \newcommand\skes{\textsc{kes}}
%% \newcommand\CASE{\textsc{case}}
%% \newcommand\CASE{\textsc{case}}
%% \newcommand\COMP{\textsc{comp}}
%% \newcommand\CONJ{\textsc{conj}}
%% \newcommand\CONN{\textsc{conn}}
%% \newcommand\CONT{\textsc{cont}}
%% \newcommand\DECL{\textsc{decl}}
%% \newcommand\FOCUS{\textsc{focus}}
%% %\newcommand\FORM{\textsc{form}} duplicate
%% \newcommand\FREL{\textsc{frel}}
%% \newcommand\GOAL{\textsc{goal}}
\newcommand\HEAD{\textsc{head}}
%% \newcommand\INDEX{\textsc{ind}}
%% \newcommand\INST{\textsc{inst}}
%% \newcommand\MODE{\textsc{mode}}
%% \newcommand\MOOD{\textsc{mood}}
%% \newcommand\NMLZ{\textsc{nmlz}}
%% \newcommand\PHON{\textsc{phon}}
%% \newcommand\PRED{\textsc{pred}}
%% %\newcommand\PRES{\textsc{pres}}
%% \newcommand\PROM{\textsc{prom}}
%% \newcommand\RELN{\textsc{pred}}
%% \newcommand\RELS{\textsc{rels}}
%% \newcommand\STEM{\textsc{stem}}
%% \newcommand\SUBJ{\textsc{subj}}
%% \newcommand\XARG{\textsc{xarg}}
%% \newcommand\bse{{\it bse\/}}
%% \newcommand\case{\textsc{case}}
%% \newcommand\caus{\textsc{caus}}
%% \newcommand\comp{\textsc{comp}}
%% \newcommand\conj{\textsc{conj}}
%% \newcommand\conn{\textsc{conn}}
%% \newcommand\decl{\textsc{decl}}
%% \newcommand\fin{{\it fin\/}}
%% %\newcommand\form{\textsc{form}}
%% \newcommand\gend{\textsc{gend}}
%% \newcommand\inf{{\it inf\/}}
%% \newcommand\mood{\textsc{mood}}
%% \newcommand\nmlz{\textsc{nmlz}}
%% \newcommand\pass{\textsc{pass}}
%% \newcommand\past{\textsc{past}}
%% \newcommand\perf{\textsc{perf}}
%% \newcommand\pln{{\it pln\/}}
%% \newcommand\pred{\textsc{pred}}


%% %\newcommand\pres{\textsc{pres}}
%% \newcommand\proc{\textsc{proc}}
%% \newcommand\nonfin{{\it nonfin\/}}
%% \newcommand\AGENT{\textsc{agent}}
%% \newcommand\CFORM{\textsc{cform}}
%% %\newcommand\COMPS{\textsc{comps}}
%% \newcommand\COORD{\textsc{coord}}
%% \newcommand\COUNT{\textsc{count}}
%% \newcommand\EXTRA{\textsc{extra}}
%% \newcommand\GCASE{\textsc{gcase}}
%% \newcommand\GIVEN{\textsc{given}}
%% \newcommand\LOCAL{\textsc{local}}
%% \newcommand\NFORM{\textsc{nform}}
%% \newcommand\PFORM{\textsc{pform}}
%% \newcommand\SCASE{\textsc{scase}}
%% \newcommand\SLASH{\textsc{slash}}
%% \newcommand\SLASH{\textsc{slash}}
%% \newcommand\THEME{\textsc{theme}}
%% \newcommand\TOPIC{\textsc{topic}}
%% \newcommand\VFORM{\textsc{vform}}
%% \newcommand\cause{\textsc{cause}}
%% %\newcommand\comps{\textsc{comps}}
%% \newcommand\gcase{\textsc{gcase}}
%% \newcommand\itkes{{\it kes\/}}
%% \newcommand\pass{{\it pass\/}}
%% \newcommand\vform{\textsc{vform}}
%% \newcommand\CCONT{\textsc{c-cont}}
%% \newcommand\GN{\textsc{given-new}}
%% \newcommand\INFO{\textsc{info-st}}
%% \newcommand\ARG-ST{\textsc{arg-st}}
%% \newcommand\SUBCAT{\textsc{subcat}}
%% \newcommand\SYNSEM{\textsc{synsem}}
%% \newcommand\VERBAL{\textsc{verbal}}
%% \newcommand\arg-st{\textsc{arg-st}}
%% \newcommand\plain{{\it plain}\/}
%% \newcommand\propos{\textsc{propos}}
%% \newcommand\ADVERBIAL{\textsc{advl}}
%% \newcommand\HIGHLIGHT{\textsc{prom}}
%% \newcommand\NOMINAL{\textsc{nominal}}

\newenvironment{myavm}{\begingroup\avmvskip{.1ex}
  \selectfont\begin{avm}}%
{\end{avm}\endgroup\medskip}
\newcommand\pfix{\vspace{-5pt}}


\newcommand{\jbsub}[1]{\lower4pt\hbox{\small #1}}
\newcommand{\jbssub}[1]{\lower4pt\hbox{\small #1}}
\newcommand\jbtr{\underbar{\ \ \ }\ }


%\fi

% cl

\newcommand{\delphin}{\textsc{delph-in}}


% YK -- CG chapter

\newcommand{\grey}[1]{\colorbox{mycolor}{#1}}
\definecolor{mycolor}{gray}{0.8}

\newcommand{\GQU}[2]{\raisebox{1.6ex}{\ensuremath{\rotatebox{180}{\textbf{#1}}_{\scalebox{.7}{\textbf{#2}}}}}}

\newcommand{\SetInfLen}{\setpremisesend{0pt}\setpremisesspace{10pt}\setnamespace{0pt}}

\newcommand{\pt}[1]{\ensuremath{\mathsf{#1}}}
\newcommand{\ptv}[1]{\ensuremath{\textsf{\textsl{#1}}}}

\newcommand{\sv}[1]{\ensuremath{\bm{\mathcal{#1}}}}
\newcommand{\sX}{\sv{X}}
\newcommand{\sF}{\sv{F}}
\newcommand{\sG}{\sv{G}}

\newcommand{\syncat}[1]{\textrm{#1}}
\newcommand{\syncatVar}[1]{\ensuremath{\mathit{#1}}}

\newcommand{\RuleName}[1]{\textrm{#1}}

\newcommand{\SemTyp}{\textsf{Sem}}

\newcommand{\E}{\ensuremath{\bm{\epsilon}}\xspace}

\newcommand{\greeka}{\upalpha}
\newcommand{\greekb}{\upbeta}
\newcommand{\greekd}{\updelta}
\newcommand{\greekp}{\upvarphi}
\newcommand{\greekr}{\uprho}
\newcommand{\greeks}{\upsigma}
\newcommand{\greekt}{\uptau}
\newcommand{\greeko}{\upomega}
\newcommand{\greekz}{\upzeta}

\newcommand{\Lemma}{\ensuremath{\hskip.5em\vdots\hskip.5em}\noLine}
\newcommand{\LemmaAlt}{\ensuremath{\hskip.5em\vdots\hskip.5em}}

\newcommand{\I}{\iota}

\newcommand{\sem}{\ensuremath}

\newcommand{\NoSem}{%
\renewcommand{\LexEnt}[3]{##1; \syncat{##3}}
\renewcommand{\LexEntTwoLine}[3]{\renewcommand{\arraystretch}{.8}%
\begin{array}[b]{l} ##1;  \\ \syncat{##3} \end{array}}
\renewcommand{\LexEntThreeLine}[3]{\renewcommand{\arraystretch}{.8}%
\begin{array}[b]{l} ##1; \\ \syncat{##3} \end{array}}}

\newcommand{\hypml}[2]{\left[\!\!#1\!\!\right]^{#2}}

%%%%for bussproof
\def\defaultHypSeparation{\hskip0.1in}
\def\ScoreOverhang{0pt}

\newcommand{\MultiLine}[1]{\renewcommand{\arraystretch}{.8}%
\ensuremath{\begin{array}[b]{l} #1 \end{array}}}

\newcommand{\MultiLineMod}[1]{%
\ensuremath{\begin{array}[t]{l} #1 \end{array}}}

\newcommand{\hypothesis}[2]{[ #1 ]^{#2}}

\newcommand{\LexEnt}[3]{#1; \ensuremath{#2}; \syncat{#3}}

\newcommand{\LexEntTwoLine}[3]{\renewcommand{\arraystretch}{.8}%
\begin{array}[b]{l} #1; \\ \ensuremath{#2};  \syncat{#3} \end{array}}

\newcommand{\LexEntThreeLine}[3]{\renewcommand{\arraystretch}{.8}%
\begin{array}[b]{l} #1; \\ \ensuremath{#2}; \\ \syncat{#3} \end{array}}

\newcommand{\LexEntFiveLine}[5]{\renewcommand{\arraystretch}{.8}%
\begin{array}{l} #1 \\ #2; \\ \ensuremath{#3} \\ \ensuremath{#4}; \\ \syncat{#5} \end{array}}

\newcommand{\LexEntFourLine}[4]{\renewcommand{\arraystretch}{.8}%
\begin{array}{l} \pt{#1} \\ \pt{#2}; \\ \syncat{#4} \end{array}}

\newcommand{\ManySomething}{\renewcommand{\arraystretch}{.8}%
\raisebox{-3mm}{\begin{array}[b]{c} \vdots \,\,\,\,\,\, \vdots \\
\vdots \end{array}}}

\newcommand{\lemma}[1]{\renewcommand{\arraystretch}{.8}%
\begin{array}[b]{c} \vdots \\ #1 \end{array}}

\newcommand{\lemmarev}[1]{\renewcommand{\arraystretch}{.8}%
\begin{array}[b]{c} #1 \\ \vdots \end{array}}

\newcommand{\p}{\ensuremath{\upvarphi}}

% clashes with soul package
\newcommand{\yusukest}{\textbf{\textsf{st}}}

\newcommand{\shortarrow}{\xspace\hskip-1.2ex\scalebox{.5}[1]{\ensuremath{\bm{\rightarrow}}}\hskip-.5ex\xspace}

\newcommand{\SemInt}[1]{\mbox{$[\![ \textrm{#1} ]\!]$}}

\newcommand{\HypSpace}{\hskip-.8ex}
\newcommand{\RaiseHeight}{\raisebox{2.2ex}}
\newcommand{\RaiseHeightLess}{\raisebox{1ex}}

\newcommand{\ThreeColHyp}[1]{\RaiseHeight{\Bigg[}\HypSpace#1\HypSpace\RaiseHeight{\Bigg]}}
\newcommand{\TwoColHyp}[1]{\RaiseHeightLess{\Big[}\HypSpace#1\HypSpace\RaiseHeightLess{\Big]}}

\newcommand{\LemmaShort}{\ensuremath{ \ \vdots} \ \noLine}
\newcommand{\LemmaShortAlt}{\ensuremath{ \ \vdots} \ }

\newcommand{\fail}{**}
\newcommand{\vs}{\raisebox{.05em}{\ensuremath{\upharpoonright}}}
\newcommand{\DerivSize}{\small}

\def\maru#1{{\ooalign{\hfil
  \ifnum#1>999 \resizebox{.25\width}{\height}{#1}\else%
  \ifnum#1>99 \resizebox{.33\width}{\height}{#1}\else%
  \ifnum#1>9 \resizebox{.5\width}{\height}{#1}\else #1%
  \fi\fi\fi%
\/\hfil\crcr%
\raise.167ex\hbox{\mathhexbox20D}}}}

\newenvironment{samepage2}%
 {\begin{flushleft}\begin{minipage}{\linewidth}}
 {\end{minipage}\end{flushleft}}

\newcommand{\cmt}[1]{\textsl{\textbf{[#1]}}}
\newcommand{\trns}[1]{\textbf{#1}\xspace}
\newcommand{\ptfont}{}
\newcommand{\gp}{\underline{\phantom{oo}}}
\newcommand{\mgcmt}{\marginnote}

\newcommand{\term}[1]{\emph{#1}}

\newcommand{\citeposs}[1]{\citeauthor{#1}'s \citeyearpar{#1}}

% for standalone compilations Felix: This is in the class already
%\let\thetitle\@title
%\let\theauthor\@author 
\makeatletter
\newcommand{\togglepaper}[1][0]{ 
\bibliography{../Bibliographies/stmue,../localbibliography,
../Bibliographies/properties,
../Bibliographies/np,
../Bibliographies/negation,
../Bibliographies/ellipsis,
../Bibliographies/binding,
../Bibliographies/complex-predicates,
../Bibliographies/control-raising,
../Bibliographies/coordination,
../Bibliographies/morphology,
../Bibliographies/lfg,
collection.bib}
  %% hyphenation points for line breaks
%% Normally, automatic hyphenation in LaTeX is very good
%% If a word is mis-hyphenated, add it to this file
%%
%% add information to TeX file before \begin{document} with:
%% %% hyphenation points for line breaks
%% Normally, automatic hyphenation in LaTeX is very good
%% If a word is mis-hyphenated, add it to this file
%%
%% add information to TeX file before \begin{document} with:
%% \include{localhyphenation}
\hyphenation{
A-la-hver-dzhie-va
anaph-o-ra
ana-phor
ana-phors
an-te-ced-ent
an-te-ced-ents
affri-ca-te
affri-ca-tes
ap-proach-es
Atha-bas-kan
Athe-nä-um
Bona-mi
Chi-che-ŵa
com-ple-ments
con-straints
Cope-sta-ke
Da-ge-stan
Dor-drecht
er-klä-ren-de
Ginz-burg
Gro-ning-en
Jap-a-nese
Jon-a-than
Ka-tho-lie-ke
Ko-bon
krie-gen
Le-Sourd
moth-er
Mül-ler
Nie-mey-er
Par-a-digm
Prze-piór-kow-ski
phe-nom-e-non
re-nowned
Rie-he-mann
un-bound-ed
with-in
}

% listing within here does not have any effect for lfg.tex % 2020-05-14

% why has "erklärende" be listed here? I specified langid in bibtex item. Something is still not working with hyphenation.


% to do: check
%  Alahverdzhieva


% biblatex:

% This is a LaTeX frontend to TeX’s \hyphenation command which defines hy- phenation exceptions. The ⟨language⟩ must be a language name known to the babel/polyglossia packages. The ⟨text ⟩ is a whitespace-separated list of words. Hyphenation points are marked with a dash:

% \DefineHyphenationExceptions{american}{%
% hy-phen-ation ex-cep-tion }

\hyphenation{
A-la-hver-dzhie-va
anaph-o-ra
ana-phor
ana-phors
an-te-ced-ent
an-te-ced-ents
affri-ca-te
affri-ca-tes
ap-proach-es
Atha-bas-kan
Athe-nä-um
Bona-mi
Chi-che-ŵa
com-ple-ments
con-straints
Cope-sta-ke
Da-ge-stan
Dor-drecht
er-klä-ren-de
Ginz-burg
Gro-ning-en
Jap-a-nese
Jon-a-than
Ka-tho-lie-ke
Ko-bon
krie-gen
Le-Sourd
moth-er
Mül-ler
Nie-mey-er
Par-a-digm
Prze-piór-kow-ski
phe-nom-e-non
re-nowned
Rie-he-mann
un-bound-ed
with-in
}

% listing within here does not have any effect for lfg.tex % 2020-05-14

% why has "erklärende" be listed here? I specified langid in bibtex item. Something is still not working with hyphenation.


% to do: check
%  Alahverdzhieva


% biblatex:

% This is a LaTeX frontend to TeX’s \hyphenation command which defines hy- phenation exceptions. The ⟨language⟩ must be a language name known to the babel/polyglossia packages. The ⟨text ⟩ is a whitespace-separated list of words. Hyphenation points are marked with a dash:

% \DefineHyphenationExceptions{american}{%
% hy-phen-ation ex-cep-tion }

  \memoizeset{
    memo filename prefix={hpsg-handbook.memo.dir/},
    % readonly
  }
  \papernote{\scriptsize\normalfont
    \@author.
    \@title. 
    To appear in: 
    Stefan Müller, Anne Abeillé, Robert D. Borsley \& Jean-Pierre Koenig (eds.)
    HPSG Handbook
    Berlin: Language Science Press. [preliminary page numbering]
  }
  \pagenumbering{roman}
  \setcounter{chapter}{#1}
  \addtocounter{chapter}{-1}
}
\makeatother

\makeatletter
\newcommand{\togglepaperminimal}[1][0]{ 
  \bibliography{../Bibliographies/stmue,
                ../localbibliography,
  ../Bibliographies/coordination,
collection.bib}
  %% hyphenation points for line breaks
%% Normally, automatic hyphenation in LaTeX is very good
%% If a word is mis-hyphenated, add it to this file
%%
%% add information to TeX file before \begin{document} with:
%% %% hyphenation points for line breaks
%% Normally, automatic hyphenation in LaTeX is very good
%% If a word is mis-hyphenated, add it to this file
%%
%% add information to TeX file before \begin{document} with:
%% \include{localhyphenation}
\hyphenation{
A-la-hver-dzhie-va
anaph-o-ra
ana-phor
ana-phors
an-te-ced-ent
an-te-ced-ents
affri-ca-te
affri-ca-tes
ap-proach-es
Atha-bas-kan
Athe-nä-um
Bona-mi
Chi-che-ŵa
com-ple-ments
con-straints
Cope-sta-ke
Da-ge-stan
Dor-drecht
er-klä-ren-de
Ginz-burg
Gro-ning-en
Jap-a-nese
Jon-a-than
Ka-tho-lie-ke
Ko-bon
krie-gen
Le-Sourd
moth-er
Mül-ler
Nie-mey-er
Par-a-digm
Prze-piór-kow-ski
phe-nom-e-non
re-nowned
Rie-he-mann
un-bound-ed
with-in
}

% listing within here does not have any effect for lfg.tex % 2020-05-14

% why has "erklärende" be listed here? I specified langid in bibtex item. Something is still not working with hyphenation.


% to do: check
%  Alahverdzhieva


% biblatex:

% This is a LaTeX frontend to TeX’s \hyphenation command which defines hy- phenation exceptions. The ⟨language⟩ must be a language name known to the babel/polyglossia packages. The ⟨text ⟩ is a whitespace-separated list of words. Hyphenation points are marked with a dash:

% \DefineHyphenationExceptions{american}{%
% hy-phen-ation ex-cep-tion }

\hyphenation{
A-la-hver-dzhie-va
anaph-o-ra
ana-phor
ana-phors
an-te-ced-ent
an-te-ced-ents
affri-ca-te
affri-ca-tes
ap-proach-es
Atha-bas-kan
Athe-nä-um
Bona-mi
Chi-che-ŵa
com-ple-ments
con-straints
Cope-sta-ke
Da-ge-stan
Dor-drecht
er-klä-ren-de
Ginz-burg
Gro-ning-en
Jap-a-nese
Jon-a-than
Ka-tho-lie-ke
Ko-bon
krie-gen
Le-Sourd
moth-er
Mül-ler
Nie-mey-er
Par-a-digm
Prze-piór-kow-ski
phe-nom-e-non
re-nowned
Rie-he-mann
un-bound-ed
with-in
}

% listing within here does not have any effect for lfg.tex % 2020-05-14

% why has "erklärende" be listed here? I specified langid in bibtex item. Something is still not working with hyphenation.


% to do: check
%  Alahverdzhieva


% biblatex:

% This is a LaTeX frontend to TeX’s \hyphenation command which defines hy- phenation exceptions. The ⟨language⟩ must be a language name known to the babel/polyglossia packages. The ⟨text ⟩ is a whitespace-separated list of words. Hyphenation points are marked with a dash:

% \DefineHyphenationExceptions{american}{%
% hy-phen-ation ex-cep-tion }

  \memoizeset{
    memo filename prefix={hpsg-handbook.memo.dir/},
    % readonly
  }
  \papernote{\scriptsize\normalfont
    \@author.
    \@title. 
    To appear in: 
    Stefan Müller, Anne Abeillé, Robert D. Borsley \& Jean-Pierre Koenig (eds.)
    HPSG Handbook
    Berlin: Language Science Press. [preliminary page numbering]
  }
  \pagenumbering{roman}
  \setcounter{chapter}{#1}
  \addtocounter{chapter}{-1}
}
\makeatother




% In case that year is not given, but pubstate. This mainly occurs for titles that are forthcoming, in press, etc.
\renewbibmacro*{addendum+pubstate}{% Thanks to https://tex.stackexchange.com/a/154367 for the idea
  \printfield{addendum}%
  \iffieldequalstr{labeldatesource}{pubstate}{}
  {\newunit\newblock\printfield{pubstate}}
}

\DeclareLabeldate{%
    \field{date}
    \field{year}
    \field{eventdate}
    \field{origdate}
    \field{urldate}
    \field{pubstate}
    \literal{nodate}
}

%\defbibheading{diachrony-sources}{\section*{Sources}} 

% if no langid is set, it is English:
% https://tex.stackexchange.com/a/279302
\DeclareSourcemap{
  \maps[datatype=bibtex]{
    \map{
      \step[fieldset=langid, fieldvalue={english}]
    }
  }
}


% for bibliographies
% biber/biblatex could use sortname field rather than messing around this way.
\newcommand{\SortNoop}[1]{}


% Doug Ball

\newcommand{\elist}{\q<\ \ \q>}

\newcommand{\esetDB}{\q\{\ \ \q\}}


\makeatletter

\newcommand{\nolistbreak}{%

  \let\oldpar\par\def\par{\oldpar\nobreak}% Any \par issues a \nobreak

  \@nobreaktrue% Don't break with first \item

}

\makeatother


% intermediate before Frank's trees are fixed
% This will be removed!!!!!
%\newcommand{\tree}[1]{} % ignore them blody trees
%\usepackage{tree-dvips}


\newcommand{\nodeconnect}[2]{}
\newcommand{\nodetriangle}[2]{}



% Doug relative clauses
%% I've compiled out almost all my private LaTeX command, but there are some
%% I found hard to get rid of. They are defined here.
%% There are few others which defined in places in the document where they have only
%% local effect (e.g. within figures); their names all end in DA, e.g. \MotherDA
%% There are a lot of \labels -- they are all of the form \label{sec:rc-...} or
%% \label{x:rc-...} or similar, so there should be no clashes.

% Subscripts -- scriptsize italic shape lowered by .25ex 
\newcommand{\subscr}[1]{\raisebox{-.5ex}{\protect{\scriptsize{\itshape #1\/}}}}
% A boxed subscript, for avm tags in normal text
\newcommand{\subtag}[1]{\subscr{\idx{#1}}}

%% Sets and tuples: I use \setof{} to get brackets that are upright, not slanted
%\newcommand{\setof}[1]{\ensuremath{\lbrace\,\mathit{#1}\,\rbrace}}
% 11.10.2019 EP: Doug requested replacement of existing \setof definition with the following:
%\newcommand{\setof}[1]{\begin{avm}\{\textcolor{red}{#1}\}\end{avm}}
% 31.1.2019 EP: Doug requested re-replacement of the above \textcolour version with the following:
\newcommand{\setof}[1]{\begin{avm}\{#1\}\end{avm}}

\newcommand{\tuple}[1]{\ensuremath{\left\langle\,\mbox{\textit{#1}}\,\right\rangle}}

% Single pile of stuff, optional arugment is psn (e.g. t or b)
% e.g. to put a over b over c in a centered column, top aligned, do:
%   \cPile[t]{a\\b\\c} 
\newcommand{\cPile}[2][]{%
  \begingroup%
  \renewcommand{\arraystretch}{.5}\begin{tabular}[#1]{c}#2\end{tabular}%
  \endgroup%
}

%% for linguistic examples in running text (`linguistic citation'):
\newcommand{\lic}[1]{\textit{#1}}

%% A gap marked by an underline, raised slightly
%% Default argument indicates how long the line should be:
\newcommand{\uGap}[1][3ex]{\raisebox{.25em}{\underline{\hspace{#1}}}\xspace}

%% \TnodeDA{XP}{avmcontents} -- in a Tree, put a node label next to an AVM
\newcommand{\TnodeDA}[2]{#1~\begin{avm}{#2}\end{avm}}

%% This allows tipa stuff to be put in \emph -- we need to change to cmr first.
%% It is used in the discussion of Arabic.
\newcommand{\emphtipa}[1]{{\fontfamily{cmr}\emph{\tipaencoding #1}}} 



 
 
\definecolor{lsDOIGray}{cmyk}{0,0,0,0.45}


% morphology.tex:
% Berthold

\newcommand{\dnode}[1]{\rnode{#1}{\fbox{#1}}}
\newcommand{\tnode}[1]{\rnode{#1}{\textit{#1}}}

\newcommand{\tl}[2]{#2}

\newcommand{\rrr}[3]{%
  \psframebox[linestyle=none]{%
    \avmoptions{center}
    \begin{avm}
      \[mud & \{ #1 \}\\
      ms & \{ #2 \}\\
      mph & \<  #3 \> \]
    \end{avm}
  }
}
\newcommand{\rr}[2]{%
  \psframebox[linestyle=none]{%
    \avmoptions{center}
    \begin{avm}
      \[mud & \{ #1 \}\\
      mph & \<  #2 \> \]
    \end{avm}
  }
}
 

% Frank Richter
\newtheorem{mydef}{Definition}

\long\def\set[#1\set=#2\set]%
{%
\left\{%
\tabcolsep 1pt%
\begin{tabular}{l}%
#1%
\end{tabular}%
\left|%
\tabcolsep 1pt%
\begin{tabular}{l}%
#2%
\end{tabular}%
\right.%
\right\}%
}

\newcommand{\einruck}{\\ \hspace*{1em}}


%\newcommand{\NatNum}{\mathrm{I\hspace{-.17em}N}}
\newcommand{\NatNum}{\mathbb{N}}
\newcommand{\Aug}[1]{\widehat{#1}}
%\newcommand{\its}{\mathrm{:}}
% Felix 14.02.2020
\DeclareMathOperator{\its}{:}

\newcommand{\sequence}[1]{\langle#1\rangle}

\newcommand{\INTERPRETATION}[2]{\sequence{#1\mathsf{U}#2,#1\mathsf{S}#2,#1\mathsf{A}#2,#1\mathsf{R}#2}}
\newcommand{\Interpretation}{\INTERPRETATION{}{}}

\newcommand{\Inte}{\mathsf{I}}
\newcommand{\Unive}{\mathsf{U}}
\newcommand{\Speci}{\mathsf{S}}
\newcommand{\Atti}{\mathsf{A}}
\newcommand{\Reli}{\mathsf{R}}
\newcommand{\ReliT}{\mathsf{RT}}

\newcommand{\VarInt}{\mathsf{G}}
\newcommand{\CInt}{\mathsf{C}}
\newcommand{\Tinte}{\mathsf{T}}
\newcommand{\Dinte}{\mathsf{D}}

% this was missing from ash's stuff.

%% \def \optrulenode#1{
%%   \setbox1\hbox{$\left(\hbox{\begin{tabular}{@{\strut}c@{\strut}}#1\end{tabular}}\right)$}
%%   \raisebox{1.9ex}{\raisebox{-\ht1}{\copy1}}}



\newcommand{\pslabel}[1]{}

\newcommand{\addpagesunless}{\todostefan{add pages unless you cite the
 work as such}}

% dg.tex
% framed boxes as used in dg.tex
% original idea from stackexchange, but modified by Saso
% http://tex.stackexchange.com/questions/230300/doing-something-like-psframebox-in-tikz#230306
\tikzset{
  frbox/.style={
    rounded corners,
    draw,
    thick,
    inner sep=5pt,
    anchor=base,
  },
}

% get rid of these morewrite messages:
% https://tex.stackexchange.com/questions/419489/suppressing-messages-to-standard-output-from-package-morewrites/419494#419494
\ExplSyntaxOn
\cs_set_protected:Npn \__morewrites_shipout_ii:
  {
    \__morewrites_before_shipout:
    \__morewrites_tex_shipout:w \tex_box:D \g__morewrites_shipout_box
    \edef\tmp{\interactionmode\the\interactionmode\space}\batchmode\__morewrites_after_shipout:\tmp
  }
\ExplSyntaxOff


% This is for places where authors used bold. I replace them by \emph
% but have the information where the bold was. St. Mü. 09.05.2020
\newcommand{\textbfemph}[1]{\emph{#1}}



% Felix 09.06.2020: copy code from the third line into localcommands.tex: https://github.com/langsci/langscibook#defined-environments-commands-etc
\patchcmd{\mkbibindexname}{\ifdefvoid{#3}{}{\MakeCapital{#3} }}{\ifdefvoid{#3}{}{#3 }}{}{\AtEndDocument{\typeout{mkbibindexname could not be patched.}}}

  \togglepaper[16]
}{}

\author{Rui Chaves\affiliation{University at Buffalo, SUNY}}
\title{Island phenomena and related matters}

%\usepackage{tipa}

% \chapterDOI{} %will be filled in at production

%\epigram{Change epigram in chapters/03.tex or remove it there }
\abstract{
Extraction constraints on long-distance dependencies  -- so-called \emph{islands} --  have been  the  subject of intense linguistic and psycholinguistic research for the last half century.   Despite  of  their importance in syntactic theory, the   heterogeneity of island constraints has posed many difficult challenges to linguistic theory, across all frameworks. The HPSG perspective of island phenomena is that they are unlikely  to be due to a unitary syntactic constraint given the fact that virtually all such island constraints have known exceptions.  Rather, it is more plausible that island constraints  result from a combination of independently motivated syntactic, semantic, pragmatic and processing phenomena.
The present chapter is somewhat different from others in this volume in that its focus in not on  HPSG analyses of some phenomena, but rather on the nature of the phenomena itself.  This is because there is evidence that most of the phenomena are not purely grammatical, and to that extent independent from HPSG or indeed any theory of grammar. One may call this view of island phenomena  `minimalist' in the sense that much of it does not involve formal grammar.}


\begin{document}
\maketitle
\label{chap-islands}

\avmoptions{inactive}


\section{Introduction} 

 This chapter provides an overview of various island effects that have received attention from members of
 the HPSG community. I begin with the extraction constraints peculiar to coordinate structures, because they not only have a special status in the history of HPSG, but also because they
illustrate well the non-unitary nature of island constraints. I then argue that, at a deeper level, some of these constraints are in fact present in many other island types, though not necessarily all.
For example, I take it as relatively clear that \emph{factive islands}  are purely pragmatic in nature \citep{Oshima:2006:FIP:1761528.1761544}, as are \emph{negative islands} \citep{kroch89,sza,abrusan,foxneg,abrusanspec},
although one can quibble about the particular technical details of how such accounts are best articulated. 
Similarly,  the \emph{NP Constraint} in the sense of \citet{horn72} is likely to be semantic-pragmatic in nature \citep{kuno87,godard92f,dubinsky2009}. Conversely, I take it as relatively uncontroversial that the \emph{Clause Non-Final Incomplete Constituent  Constraint}  is due to processing difficulty \citep{lev91,fodor92}. See also
\citet{kothari} for evidence that `bridge' verb effects in filler-gap dependencies are partly due to lack of contextualization. 

In the present chapter I focus on islands that have garnered more attention from members of the HPSG community, and that have caused more  controversy cross-theoretically. My goal is to provide an overview of the range of explanations that have been proposed to account for the complex array of facts surrounding islands, and to
show that no single unified account is likely.

\section{Background}

As already detailed in \crossrefchaptert{udc}, HPSG encodes filler-gap dependencies in terms of a set-valued feature
{\sc slash}. Because the theory consists of a feature-based declarative system of constraints, virtually all that goes on in the grammar involves constraints stating which value a given feature takes. By allowing {\sc slash} sets to be unified (or unioned),  
it follows that constructions in which multiple gaps are linked to the same filler are trivially obtained, 
as in  (\ref{cc5}).

\eal  \label{cc5}
\ex Which celebrity did  [the article insult \spc more than it praised \spc]?
\ex  Which celebrity  did you expect [[the pictures of \spc ] to bother \spc the most]?
\ex Which celebrity did you [inform \spcs [that the police was coming to arrest \spc]]?
%\item What do you think the best job [I ever got fired from \spc] was \spc?
\ex Which celebrity did you [compare [the memoir of \spc] [with a movie about \spc]?
\ex Which celebrity  did you [hire \spcs [without auditioning \spcs first]]?
\ex Which celebrity  did you [[meet \spcs at a party] and [date \spcs for a few months]]?
\zl 

But another advantage of encoding the presence of filler-gap dependencies as a feature is that
certain lexical items and constructions can easily impose idiosyncratic constraints on {\sc slash}
values. For example, to account for  languages that do not allow preposition stranding,
it suffices to state that prepositions are necessarily specified as $[$\textsc{slash} $\lbrace \rbrace]$. Thus,
their complements cannot appear in {\sc slash} instead of {\sc comps}. 
The converse also occurs. Certain uses of the verb \emph{assure}, for example, 
are lexically required to have one complement in \textsc{slash} rather
   than in \textsc{comps}. Thus, extraction is obligatory as (\ref{suggest}) shows.
   
\eal \label{suggest}   
\ex[*]{I can assure you him to be the most competent.}
\ex[]{Who$_i$ can you assure me \spcs$_i$ to be the most competent?\\
 \citep{kayne80}.}
\zl

As we shall see, it would be rather trivial to impose the classic island constraints in the standard syntactic
environments in which they arise.\footnote{In such a view, island effects could perhaps result from
grammaticized constraints, induced by parsing and performance considerations  
\citep{prichett,fodor78,fodor83}.}
 The problem is that island effects are riddled with exceptions which
defy purely syntactic accounts of the phenomena.  Hence, HPSG has generally refrained from
assuming that islands are syntactic, in contrast to mainstream linguistic theory.


\section{The Coordinate Structure Constraint}

\citet{Ross67} first observed that coordinate structures
impose various constraints on long-distance dependencies, shown in  (\ref{bors}), collectively dubbed the
\emph{Coordinate Structure Constraint}. For perspicuity,  I  follow \citet{grosu73} in referring to (i) as the \emph{Conjunct Constraint} and to (ii) as the \emph{Element Constraint}.

\ea \label{bors}
\textsc{Coordinate Structure Constraint} (CSC):

In a coordinate structure, (i) no conjunct may be moved,
(ii) nor may any element contained in a conjunct be moved out
of that conjunct
\ldots \,unless each conjunct properly contains a gap paired with the same filler.
\z 

\noindent
The \emph{Conjunct Constraint} (CC) is illustrated by the unacceptability of the extractions in (\ref{islands-cc}). No such constraint is active in other constructions like those in (\ref{com}) and (\ref{compar0}), for example. 
 

\eal
\label{islands-cc}  
\ex[*]{Which celebrity did you see [Priscilla and \spc]?\\
(cf.\ with `Did you see Priscilla and Elvis?')}
\ex[*]{Which celebrity did you see  [ \spc and Priscilla]?\\
(cf.\ with `Did you see Elvis and Priscilla?')}
\ex[*]{Which celebrity did you see  [ \spc or/and a picture of \spc]?\\
(cf.\ with `Did you see Elvis or/and a picture of Elvis?')}
%\item \bad{*}Who did you compare  [[\spc] and [\spc]]?
\zlcont 

\ealcont
\label{com}
\ex[]{Which celebrity did you see Priscilla with \spc?\\
(cf.\ with `Did you see Priscilla with Elvis?')}
\ex[]{Which celebrity did you see \spc with Elvis?\\
(cf.\ with `Did you see Priscilla with Elvis?')}
%\item Which celebrity did you prefer \spc to a picture of \spc?
%\item Who did you compare  [[\spc] with [\spc]]?
\zlcont
% the memoir of with a movie about ?


\ealcont
\label{compar0}
\ex[]{Which celebrity  is Kim as tall as \spc?}
\ex[]{Which celebrity did you say Robin arrived earlier than \spc?}
\zl


In HPSG accounts of extraction that assume the existence of traces \citep{pollardsag,levhubook} the CC must be stipulated at the level of the coordination construction, by stating that conjuncts cannot be empty elements.\footnote{See however \citet[317--318]{levine17} for the claim that  each conjunct must contain at least one stressed syllable. Given that traces are phonologically silent, nothing is there to bear stress and the CC is obtained. This raises the question of why no such stress constraint exists in P-stranding, for example, or indeed in any kind of extraction.}  On the other hand, the CC  follows immediately in a traceless account of  filler-gap dependencies  \citep{fodorsagt,bouma,ginzsag,fgsag08} since there is simply nothing to conjoin in (\ref{islands-cc}), and thus nothing else needs to be said about conjunct extraction; see \citet{sagonline} for more criticism of traces.

HPSG's\label{page-hpsg-traceless-account-arg-st-extraction-conjuncts} traceless account of the CC is semantic in nature, in a sense.  Coordinators like \emph{and}, \emph{or}, \emph{but} and so on are not regarded as heads that select arguments, and therefore have empty {\sc arg-st} and valence specifications. And given that HPSG assumes that the signs that can appear  in a given lexical head {\sc slash} values are valents, then it follows that the signs that coordinators combine with cannot instead be registered
in the coordinator's {\sc slash} feature. Hence, words like \emph{and} have no valents, no arguments
and therefore no conjunct extraction. Incidentally,  adnominal adjectives cannot be extracted either, for exactly the same reason, as they are not selected  by any head, and therefore  are not listed in any {\sc arg-st} list. 

In order to allow certain adverbials to be extractable,  \citet{ginzsag} assume that those adverbials are  members of {\sc arg-st}.
See \citet{levhubook} for more on adverbial extraction, and see  \crossrefchaptert{udc} for further discussion.\footnote{The empirical facts are less clear when it comes to adnominal PPs, however. Even PPs that are usually regarded as modifiers can sometimes be extracted, as in \emph{From which shelf am I not supposed to read any books?} In many such extractions the PP can alternatively be parsed as VP modifier, which complicates judgements. }

 
Let us now turn to the \emph{Element Constraint},  illustrated in (\ref{cc2}). As before, the constraint appears to be restricted to coordination structures, as no  oddness arises in the comitative counterparts like  (\ref{cc3}), or in comparatives like (\ref{compar}).\footnote{Although \citet[83]{winter01} and others claim that coordination imposes semantic scope islands, \citet[\S3.6]{chavesthesis} shows that this is not the case, as illustrated in examples like those below.

\eal
\ex The White House is very careful about this. An official representative [[will
              personally read each document] and [reply to every letter]].\\
              ($\forall$ doc-letter $>$$\exists$ representative / $\exists$ representative $>$$\forall$ doc-letter)

\ex We had to do this ourselves. By the end of the year,
some student [[had proof-read every document] and [corrected each
theorem]].\\
($\forall$  doc-theorem $>$$\exists$ student / $\exists$ student $>$$\forall$ doc-theorem)

\ex Your task is  to document the social interaction
between [[each female] and [an adult male]].\\
($\forall$ female $>$$\exists$ adult male / $\exists$ adult male $>$$\forall$ female)
\zllast
}


 

\eal
\label{cc2}
\ex[*]{Which celebrity did you see [Priscilla and a picture of \spc]?\\
(cf.\ with `Did you see Priscilla and a picture of Elvis?')}
\ex[*]{Which celebrity did you see [a picture of \spc and Priscilla]?\\
(cf.\ with `Did you see a picture of Elvis and Priscilla?')}
\zlcont

\ealcont 
\label{cc3} 
\ex Which celebrity did you see [the brother of \spc with Priscilla]?
\ex Which celebrity did you see [Priscilla with the brother of \spc]?
\zlcont

\ealcont
\label{compar} 
\ex Which celebrity did  [[you enjoy the memoir of \spcs more] than
                                 [any other non-fiction book]]?

\ex Which celebrity did you say that [[the sooner we take a picture of \spc ],
[the quicker we can go home]]?
\zl


The ATB exception to the CSC is illustrated by the acceptability of (\ref{cc4}), where each conjunct hosts a gap, linked to the same filler. As already noted above in (\ref{cc5}), the fact that multiple gaps can be linked to the same filler is not unique to  coordination. 

\eal\label{cc4}
\ex Which celebrity did you buy [[a picture of  \spcs and a book about \spcs]]?
\ex Which celebrity  did you [[meet \spcs at a party] and [date \spcs for a few months]]?
\zl


\citet{gazdar} and \citet{gpsg} assumed that the coordination rule  requires {\sc slash} values to be structure-shared across conjuncts and the mother node, thus  predicting  both the Element Constraint and the ATB exceptions. The failure of
movement-based grammar to predict multiple gap extraction facts was also seen
as a major empirical advantage of GPSG/HPSG. A similar constraint is assumed in \citet[202]{pollardsag} and
\citet[60]{Beavers}, among others, illustrated in (\ref{cschpsgr}). See  \crossrefchaptert{coordination} for more discussion about coordination. 

\ea
\label{cschpsgr}
{\sc Coordination Construction} (abbreviated)

\begin{avm}
{\footnotesize $coordinate$-$phr$} \impl \[synsem \| nonlocal \| slash \@{1}\\
                                                             dtrs \<  \[synsem \| nonlocal \| slash & \@{1} \], \\ 
                                                                              \[synsem \| nonlocal \| slash & \@{1} \] \>\]
\end{avm}
\z

\noindent
Because   the {\sc slash} value \avmbox{1} is structure-shared between the mother and the daughters
in (\ref{cschpsgr}),  all  three nodes must bear the same {\sc slash} value.  This predicts the CSC and the ATB exceptions  straightforwardly. The failure of mainstream Chomskyan  grammar to predict these and related multiple  gap extraction facts in a precise way is regarded as one of the  major empirical advantages of HPSG  over movement-based accounts.

But the facts about extraction in coordination structures are more complex than originally assumed, and than (\ref{cschpsgr}) allows for. A crucial  difference between the Conjunct Constraint and the Element Constraint is  that the latter is only in effect if the coordination has a symmetric interpretation  \citep{Ross67,goldsmith,lakoff86,levinprince86}, as in  (\ref{asym}).\footnote{In asymmetric coordination, the order of the conjuncts has a major effect on the interpretation.
  Thus, \emph{Robin jumped on a horse and rode into the sunset} does not 
mean the same as  \emph{Robin rode into the sunset and jumped on a horse}. Conversely,
in symmetric coordination the order of the conjuncts has no interpretational  differences,
as illustrated by the paraphrases \emph{Robin drank a beer and Sue ate a burger}
and \emph{Sue ate a burger and Robin drank a beer}.}


\eal  \label{asym}
\ex
 Here's the whiskey  which I [[went to the store] and [bought \spc]].
\ex Who did Lizzie Borden [[take an ax] and [whack \spcs to death]]?
\ex How much can you [[drink \spc] and [still stay sober]]?
\zl

% OR

\noindent
The coordinate status of (\ref{asym}) has been questioned since \cite{Ross67}. After all, if these are subordinate structures rather than coordinate structures, then the possibility for non-ATB long-distance dependencies ceases to be exceptional. But as 
\citet{schmerling72},  \citet{lakoff86}, \citet{levinepostal} and \citet{kehler}  point out,  there is no empirical reason to assume that the examples in (\ref{asym}) are anything other than coordination structures. 

Another reason to reject the idea that  the {\sc slash} values of the daughters and the mother node are are simply equated in ATB extraction is the fact that sometimes multiple gaps are `cumulatively' combined into a `pluralic gap'.\footnote{See for example  \citet{munn98atb,munn99atb},  \citet[136, 160]{postal98},
    \citet[125]{kehler},
 \citet{gawronkehlersalt}, 
 \citet{zhang}, \citet{chavessubjexp}, and \citet{Vicente2016-NELS46}.}
As an example, consider the extractions in (\ref{cumulative}). There are two possible interpretations
for such extractions: one in which the \emph{ex situ} signs (i.e.\ the gap signs) and the filler phrase are co-indexed, and therefore co-referential, and a second reading in which the two \emph{ex situ} phrases are not co-indexed even though they are linked to the same filler phrase.  Rather, the filler phrase refers to a plural
 referent composed by the referents of the \emph{ex situ} signs, as indicated by the subscripts 
in (\ref{cumulative}). For different speakers, the preferred reading is the former, and in other cases,
the latter, often depending on the example.

\eal \label{cumulative}
\ex
\,[What]$_{\lbrace i,j \rbrace}$ did Kim eat \spc$_i$ and drink \spc$_j$ at the party?\\
(answer: `Kim at pizza and drank beer')
\ex \,[Which city]$_{\lbrace i,j \rbrace}$ did Jack travel to \spc$_i$ and Sally decide to live in \spc$_j$?\\
(answer: `Jack traveled to London and Sally decided to live in Rome')
\ex \,[Who]$_{\lbrace i,j \rbrace}$ did the pictures of \spc$_i$ impress \spc$_j$ the most?\\
(answer: `Robin's pictures impressed Sam the most')
\ex \,[Who]$_{\lbrace i,j \rbrace}$ did the rivals of \spc$_i$ shoot \spc$_j$?\\
(answer: `Robin's rivals shot Sam')
\ex \,[Who]$_{\lbrace i,j \rbrace}$ did you send nude photos of \spc$_i$ to \spc$_j$?\\
(answer: `I sent photos of Sam to Robin')
%\item Where$_{\lbrace i,j \rbrace}$ did the train line run to \spc$_i$ from \spc$_j$?\\
%(possible answer: `The train line ran from NYC to Montreal')
\zl


In conclusion, the  non-ATB exceptions in (\ref{asym}) suggest that the coordination rule should not constrain {\sc slash} at all, as argued for in \citet{chaves}.  Rather,  the Element Constraint, its ATB exceptions in (\ref{cc4}a,b) and the asymmetric non-ATB exceptions in (\ref{asym}) are more likely to be the consequence of an independent semantic-pragmatic constraint that requires the filler phrase to be `topical'  relative to the clause \citep{lakoff86,kuno87,kehler,kubotalee}. Thus, if the coordination is  symmetric, then the topicality requirement  distributes over each conjunct, to require that the  filler phrase be topical in each conjunct. Consequently, extraction must
be ATB in symmetric coordination. No distribution needs to take place in asymmetric coordination, and thus both 
ATB and non-ATB extraction is licit in asymmetric coordination. For an attempt to transfer
some of Kuno's and Kehlers' insights into HPSG  see \citet{chaves}. In the latter proposal,
 the coordination rule is like most other rules in the grammar in that it says nothing about
the {\sc slash} values of the mother and the daughters, along the lines of 
\citet[354]{levhubook}. In other words, the constraints on {\sc slash} in (\ref{cschpsgr}) are unnecessary.  Rather, pragmatics is the driving force behind how  long-distance dependencies propagate one or more conjuncts, depending on the coordination being interpreted symmetrically or not. 
See \crossrefchaptert{coordination} for more discussion.

Let us take stock. The CSC does not receive a unitary account in modern HPSG, given that the Conjunct Constraint and the Element Constraint  are of a very different nature. Whereas the former does not admit ATB extraction, and is predicted by a traceless analysis, the latter allows ATB extraction as seen by the contrast between  (\ref{islands-cc}c) and  (\ref{cc4}). 
Upon closer inspection, the Element Constraint and the ATB exceptions are semantic-pragmatic in nature.  As we shall see, a similar conclusion is plausible for various other island phenomena.


\section{Complex NP Constraint}

The Complex NP Constraint concerns the difficulty in extracting out of complex NPs formed with either relative clauses (\ref{cpnpb}) or complement phrases (\ref{cpnpb2}).

\eal \label{cpnpb}
\ex[*]{ [What]$_i$ does Robin know $[$someone who has \spc$_i]$?\\
(cf.\ with `Does Robin know someone who has a drum kit?')}
\ex[*]{[Which language]$_i$ did they hire [someone [who speaks \spc$_i$]]?\\
(cf.\ with `Did they hire someone who speaks Arabic?')}
\zlcont

\ealcont  \label{cpnpb2}
 \ex[*]{[Which book]$_i$ do you believe the claim [that Robin plagiarized  \spc$_i$]?\\
(cf.\ with `Do you believe the claim that Robin plagiarized  \emph{this book}?')}
\ex[*]{What$_i$  did you believe [the rumor [that Ed disclosed \spc$_i$]]?\\
(cf.\ with `Did you believe [the rumor [that Ed disclosed \emph{that}]]?')}\todostefan{(15b)'s translation
  has markup in it, the other two examples do not have this. Is this intended?}
\zl 

\noindent
It is tempting to  prevent extractions out of adnominal clauses by simply stipulating  that the {\sc slash} value of the modifier must be empty, as (\ref{relcxt}) illustrates.
 Perhaps, along the lines of \citet{fodor78,Fodor83}, \citet{berwickwein}, and \citet{hawkins,hawbook}, 
 processing difficulties  lead to the grammaticization of such a constraint, effectively
  blocking any modified head from hosting any gaps.

\ea
{\sc Head-Modifier Construction} (abbreviated)

\begin{avm}
{\footnotesize $head$-$mod$-$phr$} \impl
                                     \[%synsem \| nonloc \| slash  $\lbrace \rbrace$\\
                                                             head-dtr  \@{1}\\
                                     dtrs  \<  \@{1},   \[synsem \[ loc \| mod  \@{1}\\ 
                                                                               nonloc \| slash $\lbrace \rbrace$ \]\] \>\]
\end{avm}
\z \label{relcxt}


However, the robustness of the CNPC  has been challenged by various
counterexamples over the years \citep{Ross67,pollardsag,kluender,postal98,saghof}.
The sample in (\ref{csubok}) involves acceptable
extractions from NP-embedded complement CPs (some of which are definite),
 and (\ref{csubok2}) involves acceptable extractions from NP-embedded relative clauses.\footnote{Counterexamples to  the CNPC can be found in a number of languages, 
including Japanese and  Korean \citep{kunojap,Nishigauchi99},  Ahan \citep{saah},  Danish  \citep[Chapter 2]{shir}, Swedish  \citep{allwood,engdahl82} Norwegian  \citep{taraldsen82} and Romance languages \citep{Cinque10}.  In some languages that support verb constructions, the CNPC is 
 not active, which goes with the analysis as complex predicates \citet{vives}.}


%Cases like ROSS have been taken to be lexically conditioned and  treated in terms of complex-predicate formation /N-V reanalysis/N-incorporation (Chomsky 1980, Kayne 1981, Cinque 1990, Davies and Dubinsky 2003).

\eal
 \label{csubok}
\ex The money which I am making [the claim [that the company squandered \spc]] amounts to \$400,000.\\
 \citep[206, 207]{pollardsag}

\ex  Which rebel leader did you hear [rumors [that the CIA assassinated \spc]]?
\ex Which company did Simon spread [the rumor [that he had started \spc]]?
\ex What did you get [the impression [that the problem really was \spc]]?\\
\citep{kluender}
\zlcont

\ealcont \label{csubok2}
\ex This is the kind of weather$_i$ that there are [many people [who like \spc$_i$]].\\
\citep{shirlappin}

\ex Violence is something$_i$ that there are [many Americans [who condone \spc$_i$]].\\
\citep[108]{mccawley81}

\ex There were several old rock songs$_i$ that she and I were [the only two [who
knew \spc$_i$]].\\
\citep{chungmc}

\ex This is the chapter$_i$ that we really need to find [someone [who understands \spc$_i$]].\\
\citep[238]{kluender92}

\ex Which diamond ring did you say there was [nobody in the world [who could
buy \spc$_i$]]?\\
\citep[206]{pollardsag}

\ex John is the sort of guy that I don't know [a lot of people [who think well of
\spc$_i$]].\\
\citep[230]{culicover99}
\zl

In the above counterexamples, the relative clauses contribute to the main assertion of the utterance, rather than expressing background information. For example, (\ref{csubok2}a) asserts `There are many people who like this kind of weather' rather than `This the kind of weather', and so on.
Some authors have argued that it is precisely because such relatives
express new information that the extraction can escape the embedded clause
  \citep{shirlappin,kuno87,Dean,goldberg13}.  If this is correct, then the proper 
  account of CNPC effects is not unlike that of the CSC. In both cases, 
  the information structural status of the clause that contains the gap is crucial
  to the acceptability of the overall long-distance dependencies.\footnote{Although it is sometimes claimed
   that
such  island effects are also active in logical form and semantic scope \citep{may85,ruys,fox,sab,katzira}, 
  there much reason to be skeptical.  For example, the universally quantified noun phrase  in (i)  is embedded in a relative clause can have wide scope over the indefinite \emph{someone}, constituting a semantic CNPC violation. Note that these relatives are not presentational, and therefore are not specially permeable to extraction.

\eal
\ex  We were able to find someone who was an expert on each of the
     castles we planned to visit.  \citep[304]{MRS}
\ex John was able to find someone  who is willing to learn every  language
    that we intend to study. \citep{chavesrnr}
\zllast}

 
 
 % footnote Han and Kim 2004  
  
In addition to pragmatic constraints,  \citet{kluender92,kluender}  proposed that 
processing factors also  influence the acceptability of    CNPC violations.
Consider for example the acceptability hierarchy in  (\ref{k});  more specific filler phrases increase acceptability, whereas    the presence of more specific phrases between the filler and the gap seem  to cause increased processing difficulty, and therefore   lower the acceptability of the sentence.  The symbol `$<$' reads as `is less acceptable than'.

\eal \label{k}
\ex What do you need to find the expert who can translate \spc?  $<$
\ex What do you need to find an expert who can translate \spc?   $<$
\ex What do you need to find someone who can translate \spc? $<$
\ex Which document do you need to find an expert who can translate \spc?
\zl


\indent
There is on-line sentence processing evidence that   CNPC   violations  with more informative  fillers are more acceptable and are processed faster at the gap site than violations with less informative fillers \citep{hofsaglang}, as in  (\ref{dd4}).
 
\eal \label{dd4}
\judgewidth{?}
\ex[?]{Who  did you say that nobody in the world could ever depose \spc?}
\ex[]{Which  military dictator did you say that nobody in the world 
could ever depose \spc?}
\zl



\noindent
The same difference in reading times  is found in sentences without CPNP violations, in fact. For example,  (\ref{hs}b) was found to be read faster at \emph{encouraged} than (\ref{hs}a). Crucially, that critical region of the sentence is not in the path of any filler-gap dependency.
 
\eal \label{hs}
 \ex The diplomat contacted the dictator who the activist looking for more contributions encouraged to preserve natural habitats and resources.
 
 \ex The diplomat contacted the ruthless military dictator who the activist looking for more contributions encouraged to preserve natural habitats and resources.
 \zl



Given that  finite tensed verbs can be regarded as definite, and infinitival verbs as indefinite \citep{partee84}, and given that finiteness can create processing difficulty 
 \citep{kluender92,gibson0000}, then acceptability clines like 
(\ref{fininf}) are to be expected. See \citet[Chapter 5]{levhubook} and \citet[308]{levine17} for more discussion.

\eal \label{fininf}
\ex Who did you wonder what Mary said to \spc?  $<$
\ex Who did you wonder what to say to \spc?  $<$
\ex Which of the people at the party did you wonder what to say to \spc? 
\zl

\subsection{On D-Linking}

The amelioration caused by more specific (definite) \emph{wh}-phrases as in  (\ref{k}d), (\ref{dd4}b) and (\ref{fininf}c)  has been called a `D-Linking' effect  \citep{pesetskydlink,pesetskybook}. It purportedly arises if the set of possible answers is pre-established or otherwise salient. But there are several problems with the D-Linking story.   First, there is currently no non-circular definition of D-Linking; see 
\citet[16]{pesetskybook},
 \citet[247--250]{ginzsag}, 
 \citet[33, 39]{chung94} and
\citet[242, 268--271]{levhubook}. Second, the counterexamples above are given out of the blue, and therefore cannot evoke any preexisting set of referents, as D-Linking 
requires.   Furthermore, nothing should prevent D-Linking with a bare \emph{wh}-item, as Pesetsky himself acknowledges, but on the other hand there is no experimental evidence that context can lead to D-linking of a bare \emph{wh}-phrase \citep{sprousediss07,villata}.\footnote{For more detailed criticism of D-Linking see \citet{hof2007}.}

 \citet{kluenderkustas}, \citet{saghof}, \citet{philcls}, \citet{philipt07} and \citet{hofsaglang} argue that more definite \emph{wh}-phrases improve the  acceptability of  extractions because they resist memory decay better than indefinites, and 
 are compatible with fewer potential gap sites. In addition, \citet{kroch89} and \citet[270]{levhubook} point out that D-Linking amelioration effects  may simply result from  the plausibility of background assumptions associated with the proposition. 

\subsection{On memory limitations}
 
  \citet{sprouse12} use $n$-back and serial recall tasks to argue that there is no evidence that working
  memory limitations correlate with island acceptability, and therefore that the `processing-based'
  account of islands put forth by  \citet{kluender92,kluender}, \citet{kluenderkustas}, \citet{hofsaglang} and others is unfounded. To be sure, it cannot be stressed enough that the accounts in \citet{kluender92} and \citet{hofsaglang} are  not strictly based on performance, and involve other factors as well, most notably plausibility and pragmatic factors. See in particular \citet[49]{hoflangreply}, where it is argued that at least some extraction constraints may be due to a combination of syntactic, semantic, pragmatic, and performance factors.  Basically, if the correct location of a gap is syntactically, semantically, or pragmatically highly unlikely in that particular utterance, then it is less likely for the sentence to be acceptable.  Indeed, there is independent experimental evidence that  speakers attend to  probabilistic information about the syntactic distribution of  filler-gap dependencies \citep{culcogsci},  and that gap  predictability  is crucial for on-line processing of islands \citep{michelt}.\footnote{More broadly, there is good  evidence that  speakers deploy probabilistic information when  processing a variety of linguistic input, including words \citep{altman99,arai,creel,delong,kutas84},  lexical categories \citep{gibson07,levy13,tabor97},  syntactic structures \citep{levyted,lau06,levy08,staub},  semantics \citep{altman99,federmeier,kamide03}, and pragmatics \citep{shankweiler,mak,roland12}.} But as \citet{reply2} point out, there is no reason to believe  that $n$-back and serial recall tasks are strongly correlated to working memory capacity to begin with. Second, one of the main points of \citet{hofsaglang} is that the literature  on experimental  island research has not systematically controlled for multiple factors that can impact the processing and comprehension of complex sentences. If the experimental items are excessively  complex, then readers are more likely to give up understanding the utterances and subtler effects  will not be measurable.  \citet{phil13}, however, regard such concerns as irrelevant.
    Although it is unclear to what extent expectations and processing constraints contribute to island effects, it is likely that they play some role in CNPC effects, as well as  other island types discussed below.


\section{Right Roof Constraint}

Rightward movement is traditionally regarded as being clause bounded. Such
 \emph{Right Roof Constraint}  \citep{Ross67} effects are illustrated in (\ref{rrc}), 
in which a phrase appears \emph{ex situ} in a position 
to the right of its \emph{in situ} counterpart; see   \citet{akma75}, \citet{baltin78}, and  \citet{stowelldiss}, among others.

\eal \label{rrc}
\ex[*]{I [met a man [who knows \spc$_i$] yesterday] [all of your songs]$_i$.}
\ex[*]{[[That a review \spc$_i$ came out yesterday] is catastrophic] [of this article]$_i$.}
\ex[*]{It was believed \spc$_x$ that $[$there walked into the room$]$ \spc$_y$ $[$by everyone$]_x$ $[$a man with  long blond hair$]_y$.\\
\citep{rochemont}}
\zl



\noindent
When treated as a form of extraction, rightward movement has been predominantly
accounted for via a feature {\sc extra(posed)} \citep{Keller95b,Noord:Bouma:96,eynde96,
kellerverb,Mueller99a,KimSag2005}, rather than by {\sc slash}.  
Thus, RRC island effects can be easily modeled by stipulating that the  {\sc extra} value of an S node must be empty. One way to do so is to state that any S dependent (valent or adjunct) must be [{\sc extra} $\lbrace \rbrace$].
Thus, no extraposed element may escape its clause. However,  the oddness of (\ref{rrc}) may not be due any such syntactic stipulation, given the acceptability of counterexamples like (\ref{longb}). Note that the adverbial interveners in such examples do not  require parenthetical prosody. Conversely, even strong  parenthetical
prosody on the adverbs in (\ref{rrc}) fails to improve those data.


\eal \label{longb}
\ex  I've [been requesting [that you pay back \spc] [ever since May]] [the money
I lent to you a year ago].\\
  \citep[251]{kayne00}

\ex I've [been wanting to [meet someone
who \textsc{knows} \spc] [ever since I was little]] [exactly what happened to Amelia Earhart].

\ex  I've been wondering  if it is possible   \spc]
[for many years now] [for anyone to memorize the Bible word for word].\\
\citep[861]{chavesrnr}
\zl


\noindent
The  durative semantics  of   \emph{I've been wanting/requesting/wondering} 
 raises an expectation about the realization of   a durative  
adverbial expression like  \emph{ever since} or \emph{for many years} that
provides information about the durative semantics of the main predicate.
Hence, the adverb is cued by the main predication, in some sense, and 
  coheres much better in a high attachment than with a lower one.

The fact that the RRC is prone to exceptions has been noted by multiple authors as the sample in 
 (\ref{unb}) illustrates. In all such cases, a phrase is right-extracted
 from an embedded clause, which should be flat out impossible if extraposition is clause-bounded. Again, the adverbial  interveners in (\ref{unb}) do not require any special prosody, which means that
  these data cannot be easily discarded as parenthetical insertions.
  
\eal \label{unb}
\ex I have  [wanted [to know \spc] for many years] [exactly what happened to Rosa Luxemburg].\\
(attributed to \citealt{
witten} in \citealt[92n]{postal74})

\ex I have  [wanted [to meet \spc] for many years] [the man who spent so much money planning the assassination of Kennedy].\\
(attributed to Janet Fodor (p.c.) in \citealt[177]{gazdar})

\ex Sue [kept [regretting \spc] for years] [that she had not turned
him down].\\
\citep{eynde96}

\ex She has been [requesting that he [return \spc] [ever since last Tuesday]] [the book
that John borrowed from her last year].\\
\citep[251]{kayne00}

\ex  Mary [wanted [to go \spc] until yesterday]  [to the public lecture].\\
(Howard Lasnik 2007 course handout\footnote{\url{http://ling.umd.edu/~lasnik/LING819\%202007/Multiple\%20Sluicing\%20819\%20.pdf}; Retr.\ 2009.})
\zl



Further evidence against a syntactic account of RRC comes from  corpora 
\citep{Mueller2004d,Mueller2007c} and experimental findings  \citep{strunk08,strunk}, which confirm
 that  extraposition does  not always obey  island constraints.  The counterexamples in (\ref{nosubza}a--c) are adapted from
 \citet{strunk08} and \citet{strunk}, and those in (\ref{nosubza}d--f) are from \citet[863]{chavesrnr}.



\eal \label{nosubza}
\ex {[In [what noble capacity \spc]] can I serve him
[that would glorify him and magnify his name]?}

\ex  {We drafted [a list of basic demands \spc] last night [that have
to be unconditionally met or we will  go on strike].}

\ex { For example, we understand that Ariva buses have won [a number
of contracts for routes in London \spc] recently,
[which will not be run by low floor accessible buses].}

\ex {Robin bought [a copy of a book \spc] yesterday [about ancient Egyptian culture].}

\ex {I'm reading [a book written by  a famous physicist
\spc]
right now, [who was involved in the Manhattan Project].}

\ex {I saw [your ad in a magazine \spc] yesterday [on the table at the dentist office].}
\zl


 \citet{grosurrc},  \citet{gazdar} and \citet{stucky} argued that the RRC is the result of
  performance  factors such as  syntactic and semantic parsing expectations and  memory resource  limitations,  not grammar proper.   Indeed, we now know that  there is a general well-known  tendency for the language processor to  prefer attaching new material
 to the more recent constituents  \citep{frazcl,gibetal,trax98,fodor02b,fernandez03}. 
Indeed, eye-tracking studies like \citet{hnps}  indicate that the
parser is reluctant to adopt  extraposition  parses.
This  explains why     extraposition in written texts
is less common in proportion to length of  the  intervening material \citep{UBDKKOS98b}:
the longer the structure, the bigger the processing burden. 
 Crucially, however, the preference for the closest attachment can be weakened by
many factors \citep{fernandez03,desmet,devic,carreras}. 
For example,  \citet{levyted}  show that relative clause extraposition creates
 significant processing difficulty when compared with  non-extraposed counterparts of 
 the same sentences,   but that a preceding context that sets up a strong expectation
for a relative clause modifying a given noun can  facilitate
comprehension of an extraposed relative clause modifying that noun.
In other words, in spite of a larger processing burden,
some extrapositions can be made easier to process by parsing expectations.

A detailed account of extraposition island phenomena  does not exist in any framework, as far as I
am aware. But the line of inquiry first proposed by  \citet{grosurrc},  \citet{gazdar} and \citet{stucky},  and later experimentally supported by \citet{levyted}, \citet{strunk08}, and \citet{strunk} seems to be on the right track. If so, then there is no syntactic constraint on {\sc extra}. Rather, RCC effects are to a large extent the result of difficulty in integrating the extraposed phrase in its  \emph{in situ} position.  

\subsection{Freezing}

A related island phenomenon also involving rightward displacement, first noted in \citet[305]{Ross67},
is \emph{Freezing}: leftward extraction (\ref{z1}a) and extraposition (\ref{z1}b) cause low acceptability when they interact, as seen in (\ref{f}). In (\ref{f}a) there is extraction from an extraposed PP,  in (\ref{f}b)  there is extraction from an extraposed NP,  and  in (\ref{f}c) an extraction from a PP crossed with direct object extraposition. 
 
 
\eal \label{z1}
\ex Who$_j$ did you [give [a picture of \spc$_j$] [to Robin]]?
\ex Did you [give \spc$_i$ [to Robin] [a picture of my brother]$_i$]?
\zlcont

\ealcont \label{f}
\ex[*]{Who$_j$ did you [give a picture  \spc$_i$]  [to Robin] [of \spc$_j$]$_i$?}    
\ex[*]{Who$_j$ did you [give \spc$_i$ [to Robin] [a picture of \spc$_j$]$_i$]?} 
\ex[*]{Who$_j$ did you [give \spc$_i$ [to \spc$_j$] [a picture of my brother]$_i$]?}
\zl


  \citet[457]{fodor78} notes that (\ref{f}c) has a syntactically highly probable temporary alternative parse  in which \emph{to} combines with the NP \emph{a picture of my brother}. The existence of this local ambiguity  likely disrupts  parsing, especially as it occurs in a portion of the sentence that contains two gaps in close succession. Indeed, constructions with two independent gaps in close proximity  are licit, but not trivial to process, as seen in (\ref{ddg}), specially if the  extraction paths cross  \citep{fodor78}, as in (\ref{ddg}b).

\eal \label{ddg}
\ex[*]{This is a problem which$_i$ John$_j$ is difficult 
to talk to  \spc$_j$ about \spc$_i$.}
\ex[*]{Who$_j$  can't  you remember which papers$_i$ 
you sent copies of \spc$_i$ to \spc$_j$?}
\zl

\noindent
A similar analysis is offered by \citet[477]{freezing}, who note that  constructions like (\ref{f}c) must cause increased processing effort since the point of retrieval and integration coincides with the point of reanalysis. The existence of a preferential alternative parse that is locally licit but globally illicit  can in turn lead to a  ``digging-in'' effect \citep{ferreirahend,ferreirahend2,tabor3}, in which the  more committed the parser  becomes to a syntactic parse, the  harder it is to backtrack and reanalyze the  input. The net effect of these factors is that the correct parse of
(\ref{f}c) is less probable and therefore harder to identify than that of  (\ref{f}b), which suffers from none of  these problems, and is  regarded to be more acceptable than
 (\ref{f}c) by \citet[453]{fodor78} and others. See \citet{chavesf} for experimental evidence that speakers
 can adapt and to some extent overcome some of these parsing biases.
 
 Finally, prosodic and pragmatic factors are likely also at  play in (\ref{f}), as in the RRC. \citet{huck} show that when an unstressed stranded preposition  is separated from its selecting head by another phrase, oddness ensues for prosodic reasons.  Finally,  \citet{huck} and  \citet{boling92} also
 argue that freezing effects are also in part due to a pragmatic conflict created by extraposition and extraction: \emph{wh}-movement has extracted a phrase leftward, focusing interest on that expression, while at the same time extraposition has moved a constituent rightward,  focusing interest on that constituent as well.  
Objects tend to be extraposed when they are discourse new, and even more so when they are heavy \citep[71]{wasowbook}. Therefore, the theme phrase \emph{a picture of John} in (\ref{f}c) is strongly biased to be discourse new,  but this clashes with the fact that an entirely different entity, the recipient, is leftward extracted, and therefore is the \emph{de facto} new information that the open proposition is about.  No such mismatch exists in (\ref{f}a) or (\ref{f}b), in contrast, where the extraposed theme is more directly linked to the  entity targeted by leftward extraction.


\section{Subject islands}

Extraction out of subject phrases like (\ref{all}) 
is broadly regarded to be impossible in several languages, including
English \citep{Ross67,chomsky73}, an effect referred to as a \emph{Subject Island} (SI).
This constraint is  much less severe  in languages like Japanese , German, and Spanish,
among others \citep{stepanov,jurkaetal,goodall11,greco,fukuda,polinsky13}. 


\eal \label{all}
\ex[*]{Who did stories about terrify John?\\
 \citep[106]{chomsky77b}}

\ex[*]{ Who was a picture of laying there?\\
\citep[114]{kayne81} }

\ex[*]{Who do you think pictures of would please John?\\
\citep[497]{huang82}}

\ex[*]{Who does the claim that Mary likes upset Bill?\\
\citep[42]{lasniksaito}}

\ex[*]{Which candidate were posters of  all over the town?\\
\citep{lasnikpark}}
\zl


\noindent
However, English exceptions were noticed early on, and have since accumulated in the literature.
In fact,  for \citet{Ross67},  English extractions like (\ref{sie}a) are not illicit,
and more recently \citet[147]{chomsky08} has added more such counterexamples.
Other authors noted that certain extractions from subject phrases are naturally attested, as in (\ref{sie}b,c). Indeed,  \citet{annerels} shows that extractions like those in (\ref{sie}c) are in fact acceptable to native speakers,
and that no such island effect exists in French either.\footnote{For completeness, other authors argue that PP extractions from
 NP subjects are illicit, such as \citet[653]{lasnikpark}, among many others.}


\eal \label{sie}
\ex[]{[Of which cars]$_i$ were [the hoods \spc$_i$] damaged by the explosion?\\
\citep[4.252]{Ross67} }

\ex[]{They have eight children [of whom]$_i$ [five \spc$_i$] are still living at home.\\
\citep[1093]{hud12} }

\ex[]{ Already Agassiz had become interested in the rich stores of the extinct fishes of Europe, especially those of Glarus in Switzerland and of Monte Bolca near Verona, [of which]$_i$, at that time, [only a few \spc$_i$] had been critically studied.\\
\citep{santorini}}
\zl

\noindent
English exceptions to the SI constraint are not restricted to PP extractions, however. Although
\citet{Ross67} claimed NP extractions from NP subjects like (\ref{rosssi}) are illicit,
it was arguably premature to generalize from such a small sample.

\eal  \label{rosssi}
\ex[]{
The hoods of these cars were damaged by the explosion. 
}
\ex[*]{Which cars were the hoods of damaged by the explosion?\\
%\citep[ft.\,31,254]{Ross67}}
\citep{Ross67}
}
\zl


\noindent
Indeed, a   number of authors  have noted that some NP extractions from subject NPs are either passable or
 fairly acceptable,   as illustrated  in (\ref{etal}). See also
 \citet[195, ft.\,32]{pollardsag},
 \citet{postal98},
  \citet[304]{sauerelb},   
  \citet[230]{culicover99},
  \citet[265]{levhubook},  \citet[470, 471]{chavesextr},
and \citet{chavesjeruen}.

  
\eal  \label{etal}
%?[Who]$_i$ did [John's talking to \spc$_i$] bother you most?\\
 %\citep{engdahl}

\ex[]{[What]$_i$ were [pictures of \spc$_i$] seen around the globe?\\
  \citep[268]{kluender} }


\ex[]{ It's [the kind of policy statement]$_i$ that [jokes about \spc$_i$] are a dime a dozen.\\
\citep[204]{levineetal}}


\ex[]{ There are [certain topics]$_i$ that [jokes about \spc$_i$] are completely unacceptable.\\
\citep[252, ft.\,6]{Levine:Sag:03}}

\ex[]{  [Which car]$_i$ did [some pictures of \spc$_i$] cause a scandal?\\
 \citep[111]{fernandez}}
 
 \ex[]{ [What]$_i$ did [the attempt to find \spc$_i$] end in failure?\\
 \citep[370]{hofsaglang}}
 
 \ex[]{  [Which president]$_i$ would [the  impeachment of \spc$_i$] cause  outrage?\\
 \citep{chavesextr}}
 
 \ex[]{ I have a question$_i$ that [the probability of you knowing the {\slshape answer to} \spc$_i$] is zero.\\
\citep{chavessubj}}

\zl



Whereas SI violations involving subject CPs are not attested,  those involving
infinitival VP subjects like  (\ref{vpsia}) are.
See  \citet[471]{chavesextr} for more natural occurrences.


\eal  \label{vpsia}
\ex []{The eight dancers and their caller, Laurie Schmidt, make up the
Farmall Promenade of nearby Nemaha, a town$_i$
that [[to describe \spc$_i$ as tiny] would be to overstate its size]. \\
 \citep[1094, ft.\,27]{hud12} }

\ex[]{ In his bedroom, [which]$_i$ [to describe \spc$_i$ as small] would be
 a gross understatement, he has an audio studio setup.}

 \ex[]{ They amounted to near twenty thousand pounds, [which]$_i$ [to pay \spc$_i$] would have ruined me. 
(Benjamin Franklin, William Temple
Franklin and  William Duane.
1834.  Memoirs of Benjamin Franklin, vol 1. 
p.58)}
\zl 




Incidentally, subject phrases are not  extraposition islands either, as in (\ref{subxxx}).
See also \citet{Gueron:May:84}. Odd examples like 
  *\emph{$[$Pictures \spc$]$ frighten people $[$of John$]$} from \citet{drummond},
  are more likely due to a digging-in effect, caused
  by speakers assuming that the subject is syntactically and semantically complete 
  by the end of the verb phrase.




\eal \label{subxxx}
\ex {}[The circulation of a rumor \spc$_i$]
has started [that Obama will not seek re-election]$_i$.

\ex {}[A  copy of a new book \spc$_i$] arrived
yesterday [about ancient Egyptian culture]$_i$.

\ex {}[Concerns about the deaths  \spc$_i$] were
raised [of two diplomatic envoys recently abducted in Somalia]$_i$.
\zl


\subsection{Clausal Subject Constraint}

Let us now consider SI effects involving more complex subjects.
Infinitival subject clauses seem to impose no SI constraint, an observation going
back to \citet{kunotakamib}, but noted elsewhere a few times:

\eal \label{vpsi}
\ex[]{ This is something [which]$_i$  -- for you to try to understand \spc$_i$  -- would be futile.\\
\citep[49]{kunotakamib}}

\ex[]{ I just met Terry's eager-beaver research assistant [who]$_i$  -- for us
to talk to \spc$_i$ about  any subject other than linguistics -- would be absolutely pointless.\\
\citep[265]{levhubook}}

\ex[]{ There are [people in this world]$_i$ that --  for me to describe \spc$_i$ as despicable 
 -- would be an understatement.\\
 \citep[471]{chavesextr}.}
\zl




\noindent
Infinitival subjects contrast dramatically with finite  subjects. The latter are renowned for being particularly hard to extract from,  as in  (\ref{ssc}).
  \citet{Ross67} dubbed this extreme kind of SI the  \emph{Sentential Subject Constraint} (SSC). See also  \citet{chomsky73}, \citet{huang82},  \citet{Chomsky86b}, and \citet{freidin92}.\footnote{That said, \citet{chavessubj} reports that some native speakers find SSC violations like  (i) to be fairly acceptable, again raising some doubt about the robustness of English SI effects:

\ea
{}[Which actress]$_i$ does [whether Tom Cruise marries \spc$_i$] make any difference to you?
\zlast
}


\eal \label{ssc}
\ex[*]{[Who]$_i$ did [that Maria Sharapova beat \spc$_i$] surprise everyone?\\
(cf.\ with `That Maria Sharapova beat Serena Williams surprised everyone')}

\ex[*]{[Who]$_i$ did [that Robin married \spc$_i$] surprise you?\\
(cf.\ with `Did that Robin married Sam surprise you?')}
\zl

%\noindent
%
%\ea
%?The hat [which]$_i$ -- [[{\slshape that I brought} \spc$_i$] seemed strange to the nurse] -- was a fedora.\\
%\citep{Ross67}
%\z



There are some functional reasons for why clausal SI violations may be so strong. First, subject clauses are notorious for being particularly difficult to process, independent of extraction. Clausal  subjects  are often  stylistically marked and difficult  to process,   as  (\ref{that}a) illustrates. Thus, it is extremely hard to embed a  clausal subject within another clausal subject, even though such constructions ought to be perfectly grammatical,  like (\ref{that}b, c). In addition, it is known that tense can induce greater processing costs
 \citep{kluender92,gibson0000}. 

\eal  \label{that}
\ex[]{ That the food that John ordered tasted good pleased him.\\
\citep{cowper76,gibson91}}

\ex[*]{That that Jill left bothered Sarah surprised Max.\\
\citep{kimball}}
 
 \ex[*]{That that the world is round is obvious is dubious.\\
\citep{kuno74}} 
\zl


\noindent
 Interestingly, clausal subjects become more acceptable if  extraposed as shown in (\ref{exps}).
The explanation offered by  \citet[356--357]{fod74} is that speakers tend to  take the initial clause in the sentence  to be the  main clause. Thus, \emph{that} is taken to be the subject, but the remainder
of the structure does not fit this pattern. Thus, a sentence like (\ref{exps}a) causes
 increased  processing  load  because  it has a  different structure 
than the parser expects.
This processing problem does not arise
in the counterpart in (\ref{exps}b).\footnote{See  \citet{gibson07} for online evidence
that  the word \emph{that} is preferentially interpreted as a determiner even in syntactic contexts where it cannot be a determiner. The use of `determiner' corresponds to the traditional term, referring to a certain category of
 prenominal constituent rather than to the whole nominal phrase including the noun and all its dependents.
 Gibson's evidence suggests that both top-down (syntactic) expectations are independent from bottom-up (lexical) frequency-based expectations in sentence processing. Thus, 
clausal subject phrase  starting with the complementizer
\emph{that} is likely to be misparsed as a matrix clause with sentence-initial  pronominal or determiner \emph{that}.}



\eal  \label{exps}
\judgewidth{?}
\ex[?]{That [it is obvious that [the world is round]] is dubious.}
\ex[]{It is dubious that [it is obvious that [the world is round]].\\
\citep{kuno74}}
\zl


\noindent
  Indeed, \citet{fodor67},  \citet{bever}, and \citet{frazier88} also
 show that extraposed clausal subject sentences
 like (\ref{ext}a) are easier to process than
their \textit{in-situ} counterparts like (\ref{ext}b). 
Not surprisingly,  the former are much more frequent than the latter,
which  explains why the parser would expect the former more than the latter.


\eal  \label{ext}
\ex It surprised Max that Mary was happy. 
\ex That Mary was happy surprised Max.
\zl


% Gibson "that"

%* Who did [the fact that the candidate supported __] upset voters?
% http://www.bu.edu/linguistics/UG/course/lx500-s06/handouts/lx500s06q-13b-proc.pdf

% That the food that John ordered tasted good pleased him
% cowper76 gibson 91


\noindent
Note that embedded conditionals like (\ref{pinker}a) are also exceedingly 
difficult to parse, although perfectly syntactically well-formed.
However,   such structures can become more acceptable with fewer embeddings to strain
working memory, and with judicious use of prosodic cues that indicate  the relevant structure, as illustrated by  the acceptability of (\ref{pinker}b).
This is  evidence that the oddness of (\ref{pinker}) is
at least in part due to performance. 


\ea \label{pinker} If -- if every time it snowed there was no school -- then there wouldn't be classes all winter.
\z 


  
  
  
  If we add a filler-gap dependency to a sentence that already is complex by
  virtue of having  a clausal  subject, the resulting structure
  may be too difficult to parse.  This   point is illustrated by
  the contrast in (\ref{that2}).



\eal \label{that2}
\judgewidth{?*}
\ex[?*]{
What does that he will come prove?
}
\ex[]{
What does his coming  prove?\\
\citep{lewis93}
}
\zl


\noindent
As argued by \citet{dubinsky2009}, the low acceptability
 of extraction in subject-auxiliary inversion sentences with clausal subjects 
  is more likely to be the result of extragrammatical factors than of grammatical conditions.
For example,  not all extractions like (\ref{v}b) are unacceptable, as
\citet[382--387]{delahunty} and
\citet[115]{dubinsky2009} point out. 

\eal \label{v}
\ex[]{That the food that John ordered tasted good pleased me.}
\ex[*]{Who did that the food that John ordered tasted good please \spcs?}
\zl

The evidence discussed so far suggests that  sentences involving extraction and clausal subjects  are odd  at least in part due to the likely cumulative effect of   various sources of  processing complexity. 
Sentences with sentential subjects are unusual structures, which can 
mislead the parser into the wrong analysis. A  breakdown in comprehension can occur
because the parser must hold complex incomplete phrases in memory
while processing the remainder of the sentence. The presence of a filler-gap dependency
will likely only make the sentence harder to process. 
It is independently known that the  more committed the parser 
becomes to a syntactic parse, the  harder it is to reanalyze the 
string \citep{ferreirahend,ferreirahend2,tabor3}.
For example, unless prosodic or contextual cues  are employed to
boost the activation  of the correct  parse,   (\ref{compzeat}) will 
be preferentially  misanalysed as having the structure [NP [V [NP]]].

\ea Fat people eat accumulates.
\z \label{compzeat}

\noindent
The garden-path effect that the digging-in causes in example (\ref{compzeat}) serves as
 an  analogy for what may be happening in particularly difficult  subject island violations. 
In both cases,  the sentences have exactly one grammatical analysis, but that parse
 is preempted by a highly preferential alternative  which  ultimately cannot yield a 
 complete analysis of the sentence.  Thus, without prosodic cues indicating
 the extraction site, sentences like (\ref{temp}) induce a significant digging-in effect
 as well.


\eal \label{temp}
\ex[*]{
Which problem will a solution to be found by you?
}
\ex[*]{
Which  disease will  a cure for be found by you?
} 
\zl

This also explains why SI violations like  (\ref{excz}) are relatively acceptable: the  processing cost of an NP-embedded 
causal  is smaller than the processing cost incurred by processing a clausal subject.\footnote{For claims that NP-embedded clausal SI violations are illicit see  \citet[42]{lasniksaito},   \citet[796]{colinphillips}, and    \citet[67]{colin_horn}.}  Clausal subjects are unusual structures, inconsistent with the parser expectations \citep{fod74},  and   the presence of filler-gap dependency in an  NP-embedded clausal subject is less likely
to cause  difficulty for the parse to go awry than a  filler-gap dependency in a clausal subject.\footnote{\citet{clausen,clausencuny} provide  experimental evidence that  complex subjects cause  a measurable increase in processing load,  with and without extraction. Moreover,  it is known that 
elderly adults  have far more difficulty repeating sentences with complex subjects than sentences with complex objects  \citep{kemper86}. Similar difficulty is  found in timed reading comprehension tasks  \citep{kynette}, and in   disfluencies in non-elderly adults \citep{clarkwasow}. 
Speech initiation times  for sentences with complex subjects are
also known to be longer than for sentences with 
 simple subjects  \citep{ferreirasubj,tsiam},
 and sentences with center-embedding in subjects 
are harder to process than sentences 
with center-embedding in objects \citep{amy,eady}.
Finally,  \citet{garnsey}, \citet{kutasetal}, and \citet{vanpetten}  show
that the processing of open-class words, particularly at
the beginning of sentences, require
greater processing effort than closed-class words.}


\eal \label{excz}
\ex  \,[Which puzzle]$_i$ did the fact that nobody could solve \spc$_i$ astonish you the most?
\ex \,[Which crime]$_i$ did the fact that nobody was accused of \spc$_i$ astonish you the most?
\ex \,[Which question]$_i$ did the fact that none of us could answer \spc$_i$  surprise you the most?
\ex \,[Which joke]$_i$ did the fact that nobody laughed at \spc$_i$ surprise you the most?
%\item Which ancient script did the fact that nobody has deciphered surprise you the most?
%\item Which poison did the fact that there is no antidote to surprise you?
%\item Which disease did the claim that nobody can cure $|$ astonish you the most? ...
%\item  Which employees did the fact that the company refused to fire $|$ astonish you? ...
%\item Which bill did the fact that some senators tried to amend $|$ surprise you? ...
\zl

\subsection{Accounts of SI effects}

This complex array of effects  suggests that the SI constraint is not due to a single factor
\citep{chomsky08,chavessubj,fernandez}, be it grammatical or otherwise.  One possibility is that SIs are partly due to pragmatic and processing constraints, perhaps not too different from those that appear to be active in the island effects discussed so far. As  \citet[495]{kluender06} notes: ``Subject Island effects seem to be  weaker when the  \emph{wh}-phrase maintains a pragmatic  association not only with the gap, but also with the main clause predicate,  such that the filler-gap dependency into the subject position is construed as of some relevance to the main assertion of the sentence''. Indeed, many authors \citep{shir-jrn,valin86,kuno87,ken,Dean,goldberg13} have argued that
extraction is in general restricted to the informational focus of the proposition, and that
SIs (among others) are predicted as a consequence. In a nutshell,
since subjects  are typically reserved for topic continuity,  subject-embedded referents are unlikely to be the informational focus of the utterance.  Although it is not easy to construct sentences where a dependent of the subject can be easily deemed as the informational focus, it is by no means impossible. 
For instance,  (\ref{kh}a) is particularly acceptable because whether or not an impeachment causes outrage crucially depends on who is impeached (cf.\ with  \emph{Would the impeachment of Donald Trump cause outrage?}). Similarly, in (\ref{kh}b) whether or not an attempt failed or succeeded crucially depends on what was attempted (cf.\ with \emph{The attempt to find the culprit ended in failure}). 

\eal  \label{kh}
\ex[]{ Which President would [the  impeachment of \spcs] cause  outrage? \\
\citep{chavesextr}}

 \ex[]{ What did [the attempt to find \spcs]  end in failure?\\
 \citep[370]{hofsaglang}}
 \zl
 
Although experimental research has confirmed that sentences with SI violations tend to be less acceptable than grammatical controls \citep{sprsat2,goodall11,crawfordwccfl,clausencuny,greco}, and that their acceptability
remains consistently low during repeated exposition  \citep{sprsat2,crawfordwccfl}, other research has found that the acceptability of  SI violations is not consistently low, and can be made to increase significantly \citep{hiramatsu00,clausencuny,chavesjeruen,chavessubjexp}. This mixed evidence is consistent with the idea that SI effects are very sensitive to the particular syntax, semantics, and pragmatics of the utterance in which they occur. If the items are too complex, or stylistically awkward, or presuppose unusual contexts, then  
SI effects are  strong.  For example, if the extraction is difficult to process because the sentence gives rise to local  garden-path and digging-in effects, and is pragmatically infelicitous in the sense that the extracted element is not particularly relevant for the proposition (i.e.\ unlikely to be what the proposition is about) or comes from the presupposition rather than the assertion, then we obtain a very strong SI effect. Otherwise, the SI effect is weaker, and in some cases nearly non-existent like (\ref{kh}), (\ref{vpsia}), or 
 the  pied-piping examples studied by \citet{annerels}. The latter involve relative clauses, in which subjects are not strongly required to be topics, in contrast to the subjects or main clauses.

This approach also explains why subject-embedded gaps often become more acceptable in the presence of a second non-island gap: since the two gaps are co-indexed, then the fronted referent is trivially relevant for the main assertion, as it is predicated by the main verb. For example, the low acceptability of (\ref{pgbq}a) is arguably caused  by the lack of   plausibility of the described proposition: without further contextual information, it is unclear how the attempt to repair  an unspecified thing $x$ is connected to the attempt causing  damage to a car. 


\eal   \label{pgbq}
\ex[*]{What did [the attempt to repair \spcs] ultimately damage the car?}
\ex[]{What did [the attempt to repair \spcs] ultimately damage \spc?\\
\citep{colinphillips}}
\zl


\noindent
The example in  (\ref{pgbq}a)  becomes more acceptable if it is contextually  established that $x$ is  a component of the car. In contrast, (\ref{pgbq}b) is felicitous even out-of-the-blue because it  conveys a  proposition  that is  readily recognized as being plausible according to world knowledge: attempting to fix $x$ can cause  damage to $x$.  If Subject Island effects are indeed contingent on how relevant the extracted subject-embedded referent is for the assertion expressed by the proposition, then a wide range of acceptable  patterns is to be expected, parasitic or otherwise. This includes cases like  (\ref{symbd}), where both gaps are in SI environments. As \citet{Levine:Sag:03}, \citet[256]{levhubook}  and \citet[161]{Culicover13} note, cases like (\ref{symbd}) should be completely unacceptable, contrary to fact.


\ea \label{symbd}
This is a man who $[$friends of \spcs $]$ think that $[$enemies of \spcs $]$ are everywhere.
\z 

The conclusion that SI effects are contingent on the particular proposition expressed by the utterance 
and its pragmatics thus seems unavoidable \citep{chavesresp}. In order to test this hypothesis, \citet{chavesresp} examine  the acceptability of sentences like (\ref{pairs}), which crucially  express nearly-identical truth conditions and have equally acceptable declarative counterparts. This way, any source of acceptability contrast must come from the extraction itself, not from the felicity of the proposition. 

\eal  \label{pairs}
\ex Which country does the King of  Spain resemble [the President of \spcs]?
\ex Which country does [the President of \spcs] resemble  the King of Spain?
\zl

\noindent
The results indicate that although the acceptability of the SI counterpart in (\ref{pairs}b)
is initially significantly lower than (\ref{pairs}a), it gradually improves.
After eight exposures, the acceptability of near-truth-conditionally-equivalent sentences like (\ref{pairs}) becomes non-statistically different.
What this suggests is that SI effects are at least in part probabilistic: the semantic, syntactic and pragmatic
likelihood of a subject-embedded gap likely matters for how acceptable such extractions are. This is 
most consistent with the claim that -- in general -- extracted phrases must correspond to the informational focus
of the utterance  \citep{shir-jrn,valin86,kuno87,ken,Dean,goldberg13}, and in particular with the intuition that
SI violations are weaker when the extracted referent is relevant for the main predication \citep[495]{kluender06}.






\section{Adjunct islands}

\citet{Ross67} and \citet{cattell}  noted that adjunct phrases  often resist extraction, 
as illustrated in  (\ref{tad}), a phenomenon usually referred to 
as \emph{The Adjunct Island Constraint} (AIC). 

\eal  \label{tad}
\ex[*]{What$_i$ did John die [whistling \spc$_i$]? }
\ex[*]{What$_i$ did John build [whistling \spc$_i$]? }
\ex[*]{Which club$_i$ did John meet a lot of girls [without going to \spc$_i$]?\\
\citep[38]{cattell}}
\ex[*]{Who$_i$ did Mary cry [after John hit \spc$_i$]?\\ 
\citep[503]{huang82}}
\zl

Although a constraint on {\sc slash} could effectively ban  all extraction from adjuncts,  the problem is that the AIC has a long history of exceptions,  noted as early as \citet[38]{cattell}, and by many others since, including 
\citet[72]{chomsky82}, \citet{engdahl}, \citet[103]{hegarty90}, \citet{cinque}, \citet{pollardsag}, 
\citet[253]{culicover87}, and \citet{borg}. A sample of representative counterexamples
is provided in (\ref{catt}). 

% Kayne 1983,
% Jones 1991, 
%Manzini 1992, 
%Levine & Sag 2003, 
%Haider 2004,
% Truswell 2007, 2009)
% Chaves


\eal \label{catt}
\ex Who did he buy a book [for \spc]?
\ex Who would you rather [sing  with \spc]?
\ex What temperature should I wash my jeans [at \spc]?
\ex That's the symphony that Schubert [died without finishing \spc].
\ex Which report did Kim [go to lunch without reading \spc]?
\ex A problem this important, I could never [go home without solving \spcs first].
\ex  What did he [fall asleep  complaining about \spc]?
\ex  What did John [drive Mary crazy trying to fix \spc]?
\ex Who did you [go to Girona in order to meet \spc]?
\ex Who did you go to Harvard [in order to work with \spc]?
\zl

 Exceptions to the AIC include  tensed adjuncts, as noted by \citet[88]{grosu81}, \citet[29]{deane}, \citet{kluender}, \citet[287]{levhubook},  \citet[144]{gold06},  \citet[471]{chavesextr}, \citet[175, ft.\,1]{truswellbook} and others. A sample is provided in (\ref{gr}).\footnote{\citet{truswellbook} argues that the AIC and its exceptions are best characterized in terms of event-semantic constraints, such that the adjunct must occupy an event position in the argument structure of the main clause verb. However, recent experimental research has been unable to validate Truswell's acceptability predictions
 \citep{kohrt}, and moreover,  such an account incorrectly predicts that extractions from tensed adjuncts is impossible \citep[175, ft.\,1]{truswellbook}.}
 
% Kohrt, Annika, Trey Sorensen, Dustin A. Chacn. 2018, to be submitted. The real-time status of semantic exceptions to the adjunct island constraint.
 

\eal \label{gr}
  \ex These are the pills that Mary died [before she could take \spc].
   \ex This is the house that Mary died [before she could sell \spc].
\ex  The person who I would kill myself [if I couldn't marry \spc] is Jane.
\ex Which book  will Kim understand linguistics better [if she reads \spc]?
\ex  This is the watch that I got upset [when I lost \spc].
\ex   Robin, Pat and Terry were the people who I lounged around at home
all day [without realizing were \spcs coming for dinner].
\ex Which email account would you be in trouble [if someone broke into \spc]?
\ex  Which celebrity did you say that [[the sooner we take a picture of \spc ],
[the quicker we can go home]]?
\zl

\noindent 
To be sure, some of these  sentences are complex and difficult to process,  
which in turn can lead speakers to prefer the insertion of an ``intrusive'' resumptive 
pronoun at the gap site, but they are certainly more acceptable than the 
classic tensed AIC violations examples like Huang's (\ref{tad}d).
Acceptable tensed AIC violations are more frequent in languages like Japanese, Korean,  and Malayalam.

Like Subject Islands, AIC violations sometimes improve ``parasitically'' in the presence of a second gap as in (\ref{parasaic}). First of all, note that these sentences express radically different propositions, and so there is no reason to assume that all of these are equally felicitous.
Second, note that (\ref{parasaic}a, c) describe plausible  states of affairs in which it is clear what the extracted referent has to do with the main predication and assertion, simply because of the fact that \emph{document}  is predicated by \emph{read}.  In contrast, (\ref{parasaic}b) describes an unusual state of affairs in that  it is unclear what the extracted referent  has to do with the main predication \emph{read the email}, out of the blue. Basically, what does reading emails have to do with filing documents?

\eal \label{parasaic}
\ex[]{Which document did John read \spcs before filing \spc? }
\ex[*]{Which document did John read the email email before filing \spc?}
\ex[]{Which document did John read \spcs before filing a complaint?}
\zl

\noindent
If AIC violations were truly only salvageable parasitically, then counterexamples
like (\ref{books}a) should not exist.  As 
\citet{Levine:Sag:03} and
\citet[256]{levhubook}
note, both gaps reside in island environments and should be 
completely out and less acceptable than (\ref{books}b, c), contrary to fact.

\eal  \label{books}
\ex[]{What kinds of books do [the authors of \spc] argue about royalties [after writing \spc]?}
\ex[*]{What kinds of books do [authors of \spc] argue about royalties after writing malicious pamphlets?}
\ex[*]{What kinds of books do authors of malicious pamphlets argue about royalties
      {}[after writing \spc]?
}
\zl

\noindent
In (\ref{books}a), there is no sense in which the gap inside the subject is parasitic on the gap inside the  adjunct, or vice-versa -- under the assumption that neither gaps is supposed to be licit without  the presence of a gap outside an island environment. In conclusion, the notion of parasitic 
gap is rather dubious. See \citet{levhubook} for a more in-depth discussion of parasitism and 
empirical criticism of null resumptive pronoun accounts.

As in the case of other island phenomena discussed so far, it is doubtful  that any purely syntactic account can describe all the empirical facts.  Rather,  extractions out of adjuncts are licit to the degree that the extracted
referent can be interpreted as being relevant for the assertion.


%\subsection{constraints on preposition stranding.
%A (brief) presentation of the way they have been/are treated in movement based approaches would be useful. 

% \subsection{Resumption}
%\section{Factive Islands}

\section{Superiority effects}

Contrasts like those below have traditionally been taken to be due to a constraint
that prevents a given phrase from being extracted if another phrase in a higher
position can be extracted instead \citep{chomsky73,chomsky80}. Thus, the highest \emph{wh}-phrase
is extractable, but the lowest is not.

\eal  \label{badsc1}
\ex[]{Who \spcs saw what?}
\ex[*]{What did who see \spc?}
\zlcont

\ealcont  \label{badsc2}
\ex[]{Who did you persuade \spcs to buy what?}
\ex[*]{What did you persuade who to buy \spc?}
\zl

Several different kinds of exceptions to this \emph{Superiority Constraint} (SC) have been 
noted in the literature. First, it is generally recognized that \emph{which}-phrases are
immune to the SC:

\eal \label{book}
\ex I wonder which book which of our students read \spc over the summer?
\ex Which book did which professor buy \spc?
\zl

\noindent
 \citet{pesetskydlink} proposed to explain the lack of SC effects in (\ref{book}) 
by stipulating that  \emph{which}-phrases are interpreted as indefinites which do not undergo LF movement. Rather, they require ``D-linking'' and obtain wide scope via an entirely different semantic mechanism called unselective binding. In order for a phrase to be D-linked, it must be associated with a salient set of referents. But as \citet[248ff.]{ginzsag} note, there is no independent evidence
for saliency interpretational differences between \emph{which} and other \emph{wh}-phrases like
\emph{what} and \emph{who}. For example, it is implausible that speakers have a specific referent
in mind for the \emph{which}-phrases in examples like (\ref{which}).

\eal  \label{which}
\ex[]{ I don't know anything about cars. Do you have any suggestions about which car -- if any -- I should buy when I get a raise?}
\ex[]{ I don't know anything about cars. Do you have any suggestions about what  -- if anything -- I should buy when I get a raise?\\
\citep[248]{ginzsag}}
\zl



Furthermore, there are acceptable SC violations involving multiple \emph{wh}"=questions such as those in (\ref{multiple}). See \citet{Bolinger78}, \citet{kayne83} and \citet[109]{pesetskydlink} for more
such examples and discussion.\footnote{\citet{gibson10} and others have found no evidence
that the presence of a third \emph{wh}-phrase improves the acceptability of a multiple
interrogative, even with supporting contexts. However, the examples in (\ref{multiple}) 
require peculiar intonation phrasings and contours, which may be difficult to elicit with written stimuli.} 

\eal \label{multiple}
\ex Who wondered what {\sc who} was doing \spc?
\ex What did {\sc who} take \spcs {\sc where}?
\ex Where did {\sc who} take {\sc what} \spc?
\zl

\noindent 
Finally, there are also SC violations that involve echo questions like (\ref{break1}) and reference
questions like (\ref{break2}). See \citet[Chapter 7]{ginzsag} for a detailed argumentation that echo questions are not fundamentally different, syntactically or semantically,   from other uses of interrogatives.

\eal \label{break1}
\ex What did Agamemnon break?
\ex What did {\sc who} break \spc?
\zlcont

\ealcont  \label{break2}
\ex What did he break?
\ex What did {\sc who} break \spc?
\zl

There are two different, yet mutually consistent, possible explanations for SC effects in HPSG circles.
 One potential factor concerns  processing difficulty \citep{arnon07}. Basically,  long-distance dependencies where a \emph{which}-phrase is fronted are  generally  more acceptable and faster to process than those where \emph{what} or \emph{who} if fronted, presumably  because the latter are semantically less informative, and thus decay from memory faster,  and are compatible  with more potential gap sites before the actual gap.
 The second potential factor is prosodic in nature. Drawing from insights by 
\citet[170--172]{Ladd96} about the
English interrogative intonation, \citet[251]{ginzsag} propose that in a multiple \emph{wh}-interrogative construction, all \emph{wh}-phrases must be in focus except  the first. Crucially, focus is typically -- but not always --  associated with clearly discernible pitch accent. Thus, (\ref{badsc1})  and (\ref{badsc2}) are odd because  the second \emph{wh}-word is unnaccented. In this account, a word like \emph{who} has two possible lexical descriptions, shown in (\ref{whoavm}).


\eal \label{whoavm}
\ex[]{\emph{Ex situ} interrogative \emph{who}:\\*
{\small \begin{avm}
 \[phon & \< \textup{who}/WHO \>\\
  synsem & \[loc  \[cat & \[head & noun\\
                                           spr & \< \>\\
                                                 comps & \< \>\]\\
                          cont  & \[
                                        ind & i\\ 
                                         restr & \{ \}\]\\
                          store & \{ \@{1}\[
                                                    ind & $i$\\ 
                                                       restr & \{ $person(i)$ \} \]  \}   \]\\
                nonloc  \[ wh &  \{ \@{1} \}\\
                  rel & \{ \}\\
                  slash & \{ \} \]\]\\
                                                                                     arg-st & \< \>
                  \]
\end{avm}}}


\ex[]{ Optionally \emph{ex situ} interrogative \emph{who}:\\*
{\small \begin{avm}
 \[phon & \< WHO \>\\
  synsem & \[loc \[cat & \[head & noun\\
                                            spr & \< \>\\
                                                 comps & \< \>\]\\
                          cont  & \[
                                        ind & i\\ 
                                         restr & \{ \}\]\\
                          store & \{\[
                                                    ind & i\\ 
                                                       restr & \{ $person(i)$ \} \]  \}   \]\\
                 nonloc \[wh &  \{ \}\\
                  rel & \{ \}\\
                  slash & \{ \} \]\]\\
                                                                                     arg-st & \< \>\]
\end{avm}}}

\zl

Since only the (optionally accented) lexical entry in (\ref{whoavm}a) is specified with a 
non-empty {\sc wh} value, the theory of  extraction proposed in \citet{ginzsag} predicts 
that (\ref{whoavm}a) must appear \emph{ex situ}.  In contrast, the accented lexical entry in
 (\ref{whoavm}b) is can  appear\emph{in situ}. For more discussion see  \citet[261]{levhubook}.

A related range of island phenomena concern extraction from \emph{whether}-clauses, 
which is traditionally assumed to be forbidden, as (\ref{whex}) illustrates.

\eal \label{whex}
\ex[*]{Which movie did John wonder whether Bill liked \spc?}
\ex[*]{Which movie did John ask why Mary liked \spc?}
\zl 

\noindent
But again, the oddness of (\ref{whex})  is unlikely to be due to syntactic constraints,
given the existence of passable counterexamples like (\ref{whexc}).

\eal \label{whexc}
\ex[]{He told me about a book which I can't figure out whether to buy \spcs or not.\\
\citep{Ross67}}
\ex[]{ Which glass of wine do you wonder whether I poisoned \spc?\\
\citep[81]{cresti1995extraction}}
\ex[]{ Who is John wondering whether or not he should fire \spc?}
\ex[]{ Which shoes are you wondering whether you should buy \spc?\\
\citep{chavesextr}}
\zl

As noted by \citet{kroch89}, the reduced acceptability of an example like (\ref{kro}a) is better explained simply
by noting the difficulty of accommodating its presupposition in (\ref{kro}b).


\eal \label{kro}
\ex How much money was John wondering whether to pay?
\ex There was a sum of money about which John was wondering whether
to pay it.
\zl


\section{The Left Branch Condition}

\citet{Ross67} discovered that the leftmost constituent of an NP cannot be extracted, 
as in  (\ref{lbcv}), a constraint he dubbed as the \emph{Left Branch Condition} (LBC).\footnote{As in other island environments discussed above, the LBC is not operative in constraining semantic scope, as illustrated below.

\ea
Someone took a picture of each student's bicycle.\\
\citep[303]{MRS}
\zlast}
 
 
\eal \label{lbcv}
\ex[*]{Whose$_i$ did you meet [ \spc$_i$ friend]?\\
 (cf. with `\emph{You met whose friend?}')}
 
\ex[*]{Which$_i$ did you buy [ \spc$_i$ book]?\\
(cf. with `\emph{You bought which book?}')}

\ex[*]{How much$_i$ did you find [ \spc$_i$ money]?\\
(cf. with `\emph{You found how much money?}')}
\zl


\noindent
These facts are accounted for in versions of HPSG like \citet{cxsag07} where Determiner Phrases (DPs) are not valents of the nominal head. If the DP is not listed in the argument structure of the nominal head, then there is no way for the DP to appear in {\sc slash}.  Rather, the DP selects the nominal head:

\ea
\begin{avm}
\[phon & $\langle$ the $\rangle$\\
synsem & \[loc & \[ cat & \[head & \[\tpv{determiner}\\
                                                       select N$'$\[ind & $i$\\ restr & $\lbrace$\@{1}$\rbrace$\]\]\\
                                        spr & $\langle \rangle$\\
                                        comps & $\langle \rangle$\\   \]\\
                             cont & \[ind  & $i$\\
                             restr &  $\lbrace \rbrace$ \]\\
                             store & \{ \[\asort{the-rel}\\ var & $i$\\ arg & \@{1}\] \} \] \\
                   non-loc & \[wh & $\lbrace \rbrace$\\
                     rel & $\lbrace \rbrace$\\
                     slash & $\lbrace \rbrace$\] \]\\
arg-st & $\langle \rangle$                    \]
\end{avm}
\zlast

\noindent\\
Analogously, in  \citet[133]{cxsag07},  genitive DPs combine with nominal heads and bind their {\sc x-arg} index via a dedicated  construction, not as valents.
For example,  in nominalizations like \emph{Kim's description of the problem} the  DP \emph{Kim's} is not a valent of \emph{description}, and  therefore the genitive DP cannot appear in {\sc slash}.
 Rather, gentive DPs are instead constructionally co-indexed with the agent role of the noun \emph{description}  via {\sc x-arg}.  
 Moreover, the clitic \emph{s} in \emph{Kim's} must lean phonologically on the NP it selects, and therefore cannot be stranded for independently motivated phonological reasons, predicting  the oddness of *\emph{It was Kim who I read 's description of the problem}.

There are various  languages  which do not permit extraction of left branches from noun phrases, 
but have a particular PP construction that appears to allow LBC violations.
This is illustrated below in (\ref{langslbc}), with French data. 

\eal
\ex \gll Combien$_i$  a-t-il  vendu  [\spc$_i$  de  livres]?\\
how-many  has-he  sold {} of  books\\
\glt `How many books did he sell?'

\ex \gll Quels$_i$  avez-vous  achet\'{e}  [\spc$_i$  livres]?\\
how-many  have-you  bought {}  books\\
\glt `How many books have you bought' 


%\ex \gll Quantos  livros$_i$  \'{e}  que  tens 
%[\spc$_i$  de  Matem\'{a}tica]?\\ 
%how-many books  is  that  you-have {} of  mathematics\\
%\glt `How many math books do you have?' \hfill (Portuguese)
 \zl \label{langslbc}


\noindent
But the LBC violation is only apparent. The {\it de livres} is in fact a post-verbal {\it de}-N$'$ nominal, and thus no LBC violation occurs  in (\ref{langslbc}a). See \citet{Abeille:Bonami:ea:04} for details.
Finally, \citet{Ross67} also  noted that some languages do not obey the LBC at all. A small sample is given in (\ref{lbcx}). However, the languages  in question lack determiners, and therefore it is possible that
the extracted phrase is has a similar independent status to the French {\it de-N}$'$ phrase in (\ref{langslbc}).
 
\ea
 \ea \gll Jak\k{a}$_i$  kupi\l{}e\'{s} { [ \spc$_i$ } ksi\k{a}\.{z}k\k{e}]\\
     what  you-bought  {} book\\
    \glt `Which book did you buy?' \hfill (Polish)
       
 \ex \gll Cju$_i$  citajes [\spc$_i$  knigu]?\\
 whose  you-are-reading {}  book\\
  \glt `whose book are you reading?' \hfill (Russian)
 
 \ex \gll Ki-nek$_i$  akarod,  hogy  halljam  {[\spc$_i$}  a  hang-j\'{a}-t]?\\
 who-{\sc dat}  you-want  that  I-hear {} the  voice-{\sc poss.3sg-acc}\\ 
\glt `Whose voice do you want me to hear?' \hfill (Hungarian)
 \z \label{lbcx}
 \z
 

 
 
 
\section{The Complementizer Constraint}

\citet{Perlmutter68} noted that subject phrases have different extraction properties than that of object phrases, as illustrated in (\ref{subobjasymm2}). The presence of the complementizer hampers extraction of the subject, but not of the complement.\footnote{There is no evidence that the Complementizer Constraint applies at the semantic level, however. The  subject phrase of the embedded clause that can outscope the subject phrase of  the matrix:

\eal
\ex Some teacher claimed that each student had cheated.
\ex Every teacher claimed that a student had cheated.
\zllast
} 

\eal   \label{subobjasymm2}
\ex[*]{[Who]$_i$ did Tom say (?that) \spc$_i$ had bought the tickets?}
%\item \,[Who] did he say \spcs bought the rutabaga?\\
\ex[*]{[Who]$_i$ do you believe (?that) \spc$_i$ got you fired? }
\ex[]{[The things]$_i$ that they believe (?that) \spc$_i$ will happen are disturbing to contemplate. }
\ex[*]{[Who]$_i$ did you ask if \spc$_i$ bought the tickets?}
%\item \,[Who] did he say \spcs bought the rutabaga?\\
\ex[*]{[Who]$_i$ do you expect for  \spc$_i$ to fire  you? }
\zl



%These examples are also found in L&H 2006, p. 63
\noindent
\citet{Bresnan:77} and others also noted that  Complementizer Constraint effects can be
reduced in the presence of an adverbial intervening between the complementizer and the gap:

\eal \label{advcirc}
\ex{[Who]$_i$ did Tom say that -- as far he could remember -- \spc$_i$ had bought the tickets?}
\ex{ [Who]$_i$ do you believe that -- for all intents and purposes --  \spc$_i$ got you fired? }
\ex{ [Who]$_i$ do you think that after years and years of cheating death \spc$_i$ finally died?}
\zl

\noindent
In \citet{bouma} and \citet{ginzsag}, extracted arguments are typed as \type{gap-ss} rather than \type{canon-ss}. Only the latter are allowed to correspond to \emph{in situ} signs and to reside in valence lists. However, subject extraction is different. If a subject phrase is extracted, then the list {\sc subj} contains the respective \type{gap-ss} sign. If one assumes that the lexical entry for the complementizer \emph{that} requires  S complements specified as $[${\sc subj} $\langle \, \rangle]$ then the oddness of  (\ref{subobjasymm2}) follows. For  \citet{bouma} and \citet{ginzsag}, the adverbial circumvention effect in (\ref{advcirc}) is the result of assuming that the construction which allows the adverb to combine with the clause forces the mother node to be [{\sc slash} $\lbrace$ $\rbrace$], a rather \emph{ad-hoc} account.\footnote{Except that in French, when the subject of the complement CP is extracted, the complementizer is {\it qui} instead of {\it que}, which could easily be captured by such an account.}

A simpler account of the  Complementizer Constraint has emerged recently, however, in principle compatible with any theory of grammar. For \citet{Kandy06,Kandy09} and others,  the Complementizer Constraint is prosodic in nature.  Complementizers must cliticize to the following phonological unit, but if a pause is made at the gap site then the complementizer cannot do so.  Accordingly, if the pronunciation of \emph{that} is
produced with a reduced vowel [\textipa{D@t}] rather than [\textipa{D\ae t}] then the
Complementizer Constraint violations in (\ref{subobjasymm2}) improve in acceptability. Though promising,
\citet{Richart} found no experimental evidence for amelioration of the Complementizer Constraint effects
either with phonological reduction of the complementizer or with contrastive focus. Further research is needed
to determine the true nature of  Complementizer Constraint effects.



\section{Island circumvention via ellipsis}

Ellipsis somehow renders island  constraints inactive, as in (\ref{slc}). A deletion-based analysis of such phenomena such as  \citet{merchantbook}  relies on moving the \emph{wh}-phrase before deletion takes place, but since  movement is assumed to be sensitive to syntactic island constraints, the  prediction is that (\ref{slc}) should be illicit, contrary to fact. 

\eal  \label{slc}
\ex[]{ Terry wrote an article about Lee and a book about someone else from
East Texas, but we don't know who$_i$ (*Terry wrote an article about \spc$_i$ Lee
and a book about \spc$_i$).\\
 \,[CSC violation]}

\ex[]{ Bo talked to the person who discovered something, but I still don't know
what$_i$ (*Bo talked to the person who discovered \spc$_i$). \\
\,[CNPC violation]}

\ex[]{ That he'll hire someone is possible, but I won't divulge who$_i$
(*that he'll  hire \spc$_i$ is possible).\\
\, [SSC violation]}

\ex[]{  She bought a rather expensive car, but I can't remember how expensive
(*she bought a \spcs car). \\
\,[LBC violation]}
\zl

The  account adopted in HPSG is one in which remnants  are assigned an interpretation based on the surrounding discourse context  \citep{ginzsag,Culicover:Jackendoff:05,jacobson08,sagn}. 
See \crossrefchaptert{ellipsis} for more detailed discussion. In a nutshell,  the \emph{wh}-phrases in (\ref{slc}) are ``coerced'' into a proposition-denoting clause via a unary branching construction that taps into contextual information.  This straightforwardly explains not only why the antecedent for the elided phrase need not correspond to  overt discourse --  e.g.\ sluices like \emph{What floor?} or \emph{What else?} --
but also  why the examples in (\ref{slc})  are immune to island constraints: there simply is no island
 environment to begin with, and thus, no extraction to violate it.



\section{Conclusion}

HPSG remains relatively agnostic about many island types, given the existence of robust exceptions.
It is however clear that many island effects are not purely due to syntactic constraints, and are more likely
the result of multiple factors, including pragmatics, semantics and processing difficulty.
To be sure, it is yet unclear how these factors can be brought together are articulate an explicit
and testable account of island effects. In particular, it is unclear how to combine probabilistic information with syntactic, semantic and pragmatic representations, although one fruitful avenue of approach to this problem may be via  \emph{Data-Oriented Parsing} \citep{NF2002a-u,NF99a,Arnold:Linardaki:07,BSS2003a-ed,Bod2009a}.  


From its inception, HPSG has been meant to be compatible with models of language comprehension and production \citep{sagser,Sag:Wasow:ta,Sag:Wasow:ta2}, but not much work has been dedicated to bridging these worlds; see \crossrefchaptert{processing}. The challenge that island effects posit to any theory of grammar is central to linguistic theory and cognitive science: how to integrate theoretical linguistics and psycholinguistic models of on-line language processing so that fine-grained predictions about variability in acceptability judgements across nearly isomorphic clauses can be explained.


 
%\section*{Abbreviations}
\section*{Acknowledgements}

Many thanks to Bob Borsley, Berthold Crysmann, and Anne Abeillé for
detailed comments about an earlier draft, and to Stefan Müller for
invaluable editorial assistance.

{\sloppy
\printbibliography[heading=subbibliography,notkeyword=this] 
}
\end{document}

