\documentclass{article}
\usepackage{forest}

\usepackage{memoize}
\memoizeset{memo dir}

\begin{document}

After loading the package with
\begin{center}
  {\tt \string\usepackage{memoize}}
\end{center}
we wrote
\begin{center}
  {\tt \string\memoizeset\{memo dir\}}
\end{center}
and now, the memo ({\tt B2E679FB208DD0D53C20B5017B4A8DAA.memo.pdf}) will appear
in folder {\tt 2-memodir.memo.dir}.

\begin{center}
  \begin{forest}
    [VP
      [DP]
      [V\rlap'
        [V]
        [DP]
      ]
    ]
  \end{forest}
\end{center}

Most of user's interaction with Memoize is through macro {\tt
  \string\memoizeset}.  It can be used anywhere in the document, but some keys
only make sense in the preamble.  The effects of {\tt \string\memoizeset}
persist until the end of the \TeX\ group.

The argument of {\tt \string\memoizeset} is a {\tt pgfkeys} keylist.  In short:
\begin{itemize}
\item A keylist is a comma-separated list of {\tt <key>=<value>} pairs, where
  the {\tt =<value>} part is not always mandatory.
\item In the keylist, spaces around keys and values are stripped away. Newlines
  inside a keylist are allowed, but empty lines are not.
\item Characters {\tt ,}, {\tt =} and {\tt /} have special meaning in a
  keylist.  If you want to include one of them in the value, surround the value
  by braces, like this: {\tt <key>=\{<value>\}}
\end{itemize}
For details, see section 88 of the PGF/TikZ manual.

\end{document}


%%% Local Variables:
%%% mode: latex
%%% TeX-master: t
%%% End:
